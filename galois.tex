\chapter{Galois Theory}
\label{Galois:Chap}

\begin{proposition}
If $K \supset E \supset k$ are fields and $K$ is a normal extension of $k$
then $K$ is a normal extension of $E$.
\end{proposition}


Let $G(L/E) = G$ be the Galois group of $L$ over $E$ and $K$ be any
intermediate field $E \varsubsetneq K \varsubsetneq L$.\footnote{For a more
detailed discussion of Galois theory see \cite{Lang:Algebra}} $L$ is Galois
over $K$ and the Galois group of $L/K$ is a subgroup of $G$.  Let $H$ be a
subgroup of $G$.  Then the fixed field of $H$, $L^H$, is a subfield of $L$
and $L$ is Galois over $L^H$ with Galois group $H$.  If $H$ is a normal
subgroup of $G$, then $L^H$ will be Galois over $E$ and the Galois group of
$L^H/E$ will be $G/H$.  There is a one to one correspondence between the
subfields of $L$ containing $E$ are the subgroups of the Galois group of
$L$ over $E$.

In particular, if $E[\beta]$ lies between $F$ and $E$ as shown in
\figref{Decomposition:Fields:Fig}, then $L$ will be Galois over
$E[\beta]$ and $G(L/F) \varsubsetneq G(L/E[\beta])$.  Our approach to
finding a subfield of $F$ is to find a proper subgroup of $G(L/E)$ that
contains $G(L/F)$.

We illustrate this with a very simple example.  Let $f(x)= (x^7-1)/(x-1)$,
which is normal over $\Q$.  Let $\zeta_7$ denote a primitive seventh root
of unity.  Then $f(x)$ factors over $\Q[\zeta_7]=L$ into
\[
f(x) = (x - \zeta_7) (x - \zeta_7^2) (x - \zeta_7^3) (x - \zeta_7^4) (x -
\zeta_7^5) (x - \zeta_7^6).
\]
The Galois group of $L/\Q$ is cyclic of order $6$.  The action of each
element of $G(L/\Q)$ is characterized by its effect on $\zeta_7$.  We can
denote the elements of $G(L/\Q)$ by $\sigma_i(\zeta_7) = \zeta_7^i$, where
$\sigma_1$ is the identity.  $G(L/\Q)$ is generated by $\sigma_3$, while
$\sigma_2$ generates a subgroup of order $3$, which we will call $H$.
\[
\begin{aligned}
& \zeta_7 \mapsto^{\sigma_3} \zeta_7^3 \mapsto{\sigma_3} \zeta_7^2
\mapsto{\sigma_3} \zeta_7^6 \mapsto{\sigma_3} \zeta_7^4
\mapsto{\sigma_3} \zeta_7^5 \mapsto{\sigma_3} \zeta_7 \\ & \zeta_7
\mapsto{\sigma_2} \zeta_7^2 \mapsto{\sigma_2} \zeta_7^4
\mapsto{\sigma_2} \zeta_7
\end{aligned}
\]

The subfield of $L$ fixed by $H$, $L^H$ is of degree $2$ over $\Q$ and
$\fieldDegree{L}{L^H}=3$.  The fixed field of $H$ contains all symmetric
functions in $\{ \zeta_7, \zeta_7^2, \zeta_7^4 \}$.  Therefore, the
coefficients of
\[
(x - \zeta_7)(x - \zeta_7^2)(x - \zeta_7^4) = x^3- (\zeta_7 + \zeta_7^2 +
\zeta_7^4)x^2 + (\zeta_7^3 + \zeta_7^5 + \zeta_7^6)x - 1
\]
are in $L^H$.  Similarly the coefficients of
\[
(x - \zeta_7^3)(x - \zeta_7^5)(x - \zeta_7^6) = x^3- (\zeta_7^3 + \zeta_7^5
+ \zeta_7^6)x^2 + (\zeta_7 + \zeta_7^2 + \zeta_7^4)x - 1
\]
lie in $L^H$.  Thus
\[
L^H = \Q[\zeta_7 + \zeta_7^2 + \zeta_7^4, \zeta_7^3+\zeta_7^5+\zeta_7^5] =
\Q[\beta_1, \beta_2].
\]
where $\beta_1$ and $\beta_2$ are the conjugate roots of the polynomial
$x^2+x+2$.

\section{Kummer Theory}
\label{Kummer:Sec}

The first work by computer algebraists on simplifying expressions
involving radicals was that of {\Caviness}
\cite{Caviness:Thesis,Caviness:Canonical} and later {\Fateman}
\cite{Fateman:Thesis}.  This work was revised and summarized in
{\Caviness} and {\Fateman} \cite{Caviness:Fateman}.  Both {\Caviness}
and {\Fateman} base their results on simplification of algebraic
expressions on the following results.  Let $k$ be a field, $k(\theta)$
an algebraic extension, contained in an algebraically closed field
$L$.  Let $\alpha$ be an element of $L$ that has $g(x)$ as its minimal
polynomial over $k$.  If $g(x)$ is irreducible over $k(\theta)$ then
$\fieldDegree{k(\theta,\alpha)}{k(\theta)}$ is equal to the degree of
the polynomial $g(x)$, $\deg g$, and the monomials in $\theta$ and
$\alpha$ of low degree are linearly independent.  Thus if we are able
to show that $x^2 - 2$ is irreducible over $\Q(\sqrt{3})$, we will
know that $\{1, \sqrt{2}, \sqrt{3}, \sqrt{2} \sqrt{3}\}$ is a
$\Q$-linear basis.

In general, {\Caviness} and {\Fateman} need to find conditions for
which $x^n - a$ is irreducible over $k$, over some ground field $k$.
To do this they make use of the following theorem which was first
proved by {\Capelli} \cite{Capelli} for $\Char k = 0$, and by {\Redei}
\cite{Redei} in general.

\begin{proposition}[{\Capelli}] \label{Capelli:Prop}
Let $k$ be a field and $n$ an integer.  Let $a$ be a non-zero element
of $k$.  Assume for all prime numbers $p$ that divide $n$ we have $a
\notin k^p$, and if $4$ divides $n$ then $a \notin -4k^4$.  Then $X^n
- a$ is irreducible in $k[X]$.
\end{proposition}

\noindent
The special case for $p=4$ is due to the factorization 
\[
x^4+4 = (x^2-2x+2)(x^2+2x+2).
\]

{\Caviness} uses this result to prove the following proposition.
{\Capelli}'s theorem implies that $\fieldDegree{k(\root n\of{a})}{k} =
n$ if $a$ satisfies the conditions of the theorem.  Needless to say,
these conditions are easier to verify than proving that $X^n -a$ is
irreducible.

\begin{proposition}
Let $\ell$ be a positive integer, $m$ an odd positive integer, and
$p_1, \ldots, p_k$ distinct positive prime integers.  Then the field
$\Q(\omega_{\ell}, \root m \of {p_1}, \ldots, \root m \of {p_k})$ is
of degree $m^k$ over $\Q(\omega_{\ell})$ where $\omega_{\ell}$ is a
primitive $\ell$\th root of unity.
\end{proposition}

{\Besicovitch} \cite{Besicovitch:Radicals} proved a version of this
theorem that does not include roots of unity.  A simple proof using
Galois theory has been presented by {\Richards} \cite{Richards:Radicals}.

\begin{proposition}
Let $n$ be a positive integer and $p_1, \ldots, p_m$ distinct positive
prime integers.  Then the field $\Q(\root n\of{p_1}, \ldots, \root
n\of{p_k})$ is of degree $n^k$ over $\Q$.
\end{proposition}

\noindent
{\Fateman} strengthened this by letting the $p_i$ be pairwise
relatively prime polynomials and leaving out the roots of unity.

\begin{proposition}
Let $n$ be a positive integer, and $B_1,\ldots, B_k$ be non-constant,
square-free pairwise relatively prime polynomials over the ring
$\Q[x_1, \ldots, x_m]$.  Then the field $\C(x_1, \ldots, x_n)(\root n
\of {B_1}, \ldots, \root n \of {B_k})$ is of degree $n^k$ over
$\C(x_1, \ldots, x_m)$ and $\root n \of {B_i}$ denotes any one of the
roots of $y^n - B_i = 0$.
\end{proposition}

In this paper we will make use of the more powerful results of
{\Kummer} theory to remove the non-nested restrictions of the previous
theorems and to strengthen them somewhat.  In particular we will
present a simple algorithm for generating a linearly independent basis
for an arbitrary radical extension.

\medskip
Let\Marginpar{Transition here} $r_1, \ldots, r_m$ be rational
integers, $k$ a field with characteristic prime to the $r_i$. Consider
the elements of the field $K = k(\alpha_1^{1/r_1}, \ldots,
\alpha_m^{1/r_m})$ where the $\alpha_i$ are distinct and different
from 1.  $K$ can be considered to be a $k$-vector space of dimension
$n = \prod_i r_i$ with basis
\[
\{\,\alpha_1^{s_1/r_1} \cdots \alpha_m^{s_m/r_m} \mid 0 \le s_i < r_i\,\}.
\]
However, the degree of $K$ over $k$ can be less than $n$.  For
instance, $k(\sqrt{2}, \sqrt{3}, \sqrt{6})$ has degree 4 over $k$
since $\sqrt{6} = \sqrt{2} \sqrt{3}$.  In general, if $\{\omega_i\}$
is this naive basis, any of $\{\,a_1 \omega_1 + \cdots + a_n \omega_n
\mid a_i \in k\,\}$ could be zero.  What we prove in this section is that
if the $\omega_i$ are multiplicatively independent over $k$ then they
are linearly independent.  Equivalently, all linear dependencies are
generated by linear dependencies of two terms.

\sectref{Group:Cohomology:Sec} presents the basic techniques that underlie
group cohomology, which we use later.  The key propositions upon our
results on the simplification of radicals are part of Kummer theory.  In
the \sectref{Basic:Kummer:Theory:Sec} we shall summarize the relevant
results we shall need and prove the general structure theorem upon which
this section is based.  For a more thorough study of Kummer theory we
suggest \cite{Artin:Tate,Lang:ANT,Serre:Corps:Locaux}.  An elementary
version of the proof of the main theorem may be found in the beautiful
monograph by Artin \cite{Artin:Galois}.

The \sectref{General:Reduction:Sec} presents the basic algorithm we use to
determine the independent basis, and \sectref{Kummer:Example:Sec} gives
some examples of its use.  The reader unfamiliar with Galois theory might
find it easier to skip immediately to \sectref{General:Reduction:Sec} for
the main results, skipping the proof of \propref{Kummer:Radical:Prop}.

\section{Group Cohomology}
\label{Group:Cohomology:Sec}

Group cohomology is a deep and beautiful branch of mathematics.
Originally introduced by {\Grothendieck} \cite{Grothendieck:Tohoku},
it has been put to important use in algebraic number theory and
algebraic geometry.  Presentations we have found informative include
those of {\Lang} \cite{Lang:ANT,Lang:Cohomology}, {\SchatzS}
\cite{Schatz:Profinite} and {\Serre} \cite{Serre:Corps:Locaux}.  In
order to make the presentation here somewhat self contained we have
simplified many of the results.  In particular, we have assumed the
Galois groups used are finite rather than profinite.

We begin with some algebraic definitions.  Let $A$, $B$ and $C$ be
modules and let $f$ and $g$ be homomorphisms
\[
A \mapsto{f} B \mapsto{g} C.
\]
This sequence of modules and homomorphisms is said to be {\em exact} at $B$
if the image of $f$ ($f(A)$) is equal to the kernel of $g$ (elements of $B$
mapped to the identity in $C$).\index{sequence! exact} We use $0$ to
represent the module of one element, the identity.  If the sequence
\[
0 \longrightarrow A \mapsto{f} B \mapsto{g} C \longrightarrow 0
\]
is exact then the map $f$ is an injection (only the the identity of $A$
is mapped to the identity in $B$) and $g$ is a surjection (every element
of $C$ is the image of some element of $B$).

\medskip 
The goal behind \key{group cohomology} is to produce a map $\delta$ and
a \keyi{$\delta$-functor} $H^{-}(-)$ with the following property: Given an
exact sequence of modules:
\[
0 \longrightarrow A \longrightarrow B \longrightarrow C \longrightarrow 0
\]
we can compute a corresponding infinite exact sequence of groups
\[
 \cdots \longrightarrow H^{p-1}(C) \mapsto{\delta} H^p(A) \longrightarrow
H^p(B) \longrightarrow H^p(C) \mapsto{\delta} H^{p+1}(A)
\longrightarrow \cdots.
\]
The structure of the cohomology groups gives a great deal of insight into
the modules.  The exact sequence they are part is a powerful tool for
proving results about the structure of the modules.  The main theorems of
Kummer theory result from degree arguments on a small section of a
cohomology sequence.

We begin by defining the cohomology groups.  Let $G$ be a finite
multiplicative group and $A$ an additive group upon which $G$ acts.  We
call $A$ a {\em $G$-module} if there exists a map $G \times A \rightarrow
A$ that sends $(\sigma, a)$ to $\sigma a$ satisfies the following
conditions:

\begin{enumerate}
\item $1 \cdot a = a$ where $1$ is the identity element of $G$,
\item $\sigma(a + a') = \sigma a + \sigma a'$,
\item $(\sigma \tau)(a) = \sigma(\tau a)$.
\end{enumerate}

If $K$ is a Galois extension of $k$ with Galois group $G$, the action of
$G$ on $K$ induces a $G$-module structure on $K$.  The elements of $G$ will
be denoted by $\sigma_i$.

Let $G^n$ denote the cartesian product of $G$ with itself $n$ times.
$G^0$ is the set with one element, $\{\phi\}$.  Assume that $A$ is a
$G$-module.  We define the {\em
$n$-cochains}\index{cochains@$n$-cochains} of $A$, $C^n(G, A)$ to be 
the set of maps from $G^n$ to $A$.  The cochains form a group under
the addition operation
\[
(f + g)(\sigma_1, \ldots, \sigma_n) = 
f(\sigma_1, \ldots, \sigma_n) + g(\sigma_1, \ldots, \sigma_n),
\]
where $f$ and $g$ are cochains.  The element of $C^n(G, A)$ that maps
$G^n$ to the identity of $A$ is the identity element of the group.  
Thus $C^0(G, A)$ is isomorphic to $A$.

Given an exact sequence 
\[
0 \longrightarrow A \longrightarrow B \longrightarrow C \longrightarrow 0
\]
it is easy to show that there is a sequence of exact sequences
\[
0 \longrightarrow C^n(G, A) \longrightarrow C^n(G, B) \longrightarrow
C^n(G, C). 
\]

To construct the infinite cohomology sequence we want to find a map
that connects $n$-cochains to $n+1$-cochains.  The map we are looking
for is denoted by $\delta$.  Define the map $\delta : C^n(G, A)
\rightarrow C^{n+1}(G,A)$ by
\[
\displaylines{(\delta f)(\sigma_1, \ldots, \sigma_{n+1}) =
\sigma_1 f(\sigma_2, \ldots, \sigma_{n+1})\hfill\cr
\hfill{} + \sum_{j=1}^n (-1)^j f(\sigma_1, \ldots, \sigma_j \sigma_{j+1}, 
\ldots, \sigma_{n+1}) + (-1)^{n+1} f(\sigma_1, \ldots, \sigma_n).\cr}
\]
This map has been carefully chosen so that $\delta^2 f = 0$

By an {\em $n$-cocycle} \index{cocyle} we mean an element of the kernel
of $\delta : C^n \rightarrow C^{n+1}$ and by an {\em $n$-coboundary}
\index{coboundary}an element of the image of $\delta : C^{n-1}
\rightarrow C^n$.  The $n$-cocycles
($Z^n (G, A)$) and $n$-coboundaries ($B^n(G, A)$) are groups.  The
$0$-cocycle group is defined to be the trivial group.
Since $\delta^2 f = 0$, the $n$-coboundary group is a subgroup of the 
$n$-cocycle group.  The {\em $n$\th\ cohomology group} ($H^n(G,A)$)
\index{cohomology group} is the quotient group of the cocycles by the
coboundaries.

Let $f$ be a $0$-cochain.  Then $\delta f$ is a $1$-cochain and is
defined by
\[
(\delta f)(\sigma) = \sigma f() - f().
\]
$f$ is a $0$-cocycle if $\delta f = 0$.  That is, for all $\sigma \in G$
\[
\sigma f() - f() = 0.
\]
This means that the element of $A$ that corresponds to the cocycle $f$
is invariant under $\sigma$.  Thus the $0$-cocycle group is isomorphic
to the group of elements of $A$ that are fixed by $G$, which is
written $A^G$.  Since the $0$-coboundary group is trivial, we have
\[
C^0(G, A) \simeq H^0(G, A) \simeq A^G.
\]

The $1$-cocycles and $1$-coboundaries are more complex.  $f$ is a 1-cocycle if
\[
0 = (\delta f)(\sigma , \tau) = \sigma f(\tau) - f(\sigma \tau) +
f(\sigma),
\]
\ie, $f(\sigma \tau) =  \sigma f(\tau ) + f(\sigma)$.  
If we identify the $0$-cochains with the elements of $A$ then $f$ is
a $1$-coboundary if $f(\sigma) = \sigma a - a$ for every $\sigma$ and some
fixed $a \in A$ (a $0$-cochain) corresponding to $f$.

\medskip
If $k$ is a field, then $k^{\ast}$ will be used to represent
the multiplicative subgroup of  $k$, \ie, the non-zero elements of $k$.
We shall let $k^n$ represent the set of $n$\th\ powers of elements 
of $k$.  Notice that $G^n$ is the cartesian product when $G$ is a group.

Let $K$ be an algebraic extension of $k$ and $G$ the Galois group of
$K$ over $k$.  There are two $G$ modules of $K$ that lead to
interesting cohomologies.  The additive group of $K$ is a $G$ module
and leads to a cohomology sequence $H^p_{+}(G, -)$.  For our purposes
we will be more interested in the $G$ module formed by the
multiplicative group $K^{\ast}$.

Now the cocycle and coboundary conditions need to be written
multiplicatively rather than additively.  Let $f \in C^0(G,
K^{\ast})$.  Then $f$ is a $0$-cocycle if
\[
1 = (\delta f)(\sigma) = {\sigma f() \over f()}.
\]
Once again, we see that the $0$-cocycles are isomorphic to the elements
of $K^{\ast}$ left fixed by $G$ and $H^0(G, K^{\ast}) = K^{\ast G}$.

The $1$-cocycles are those 
$f \in C^1(G, K^{\ast})$ such that
\[
1 = (\delta f)(\sigma, \tau) 
    = {\sigma f(\tau) \over f(\sigma\tau)} f(\sigma),
\]
\ie, $f(\sigma\tau) = (\sigma f(\tau)) f(\sigma).$
$f$ is a $1$-coboundary if there exists an $a$ in $K^{\ast}$
corresponding to $f$ such that for every $\sigma$, $f(\sigma) = \sigma a / a$.
This is the multiplicative formulation of group cohomology.

Now that we have complete characterizations of the zeroth cohomology
groups, our next problem is to study the first cohomology groups.  The
first major result is a famous theorem due to {\Hilbert}
\cite{Hilbert:ANT} that characterizes the first cohomology group when
$G$ is cyclic.

\begin{proposition}[Hilbert's theorem 90]
\label{Hilbert:Ninety:Prop}
Let $K/k$ be a cyclic normal extension and $\sigma$ a generator of the
Galois group of $K$ over $k$.  If an element $a$ of $K$ has norm $1$
there is an element $b$ of $K$ such that $a= b/\sigma b$.
\end{proposition}

\begin{proof}
\end{proof}

What does this say about the cohomology group of $K^{\ast}$?
Let $\sigma$ be a generator of $G$ and $f$ be a $1$-cocycle.  Then
\[
f(\sigma^r) = f(\sigma \sigma^{r-1}) = \sigma(f(\sigma^{r-1}))
f(\sigma).
\]
$f$ is fully specified by the value of $f(\sigma)$, which is an element
of $K^{\ast}$.  Let the degree of $K$ over $k$ be $n$.  Then 
$\sigma^n = 1$ and
\[
\begin{aligned}
1 = f(\sigma^n) &= \sigma(f(\sigma^{n-1})) f(\sigma) \\
&= \sigma^2(f(\sigma^{n-2})) \sigma(f(\sigma)) f(\sigma) \\
& = \sigma^{n-1}(f(\sigma)) \sigma^{n-2}(f(\sigma)) \cdots f(\sigma)\\
& = \Norm(f(\sigma))
\end{aligned}
\]
Thus the $1$-cocycles are those cochains that map $\sigma$ to an element
of $K$ with norm $1$.  The statement that this element can be written in
the form $b / \sigma b$ is equivalent to saying that $f$ is also a
$1$-coboundary.  Thus Hilbert's theorem 90 says that $H^1(G, K^{\ast}) =
0$ when $G$ is cyclic.  Noether's generalization is that the first
cohomology group is zero as long as $K/k$ is normal.

\begin{proposition}
Let $L/K$ be a normal extension.  If a map $\sigma \rightarrow
f(\sigma)$ of $G(L/K)$ into $L^{\ast}$ satisfies the condition
$f(\sigma \tau)= f(\sigma) \sigma f(\tau)$ for all $\sigma, \tau \in
G$ then there is an element $b$ of $L^{\ast}$ such that $f(\sigma) = b
/ \sigma b$.  That is $H^1(G,L^{\ast}) = 0$.
\end{proposition}

\begin{proof}
\end{proof}

The following corollary is often useful.

\begin{corollary}
If $k$ contains a primitive $n$\th\ root of unity $\zeta$, and $K$ is
a cyclic normal extension of $k$ of degree $n$, then there is an
element $\alpha$ of $K$, such that $K = k(\alpha)$ and $\alpha^n$ is
an element of $k$.
\end{corollary}

\begin{proof}
$\zeta$ has norm $1$, therefore there is an $\alpha$ in $K$ such
that $\sigma \alpha = \zeta \alpha$.
\[
\sigma(\alpha^n) = (\zeta a)^n \alpha^n = \alpha^n,
\]
so $\alpha^n$ is an element of $k$.  If degree of $\alpha$ was less
than $n$, there would be some $r$ such that $\sigma^r(\alpha) =
\alpha$ with $r < n$.  But $\sigma^r(\alpha) = \zeta^r \alpha$.  Since
$\zeta$ is a primitive $n$\th\ root of unity, this is possible only if
$r = n$.  Thus the degree of $\alpha$ is $n$, and $K = k(\alpha)$.
\end{proof}

\section{Basic Kummer Theory}
\label{Basic:Kummer:Theory:Sec}

The basic theorems we prove here can be derived by elementary techniques,
although the proofs will be long and involved.  We have chosen to use the
techniques of group cohomology of the previous section because the
resulting proofs are simple and short and, more importantly, provide the
machinery for dealing with more complex algebraic extensions.

Let $k$ be a field of characteristic $0$.  Assume $k$ contains a
primitive $n$\th\ root of unity and let $E_n$ be the subgroup of
$k^{\ast}$ generated by $n$\th\ roots of unity in $k$.  Let $\Delta$ be a
multiplicatively closed subset of $k^{\ast}$ that contains $k^{\ast n}$
(the $n$\th\ powers of elements of $k^{\ast}$).  For instance if $k = \Q$
we might let $n = 3$ and  
\[
\Delta = 
  \{\,a^3 b \mid a \in k, b \in \{1, 2, 3, 4, 6, 9, 12, 18, 36\}\,\}.
\]
Let $K = k(\Delta^{1/n})$ be the field obtained by adjoining the 
$n$\th\ roots of elements of $\Delta$ to $k$.  Thus in our example $K 
= k(\root 3 \of{2}, \root 3 \of {3})$.  Since $k$ contains the third
roots of unity, it doesn't matter which cube root is taken.  Once one is
included in $K$, all are included.  More generally, since $k$
contains a primitive $n$\th\ root of unity, $K$ is normal over $k$.
Let $G$ be the Galois group of $K$ over $k$.

Because the structure of the extension $K/k$ is so simple, it is easy to
determine the structure of $G$.  The purpose of Kummer theory is to relate
the structure of $G$ to the structure of $\Delta$.  First, we will
summarize some basic results about the structure of $K/k$ (along the same
lines as Artin and Tate \cite{Artin:Tate}) and then proceed to the main
theorem of Kummer theory.

$K$ is generated as a $k$-module by some subset of $\Delta^{1/n}$.  Let
$\alpha$ be an element of $\Delta^{1/n}$.  Its degree over $k$ is a divisor
of $n$.  Let $\sigma$ and $\tau$ be elements of the Galois group $G$.
Since $\alpha^n$ is an element of $k$,
\[
(\sigma\alpha)^n = \sigma(\alpha^n) = \alpha^n.
\]
Thus, $\sigma\alpha = \zeta_{\sigma} \alpha$, where $\zeta_{\sigma}$ is an
$n$\th\ root of unity.  If the degree of $\alpha$ over $k$ is $n$ then
$\zeta_{\sigma}$ will be a primitive $n$\th\ root of unity.  In any case,
it is easy to see that $G$ is abelian,
\[
\sigma\tau\alpha = \zeta_{\sigma} \zeta_{\tau} \alpha = 
\zeta_{\tau} \zeta_{\sigma} \alpha = \tau \sigma \alpha
\]
so $G$ is abelian.  Also, for all $\sigma \in G$,
$\sigma^n$ is the identity so the exponent of $G$ is $n$. 

Let $E_n$ be the multiplicative group of the $n$\th\ roots of unity.  We
can form the exact sequence
\[
0 \longrightarrow E_n \longrightarrow K^\ast \mapsto{\wp}
K^{\ast n} \longrightarrow 0,
\]
where the surjective map is $\wp : a \mapsto a^n$.  The cohomology groups
of this sequence tell us what we need to know about $G$.  First, we need to
determine the structure of $H^1(G, E_n)$.

A $1$-cochain of $E_n$ is a map $\chi : G \rightarrow E_n$ (also called a
{\em character} of $G$).  Let $\chi$ be a 1-cocycle of $H^1(G, E_n)$.  Then
(see appendix)
\[
1 = \chi(\sigma) \chi(\sigma \tau)^{-1} \sigma(\chi(\tau))
= \chi(\sigma) \chi(\sigma \tau)^{-1} \chi(\tau).
\]
Thus $\chi$ is a homomorphism of $G$ to $E_n$, so we may identify $H^1(G,
E_n)$ with the dual group of $G$.  If, in addition to being an abelian
extension of $k$, $K$ is also of finite degree over $k$, then $H^1(G, E_n)
= G$.

\medskip
We now come to the main theorem of this section.  The proof is made
quite simple by using the results of group cohomology.

\begin{proposition}
\label{Kummer:Radical:Prop}
Assume $k$ contains the $n$\th\ roots of unity, and $k^{\ast n}
\subseteq \Delta \subseteq k^{\ast}$ a multiplicatively closed set.
$G$ is the Galois group of the normal extension $K = k(\Delta^{1/n})$
over $k$.  Then $G$ is isomorphic to $\Delta/k^{\ast n}$.  In
particular $\fieldDegree{K}{k} = \groupDegree{\Delta}{k^{\ast n}}$.
\end{proposition}

\begin{proof}
The following sequence is exact
\[
0 \longrightarrow E_n \longrightarrow K^{\ast} \longrightarrow K^{\ast
n} \longrightarrow 0
\]
where the surjection is the $n$\th\ power map.  A piece of the corresponding
cohomology sequence is:
\[
 \cdots \longrightarrow K^{\ast G} \longrightarrow (K^{\ast n})^G
\longrightarrow H^1(G, E_n) \longrightarrow H^1(G, K^{\ast})
\longrightarrow \cdots
\]
By Hilbert's theorem 90, \propref{Hilbert:Ninety:Prop}, $H^1(G,
K^{\ast}) = 0$. By the 
previous remarks $G \simeq H^1(G, E_n)$.  $K^{\ast G}$ is $k^{\ast}$ by
the definition of the Galois group.  Thus we have the exact sequence
\[
\cdots \longrightarrow k^{\ast} \longrightarrow (K^{\ast n})^G \longrightarrow G
\longrightarrow 0.
\]
And so $G \simeq (K^{\ast n})^{G}/k^{\ast n}$ by the definition of exactness.
Since $\Delta$ is contained in $(K^{\ast n})^G$,
\[
\groupDegree{(K^{\ast n})^G}{k^{\ast n}} \ge \groupDegree{\Delta}{k^{\ast n}}.
\]
On the other hand
$K$ may be considered as a $k$-module spanned by $\Delta^{1/n}/k^{\ast}$.
Thus $\fieldDegree{K}{k} \le \groupDegree{\Delta^{1/n}}{k^{\ast}}
 = \groupDegree{\Delta}{k^{\ast n}}$.
Therefore $\groupDegree{(K^{\ast n})^G}{k^{\ast n}} 
= \groupDegree{\Delta}{k^{\ast n}}$
and finally 
\[
\Delta/ k^{\ast n} = (K^{\ast n})^G / k^{\ast n} = G.
\]
\end{proof}

This theorem shows that $K^\ast$ is isomorphic to the $k$ module
generated by $n$\th\ roots of elements of $\Delta/k^{\ast n}$---which is
the basis set required by our first problem.  Notice that the structure of
$\Delta/k^{\ast n}$ can be computed using only operations over $k$.

Two examples suggested by James {\DavenportJ} are useful in making
these ideas clearer.  Consider the field $\Q[\root4\of{4}]$ over $\Q$.
Since we are working with $4$\th\ roots, we must have $4$\th\ roots of
unity in the ground field.  Let $k = \Q[i]$ and $K = k[\root4\of4] =
\Q[i, \root4\of4]$.  We begin with 
\[
\Delta = \{\,a^4 b \mid a \in k, b \in \{1, 4, 16, 64\}\},
\]
looking for perfect fourth powers.  The elements of $\Delta$ correspond
to ${\root4\of4}^0$, ${\root4\of4}^1$, ${\root4\of4}^2$ and
${\root4\of4}^3$ respectively.  Since $16$ is a perfect fourth power,
$1$ and $16$ are equivalent as are $4$ and $64$.  We can reduce $\Delta$
to $\{1, 4\}$.  Because of the correspondence between $16$ and 
${\root4\of4}^2$ we know that ${\root4\of4}^2$ is an element of $k$;
it is $\zeta_4 2$, for some $4$\th\ root of unity $\zeta_4$, thus
$\root4\of4 = \sqrt{\zeta_4 2}$.  Now we need to determine which
root of unity to use.  The minimal polynomial of $\root4\of4$ is
$X^4-4$,  thus $\sqrt{\zeta_4 2}^4 - 4 = 0$, or $\zeta_4^2-1 = 0$.  So
$\Q[\root4\of4]$ is isomorphic to either $\Q[\sqrt{2}]$ or
$\Q[\sqrt{-2}]$ depending on which fourth root of $4$ is meant by
$\root4\of4$.  This ambiguity is expected because $X^4-4$ factors into
$X^2-2$ and $X^2+2$ which generate different but isomorphic fields.

Notice that in this calculation, the structure of 
$\Delta /k^4$ is identical to the structure of
$\Delta / \Q^4$.  The introduction of fourth roots of unity
required by theorem {\bf 5}, was not really necessary.  Even the
$\zeta_4$ needed at the last step was only a square root of unity which
is an element of $\Q$.  This is not always the case.  

For instance, consider the two fields
$\Q[\root4\of{-4}]$ and $\Q[\root4\of{-9}]$.  Though superficially they
look quite similar, they have very different structure.  Since $X^4+9$
is irreducible, $\Q[\root4\of{-9}]$ is of degree $4$ over $\Q$.  But,
\[
X^4+4 = (X^2 - 2X +2)(X^2 +2X+2)
\]
so $\Q[\root4\of4]$ only has degree 2 over $\Q$.

The residue groups for these two fields are, respectively:
\[
\begin{aligned}
  \Delta_{-4} &= \{\, a^4 b \mid a \in k, b\in \{1, -4, 16, -64\}\,\} \\
  \Delta_{-9} &= \{a^4 b \mid a \in k, b \in \{1, -9, 81, -729\}\,\}.
\end{aligned}
\]
When considered modulo $\Q^4$ there are two problems.  First, how do we
correlate the fact that $\fieldDegree{\Q[\root4\of{-9}]}{\Q}$ is four with the
fact that the residue group is of order 2? Second, both residue groups
are of order 2 since $16$ and $81$ are perfect fourth powers, and yet
the have radically different structure.  

Consider the first problem.  How can $\Q[\root4\of{-9}]$ be of degree
four over $\Q$ when its residue group is of order 2?  Again, $k$ is not
$\Q$ but, $\Q[i]$ and the residue group gives the structure of 
$K = \Q[i, \root4\of{-9}]$ over $k$.  Since $81 = (-9)^2$ is a perfect
fourth power in $k$, we know that $\root4\of{-9}^2 = \zeta_4 3$, or 
$\root4\of{-9} = \sqrt{\zeta_4 3}$.

To determine which root of unity is needed we again substitute
$\sqrt{\zeta_4 3}$ into $X^4+9=0$.  Thus $\zeta_4^2+1=0$ so $\zeta_4$
must be a primitive fourth root of unity.  This field is of degree $2$
over $k$ and is generated by $\sqrt{3i}$.

Now we need to look at $\Delta_{-4}$ modulo $k^4$.  Unlike $\Delta_{-9}$
all the elements of $\Delta_{-4}$ are perfect fourth powers since
$(1+i)^4 = -4$.  So $\root4\of{-4} = \pm 1\pm i$, depending on which
fourth root of unity is chosen. 

Unfortunately, the techniques discussed require the $n$\th\ roots
of unity be in $k$.  If $k$ does not contain the $n$\th\ roots of unity
it can be enlarged to a new field $k'$ which does, and then theorem {\bf
5} can be applied to $Kk'$ over $k'$ (this is how we handled $\Q(\root
4\of 3)$ above).  Since $k'$ is an algebraic
extension of $k$, it will be more significantly more difficult to
compute the structure of $\Delta/k^{\prime \ast n}$ than 
$\Delta/k^{\ast n}$.  Sometimes however, the structure of
$Kk'$ over $k'$ can be deduced from $\Delta/k^{\ast n}$.  In particular
$\Delta/k^{\prime\ast n}$ is equal to $\Delta/k^{\ast n}$ if $K \cap k'$
is equal to $k$.  This can be seen from the following theorem from
Lang \cite{Lang:Algebra}. 

\begin{proposition}
Let $F_1$, $F_2$ and $k$ be subfields of some common field, $F_1$ a Galois
extension of $k$ and $F_2$ and arbitrary extension of $k$.  $F_1$ is Galois
over $F_1 \cap F_2$ and $F_1 F_2$ is Galois over $F_2$.  Further more the
Galois group of $F_1$ over $F_1 \cap F_2$ isomorphic to the Galois group of
$F_1 F_2$ over $F_2$.
\end{proposition}

Here we identify $F_1$ with $k'$ and $K$ with $F_2$.

The reason for the complications in the last two problems was that the
intersections between $\Q[i]$ and $\Q[\root4\of{-4}]$ and
$\Q[\root4\of{-9}]$ was greater than $\Q$.  That is, the
intersection between $K$ and $k'$ was not $k$.

The following corollary gives one case where $K \cap k'$ must be equal
to $k$.  We let $\varphi(r)$ be the Euler $\varphi$-function (the number
of integers less than $r$ that are relatively prime to $r$).  It is
well known that the degree of the extension generated by a primitive
$r$\th\ root of unity over $\Q$ is $\phi(r)$.

\begin{corollary} Let $r$ be a rational integer that is relatively
prime to $\varphi(r)$.  Let
\[
K = k(\alpha_1^{1/r}, \ldots, \alpha_n^{1/r})
\qquad\hbox{and }\qquad
\Delta = \{\,\alpha_1^{s_1} \cdots \alpha_n^{s_n} \mid 0 \le s_i < r\,\},
\]
then the degree of $K$ over $k$ is $\groupDegree{\Delta}{k^{\ast r}}$.
\end{corollary}

\begin{proof}
Let $k' = k(\zeta_r)$ where $\zeta_r$ is a primitive $r$\th\ root of unity.
$k'$ is a Galois extension of $k$.  By the previous proposition $Kk'$ is
Galois over $K$, and the Galois groups of $Kk'/K$ and $k'/K\cap k'$ are
isomorphic.  The degree of $k'$ over $k$ must divide $\phi(r)$.  Since $r$
and $\phi(r)$ are relatively prime, $\fieldDegree{k'}{k}$ is relatively
prime to $\fieldDegree{K}{k}$, and so the intersection of $K$ and $k'$ is
$k$.  Applying the last theorem to the extension $K'/ k'$ we have the
corollary.
\end{proof}

There has been other work in developing a theory of radical extensions that
does not require the extensions to be Galois.  Kneser
\cite{Kneser:Radicals} has characterized which roots of unity are actually
necessary in the following theorem:

\begin{proposition}[Kneser]
\label{Kneser:Prop} Let $K/k$ be a separable extension and let $M$ be a
subgroup of $K^{\ast}$ such that $\fieldDegree{Mk^{\ast}}{k^{\ast}}$ is
finite.  Then
\[
\fieldDegree{K(M)}{k} = \fieldDegree{Mk^{\ast}}{k^{\ast}}
\]
if and only if (i) for all primes $p$, $\zeta_p \in k^{\ast}M$ implies
$\zeta_p \in k$ and $1+ \zeta_4 \in k^{\ast} M$.
\end{proposition}

In this proposition $M$ corresponds to $\Delta^{1/n}$ in theorem {\bf 5}.
A proof and discussion of \propref{Kneser:Prop} and Capelli's theorem
(\propref{Capelli:Prop}) are contained in Schinzel
\cite{Schinzel:Polynomials}.  Other related theorems have been presented by
Halter-Koch \cite{HalterKoch}.

\section{General Reduction Techniques}
\label{General:Reduction:Sec}

In the general case of simplifying multiply nested radicals
we have a tower of abelian extensions that may be refined so that
\[
K = K_m \supset K_{m-1} \supset \cdots \supset K_0 = k
\]
each of which is of the form 
\[
K_i = K_{i-1}(\beta_{i1}^{1/r_i}, \ldots, \beta_{i n_i}^{1/r_i})
\]
where the $\beta_{ij}$ are elements of $K_i$ and we require either that
$(\varphi(r_i), r_i) = 1$\Marginpar{what if $r_i$ is a power of an integer}
or that $K_{i-1}$ contains a primitive $r_i$\th\ root of unit.  The
algorithm we present relies heavily upon factoring over $K_{i - 1}$.  Since
algebraic factoring is so expensive it is advisable to minimize the height
of the tower from $K$ to $k$.

By the main theorem of the previous section we saw that the degree of the
extension $K_i/K_{i - 1}$ could be computed.  In this section we shall show
that this computation also leads to the construction of a $K_{i -
1}$-algebra basis for $K_i$.  Given these bases for each extension in the
tower, it is trivial to construct the basis for $K$ over $k$.  So we need
only consider the simple case of an extension of the form $K =
k(\alpha_1^{1/r}, \ldots, \alpha_\ell^{1/r})$.

Clearly $\Delta = 
\{\,\alpha_1^{s_1} \cdots \alpha_n^{s_\ell} \mid 0 \le s_i < r\,\}$
is a group under multiplication modulo the elements $\alpha_i^r$.  By the
results of the last section we can determine the degree of $K$ over $k$ by
determining the subgroup of $\Delta$ that is free of $r$\th\ powers of
elements of $k$, \ie, $\Delta' = \Delta/(\Delta \cap k^{\ast r})$.
Furthermore, the $r$\th\ roots of elements of $\Delta'$ spans $K$ as a
$k$-algebra and is the basis of $K$ we are looking for.  Since $\Delta'$ is
a subgroup of $\Delta$ can be written as 
\[
\Delta' =
\{\,\alpha_1^{s_1} \cdots \alpha_n^{s_\ell} \mid 0 \le s_i < r_i\,\},
\]
where the $r_i$ are less than or equal to $r$ and some may be equal to
$1$.  We can then write $K$ as
$k(\alpha_1^{1/r_1}, \ldots, \alpha_\ell^{1/r_\ell})$.

We can determine whether $\theta \in k$ is a perfect $r$\th\ power by
looking for a linear factor $x - \lambda$ of $x^r - \theta$.  This is
done by factoring $x^r - \theta$ over $k$ using one of the algebraic
factoring algorithms, Rothschild and Weinberger
\cite{Rothschild:Weinberger}, {\Trager} \cite{Trager76a} and
{\WangP} \cite{Wang76a}. The algorithm presented by
{\Trager} \cite{Trager76a}, an improvement of one presented by
{\Waerden} \cite{Waerden:Algebra}, permits factoring over an arbitrary
function field.

Now assume that we can determine that an element of $\Delta$ is a
perfect $r$\th\ power.  We write $a \approx b$ if $a/b$ is a perfect
$r$\th\ power.  Any element that is a perfect $r$\th\ power generates a
subgroup of perfect $r$\th\ powers.  Consider the following example: let
$\ell= 3$, $r = 6$ and assume all the $r_i$ are also 6.  $\Delta$ has
$6 \times 6 \times 6 = 216$ elements. If we discover that
$\alpha_1^2 \alpha_2^3$ is a perfect 
sixth power then we also have 
\[
\begin{aligned}
(\alpha_1^2 \alpha_2^3)^2 &\approx \alpha_1^4\\
(\alpha_1^2 \alpha_2^3)^3 &\approx \alpha_2^3\\
(\alpha_1^2 \alpha_2^3)^4 &\approx \alpha_1^2\\
(\alpha_1^2 \alpha_2^3)^5 &\approx \alpha_1^4 \alpha_2^3
\end{aligned}
\]
All are perfect sixth powers; that is, $\alpha_1$ is a perfect cube
and $\alpha_2$ is a perfect square.  Thus $\alpha_1^2 \alpha_2^3$
generates the subgroup 
\[
\{1, \alpha_1^2, \alpha_1^4, \alpha_2^3, \alpha_1^2 \alpha_2^3,
\alpha_1^4 \alpha_2^3\}
\]
Removing this subgroup reduces the order of $\Delta$ by a factor of 6.

There are many techniques available for finding the ``quotient group''
as it is called.  We present one that is particularly suggestive in
our case.  With notation as before, each of the $\alpha_i$ ($1 \le i
\le n$) are of order $r_i$.  Let $\alpha_1^{m_1} \ldots
\alpha_\ell^{m_\ell}$ be an element of $\Delta_1$ that is a perfect
$r$\th\ power in $k$ and assume $m_i \ne 0$.  Let $w$ be the solution
of
\[
(r_1, m_1) \equiv w\, (r_1 - m_1) \mod{r_1}.
\]
If $(r_1, m_1)$ is not $1$ then it divides both $r_1 - m_1$ and $r_1$.
Thus the solution of the equation is
\[
w \equiv {(r_1, m_1) \over (r_1 - m_1)} \mod{r_1 \over (r_1, m_1)}.
\]

Using this value of $w$,
\[
(\alpha_2^{m_2} \cdots \alpha_n^{m_n})^w \approx 
   (\alpha_1 ^{r_1 - m_1})^w \approx \alpha_1^{(r_1, m_1)}
\]
and we have reduced $\alpha_1$'s order to $(r_1, m_1)$.  Notice that this
equation  gives a new polynomial that $\alpha_1$ must satisfy of lower
degree than any  that has occurred before.  We also know that 
\[
(\alpha_1^{m_1} \alpha_2^{m_2} \cdots \alpha_n^{m_n})^{r_1/(r_1, m_1)}
  \approx (\alpha_2^{m_2} \cdots \alpha_n^{m_n})^{r_1/(r_1, m_1)}
  \approx 1,
\]
so we may repeat the procedure with this smaller expression
and obtain further reductions.  This results in a sequence of 
polynomials equalities of the form 
\[
P_i(\alpha_i) = \alpha_i^{s_i} - f(\alpha_{i+1}, \ldots, \alpha_n) = 0
\]
where the product of the $s_i$ is the degree of the extension.  Thus each
of the $P_i$ must be irreducible.

\def\dt{\mathop{\Delta_{\rm table}}}

In summary this algorithm consist of two parts.  The first, and
most time consuming portion, determines which elements of $\Delta$ are
actually perfect $r$\th\ powers.  The second phase takes each of 
these elements of $\Delta$ and determines a list of new minimal polynomials
for some of the $\alpha_i$ that comprise the perfect $r$\th\ power.
The following loose algorithmic description should clarify the details of
this algorithm.  We define two procedures here, {\bf Find-$r$\th-powers} and
{\bf Reduce-$\Delta$}.  {\bf Find-$r$\th-powers} is the driver; it searches
through $\Delta$ to find those elements that are $r$\th\ powers and then
calls {\bf Reduce-$\Delta$} for each $r$\th\ power it finds
to reduce $\Delta$ modulo that element.
To aid in the search {\bf Find-$r$\th-powers} maintains an array,
$\dt$
with one location $\dt(\gamma)$ for each element $\gamma$
of $\Delta$ indicating one of the following 

\begin{enumerate}

\item $\gamma$ is known to be a perfect $r$\th\ power, 
\item $\gamma$ is known not to be a perfect $r$\th\ power or 
\item it is not known whether $\gamma$ is a perfect $r$\th\ power or not.
\end{enumerate}

Initially all the elements of $\dt$ are in state (3).
Notice that if $\dt(\gamma)$ indicates state (1) then $\dt(\gamma) =
\dt(\gamma^i)$ for all $i$.  If $\dt(\gamma)$ indicates state (2 ) then
$\dt(\gamma) = \dt(\gamma^i)$ for all $i$ relatively prime to $r$.

\medskip
\noindent
{\bf Algorithm Find-$r$\th-powers} takes as input: a list of $\ell$ variables
$\gamma_i$  and corresponding elements of $k$, $\alpha_i$, and integer
$r$.  The $\gamma_i$ represent the radicals $\root r \of \alpha_i$.
This algorithm returns a list of minimal polynomials for the $\gamma_i$
such that the product of their degrees is
$\fieldDegree{k(\gamma_1, \ldots, \gamma_\ell)}{k}$.

\begin{algorithm}{A}
\item {[Initialization]} $\dt$ is an $\ell$ dimensional table of
order $r$ in each dimension.  Set each element of $\ell$ to state (3).
{\em Min-polys} to a list of the minimal polynomials of the $\gamma_i$.
Initialize {\em Min-polys} to the polynomials $X_i^r - \alpha_i$ for $1 \le
i \le \ell$.  $r_i$ is the order of $\gamma_i$.  Set $r_i = r$. 

\item {[Loop]}  While there is $m_1, m_2, \ldots, m_\ell$ with each
$m_i$ less than $r_i$ and such that $\dt(m_1, m_2, \ldots, m_\ell)$
indicates state (3) perform the following two steps.  Finally return
{\em Min-polys\/}.

\item {[Check $r$\th\ power]  }
For all positive $k$ less than $r$ set $\dt(k m_1 \mod{r_i}, \ldots, k
m_\ell \mod{r_i})$ to (1) if
$\alpha_1^{m_1} \cdots \alpha_\ell^{m_\ell}$ is a perfect $r$\th\ power.
If not, then for all positive $k$ less than $r$  and relatively prime to
$r$ set $\dt(k m_1 \mod{r_i}, \ldots, k m_\ell \mod{r_i})$ to (2).

\item {[Update $\dt$]}  Call {\bf Reduce-$\Delta$} with $m_1,
\ldots, m_\ell$ to reduce the degree of some of the minimal
polynomials.

\end{algorithm}

\medskip
\noindent {\bf Algorithm Reduce-$\Delta$}
takes as arguments $m_k, \ldots, m_\ell$ corresponding to 
$\gamma = \alpha_k^{m_1} \cdots \alpha_k^{m_\ell}$ where the order of
the $\alpha_i$ is $r_i$.  It factors out the the subgroup generated by
$\gamma$ from $\Delta$. 

\begin{algorithm}{R}
\item {[Determine $w$]} Set $w = (r_k, m_k)/(r_k - m_k) \mod{r_k/(r_k, m_k)}$.

\item {[Update {\em Min-Polys}]}  Make 
$X_k^{(m_k, r_k)} \approx \alpha_{k+1}^{m_{k+1} w} \cdots \alpha_\ell^{m_\ell w}$
reduced modulo the relations in {\em Min-Polys\/} be he new minimal
polynomial for $\alpha_k$ and add it to  {\em Min-Polys\/}.  The
coefficient of the constant term will be an $r_k$\th\ root of unity.
Just which one is determined by the technique of the last section or by
knowledge of which branch was taken by $\alpha_k$.

\item [Recurse]  Call {\bf Reduce-$\Delta$} with
\[ 
m_{k+1} r_k/(m_k, r_k), m_{k+2} r_k/(m_k, r_k), \ldots,
m_{\ell} r_k/(m_k, r_k)
\]
to reduce the other $\alpha_i$.

\end{algorithm}

\medskip
The most time consuming portion of this algorithm is determining
which of the elements $\Delta$ are actually perfect $r$\th\ powers.  
In general it will only be necessary to determine if
approximately $\prod_i(1 + d(r_i))$ elements of $\Delta$ are perfect
$r$\th\ powers where $d(n)$ is the number of divisors of $n$.
Thus to find a $k(\alpha_1^{1/r}, \ldots, \alpha_k^{1/r})$
we will need to determine if roughly $(1 + \log r)^k$ elements of
$\Delta$ are perfect $r$\th\ powers.  Since not every element tried will
be an $r$\th\ power, it appears that the algebraic factoring step will 
dominate.  So any improvements to step 3 of {\bf Find-$r$\th-powers} would
lead to dramatic improvements in the algorithm.  

If $k$ is a Euclidean domain and 
$K = k(\root r \of {\alpha_1}, \ldots, \root r \of {\alpha_n})$
then we can compute the basis by only taking GCD's.
This is the type of problem that Caviness and Fateman can handle, and this
is the way they handle it.  Caviness and Fateman
\cite{Caviness:Fateman} covers this thoroughly, but note that by our
theorem their techniques are valid in any unique factorization domain
in which gcd's can be computed effectively.  If $k$ is not a unique
factorization domain, however, and if $p_1 p_2 = p_3 p_4$ are four
irreducible elements of $k$, then the four radicals $\root r \of
{p_i}$ are not independent. 

\section{Examples}
\label{Kummer:Example:Sec}

Here we present a few examples of the usage of this technique.  First,
let $k = \Q$ and $K = k(\sqrt{2}, \sqrt{3}, \sqrt{6})$; $\alpha_1=2$, 
$\alpha_2=3$, and  $\alpha_3=6$.  The elements of $\Delta$ and their
corresponding $m_i$ are  
\[
\vcenter{\halign{\hfil$(#)$&$\null = #$\hfil\qquad&\hfil$(#)$&$\null =
#$\hfil\cr
0,0,0& 1&1,1,0& 6\cr 
1,0,0& 2&1,0,1& 12\cr
0,1,0& 3&0,1,1& 18\cr
0,0,1& 6&1,1,1&36\cr}}
\]
The only element of $\Delta$ that is a perfect square is $36$.  So we
have
\[
\alpha_1 \alpha_2 \alpha_3 = 36 \approx 1.
\]
This can be rewritten as 
\[
36^{-1} \alpha_1^2 \alpha_2 \alpha_3 = \alpha_1^{r_1 -1} = \alpha_1.
\]
So we can write the minimal polynomial for $\sqrt{\alpha_1}$ as
$X_1^2 - 36^{-1} \alpha_1^2 \alpha_2 \alpha_3$.  
The new minimal polynomial for $\sqrt{\alpha_1}= \sqrt{2}$ is now 
\[
\begin{aligned}
0 &= X_1^2 - 36^{-1} \alpha_1^2 \alpha_2 \alpha_3\\
& = X_1 - \sqrt{{4 \over 36} \alpha_2 \alpha_3}\\
& =X_1 - {1 \over 3} \sqrt{3} \sqrt{6},
\end{aligned}
\]
Now $r_1 = 1$ and we see the degree of the extension is actually $4$.

\medskip
Now a more complex example.  Let $k = \Q[\sqrt{-5}]$, and 
\[
K = k[\sqrt{2}, \sqrt{3}, \sqrt{1+\sqrt{-5}}, \sqrt{1-\sqrt{-5}}].
\]
Notice that $2$, $3$, $1+\sqrt{-5}$ and $1-\sqrt{-5}$ are all primes in
$k$, but furthermore, $k$ is not a unique factorization domain.
Examining $\Delta$ for this domain we discover that
\[
2\cdot 3\cdot(1+\sqrt{-5})\cdot(1-\sqrt{-5})= 36 
  = 6^2 \in k^{\ast2}.
\]
This leads to the relationship
\[
\sqrt{1-\sqrt{-5}} 
  = {1 - \sqrt{-5} \over 6} \sqrt{2} \sqrt{3} \sqrt{1+\sqrt{-5}}.
\]

\section{Computing Galois Groups}

{\em Don't forget \cite{Mckay79}}

\section{\v{C}ebotarev Density Theorem}
\label{Cebotarev:Sec}

{\em Tie to irreducibility results and give Hilberts example.
Probably want to use MacCluer's proof \cite{MacCluer68} and Moreno's
results also \cite{Moreno:Chebotarev}.  Also Oesterl\'e's computations
\cite{Oesterle79}.  The basic idea here is due to Peter Weinberger
\cite{Weinberger84}.} 


One of the key function in the study of the distribution of prime
numbers is the $\pi(x)$ which is the number of rational primes less
than $x$.  This function can be extended to a measure of the
distribution of prime ideals in number fields.  Let $k$ be a finite
algebraic extension of $\Q$, with ring of integers $A$.

\begin{proposition}[\Chebotarev] 
\label{Cebotarev:Density:Prop}
Let $L$ be a finite normal extension of a number field $k$, with Galois
group $G$, and let $C$ be a conjugacy class of $G$.  Those primes $\mathfrak{p}$ of $k$ which are unramified in $L$ and for which there exists $\mathfrak{P} |\mathfrak{p}$ such that
\[
C = \left[\frac{L/k}{\frak P}\right]
\]
have density equal to $|C|/|G|$.
\end{proposition}

This proposition was made effective by {\Lagarias} and {\Odlyzko}
\cite{Lagarias77}.

We define $\pi_{C}(x, L/k)$ as the size of the set
\[
\{\, {\frak p} \mid {\frak p} \enspace\mbox{is unramified in $L$},
\left[\frac{L/k}{\frak p}\right] = C, \Norm_{k/\Q}{\frak p} \le x\,\}
\]

\begin{proposition}[Lagarias and Odlyzko]
If the generalized \key{Riemann hypothesis} holds for the Dedekind
zeta function for $L$, then there exists an effectively computable
constant $c_{1}$ such that for all $x>2$
\[
\left|\pi_{C}(x, L/k) - \frac{|C|}{|G|} \Logint(x)\right|
\le c_{1} \left[ \frac{|C|}{|G|} x^{1/2} \log (d_{L} x^{n_{L}}) + \log d_{L}\right]
\]
\end{proposition}

The following result was proven in \cite{Lagarias79}

\begin{proposition}[Lagarias, Montgomery and Odlyzko] There is an
absolute, effective computable constant $A_1$ such that for every
finite extension $k$ of $\Q$, every Galois extension $L$ of $K$ and
every conjugacy class $C$ of $\Gal(L/K)$, there exists a prime ideal
${\frak p}$ of $K$ which is unramified in $L$ and for which 
\[
\left[ \frac{L/K}{\frak p} \right ] = C,
\]
for which $\Norm_{K/\Q} {\frak p}$ is a rational prime and which
satisfies the bound
\[
\Norm_{K/\Q} {\frak p} \le 2 d_L^{A_1}.
\]
\end{proposition}

\section*{Note}

\footnotesize

\notesectref{Kummer:Sec} Albu's work looks relevant \cite{Albu92}.

\normalsize


