%$Id: zero-test.tex,v 1.1 1992/05/10 19:35:22 rz Exp rz $
\chapter{Zero Equivalence Testing}
\label{Zero:Testing:Chap}

This chapter begins the discussion of ``modern'' algorithms that take
advantage of the sparsity of multivariate polynomials.  These
algorithms do not perform arithmetic operations with the polynomials
directly, but rather compute with the values of the polynomials at
selected points.  They produce the value of the final answer at these
points and then reconstruct the answer from the values.  This approach
eliminates \key{intermediate expression swell} and, when done
properly, only requires time polynomial in the size of the answer.
This is often called a \keyi{black box} model of computation since the
polynomials are treated as opaque boxes---how the values of the
polynomial are computed is not available to us.

In order to highlight the most important points this chapter
considers the simplest problem to which these methods
apply---determining when a polynomial is identically zero.  We call
this the \keyi{zero equivalence problem}.
\chapref{Interpolation:Chap} discusses interpolation schemes that
reconstitute univariate polynomials from their values and
\chapref{Sparse:Interp:Chap} discusses how this is done for
multivariate polynomials by taking sparsity into account.  In
\chapref{Poly:GCD:Chap} these methods are applied to the computation
of polynomial {\sc gcd}'s.  The discussion of the other major modern
scheme for dealing with sparsity begins with \chapref{Hensel:Chap}.

The study of large sparse problems is very much like the study of very
large physical objects (galaxies or black holes) or very small
physical objects (nuclear particles).  They are largely intractable,
and cannot be studied directly.  However, information about the
objects can be obtained indirectly, by their effect on other
computations.  Consider a situation where there is a quantity we wish
to study $P(X_1, \ldots, X_v)$, but where direct computation of $P$
is impractical.  By evaluating $P$ at certain points, we can still
study $P$ indirectly.

As an example of how this might arise, assume we are given $M$, a
square matrix over $\F_p[X_1, \ldots, X_v]$, with determinant $P(X_1,
\ldots, X_v)$.  How would we determine if $M$ is singular; \ie, if
$P$ is identically zero?  If $M$ is large (\eg, $100 \times 100$ with 
polynomial entries), explicitly computing $P$ is impractical due to
\key{intermediate expression swell}.  However, given elements of
$\F_p$, $a_1, \ldots, a_m$, it is relatively easy to determine $P(a_1,
\ldots, a_m)$, since it is an element of $\F_p$.  (We can compute the 
determinant of a $100\times 100$ matrix over a finite field directly using 
Gaussian elimination.)  Thus it is relatively easy to compute the values 
of $P$, but not $P$ itself.  The matrix $M$ can be viewed as a 
representation of a function that computes the value of $P$ at a point.

In this situation, we say that $P$ is represented by a \keyi{black
box} that returns the value of $P$ when given values for its
arguments.  When using a black box representation of a polynomial,
evaluation using the black box is the only way to inspect the
polynomial.  To be more precise, let $P(X_1, \ldots, X_v)$ be some
symbolic expression over a ring $R$.  ${\cal B}_P$ is a \keyi{black
box} representing $P$ if, for $x_1, \ldots, x_v \in \R$,  ${\cal
B}_P(x_1, \ldots, x_v)$ returns $P(x_1, \ldots, x_v)$.  Most often
$P(X_1, \ldots, X_v)$ is a polynomial, but rational and algebraic
functions are also possible.\Marginpar{This paragraph needs work.}

Let $\vec{a}$ be a $v$-tuple.  If ${\cal B}_P(\vec{a})$ is non-zero
then $P$ is not the zero polynomial.  However, the converse is not
true.  However unlikely, $\vec{a}$ may still accidently be chosen to
be a zero of $P$.  By quantifying the likeliness of accidently
choosing a zero of $P$, we can develop a {\em probabilistic}
algorithm for zero equivalence.  These techniques are discussed in
\sectref{Zero:Probabilistic:Sec}.

When we have a bound on the maximum number of non-zero terms in $P$,
the zero equivalence problem is called the {\em zero equivalence
problem with term bounds\/}.\index{zero equivalence problem! with term
bounds} This problem can be solved by deterministic algorithms that
take time polynomial in the size of $P$.  These techniques, which are
discussed in \sectref{Zero:Deterministic:Sec}, can be used to prove
that certain problems can be solved in polynomial time.  For most
practical problems, however, the probabilistic algorithms are more
useful.

Throughout this chapter and the next we assume polynomials are
represented in expanded form, as a list of monomials (pairs of
exponent-vectors and coefficients), and that monomials with zero
coefficients are omitted.  We denote the number of variables in a
polynomial by $v$, so the exponent vectors are $v$-tuples.  The
maximum degree of $X_i$ in $P$ is denoted by $d_i$ and the maximum of
the $d_i$ is denoted by $d$.  Occasionally, it is necessary to work
with polynomials for which we have \keyi{total degree bounds} instead
of individual degree bounds for each variable.  This bound is usually
given by $D$.  We always have
\[
d \le D \le vd.
\]

The number of non-zero monomials of the polynomial $P$ is usually
denoted by $t$ or $\terms P$.  The number of terms in a possibly
sparse polynomial is bounded by the number of terms in a dense
polynomial:
\[
\terms P \le (d+1)^v.
\]
Occasionally we will need to distinguish between the actual number of
terms in a polynomial and an {\em a priori} bound on the number of
terms in a polynomial.  We will use $T$ to denote the {\em a priori}
bound, and $t$ to denote the actual number of non-zero terms present
in $P$.
\addsymbol{$\terms P$}{The number of terms in the polynomial $P$}

There is a problem when analyzing the cost of algorithms that use
black boxes: How should we account for the cost of the black box
computation?  Different black boxes will require different amounts of
computation.  At best, we can parameterize this cost.  The most common
approach is merely to count the number of black box invocations.  For
example we might say that an algorithm requires ``$O(n^3)$ arithmetic
operations and invokes the black box $O(n)$ times.''  The second
approach is to assume that there is an exponent $\BBoxExp$ such
that the cost of a black box invocation on an input of size $w$ is
bounded by $O(w^{\BBoxExp})$.  With this convention, the cost of the
same algorithm might be $O(n^3 + n^{\BBoxExp})$.  If $\BBoxExp \le 3$,
then the resulting algorithm requires $O(n^3)$ operations. 


\section{Probabilistic Techniques}
\label{Zero:Probabilistic:Sec}

Without any bounds on the number of terms or the degree of the
variables in the polynomial $P$ and given only the black box ${\cal
B}_P$, one can provide substantial evidence that $P$ is the zero
polynomial by showing that ${\cal B}_P(\vec{a})$ returns zero for many
randomly chosen $v$-tuples $\vec{a}$.  In order to provide a
probabilistic algorithm for zero equivalence, we need to quantify how
strong this evidence actually is.

The only way that the evidence provided by ${\cal B}_P$ can be
misleading is if ${\cal B}_P$ returns $0$ when $P$ is actually
different from zero.  The following proposition quantifies how often
this can occur. It is the basis of nearly all the probabilistic
algorithms in computer algebra.

\begin{proposition}\label{Prob:Zero:Prop}
Let $A$ be an integral domain, $P$ a non-zero element of
$A[X_1,\ldots, X_v]$ and the degree of $P$ in each of $X_i$ be bounded
by $d_i$.  Let $Z_n(B)$ be the number of zeroes of $P$, $\vec{x}$ such
that $x_i$ is chosen from a set with $B$ elements, $B \gg d$.  Then
\[
\begin{aligned}
  Z_n(B) &\le B^v - (B-d_1)(B-d_2) \cdots (B-d_v)\\
    & \le (d_1 + d_2 + \cdots + d_v) B^{v-1}.
\end{aligned}
\]
\end{proposition}

\begin{proof}
There are at most $d_v$ values of $X_v$ at which $P$ is identically
zero.  So for any of these $d_v$ values of $X_v$ and any value for the
other $X_i$, $P$ is zero.  This comes to $d_vB^{v-1}$.  For all other
$B-d_v$ values of $X_v$ we have a polynomial in $v-1$ variables.  The
polynomial can have no more than $Z_{v-1}(B)$ zeroes.  Therefore,
\[
Z_v(B) \le d_v B^{v-1} + (B - d_v) Z_{v-1}(B).
\]

Rather than solving this recurrence for $Z_v$, we solve it for $N_v =
B^v - Z_v$.  Since $Z_1$ is less than or equal to $d_1$, $N_v \ge (B -
d_1)$.  This is the basis step of the inductive proof.  Writing the
recurrence in terms of $N_v$ we have
\[
B^v - N_v(B) \le d_v B^{v-1} + (B - d_v) (B^{v-1} - N_{v-1}(B))
\]
or
\[
N_v(B) \ge (B - d_v) N_{v-1}(B),
\]
from which the proposition follows. 
\end{proof}

The following polynomial actually has $B^v - (B-d_1) \cdots (B-d_v)$
zeroes with components less than $B$
\[
P(X_1, \ldots, X_v) =
\prod_{0 \le i_1 \le d_1}(X_1 - x_{1i}) \quad \cdots 
\prod_{0 \le i_v \le d_1}(X_v - x_{1v}).
\]
Thus the first inequality in the proposition cannot be further
strengthened.  

For the analysis of some algorithms the results of this theorem are
more useful when expressed in terms of probabilities.  We denote the
probability that $P$ is zero for $(x_1, \ldots, x_v)$, chosen randomly,
by
\[
{\cal P}(P(x_1, \ldots, x_v) = 0 \mid x_i \in {\cal S})
= \frac{Z_v(B)}{B^n} \le \frac{d_1 + \cdots d_v}{B}.
\]

Sometimes a similar bound is needed, but in terms of the total degree of
the polynomial $P$.\index{total degree, of a polynomial}  This bound
is provided by the following proposition.

\begin{proposition} \label{Prob:Total:Zero:Prop}
Let $P \in A[X_1, \ldots, X_v]$ be a polynomial of
total degree $D$ over an integral domain $A$.  Let ${\cal S}$ be a
subset of $A$ of cardinality $B$.  Then
\[
{\cal P}(P(x_1, \ldots, x_v)=0 \mid x_i \in {\cal S}) \le \frac{D}{B}.
\]
\end{proposition}

\begin{proof}
We use induction on the number of variables as was done in the proof
of the previous proposition.  For $v= 1$, $f$ is a univariate
polynomial of degree $D$ and can have no more than $D$ zeroes in $A$,
so 
\[
{\cal P}(P(x_1)=0 \mid x_1 \in {\cal S}) \le \frac{D}{B}.
\]

Assume the proposition is true for polynomials in $v-1$ variables.
Let the degree of $P$ in $X_v$ be $d_v$ and denote the leading
coefficient of $f$ with respect to $X_v$ by $f_0$, \ie,
\[
P = p_0(X_1, \ldots, X_{v-1}) X_v^d + \cdots.
\]
The total degree of $p_0$ is no more than $D - d$, so the probability
that $p_0 = 0$ is
\[
{\cal P}(p_0(x_1, \ldots, x_{v-1})=0 \mid x_i \in {\cal S}) \le \frac{D-d}{B}.
\]

Omitting the arguments of $x_1, \ldots, x_v$ and $x_1, \ldots,
x_{v-1}$ for brevity, we can write
\[
\begin{aligned}
{\cal P}(P =0) &= 
  {\cal P}(P =0 \wedge p_0=0) \cdot {\cal P}(p_0=0) \\
  & \qquad \qquad+
  {\cal P}(P=0 \wedge p_0\not=0) \cdot {\cal P}(p_0\not=0), \\
   & \le {\cal P}(p_0=0) + {\cal P}(P=0 \wedge p_0\not=0).
\end{aligned}
\]
Assume that $p_0(x_1, \ldots, x_{v-1})\not=0$. $P(x_1, \ldots,
x_{v-1}, X_v)$ is a polynomial of degree $d$, so there are at most $d$
$x_v \in {\cal S}$ such that $P(x_1, \ldots, x_v) = 0$. Consequently,
\[
{\cal P}(P(x_1, \ldots, x_{v})=0 \mid x_i \in {\cal S}) \le \frac{D -
d}{B} + \frac{d}{B} = \frac{D}{B}.
\]
\end{proof}


The following proposition phrases the result of
\propref{Prob:Zero:Prop} as a probability.

\begin{proposition} \label{Zero:MPoly:Prop}
Let $f_1, f_2, \ldots, f_s$ be elements of $A[X_1, \ldots, X_v]$, $A$
an integral domain, where the degree in each variable is bounded by
$d$.  Let ${\cal P}(f_1, \ldots, f_s)$ be the probability that a
randomly chosen point $\vec{x}$ is a zero of any of the $f_i$, where
$x_i$ is an element of a set with $B$ elements.  Then
\[
{\cal P}(f_1, \ldots, f_s) < \frac{vds}{B}.
\]
\end{proposition}

\begin{proof}
Let $f = f_1 f_2 \cdots f_s$.  The degree of each variable in $f$ is
bounded by $ds$.  Applying the previous proposition, we see that the
number of zeroes of $f$ is bounded by $vdsB^{v-1}$, for sufficiently
large $B$.  Since there are $B^v$ possible $\vec{x}$ from which to
choose, we have the corollary.
\end{proof}

\medskip
\propref{Zero:MPoly:Prop} gives a probabilistic algorithm for zero
equivalence.  We only indicate that a polynomial is non-zero when
${\cal B}_P$ returns a non-zero value, proving that $P$ is non-zero.
We want to know the probability that ${\cal B}_P$ returns zero even
though $P$ is not identically zero.  Assume that $P$ is not the zero
polynomial.

Define the set ${\cal S}_B$ to be
\[
{\cal S}_B = \{\,(x_1, \ldots, x_v) \mid 0 \le x_i < B\,\}.
\] 
Let $\vec{x}$ be an element of ${\cal S}_B$ such that ${\cal B}_P$
returns zero.  Then $\vec{x}$ is one of the at most $Z_v(B)$ zeroes in
of $P$ in ${\cal S}_B$.  If we choose $\vec{x}$ randomly, the
probability that we will get a zero of $P(\vec{X})$ is 
\[
{vD \over B},
\]
where the $D \ge \max(d_i)$.  The probability that $k$ randomly chosen
elements of ${\cal S}_B$ would all yield a value of zero (even though
$P$ is not the zero polynomial) is less than $(vD/B)^k$.

To verify that a polynomial $P(X_1, \ldots, X_v)$, whose degree in
each variable is less than $D$, is zero with probability less than
$\epsilon$ we need to perform $k$ evaluations with random elements of
${\cal S}_B$ where
\begin{equation}
\epsilon < \left({v D \over B}\right)^k. 
\label{Bound:Eq}
\end{equation}
A single evaluation will suffice if $B$ is chosen such that $B \ge
vD/\epsilon$.  The cost of the single evaluation is the cost of the
black box evaluation.  Recall that the cost of a black box evaluation
with $v$-tuple components of size $w$ is $O(w^{\BBoxExp})$, for some
$\BBoxExp$.  Since the size of the components of the random $v$-tuple is
$\log vD/\epsilon$, the cost of a single evaluation is
$O(\log^{\BBoxExp} v D/\epsilon)$.  If $k$ different evaluations are
used, then we need to choose $B$ such that \eqnref{Bound:Eq} is
satisfied.  So,
\[
\log B < \log vD + {1 \over k} \log {1 \over \epsilon}.
\]
The cost of a single evaluation when $k$ are intended is
\[
O\left(\log^{\BBoxExp} {vD \left({1 \over \epsilon}\right)^{1/k}}\right),
\]
and the total time required for $k$ evaluations using the
probabilistic approach will be 
\[
C_{\rm prob} 
   = O\left(k \log^{\BBoxExp} {vD \left({1 \over \epsilon}\right)^{1/k}}\right).
\]
This quantity is minimized when
\begin{equation}\label{Zero:PCount:Eq}
k = {({\BBoxExp} - 1) \log{1 \over \epsilon} \over \log vD}.
\end{equation}
Since $1/\epsilon \gg vD$, using multiple evaluations with relatively
small evaluation points is better than a single evaluation at a
large evaluation point. 

Replacing $k$ by this quantity in $C_{\rm prob}$ we have
\[
\begin{aligned}
C_{\rm prob} &= 
O\left({({\BBoxExp}-1) \log {1\over \epsilon} \over \log vD}
 \log^{\BBoxExp} {vD \left({1 \over
\epsilon}\right)^{ \log vD \over ({\BBoxExp}-1) \log {1\over \epsilon}}}\right)
\\
& = O\left({\log {1 \over \epsilon} \over \log vD} 
  \log^{\BBoxExp}  (vD)^{1+{1\over {\BBoxExp}-1}}\right)\\
& = O(\log {1 \over \epsilon} \times \log^{{\BBoxExp}-1} vD).
\end{aligned}
\]
Notice that the cost of the probabilistic algorithm is linear in the
logarithm of the error.  This is a good characterization of what it
means for a probabilistic algorithm to be polynomial time.

\medskip
Using the ideas in this section, the following routine
probabilistically checks to see if ${\cal B}_P$ represents a polynomial
that is identically zero.  The probability that this routine returns
an incorrect answer is controlled by the parameter $\epsilon$.  The
number of variables and the degree of the variables in $P$ is
indicated by $v$ and $D$ respectively.
\begindsacode
PZeroEquiv(${\cal B}_P$, $v$, $D$, $\epsilon$) := $\{$\\
\> $k \leftarrow  4 (\log 1/\epsilon)/(\log vD)$; \\
\> loo\=p for $0 \le i < k$ do \{ \\
\>\> if ${\cal B}_P(2^i, 3^i, \ldots, p_v^i) \not=0$ then return(false); \\
\>\> $\}$\\
\> return(true);\\
\> $\}$
\enddsacode

\noindent
Notice that in \keyw{PZeroEquiv} we have replaced $\BBoxExp - 1$ in
\eqnref{Zero:PCount:Eq} by $4$.  From a practical point of view this is
quite reasonable.  It is unlikely that the code in ${\cal B}_P$ would
have an exponent larger than $5$.

\section{Deterministic Results}
\label{Zero:Deterministic:Sec}

The probabilistic zero equivalence algorithm discussed in the previous
section is, from a practical point of view, quite sufficient.  However,
from a theoretical perspective it is desirable to know when the zero 
equivalence problem can be decided in \keyi{deterministic polynomial 
time}.  Given ${\cal B}_P$, a black box for the polynomial 
$P(X_1, \ldots, X_v)$ and degree bounds on $P$, only probabilistic 
algorithms are possible.  In essence this is because there are too many 
possible coefficients in $P$ and thus too many degrees of freedom to 
eliminate in polynomial time.  This is made more precise in 
\sectref{Zero:Negative:Sec}.  However, given a bound on the number of 
non-zero terms in $P$, there exist deterministic algorithms for zero 
equivalence. 

The first deterministic zero equivalence algorithm was
developed by {\Grigoriev} and {\Karpinski} \cite{Grigoriev87}.  It is
quite simple and does not even require degree bounds for the
polynomial---only a term bound.  However, this algorithm only works for 
polynomials over $\Z$ (though it is easily generalized to unique 
factorization domains).  The second algorithm, due to {\Zippel} 
\cite{Zippel90}, works over more general domains, but requires a degree 
bound in addition to the term bound. 

Assume we are given a black box for a polynomial, ${\cal B}_P$ and
want to determine if $P$ is identically zero.  The
only interaction the algorithm has with the black box is to pass it an
evaluation point and look at the result.  If the black box ever
returns a non-zero value, then $P$ is not identically zero and the
algorithm can terminate.  The sequence of test values used by the
algorithm cannot depend in any adaptive fashion on the responses from
${\cal B}_P$, since the responses will always be zero and thus do not depend
upon ${\cal B}_P$.  The sequence of test values {\em is} the
algorithm.  A sequence of test values that identifies the character of
a class of black boxes is called a \keyi{characterizing sequence}.  In
\sectref{Zero:Probabilistic:Sec} we proved that random sequences are
{\em likely} to characterize a black box, but this is not assured.

The two algorithms discussed here use different characterizing
sequences, but in both cases the sequences are based on the following 
proposition.

\begin{proposition}\label{Zero:Mon:Prop}
Let $P(\vec{X})$ be a non-zero polynomial in $R[\vec{X}]$ with at most
$T$ terms and with monomial exponent vectors $\vec{e}_i$.  Assume there
exists an $n$-tuple $\vec{x}$ (in some $R$-module) such that the
$\vec{x}^{\vec{e_i}}$ are distinct.  Then not all of $P(\vec{x}^0),
P(\vec{x}), P(\vec{x}^2), \ldots, P(\vec{x}^{T-1})$ are zero.
\end{proposition}

\begin{proof}
Denote $\vec{x}^{\vec{e}_i}$ by $m_i$.  By assumption, each of the
$m_i$ are distinct.  If $P$ vanished at each of the $\vec{x}^i$ then
the following system of linear equations would hold.
\[
\begin{aligned}
  c_1 + c_2 + \cdots + c_T &= 0 \\
  c_1 m_1 + c_2 m_2 + \cdots + c_T m_T &= 0\\
  c_1 m_1^2 + c_2 m_2^2 + \cdots + c_T m_T^2 &= 0\\ \vdots\\
  c_1 m_1^{T-1} + c_2 m_2^{T-1} + \cdots + c_{T} m_T^{T-1}&=0
\end{aligned}
\]
Since this is a Vandermonde system and we have assumed that the $m_i$
are distinct, the system of equations is non-singular.  Thus the $c_i$
must all be zero, and $P$ must be identically zero for all of
$P(\vec{x}^0), P(\vec{x}), P(\vec{x}^2), \ldots, P(\vec{x}^{T-1})$ to 
vanish.
\end{proof}


\paragraph{Without Degree Bounds}

The remaining problem is finding a substitution that keeps the
monomials distinct.  The technique of {\Grigoriev} and {\Karpinski} is
quite simple.  It is based on the observation that the rational
integers are a unique factorization domain. Thus if $p$ and $q$ are
\key{prime numbers} then $p^{e_1} q^{f_1}$ is equal to $p^{e_2} q^{f_2}$ if 
and only if $e_1= e_2$ and $f_1 = f_2$.  Since this keeps the terms 
distinct, by \propref{Zero:Mon:Prop}, it is a distinguishing set of 
evaluation points.  The following proposition makes this precise.

\begin{proposition}[{\Grigoriev} and {\Karpinski}]
\label{Deterministic:Zero:Prop}
Let $P(\vec{X})$ be a polynomial in $v$ variables over a ring of
characteristic zero, $A$, and assume that $P$ has no more than $T$
monomials.  Then there exists a set of $v$-tuples, $\{\vec{x}_0,
\ldots,
\vec{x}_{T-1}\}$ such that either $P(\vec{x}_i) \not= 0$ for some 
$\vec{x}_i$ or $P$ is identically zero.
\end{proposition}

\begin{proof}
Let $\vec{x} = (2, 3, 5, \ldots, p_v)$, where the entries are the 
canonical images of the prime numbers of $\Z$ in $A$.  By unique 
factorization of $\Z$, the monomials $\vec{x}^{\vec{e}_i}$ are distinct, 
and thus by \propref{Zero:Mon:Prop} either $P$ is identically zero or 
does not vanish at every element of the set $\{\vec{x}^0, \ldots,
\vec{x}^{T-1}\}$.
\end{proof} 

\propref{Deterministic:Zero:Prop} gives a simple deterministic
solution of the zero equivalence problem with term bounds.  We merely
pass each of the trial points $\vec{x}_0, \ldots, \vec{x}_{T-1}$ to ${\cal
B}_P$. If ${\cal B}_P$ returns an answer different from zero then $P$
is not the zero polynomial.  If ${\cal B}_P$ returns zero for all of
the trial points then it is the zero polynomial.
\begindsacode
GKZ\=eroEquiv(${\cal B}_P$, $n$, $T$) := $\{$\\
\> loo\=p for $0 \le i < T$ do \{ \\
\>\> if ${\cal B}_P(2^i, 3^i, \ldots, p_v^i) \not=0$ then return(false); \\
\>\> $\}$\\
\> return(true);\\
\> $\}$
\enddsacode

\keyw{GKZeroEquiv} requires $T$ evaluations using ${\cal B}_P$, the
last of which involves numbers as large $p_v^{T-1}$.  Since we have no
idea what computation is going on inside of ${\cal B}_P$ there is no
way to estimate the cost of each of the $T$ evaluations.  Thus the
most precise statement we can make about cost of {\tt GKZeroEquiv} is
that it requires $T$ evaluations and no other arithmetic operations.

Nonetheless, we can get some understanding of the \keyi{bit complexity} 
of using such 
large evaluation points by assuming that ${\cal B}_P$ simply
computes each of the monomials in $P$ and then adds up the values.  On
average, the $n$-th prime is no larger than $n \log n$ by the prime
number theorem.  So, the $k$-th evaluation by ${\cal B}_P$ will
require $O\left(M(k \log (n \log n))\right)$ time, where $M(K)$ is the
time to multiply two $K$ bit numbers.  For simplicity, we assume $M(K)
= K^2$.  Thus all $T$ evaluations require
\[
\sum_{k=0}^{T-1} O(k^2 \log^2 (v \log v))  = O(T^{3} \log^2 (v \log v))
\]
time.  Thus the complexity of \keyw{GKZeroEquiv} is dominated by the 
number of non-zero terms, not the degree of the polynomials.

The following proposition generalizes \propref{Deterministic:Zero:Prop} to 
the zero avoidance  problem for several polynomials.\Marginpar{Use development from lecture notes 4/18/95 for the following proposition.}

\begin{proposition}
\label{Interp:15:Prop}
Let $P_1(\vec{X}), \ldots, P_s(\vec{X}) \in A[X_1, \ldots, X_v]$ be
non-zero polynomials over a ring of characteristic zero, $A$, and
assume that 
\[
\terms P_1 + \cdots + \terms P_s = T.
\]
Let $\vec{x}$ be the image of $(2, \ldots, p_n)$ in $A^n$.  Then for some 
integer $j$, $0 \le j \le T$, all of $P_i(\vec{x}^j)$ are different from zero.
\end{proposition}

\begin{proof}
Denote the points $\{ \vec{x}^0, \vec{x}, \ldots, \vec{x}^T\}$ by
${\cal S}$.  $P_i$ cannot vanish at more than $\terms P_i$ elements of
${\cal S}$ by \propref{Deterministic:Zero:Prop}.\Marginpar{Do I need the generalized vanderonde results here? (prop. 114)}  Since ${\cal S}$
contains $T+1$ points, there must be one for which none of the $P_i$
vanish. 
\end{proof}

\paragraph{With Degree Bounds}

An alternate approach to the zero equivalence problem was developed 
independently by {\Grigoriev}, {\Karpinski} and {\Singer} 
\cite{Grigoriev90} and {\Zippel} \cite{Zippel90}.  The approaches in 
these two papers are almost 
identical, but with one slight variation that will be mentioned later.
The basic idea is that rather than showing that $P(\vec{X})$ is
non-zero, we choose a substitution $X_i \mapsto Z^{e_i}$ and show that
$P(Z^{\vec{e}})$ is non-zero.  This polynomial is univariate and has no
more non-zero terms than the original multivariate polynomial.
Proving that it is non-zero is relatively easy.

The crucial part of this technique is showing that the substitution
$\vec{X} \mapsto Z^{\vec{u}}$ does not send the non-zero polynomial
$P(\vec{X})$ to the zero polynomial, and furthermore that the degree
of $P(Z^{\vec{u}})$ is not too large.  To do this several different
substitutions are required.  All that we can prove is that if
$P(\vec{X})$ is not identically zero then one of the resulting
univariate polynomials will not be zero.

To illustrate the basic ideas, consider a simple version of this
technique that follows the approach used in the previous section.
Assume $R$ is a field, so $R[Z]$ is a unique factorization domain and
$Z+1, Z+2, \ldots$ are primes. Denote by $\vec{Z}$ the vector $(Z+1,
Z+2, \ldots, Z+v)$.  Thus the $\vec{Z}^{\vec{e}_i}$ are distinct.
Sending
\begin{equation}\label{Zero:LinearSubs:Eq}
(X_1, \ldots, X_v) \mapsto (Z+1, \ldots, Z+v) = \vec{Z}
\end{equation}
maps $P(\vec{X})$ into a univariate polynomial.  Since the $Z+i$ are
primes in a unique factorization domain, $\vec{Z}^0, \vec{Z}^1,
\ldots, \vec{Z}^{T-1}$ is characterizing sequence for any polynomial
$P$, with $T$ non-zero terms by \propref{Zero:Mon:Prop}.  In this
section we call the substitution of \eqnref{Zero:LinearSubs:Eq} a
\keyi{linear substitution}.

Unfortunately, this characterizing sequence yields a sequence of
univariate polynomials $P(\vec{Z}^i)$, each of whose degree is bounded
by $ivD$.  Since a univariate polynomial of degree $N$ cannot vanish
at more than $N$ values, each $P(\vec{Z}^i)$ can be characterized by
$ivD+1$ different values of $z$.  Combining these two steps gives the
following deterministic algorithm

\begindsacode
SD\=ZeroEquiv(${\cal B}_P$, $v$, $D$, $T$) := $\{$\\
\>loo\=p for $0 \le i < T$ do \{ \\
\>\> loo\=p for $0 \le z \le ivD$ do \{ \\
\>\>\> if \=${\cal B}_P((z+1)^i, (z+2)^i, \ldots, (z+v)^i) \not= 0$ \\
\>\>\>\>then return(false); \\
\>\>\> \}\\
\>\> \}\\
\> return(true); \\
\> \}
\enddsacode

\keyw{SDZeroEquiv} is a deterministic algorithm for zero equivalence,
but it requires that $R$ contain $vDT+1$ distinct elements if the
inner loop is modified appropriately.  When this is possible, it
generates a characterizing sequence of length $O(vDT^2)$.

When the coefficient field is contained in $\R$, we can improve the
number of evaluations required to characterize a univariate
polynomial.  The key is {\Descartes}' rule of signs:

\begin{proposition}[Descartes]
\label{Descartes:Sign:Prop}
Let $C$ denote the number of sign changes in the polynomial $p(x) \in
\R[x]$ and $Z$ the number of positive real zeroes.  Then $C - Z$ is a
non-negative even integer.
\end{proposition} 


Let $P(Z)$ be a univariate polynomial with $T$ non-zero
terms and denote the number of sign changes among its coefficients by
$C$.  Then $C$ is bounded above by $T-1$.  Applying
\propref{Descartes:Sign:Prop} gives 
\[
0 \le C - N_Z \le T - 1 - N_Z,
\]
which is stated as the following proposition.

\begin{proposition}
\label{Positive:Zeroes:Prop}
Let $P(x)$ be a univariate polynomial with coefficients in $\R$.  The
number of positive real zeroes of $P(x)$ is less than $\terms(P)$.
\end{proposition}

By choosing the evaluation points to be positive numbers the length of
the characterizing sequence for $P$ can be reduced from $\deg P$ to
$\terms P$, which can be much smaller.  The technique of {\tt
SDZeroEquiv} unfortunately, generates dense univariate polynomials.
By using a more sophisticated substitution we can reduce a
multivariate polynomial to a univariate polynomial that has no more
non-zero terms than the original multivariate polynomial.  This
technique will yield a characterizing sequence of length $O(vT^2)$ for
polynomials over $\R$.

\medskip
Instead of using the simple linear substitution of
\eqnref{Zero:LinearSubs:Eq}, we use:
\begin{equation} \label{Zero:MultiSub:Eq}
(X_1, X_2, \ldots, X_v) \longrightarrow
  (Z^{u_1}, Z^{u_2}, \ldots, Z^{u_v})
\end{equation}
where the $u_i$ are positive integers.  We call
\eqnref{Zero:MultiSub:Eq} a \keyi{nonlinear substitution}.  The
nonlinear substitution sends monomials in $P(\vec{X})$ to univariate
monomials in $Z$, so that $P(Z^{\vec{u}})$ has no more non-zero terms
than $P(\vec{X})$.  The difficulty is finding a vector $\vec{u}$ such
that $P(Z^{\vec{u}})$ is not identically zero.

The key idea is that the exponents $u_1, \ldots, u_v$ should come from a
large set of maximally independent $v$-tuples.

\begin{definition}
Let $\cal U$ be a set of $v$-tuples with components in $\Z$.
$\cal U$ is said to be \keyb{maximally independent} if every subset of
$v$ elements of $\cal U$ is $R$-linearly independent.
\end{definition}

In our situation, each element of $\cal U$, $\vec{u}$, corresponds to
a substitution $X_i \mapsto Z^{u_i}$.  If ${\cal U}$ is sufficiently
large, then some element $\vec{u} \in {\cal U}$ will lead to a
polynomial $P(Z^{\vec{u}})$ that is not identically zero.  The
following proposition shows that there exist arbitrarily large sets of
maximally independent $v$-tuples and that the components of the
$v$-tuples are not large.

\begin{proposition}
\label{SPMod:Lemma:1:Prop}
Let $S$ be a positive integer and $p$ be the smallest prime number
larger than $S$.  There exists a maximally independent set of $S$
$v$-tuples where each of the components of the $v$-tuples is less than
$p$.
\end{proposition}

\begin{proof}
First we show that we can construct arbitrarily large maximally
independent sets of $v$-tuples.  Then by reducing them modulo a prime
we get $v$-tuples of the size required by the proposition. Consider
$v$-tuples of the form $(1, k, k^2, \ldots, k^{v-1})$.  For $v$ of
these to be independent the determinant of the matrix
\[
\begin{pmatrix}
1& k_1 & k_1^2 & \cdots & k_1^{v-1}\\
1& k_2 & k_2^2 & \cdots & k_2^{v-1}\\
\vdots& &\vdots& & \vdots\\
1& k_v & k_v^2 & \cdots & k_v^{v-1}\end{pmatrix}
\]
must not be zero.  Since this is a \key{Vandermonde matrix}, it is
nonsingular if and only if the $k_i$ are different.  This and further
results on Vandermonde matrices are given in
\sectref{Vandermonde:Sec}.

Thus, if the $k_i$ are distinct then the vectors they generate will be
linearly independent.  In particular if we let $\vec{u}_k = (1, k,
\ldots, k^{v-1})$ then any subset of $n$ of the $\vec{u}_k$ will be
linearly independent.  Furthermore, if we reduce the elements of $\vec
u_k$ by $p$ the $k_i$ remain distinct modulo $p$ then the $v$-tuples
remain maximally independent.\Marginpar{This needs to be clarified a bit.}
\end{proof}

The size of the elements of the maximally independent set depend upon
the size of $p$.  By \key{Bertrand's postulate}
(\longpropref{Bertrands:Post:Prop}) there exists a $p$ greater than
$S$ and less than $2S$.  

In \cite{Grigoriev90} essentially identical techniques are used, but
the Vandermonde matrix used in the previous proposition is replaced by
the \keyi{Cauchy matrix} whose determinant is
\begin{equation}\label{Zero:Cauchy:Eq}
\det\left|
  \begin{array}{cccc}
\frac{1}{a_1+b_1} & \frac{1}{a_1+b_2} & \cdots & \frac{1}{a_1+b_v} \\[3pt]
\frac{1}{a_2+b_1} & \frac{1}{a_2+b_2} & \cdots & \frac{1}{a_2+b_v} \\[3pt]
\vdots & \vdots & & \vdots \\[3pt]
\frac{1}{a_v+b_1} & \frac{1}{a_v+b_2} & \cdots & \frac{1}{a_v+b_v} 
\end{array}\right|
=
\prod_{1\le i < j \le v}\frac{(a_j - a_i) (b_j - b_i)}{(a_i + b_j)}.
\end{equation}
As long as the $a_i$ and $b_j$ are (separately) distinct every square
submatrix of the Cauchy matrix will be non-singular and thus its rows will
be linearly independent.  Letting $a_i = 1, \ldots, S$ and $b_j = 1,
\ldots, v$ we get a sequence  of $S$ maximally independent vectors:
\[
(\frac{1}{2}, \frac{1}{3}, \ldots, \frac{1}{v+1}), 
(\frac{1}{3}, \frac{1}{4}, \ldots, \frac{1}{v+2}), \ldots
(\frac{1}{S+1}, \frac{1}{S+2}, \ldots, \frac{1}{S+v}).
\]
Again choosing a prime $p > S$ and reducing the components of each
vector by $p$ gives a sequence of vectors that satisfy
\propref{SPMod:Lemma:1:Prop}.  

For simplicity, we define ${\cal U}_{S,v}$ to be a set of maximally
independent $v$-tuples, where the components of each vector are
positive and less than $2S$.  Let $p$ be a prime such that $S < p
<2S$.  The previous discussion shows that either of the following
constructions of ${\cal U}_{S,v}$ can be used:
\[
{\cal U}_{S,v} =
\begin{array}{l}
  \{\, (1, i, i^2\mod p, \ldots, i^{v-1} \mod p) \mid 1 \le i \le v \,
\} \\[4pt]
  \{\, ((i+1)^{-1} \mod p, \ldots, (i+v)^{-1} \mod p) 
      \mid 1 \le i \le v \,\}
\end{array}
\]

\medskip
The following proposition uses the elements of the set ${\cal
U}_{nT,v}$ to construct univariate characterizing sequences for sparse 
polynomials.

\begin{proposition}\label{Sparse:ZeroPoly:Prop}
For every non-zero polynomial $P(X_1, \ldots, X_v)$ with no more than
$T$ non-zero terms and the degree of each $X_i$ bounded by $D$ there is a $\vec{u}$ in ${\cal U}_{vT,v}$ such that
$P(Z^{\vec{u}})$ is not identically zero.  Furthermore, the degree of
$P(Z^{\vec{u}})$ is less than $2v^2DT$ and $P(Z^{\vec{u}})$ has no
more than $T$ non-zero terms.
\end{proposition}

\begin{proof}
Let the non-zero terms of $P$ be
\[
P(\vec{X}) = c_1 \vec{X}^{\vec{e}_1} + c_2 \vec{X}^{\vec{e}_2} 
+ \cdots + c_T \vec{X}^{\vec{e}_T}.
\]
The substitution $X_i \mapsto Z^{u_i}$ transforms this polynomial into
\[
P(Z) = c_1 Z^{\vec{e}_1 \cdot \vec{u}} + c_2 Z^{\vec{e}_2 \cdot \vec{u}} +
\cdots +
c_T Z^{\vec{e}_T \cdot \vec{u}}.
\]
To find a substitution for which $P(Z^{\vec{u}})$ is not identically
zero we require $\vec{u}$ to satisfy
\[
\vec{e}_1 \cdot \vec{u} \not= \vec{e}_i \cdot \vec{u},
\]
or equivalently $(\vec{e}_i - \vec{e}_1) \cdot \vec{u} \not= 0$, for $2
\le i \le T$.  Let $d = \vec{e}_1 \cdot \vec{u}$.  With such a
substitution only the $c_1 \vec{X}^{\vec{e}_1}$ monomial of
$P(\vec{X})$ will be mapped to a term in $P(Z)$ of degree $d$.  Since
$c_1 \not= 0$, $P(Z)$ cannot be identically zero; it must contain a
$Z^d$ term.

Letting $L_i(\vec{w}) = (\vec{e}_i - \vec{e}_1)\cdot
\vec{w}$, $2 \le i < T$ we want to find a $\vec{w}$ at which none of
the $L_i$ vanish. 

Let $\vec{w}_1, \ldots, \vec{w}_v$ be distinct elements of ${\cal
U}_{vT, v}$, so
\[
\begin{pmatrix}\vec{w}_1 \\ \vdots \\ \vec{w}_v \end{pmatrix} \cdot
 (\vec{e}_i - \vec{e}_1)^T
 = 
\mathbf{A}\cdot (\vec{e}_i - \vec{e}_1)^T = 
  \begin{pmatrix}L_i(\vec{w}_1) \\ \vdots \\ L_i(\vec{w}_v) \end{pmatrix}.
\]
Since $\mathbf{A}$ is non-singular, the right hand side can only be zero
if $L_i$ is identically zero.  Thus, $L_i$ cannot vanish for more than
$n-1$ of the elements of ${\cal U}_{vT,v}$.  There are $T-1$ $L_i$'s.
Since $(v-1)\cdot (T-1)$ is less than $vT$, there must be at least one
element of ${\cal U}_{vT, v}$ for which none of the $L_i$ vanish as
desired.  We denote such an element by $\vec{u}$.  Each of the
components of $\vec{u}$ is less than $2nT$, while the elements of
$\vec{e}_i$ are less than $D$.  Thus the degree of $P(Z^{\vec{u}})$ is
less than $2v^2DT$.
\end{proof}

Using the univariate polynomial constructed in
\propref{Sparse:ZeroPoly:Prop} and the univariate polynomial zero test 
given by \propref{Positive:Zeroes:Prop}, we can construct the
following zero equivalence algorithm for black box polynomials over
the reals.

\begindsacode
RDZeroEquiv(${\cal B}_P$, $v$, $T$) := $\{$\\
\>loo\=p for $\vec{u} \in {\cal U}_{vT,v}$ do \{ \\
\>\> loo\=p for $1 \le z \le T$ do \{ \\
\>\>\> if \=${\cal B}_P(z^{u_1}, z^{u_2}, \ldots, z^{u_v}) \not= 0$ \\
\>\>\>\>then return(false); \\
\>\>\> \}\\
\>\> \}\\
\> return(true); \\
\> \}
\enddsacode

\begin{figure}
\begin{center}
\begin{tabular}{l|c|c|c|c|}
\multicolumn{1}{c}{} 
& \multicolumn{1}{c}{\# poly} & \multicolumn{1}{c}{\# terms} &
 \multicolumn{1}{c}{degree} & \multicolumn{1}{c}{\# points} \\ \cline{2-5}
Linear & $T$ & $\le vDT$ & $\le vDT$ & $vDT^2+T$ \\ \cline{2-5}
Nonlinear & $vT$ & $\le T$ & $\le v^2DT$ & $vT^2$ \\ \cline{2-5}
\end{tabular}
\end{center}
\caption{Complexity of different multivariate to univariate substitutions\label{Mult:Uni:Fig}}
\end{figure}

The table in \figref{Mult:Uni:Fig} illustrates the difference between
these two methods for black boxes over rings of characteristic zero.
Using the linear substitution \eqnref{Zero:LinearSubs:Eq} requires
generating $T$ polynomials of degree as large as $vDT$.  Since the
polynomials are dense the characterizing sequence of each must be of
length $vDT+1$ and the entire characterizing sequence for the black
box is of length $vDT^2+T$.  The non-linear substitution
\eqnref{Zero:MultiSub:Eq} produces more polynomials ($vT$) and
polynomials of greater degree, but each has no more than $T$ non-zero
terms.  By using positive integers as evaluation points, we get a
characterizing sequence of length $vT^2$.

\paragraph{Finite Fields}

If the characteristic of the coefficient domain is positive, then the
zero testing problem becomes a bit harder.  Consider, for instance,
the polynomial $M(X) = X^p - X$.  Even though $M(X)$ only has two
terms it vanishes for every element of $\F_p$.  Nonetheless,
\propref{Zero:Mon:Prop} is true for polynomial over fields of positive
characteristic.  The difficulty in this case is that there are no
values of $X$ that distinguish the two monomials of $M(X)$, $X^p$ and
$X$.  This problem does not arise for polynomials over fields of
characteristic zero, since there are many elements that distinguish
$X^m$ and $X^n$ when $m \not= n$.

This issue means that it is not possible to do {\em deterministic}
zero testing for polynomials over a finite field {\em without degree
bounds\/}.  However, the problem is solvable if we have degree bounds
on the black box.  

As usual, let ${\cal B}_Q$ be a black box for a polynomial $Q$.
Assume $Q$ is a univariate polynomial of degree $d$, with $T$ terms,
with coefficients in $\F_p$:
\[
Q(X) = q_1 X^{e_1} + q_2 X^{e_2} + \cdots + q_T X^{e_T},
\]
where $e_i \le d$.  Using \propref{Zero:Mon:Prop}, the sequence of
evaluation points, $1, m, m^2, \ldots$ will be a distinguishing
sequence if each of the values 
\[
m^{e_1}, m^{e_2}, \ldots, m^{e_T}
\]
are distinct. If the multiplicative order of $m$ is greater than $d$,
then these values are certainly distinct.

Now the problem is finding element of multiplicative order greater
than $d$.  In the case of $M(X) = X^p - X$, there is no such element
in $\F_p$ and thus there is no way to distinguish $M(X)$ from $0$
using a test sequence chosen from $\F_p$.  The solution to this
problem is to enlarge the ground field $\F_p$ to $\F_{p^k}$ which does
have elements of order $d$.  For instance, $\F_{p^2}$ has $\phi(p^2 -1)$
elements of order $p^2 -1$ any of which can be used to distinguish $M(X)$
from $0$.

In general there are three cases.  First, if the characteristic of the
ground field is very large, $p > 2^d$, then $m = 2$ will suffice.

Second, if $p$ is small we allow algorithms that are polynomial in
$p$.  The ground field $\F_p$, can be expanded by adjoining an element
of degree $k$ over $\F_p$, where $p^k > d$.  By
\propref{FF:AlgExtOrder:Prop} the field $\F_{p^k}$ has 
\[
\varphi(p^k -1) \approx \frac{6}{\pi^2} p^k -1
\]
elements of order $p^k -1$.  So with approximately $\lceil
\pi^2/6\rceil = 2$ random choices we should be able to find an
element of $\F_{p^k}$ whose multiplicative order is $p^k -1$.

Third, if $p$ is very large we cannot afford to find a multiplicative
generator of $\F_{p^k}$.  In this case we take the opposite approach,
and construct a very large degree extension of $\F_p$, one of degree
$K$ where $K > d$.  The generator of this extension will serve a good
generator for the irreducibility test.  
{\Adleman} and {\LenstraH} \cite{Adleman86} have shown that there
exists a positive integer $c$ such 
that using $(k \log p)^c$ steps one can find 
an irreducible polynomial of degree $d$ such that
\[
\frac{k}{c \log p} < d \le k.
\]
So, sufficiently large extensions can be found in polynomial
time.  %%\Marginpar{Needs work}

\section{Negative Results}
\label{Zero:Negative:Sec}

The zero equivalence problem with only degree bounds, and no bound on
the number of terms, is not solvable in deterministic polynomial time.
This is because it is too easy to find polynomials that vanish at a
large number of prescribed points and which still have small degree.
This is shown in the following proposition.

\begin{proposition}
\label{Key:Negative:Prop}
Let $D$, $T$ and $v$ be integers such that $D^v > T$.  Let ${\cal S} =
\{\vec{a}_i \}$ be a set of $T-1$ $v$-tuples.  There exists a
polynomial with rational integer coefficients with no more than $T$
non-zero monomials and that vanishes at each point in $\cal S$.
\end{proposition}

\begin{proof}
In fact, there are many polynomials that satisfy the requirements.
Choose a set of $T$ distinct $v$-tuples, $\vec{e}_1, \ldots, \vec{e}_T$,
where each component is an element of $\{0, 1, \ldots, D\}$.  The
polynomials with these exponents are all of the form 
\[
P(\vec{X}) = c_1 {\vec{X}}^{\vec{e}_1} + c_2 {\vec{X}}^{\vec{e}_2} 
+ \cdots + c_T {\vec{X}}^{\vec{e}_T}.
\]
We claim that at least one of these polynomials vanishes at each point
in ${\cal S}$.  For $P$ to vanish at $\vec{a}_i \in {\cal S}$ the $c_j$
must satisfy the following linear equation:
\[
c_1 {\vec{a}_i}^{\vec{e}_1} + c_2 {\vec{a}_i}^{\vec{e}_2} + \cdots +
c_T {\vec{a}_i}^{\vec{e}_T} = 0.
\]
Thus $P(\vec{X})$ vanishes at each element of $\cal S$ if the $c_j$
satisfy the following system of linear equations.
\[
\begin{aligned}
c_1 {\vec{a}_1}^{\vec{e}_1} + c_2 {\vec{a}_1}^{\vec{e}_2} + \cdots +
c_T {\vec{a}_1}^{\vec{e}_T} & = 0\\
c_1 {\vec{a}_2}^{\vec{e}_1} + c_2 {\vec{a}_2}^{\vec{e}_2} + \cdots +
c_T {\vec{a}_2}^{\vec{e}_T} & = 0\\
\vdots&\\
c_1 {\vec{a}_{T-1}}^{\vec{e}_1} + c_2 {\vec{a}_{T-1}}^{\vec{e}_2} 
+ \cdots +
c_T {\vec{a}_{T-1}}^{\vec{e}_T} & = 0
\end{aligned}
\]
Since these equations are homogeneous and there are more variables than
equations, the system possesses a non-trivial solution.
\end{proof}

\propref{Key:Negative:Prop} directly implies the non-existence of a
deterministic algorithm for zero recognition.

\begin{proposition}
\label{Black:Box:Prop}
Given a black box representing a polynomial $P(\vec{X})$ in $v$ variables
and of degree less than $D$ in each variable, any deterministic algorithm
that determines if $P$ is the zero polynomial runs in time at least
$O(D^v)$.
\end{proposition}

\begin{proof}
The time required by the algorithm is at least as large the number of
trial evaluations used.  Since no polynomial of degree less than $D$
has more than $D^v$ terms, $D^v$ trials suffice.  Thus the problem can
be solved in exponential time.  By \propref{Key:Negative:Prop}, if the
algorithm uses less than $D^v$ trials there will be polynomials that
meet the degree bounds and that are indistinguishable from the zero
polynomial.  Thus at least $D^v$ trial evaluations are required.
\end{proof}

Though the zero equivalence problem, given degree bounds, is not solvable
in deterministic polynomial time, notice that a polynomial that vanishes at
each of $O(D^v)$ trial points, constructed as in the proof of
\propref{Black:Box:Prop} would have $O(D^v)$ non-zero terms.  It would be
very interesting to know if there exists a polynomial that vanishes at
those trial points and that has a more succinct representation
(straight-line program, for instance).

\begin{figure}
{
\renewcommand{\arraystretch}{1.4}
\begin{center}
\begin{tabular}{l|c|c|}
\multicolumn{1}{c}{} & \multicolumn{1}{c}{Probabilistic}
   & \multicolumn{1}{c}{Deterministic}\\ 
\cline{2-3}
degree bounds & $\log {1 \over \epsilon} \times \log^{r-1} vD$ 
   & $D^v \log^r D$ \\ \cline{2-3}
term bounds & & $T^{r+1} \log^r v$ \\ \cline{2-3}
\end{tabular}
\end{center}
}
\caption{Complexity of Zero Testing\label{Zero:Comp:Fig}}
\end{figure}

The results of this section are summarized in the table in
\figref{Zero:Comp:Fig}.  The columns correspond to probabilistic and
deterministic algorithms respectively.  The rows correspond to whether
degree bounds or term bounds are given.  In the deterministic, bounded
degree case, we have given a lower bound on the time required, while
for the other two entries we have demonstrated algorithms that achieve
the indicated performance.  Also recall that $r$ is a constant
corresponding to the type of arithmetic being used by ${\cal B}_P$.
For classical arithmetic $r=2$; for fast arithmetic $r$ is slightly
greater than $1$.

As remarked earlier, the zero equivalence problem with degree bounds
cannot be solved in deterministic polynomial time, yet it can be
solved in probabilistic polynomial time.  We find it very curious that
with a slight change to the way information is provided, the problem
can be solved in deterministic polynomial time.  The zero equivalence
problem seems to lie on the boundary between those problems that can and
cannot be solved in deterministic polynomial time.  Though earlier
work has shown that random and deterministic polynomial time can be
separated via an oracle \cite{Heller86}, we find this example
interesting because it arises in a common practical problem.

By representing the polynomial as a black box, we have swept the issues of
the size of the computation required to compute $P(\vec{x})$ under the rug.
If we could look inside ${\cal B}_P$ and examine the ``program'' used to
compute $P(\vec{x})$ we might be able to show that ${\cal B}_P$ represents
the zero polynomial without any bounds on $P$.  For example, it seems
likely that one can prove that a straight-line program for a polynomial (in
the sense of {\Kaltofen} \cite{Kaltofen85d}) can be deterministically shown
to represent the zero polynomial in time polynomial in the size of the
straight-line program.



\section*{Notes}

\small

\notesectref{Zero:Probabilistic:Sec} The earliest mention of
probabilistic zero testing that we have been able to find appears in a
1978 note of {\DeMillo} and {\Lipton} on testing algebraic programs
\cite{Demillo78}.  They essentially reproduce \propref{Prob:Zero:Prop}.

Unaware of the prior work of {\DeMillo} and {\Lipton}, {\SchwartzJ}
\cite{Schwartz80} and {\Zippel} \cite{Zippel79a} simultaneously and
independently also proved \propref{Prob:Zero:Prop}.  (Apparently the time 
was ripe for this proposition.)  Their solutions to this problem were 
given in two successively presented papers at the EUROSAM '79 
conference in Marseille during the summer of 1979.  The slightly weaker 
result of the last line of the proposition was proven and used in the 
work of {\SchwartzJ} \cite{Schwartz80}, while that of the first line was 
given by  {\Zippel} \cite{Zippel79a}.

\notesectref{Zero:Deterministic:Sec}
The probabilistic techniques discussed in
\sectref{Zero:Probabilistic:Sec} were a dramatic improvement to the
state of the art in symbolic computation when they were introduced in
1979.  From the theoretical perspective, the work in
\cite{Zippel79b} essentially reduced a number of problems
(polynomial greatest common divisor and polynomial factorization) to
random polynomial time.  Moreover, the reduction led to algorithms
that were more efficient than the existing algorithms.  

For several years, it was believed that these problems were
intrinsically harder than deterministic polynomial time. In
particular, it was believed the zero equivalence problem could not be
solved in deterministic polynomial time.  The results of
\sectref{Zero:Negative:Sec}, which were well understood during the
1980's reinforced this belief.

In 1987 {\Grigoriev} and {\Karpinski}, as part of their work on
bipartite graph matching \cite{Grigoriev87}, developed a deterministic
algorithm for the zero equivalence problem for polynomials with
integer coefficients given a bound on the number of {\em non-zero
terms}.  These ideas where then used by {\BenOr} and {\Tiwari} in 1988
\cite{BenOr88} to produce a deterministic algorithm for interpolation
of sparse polynomials, which is discussed in
\sectref{Interp:BenOr:Sec}.  In 1988, {\Zippel} independently
developed a deterministic algorithm for the zero equivalence problem
with term bounds \cite{Zippel90} (as well as an interpolation
algorithm), which did not have the restrictions of {\Grigoriev} and
{\Karpinski} algorithm.

{\Risler} and {\Ronga} \cite{Risler90} independently developed an
approach similar to that of \cite{Zippel90,Grigoriev90}.

Building on \cite{Heintz80a}, {\Heintz} and {\Schnorr}
\cite{Heintz82a} have shown that there exist short characterizing
sequences for polynomials expressed as straight line programs.
Unfortunately, their techniques do not provide a way to generate these
characterizing sequences.  Determining these sequences is currently
the most important outstanding problem in zero testing.

\normalsize
