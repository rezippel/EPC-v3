%$Id: poly-factor.tex,v 1.1 1992/05/10 19:41:26 rz Exp rz $
\chapter{Univariate Factorization}
\label{UFactoring:Chap}
\index{factorization!of univariate polynomials|(}

This chapter discusses the factorization of univariate polynomials
over the rational integers, $\Z$.  This problem has been attacked by a 
combination of the Hensel lifting techniques and one of the algorithms 
for factorization over finite fields.  For many problems this approach 
has been quite effective, but in the worst case can take exponential 
time.  In 1982, A.~K.~{\LenstraA}, H.~W.~{\LenstraH} and {\Lovasz} 
\cite{LenstraAK82} developed an algorithm for factoring univariate 
polynomials over $\Z$ in polynomial time by using basis 
reduction techniques for lattices.  Suddenly, the complexity of factoring 
polynomials and a number of other problems was reduced to polynomial 
time, \eg, \cite{Landau85a}.  However, experience with actual implementations has
shown that the lattice reduction techniques are not faster than the
older exponential time techniques for any factorization problem that
can be completed in reasonable time today.

\sectref{UF:Reductions:Sec} discusses some simple reductions of the
factorization problem that can be used independently of the core 
factorization algorithm.  In \sectref{UF:Simple:Sec} we discuss the 
classical, exponential time factorization algorithm, which is probably
still the optimal approach.  Finally, in \sectref{UF:LLL:Sec} we
discuss the asymptotically good algorithms that make use of lattice
reduction techniques. \Marginpar{Need algorithm for factoring $X^n-1$.}

\section{Reductions}
\label{UF:Reductions:Sec}

When factoring a univariate polynomial, 
\[
F(Z) = f_0 Z^d + f_1 Z^{d-1} + \cdots + f_d,
\]
over $\Z$, two simplifications are effective.  First, removing the
integer content of $F(Z)$ decreases the size of the coefficients of
$F(Z)$, and does not change the number of non-zero terms.  For all
univariate factorization algorithms known at this writing, it is
worthwhile.

The case for the second simplification is not so clear.
As with all univariate problems, we define the size of a univariate
polynomial, $F(Z)$, to be $(\deg F)\cdot \|F\|_{\infty} = d \cdot
|F|$.\index{size, of univariate polynomial} Since the sum of the
degrees of the factors of $F$ is equal to the degree of $F$, the size
of the factorization of $F$ can only be larger than that of $F$
because of growth in the size of the coefficients.  By
\longpropref{Factor:CBound:Prop} the size of the factorization can
only increase by about a factor of $d$.  Furthermore, no factorization
of $F$, into irreducible components or not, is more than a factor of
$d$ larger than the size of $F$.  


\sectref{FFac:Sqfr:Sec} noted that square free decompositions of
polynomials over integral domains are easily reduced to {\sc gcd}
computations.  For univariate polynomials over $\Z$ the {\sc gcd}'s
can be effectively computed using the subresultant polynomial
remainder sequence\index{polynomial remainder sequence!subresultant}
discussed in \sectref{Subresultant:PRS:Sec}.  However, for large
degree polynomials the coefficient growth can be excessive.  In this
case, the modular interpolation algorithm discussed in
\sectref{PGCD:Uni:Sec} can be used.  Thus, computing a \key{square free 
decomposition} of $F(Z)$ will not increase the cost of the factorization 
significantly.  In \sectref{UF:Simple:Sec} we will show how a Hensel process 
can be used to lift the square free decomposition and the factorization 
in irreducibles together. 

Although one could also try to monicize the $F(Z)$:
\[
f_0^{d-1}F(Z) = (f_0 Z)^d + f_1 f_0 (f_0 Z)^{d-1} + \cdots + f_d
f_0^{d-1},
\]
the size of the coefficients of $F$ would be increased by as much as
$f_0^{d-1}$.  Since neither of the two factoring algorithms discussed
benefit significantly from being restricted to monic polynomials this
is not usually done.

\section{Simple Algorithm}
\label{UF:Simple:Sec}

We assume that $F(Z)$ is a primitive univariate
polynomial\index{primitive polynomial} over $\Z$
of degree $d$.  The classical univariate factorization algorithm
consists of three steps:
\begin{enumerate}
\item Choose a ``good'' random rational prime $p$ and factor $F(Z)$
into irreducible factors modulo $p$ 
\[
F(Z) = F_1^{e_1}(Z) F_2^{e_2}(Z) \cdots F_k^{e_k}(Z) \pmod{p}
\]

\item Use Newton's iteration to lift the $F_i$ to factors modulo
$p^{\ell}$,
\[
F(Z) = \hat{F}_1^{e_1}(Z) \cdot \hat{F}_2^{e_2}(Z) \cdots \hat{F}_k^{e_k}(Z)
\pmod{p^{\ell}}
\]

\item Combine the $\hat{F}_i$, as needed, into true divisors of $F$
over $\Z$.
\end{enumerate}

Each of these steps needs a bit of clarification.  The behavior of
this algorithm may be exponential when the final stage needs to
try many combinations of the $\hat{F}_i$ to get true factors over
$\Z$. Thus the best primes in the first step, are those for which the
factorization of $F(Z)$ modulo $p$ is as close as possible to the
factorization of $F(Z)$ over $\Z$.  Since univariate factorization
over finite fields is efficient, one usually tries several primes $p$
and picks the one that gives the coarsest factorization.  Depending on
the particular constants of the finite field factorization algorithm,
primes near the word size of the processor are usually chosen to
minimize the number of Newton steps required.  In current
implementations, between five and ten different primes may be tried.

When a prime $p$ is chosen the square free decomposition of $F(Z)$ is
first computed and compared with the square free decomposition of
$F(Z)$ computed with previous primes.  If any of the degrees of the
square free components modulo $p$ are larger than the degrees
previously encountered then $p$ has introduced extraneous repeated
factors and $p$ should be discarded.  If any of the degrees in the
previous square free decompositions are larger than those that occur
modulo $p$ then the decompositions with respect to the previous primes
introduced extraneous repeated factors and they should be discarded.  

Once we have several square free decompositions that have the same
degree distributions, we can apply a univariate finite field
factorization algorithm, \eg, \keyw{CZFactor} to the square free parts
of each of the square free decompositions.  The coarsest of the
resulting factorizations is then used as the starting point for
Newton's method. 

The coefficients of the divisors of $F(Z)$ are bounded in absolute
value by
\[
B = \sqrt{d+1} \cdot 2^d |F|.
\]
Since the coefficients may be either positive or negative $p^{\ell}$
must exceed $2B+1$.  Finally, notice that the finite field
factorization will yield monic polynomials.  Thus the coefficients of
$\hat{F}$ are images of $f_i/f_0$.  In order to recover both the
numerator and denominator we need an additional factor of $2$.  So
$\ell$ must be chosen such that
\[
p^{\ell} > 2(2B+1).
\]

At this point the system of equations formulation of Hensel's lemma
can be used to lift the factorization modulo $p$ to one modulo
$p^{\ell}$.  Either quadratic or linear iterations can be used, but
since $\ell$ can be rather large, the quadratic iteration of
\eqnref{Quasi:MNewton:Eq} is often worth the effort.

Some implementations prefer to use the Zassenhaus formulation of
Hensel's lemma instead of the system of equations formulation because
the overhead of the Zassenhaus formulation is often smaller in the
univariate case.  However, if the Zassenhaus formulation is used
either a square free decomposition of $F(Z)$ must be done over $\Z$ or
the Zassenhaus formulation must be adapted to lift the square free
decomposition as well \cite{Trager79}.  This approach is rather
complex and thus we prefer the simpler system of equations
formulation.

\medskip
The most time consuming step is the last one.  It actually has two
parts, once we have a potential factor of $F$ modulo $p^{\ell}$, we
need to convert it to a factor over $\Z$ and then must do a test division 
to see if it is actually a factor.

The factors modulo $p^{\ell}$ are monic.  One way to convert them to
potential divisors over $\Z$ is to multiply the divisor by the leading
coefficient of $F$, $f_0$ and convert to a \key{balanced
presentation}.  The primitive part\index{primitive part!of a
polynomial} of the resulting polynomial is the potential divisor.  If
the absolute value of any of the coefficients is larger than the
coefficient bound then there is no need to even attempt the
\key{trial division}.

Trial division is also used in the sparse modular {\sc gcd}
algorithm where it is expected that most trial divisions will be
successful.  With polynomial factorization the opposite is true.
Very often the \key{trial division} will be unsuccessful.  The following
observation can be used to significantly minimize the cost of trial
divisions.  If the polynomial $G(Z)$ divides $F(Z)$ then,
for all integers $r \in \Z$, $G(r)$ divides $F(r)$.  There is no point
in performing a polynomial trial division if this integer division fails.
Experience has shown \cite{abbot:1985:rf} that using $r= 0$ and $r=1$ will
often obviate the need to perform any polynomial test divisions.

\section{Asymptotically Good Algorithms}
\label{UF:LLL:Sec}

To illustrate the ideas behind the asymptotically good, lattice
reduction algorithms we present a simple factorization algorithm that
uses lattices in the same spirit as the Lenstra-Lenstra-Lov\'asz
algorithm.\Marginpar{Should we really include the LLL algorithm?}

Let $F(Z)$ be a polynomial over $\Z$:
\[
F(Z) = f_0 Z^d + f_1 Z^{d-1} + \cdots f_d,
\]
and let $\alpha$ be an approximation of some zero of $F(Z)$ that has
been computed to some precision.  The minimal polynomial for $\alpha$
is an irreducible polynomial that divides $F(Z)$, say $G(Z)$.
Repeating this process with $F(Z)/G(Z)$ gives the factorization of
$F(Z)$.

The minimal polynomial for $\alpha$ is found by searching for a
linear polynomial satisfied by $\alpha$, then a quadratic and so on.
To determine if there is a polynomial of degree $k$ that is satisfied
by $\alpha$, we create the $k+1$ dimensional lattice ${\cal L}_k$ that
has a basis of:
\[
(\alpha^k, 0, \ldots, 0), \quad(0, \alpha^{k-1}, 0, \ldots, 0), 
\quad\ldots, \quad(0, \ldots, 0, 1).
\]
The Lov\'asz basis reduction algorithm of \sectref{Lovasz:Basis:Sec}
can then be used to find a small vector in ${\cal L}_k$, \ie, rational
integers $g_0, \ldots, g_k$ such that 
\[
\left|g_0 \alpha^k + g_1 \alpha^{k-1} + \cdots + g_k\right| =
\epsilon_k
\]
is small.  If $\epsilon_k$ is sufficiently small then the zero that
$\alpha$ represents is actually a zero of
\[
G(Z) = g_0 Z^k + g_1 Z^{k-1} + \cdots + g_k
\]
and $G(Z)$ is an irreducible factor of $F(Z)$.


\medskip
The difficulty of using high precision floating point arithmetic and
the fact that not all polynomials have real zeroes forces the use of a
different lattice for factoring polynomials.  We begin by describing
this lattice and then prove that we have an algorithm.

Choose a prime $p$ for which $F(Z)$ is square free over $\F_p$.  Using
the Cantor-Zassenhaus algorithm, factor $F(Z)$ into irreducibles over
$\F_p$ and using Newton's method lift the factorization to one modulo
$p^m$ for some large $m$.  Let
\[
H(Z) = h_0 Z^{\ell} + h_1 Z^{\ell-1} + \cdots + h_{\ell}
\]
denote one of the irreducible factors of $F(Z)$ modulo $p^m$, viewed
as an element of $\Z[Z]$.  Although $H(Z)$ is not a factor of $F(Z)$
over $\Z$, it divides an irreducible factor, $\bar{H}(Z)$ of $F(Z)$ of
degree less than or equal to $r$.  We claim that the set of
polynomials in $\Z[Z]$ of degree less than or equal to $r$ that are
multiples of $H(Z)$ modulo $p^m$ form a lattice.  We have a bound on
the size of the coefficients of $\bar{H}(Z)$, so we will pick $m$
sufficiently large that $\bar{H}(Z)$ will correspond to an especially
short vector in the lattice.  The {\Lovasz} reduction algorithm is
then used to find a short vector in the lattice, which will correspond
to an irreducible factor of $F(Z)$.

We use the following mapping to embed the polynomials of degree no 
greater than $r$ in $\R^{r+1}$:
\[
\begin{aligned}
G(Z) &= g_0 Z^{r} + g_1 Z^{r-1} + \cdots + g_{r}, \\
G & \mapsto (g_0, g_1, \ldots, g_r),
\end{aligned}
\]
and use $G$ and $(g_0, \ldots, g_r)$ interchangeably.

Observe that any multiple of $H(Z)$ modulo $p^m$, and of degree no 
greater than $r$ can be written in the form 
\[
P(Z) \cdot H(Z) + p^m Q(Z)
\]
where the degree of $P(Z)$ is no more than $r - \ell$ and the degree
of $Q$ is less than $\ell$.  We claim that these polynomials (vectors) 
form a lattice ${\cal L}_H$.  To do this we must exhibit a basis.

Consider the vectors:
\begin{equation} \label{First:LLL:Eq}
\begin{aligned}
Z^{r-\ell} H(Z) & \mapsto (h_0, h_1, \ldots, h_{\ell}, 0, \ldots, 0),
\\
Z^{r-\ell -1} H(Z) & \mapsto (0, h_0, \ldots, 0, h_{\ell}, \ldots, 0), \\
& \vdots\\
Z H(Z) & \mapsto (0, \ldots, 0, h_0, h_1, \ldots, h_{\ell}, 0), \\
H(Z) & \mapsto (0, \ldots, 0, 0, h_0, \ldots, h_{\ell -1}, h_{\ell}). \\
\end{aligned}
\end{equation}
Any vector that corresponds to a multiple of $H(Z)$, and is of degree
$\le r$ is a linear combination of these vectors.  The coefficients in
the linear combination correspond to $P(Z)$.  

The following set of vectors is a basis for the polynomials $p^m
Q(Z)$:
\begin{equation} \label{Second:LLL:Eq}
\begin{aligned}
p^m Z^{\ell -1} & \mapsto (0,  \ldots, p^m, 0, \ldots, 0, 0), \\
p^m Z^{\ell -2} & \mapsto (0,  \ldots, 0, p^m, \ldots, 0, 0), \\
& \vdots\\
p^m Z & \mapsto (0, \ldots, 0, 0, \ldots, p^m, 0), \\
p^m & \mapsto (0, \ldots, 0, 0, \ldots, 0, p^m).
\end{aligned}
\end{equation}

Together, the set of vectors in \eqnref{First:LLL:Eq} and
\eqnref{Second:LLL:Eq} form a basis for the lattice ${\cal L}_H$.  Since 
$H$ is monic, $h_0 = 1$, the lattice determinant of ${\cal L}_H$ is $p^{m 
\ell}$. 

In \cite{LenstraAK82} the following proposition is proven

\begin{proposition}[Lenstra-Lenstra-{\Lovasz}]  With ${\cal L}_H$ defined as
above, if $G \in {\cal L}_H$ satisfies
\begin{equation} \label{LLL:Factor:Eq}
p^{m\ell} > \|F\|^r \cdot \|G\|^d
\end{equation}
then $\bar{H}(Z)$ divides $G(Z)$.  In particular, $\gcd(F(Z), G(Z)) =
\bar{H}(Z)$.
\end{proposition}

First, we need to establish a relationship between $\|\bar{H}\|$ and
$\|F\|$.  By \propref{Factor:CBound:Prop} we have
\[
\|\bar{H}\| < (d+1)^{1/2} 2^d |F|
\]
where $|\,|$ is the infinity or absolute value norm on vectors.
Applying \propref{PB:HeightRel:Prop} to both sides gives
\[
\frac{\|\bar{H}\|}{(r+1)^{1/2}} <  (d+1)^{1/2} 2^d \|F\|.
\]
Since, $d > r$ we have
\begin{equation} \label{Factor:2Norm:Eq}
\|\bar{H}\| < (d+1) 2^d \|F\|.
\end{equation}

$\bar{H}$ is a vector in the lattice ${\cal L}_H$.  So, if $\vec{b}_1,
\ldots, \vec{b}_{r+1}$ is a reduced basis of ${\cal L}_H$, 
\[
\|\vec{b}_1\| < 2^r \| \bar{H} \| < (d+1) 2^{d+r} \|F\|.
\]
Substituting this inequality into \eqnref{LLL:Factor:Eq} we see that
if $m$ satisfies
\[
p^{m\ell} > \|F\|^r \cdot \left( (d+1) 2^{d+r} \|F\|\right)^r
 = \|F\|^{r+d} (d+1)^d 2^{d(d+r)},
\]
then $\vec{b}_1$ will be a multiple of $\bar{H}$.  So we should choose
$m$ such that $m$ is larger than $2d \log \|F\|$.  Consequently, we can
find $\bar{H}$ by taking the {\sc gcd} of $\vec{b}_1$ and $F(Z)$.

A number of refinements of this algorithm can be
introduced.\Marginpar{Include some refinements.}  When
this is done, one can produce an algorithm that takes $O(d^9 \log
\|F\|)$ bit operations.  

\section*{Notes}

\small

\notesectref{UF:Simple:Sec} In \cite{Abbot87} it is observed that when
lifting the coefficients of the factorization using Hensel's
lemma, one should use a quadratic lift until reaching approximately
the cube root of the coefficient bound, \ie, until $p^{\ell/3}$ and
then use a linear lift from then on.  

\notesectref{UF:LLL:Sec} {\Schoenhage} \cite{Schonhage84} has given an
improvement to this algorithm that only requires $O(d^{6+\epsilon}+
d^4 \log^2 \|F\|)$ bit operations.  An alternative factorization
method, also based on lattice bases, was by {\Kannan}, {\LenstraA} and
{\Lovasz} \cite{Kannan:Lenstra:Lovasz}.  Recently, using relation
finding techniques {\MillerV} \cite{MillerV92} has shown that this can
be reduced to about $O(d^{5+\epsilon} \log
\|F\|)$.

Despite the practical difficulty of using the lattice algorithms for
factoring univariate polynomials over $\Z$, some explorations indicate
that when carefully implemented it may be a good technique for
factoring polynomials over algebraic number extensions \cite{Abbot87}.
Other approaches for factoring polynomials over algebraic extensions
are given in \cite{Trager76a,Rothschild:Weinberger}. 

\index{factorization!of univariate polynomials|)}

\normalsize
