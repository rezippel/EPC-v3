%$Id: ff-factor.tex,v 1.1 1992/05/10 19:40:35 rz Exp rz $

\chapter{Factoring over Finite Fields}
\label{Poly:FF:Factor:Chap}
\index{factorization!of univariate polynomials over finite fields|(}


Factoring polynomials is one of the most challenging problems in
algebraic computation.  The complete algorithm for factoring
multivariate polynomials over the rational integers uses nearly all of
the techniques developed in this book.  This chapter considers the
simpler problem of factoring univariate polynomials over the finite
fields.  This problem arises in coding theory, computational number 
theory and is a step in the practical algorithms for factoring 
polynomials over the rational integers.

Let $F(X)$ be a univariate polynomial with coefficients in the finite
field $\F_q$, where $p$ is a prime and $q=p^r$.  Two simple steps can
be taken that simplify the factorization problem.  In
\sectref{FFac:Sqfr:Sec} we show how to compute a \keyi{square free
factorization} of $F$, \ie, factor $F$ so that
\[
F(X) = F_1(X) F_2(X)^2 F_3(X)^3 \cdots F_k(X)^k,
\]
where the $F_i$ are pairwise relatively prime and $F_i$ and $F_i'$ are
relatively prime.  Square free decomposition of polynomials is a
simple sequence of {\sc gcd} computations and for univariate
polynomials direct computation using the \key{Euclidean algorithm} is almost
always efficient

The second simplification is to compute a \keyi{distinct degree
factorization} of the $F_i$, \ie,
\[
F_i(X) = F_{i1}(X) F_{i2}(X) F_{i3}(X) \cdots F_{i\ell}(X),
\]
where the $F_{ij}(X)$ are products of irreducible polynomials of
degree $j$.  Again this is a relatively simple process for univariate
polynomials over finite fields.

The most expensive part of univariate factorization over finite fields
is factoring the $F_{ij}$.  Factoring $F_{i1}$ is just zero finding in
$\F_q$.  A nice probabilistic algorithm for this is given in
\sectref{FFac:Linear:Sec}.  Determining the factors of $F_{ij}$
corresponds to finding zeroes in $\F_{q^j}$.  A technique for this is
discussed in \sectref{PolyFF:Cantor:Sec}.

A word about the complexity analysis of univariate factoring
algorithms is in order here.  The size of a univariate polynomial
$F(X)$ is $(\deg F) \cdot |F|$. Since the coefficients are elements of
the finite field $\F_q$, the size of $F$ is bounded by $(\deg F) \cdot
r \cdot (\log p)$.  Thus ``fast'' algorithms should be polynomial in
the degree of $F$ and polynomial in $\log p$.

\section{Square Free Decomposition}
\label{FFac:Sqfr:Sec}
\index{square free decomposition|(}

The square free decomposition algorithm described here is
valid for polynomials over any integral domain.  For simplicity
however, we describe the technique for polynomials over a field
$k$, even though the only field for which we use these algorithms is
$\F_q$.  

%\sectref{GFac:Sqfr:Sec} describes how to combine the
%techniques of this section with the lifting techniques of
%\chapref{Sparse:Hensel:Chap} to compute square free decompositions of
%multivariate polynomials and polynomials over the rational integers.

Let $F(X)$ be a monic polynomial over the field $k$,
\[
F(X) = X^n + f_1 X^{n-1} + \cdots + f_n.
\]
The \keyi{square
free decomposition} of $F(X)$ is a set of polynomials $F_1, \ldots,
F_k$ such that
\begin{equation}\label{FFac:Sqfr:Eq}
F(X) = F_1(X) \cdot F_2(X)^2 \cdot F_3(X)^3 \cdots F_k(X)^k,
\end{equation}
the $F_i$ are relatively prime and, for each $i$, $F_i$ and $F_i'$ are
relatively prime.  

The derivative of $F(X)$ is
\[
F'(X) = nX^{n-1} + (n-1) f_1 X^{n-2} + \cdots + f_{n-1}.
\]
If $F(X)$ is a constant, its derivative is zero.  If the characteristic 
of $k$ is positive, then $F'$ is also zero if the exponents of $X$ are 
multiples of $p$, \ie, when $F(X) = \hat{F}(X^p)$.  For instance, for $k= 
\F_7$ the derivative of 
\[
A(X) = X^{49} + 3 X^{14} + 3
\]
is zero.  However, $A(X) = 
\hat{A}(X^7) = (\hat{A}(X))^7$, where $\hat{A}(X) = X^7 +3X^2 + 3$ since, 
for $a$ in $\F_7$, $a^7 = a$.  This may not be true for more general 
fields.  For instance, if $B(X) = X^7 +Y$ and $k = \F_7(Y)$ then 
$B'(X) = 0$, but 
\[
X^7 + Y \not= (X+Y^{1/7})^7.
\]
So, an algebraic extension is needed in this case.  Such polynomials are 
called {\em inseparable} and their inclusion causes many problems with 
factoring algorithms.\index{inseparable polynomials}  For the most part, 
these polynomials are ignored in this book. 

The condition that $F_i$ and $F_i'$ be relatively prime is equivalent
to saying that the $F_i$ have no multiple zeroes.  For instance, let
$G(X) = (X - \alpha)^r H(X)$, where $(x - \alpha)$ does not divide
$H(X)$ and $r \ge 0$.  We say that $\alpha$ is a {\em zero of
multiplicity}\index{multiplicity!of polynomial zeroes} $r$ in this
case.  The multiplicity of a polynomial factor\index{multiplicity!of
polynomial factor} is defined in precisely the same fashion.  Each
of the non-constant $F_i$ in \eqnref{FFac:Sqfr:Eq} has multiplicity $i$
in $F(X)$.

Assume that $r$ is positive and the derivative of $G$ is not zero:
\[
\begin{aligned}
  G'(X) &= r (X - \alpha)^{r-1} H(X) + (X - \alpha)^r H'(X), \\
  & = (X - \alpha)^{r-1}\left[r H(X) + (X - \alpha) H'(X)\right].
\end{aligned}
\]
Since $(X - \alpha)$ does not divide $H(X)$, $\alpha$ has multiplicity
at least $r-1$ in $G'(X)$.\Marginpar{Need some comments about the
characteristic and $r$.}  If the multiplicity of $\alpha$ in $G$ is
greater than one then $\gcd(G, G') \not = 1$.  So $(G, G') = 1$
implies that $G(X)$ has no multiple zeroes. Notice that the
requirement that $G'(X) \not= 0$ is also covered by $(G, G') = 1$.

The following proposition generalizes this result to polynomial
factors.  Its proof is the same, but requires that each irreducible
factor possess a zero.  This is true in the algebraic closure of $k$.

\begin{proposition}
Let $G(X)$ be an irreducible factor of $F(X)$, a polynomial over a
field.  If the {\sc gcd} of $F(X)$ and $F'(X)$ is $1$ then the multiplicity
of $G(X)$ in $F(X)$ is exactly $1$.
\end{proposition}

\medskip
With these ideas, we can construct an algorithm for computing the
square free decomposition of $F(X)$.  By the previous reasoning
$F_i(X)$ has multiplicity at least $i-1$ in $F'(X)$, so
\[
(F(X), F'(X)) = F_2 F_3^2 \cdots F_k^{k-1} = G(X).
\]
Repeating this process with $G(X)$ gives
\[
(G(X), G'(X)) = F_3 F_4^2 \cdots F_k^{k-2} = H(X).
\]
Therefore,
\[
\frac{F}{G} = F_1 F_2 \cdots F_k \quad\mbox{and}\quad
\frac{G}{H} = F_2 \cdots F_k.
\]
The ratio of these quantities is
\[
\frac{F \cdot H}{G \cdot G} = F_1.
\]
Repeating this process, we can compute the the $F_i$ in sequence.

The routine \keyw{PolySQFR} in \figref{SQFR:Fact:Fig} implements this
approach.  Note that we have assumed that $F'$ is not zero.

\begin{figure}
\begindsacode
PolySQFR ($F(X)$) := $\{$ \\
\> $G \leftarrow \mbox{PolyGCD}(F, \mbox{PolyDeriv}(F, X))$;\\
\> $H \leftarrow \mbox{PolyGCD}(G, \mbox{PolyDeriv}(G, X))$;\\
\> $i \leftarrow 1$;\\
\> loo\=p while $\deg F > 0$ do $\{$\\
\> \> $F_i \leftarrow (F/G)/(G/H)$; \\
\> \> $(F, G, H) \leftarrow \mbox{PolyGCD}(G, H)$;\\
\>\> $i \leftarrow i + 1$; \\
\>\> $\}$\\
\> $\}$
\enddsacode
\caption{Square free factorization algorithm\label{SQFR:Fact:Fig}}
\end{figure}
\index{square free decomposition|)}

\section{Distinct Degree Factorization}
\label{FFac:Distinct:Sec}
\index{distinct degree factorization|(}

While the square free decomposition algorithm applies to polynomials
over any ring, distinct degree factorization is an operation that
appears practical only for univariate polynomials over finite fields.
At this point, we assume that $F(X)$ is a square free monic polynomial
over $\F_q$ of degree $n$, where $q= p^r$.  A \keyi{distinct degree
factorization} of $F(X)$, is a set of polynomials $F_1, \ldots,
F_{\ell}$ such that
\[
F(X) = F_1(x) \cdot F_2(X) \cdots F_k(X)
\]
and $F_j$ is the product of irreducible polynomials of degree $j$.

By Fermat's little theorem, each element of $\F_q$ is a zero of $X^q-
X$, \ie, 
\[
\prod_{\alpha \in \F_q} (X - \alpha) = X^q - X.
\]
So $X^q- X$ is the product of all the monic linear polynomial over
$\F_q$.  Since we have assumed $F(X)$ is square free, $F_1$ is the
{\sc gcd} of $F(X)$ and $X^q - X$.

Similarly, the product of all monic polynomials of degree less than
$\ell$ is $X^{q^\ell} -X$ and we have\Marginpar{This needs to be justified.}
\[
(F(X), X^{q^{\ell}} - X) = F_1(X) \cdots F_{\ell}(X).
\]

Distinct degree factorization is thus just a matter of computing a few
{\sc gcd}'s.  However, a trick is necessary.  For large values of $q$,
the standard remainder algorithm for the first step of the Euclidean
algorithm with $X^{q^{\ell}}-X$ by $F(X)$ will take $O(q^{\ell})$
steps, which is too costly.  Instead, it is preferable to compute
$X^{q^{\ell}}$ by repeated squaring modulo $F(X)$.  

\begin{figure}
\begindsacode
PolyExptMod (base, $M$, $F$) := $\{$ \\
\> $\mbox{prod} \leftarrow 1$; \\
\> $\mbox{base} \leftarrow \mbox{PolyRemainder}(\mbox{base}, F)$; \\
\> loo\=p do $\{$\\
\>\> if $\mbox{oddp}(M)$ then $\mbox{prod} \leftarrow \mbox{PolyRemainder}(\mbox{prod} \times \mbox{base}, F)$;\\
\>\> $M \leftarrow \lfloor M/2 \rfloor$; \\
\>\> if $M = 0$ then $\mbox{return}(\mbox{prod})$;\\
\>\> else $\mbox{base} \leftarrow \mbox{PolyRemainder}(\mbox{base}
  \times\mbox{base}, F)$; \\
\>\> $\}$\\
\>$\}$
\enddsacode
\caption{Exponentiation of a polynomial modulo
another\label{PolyExptMod:Fig}}
\end{figure}

The routine \keyw{PolyExptMod} in \figref{PolyExptMod:Fig} uses
this technique to compute $\mbox{\tt base}^M
\mod{F}$.\index{polynomial exponentiation}  The loop
inside \keyw{PolyExptMod} is executed $O(\log M)$ times.  During each
loop at most two dense polynomial multiplies and one polynomial
remainder are performed.  Each polynomial involved is of degree no
larger than $\deg F = n$.  Thus the number of coefficient operations
performed is $O(n^2 \log M)$.

The {\sc gcd} of $F(X)$ and $X^{q^\ell} -X$ is computed as follows
\[
\mbox{\tt PolyGCD}(F(X), \mbox{\tt PolyExptMod}(X, q^{\ell}, F(X)) - X)
\]
That is, the first remainder is computed using repeated squaring, and
then the rest of the computation is done using the Euclidean algorithm.

Combining these ideas we have the following routine for computing the
distinct degree factorization of $F$.
\begindsacode
PolyDistinctDegree ($F$) := $\{$ \\
\> declare $F \in \F_q[X]$; \\
\> $i = 1$; \\
\> loo\=p while $F \not= 1$ do $\{$ \\
\>\> if \=$i | \deg F$ then $\{$ \\
\>\>\> $F_i \leftarrow
      \mbox{PolyGCD}(F(X), \mbox{PolyExptMod}(X, q^i, F(X)) - X)$;\\
\>\>\> $F \leftarrow F/F_i$ ; \\
\>\>\> $\}$ \\
\>\> $i \leftarrow i + 1$;\\
\>\> $\}$ \\
\> $\}$
\enddsacode

\noindent
Notice that  as each $F_i$ is determined, it is factored out of $F$.
This makes the succeeding {\sc gcd}'s a bit smaller and simplifies dealing
with factors of degree less than $i$.  

Let $n$ denote the degree of $F$ in $X$.  Asymptotically, the \Marginpar{This analysis is wrong.}
computation of each $F_i$ takes time $O(n^2 \log q)$.  The loop in
\keyw{PolyDistinctDegree} is repeated at most $d(n)$ time, where
$d(n)$ is the number of divisors of $n$.  Since $d(n) = O(n^{1/2})$ by
\longpropref{NT:Divisor:Est:Prop} the total time taken is bounded by
$O(n^{5/2}(\log q))$.

\index{distinct degree factorization|)}
\section{Finding Linear Factors}
\label{FFac:Linear:Sec}

Using the techniques of the last two sections, we are left with the
problem of factoring a univariate polynomial, all of whose factors are
of the same degree.  Splitting polynomials of this type into
irreducible components essentially reduces to zero finding in $\F_q$
or some extension of $\F_q$.  This section illustrates the basic ideas
involved and their analysis, by focusing on the case where $F(X)$ is
the product of linear factors.  The next section considers this
problem in more generality.

We assume that $F(X)$ is a monic, square free polynomial over $\F_q$
that is the product of linear factors.  Further assume that $q$ is
odd and denote by $n$ the degree of $F(X)$.  Since $F(X)$ is both square
free and the product of linear factors, it must divide $X^q - X$.  This
polynomial is easily factored:
\[
X^q - X = X\cdot (X^{(q-1)/2} - 1)\cdot (X^{(q-1)/2} + 1) 
  = X \cdot R(X) \cdot N(X)
\]
If the zeroes of $F(X)$ are uniformly distributed then about half of
them will be zeroes of $N(X)$ and half of $R(X)$, \ie, half will be
quadratic residues and half will not be.  Denote the {\sc gcd} of
$F(X)$ and $X^{(q-1)/2} - 1$ by $F_1(X)$.  The expected degree of
$F_1(X)$ is $n/2$.  Notice that by using \keyw{PolyExptMod} we can do
this in $O(n \log q)$ time.  So we have reduced the problem of
factoring a polynomial of degree $n$ to that of factoring one of
degree $n/2$.  If we can continue this splitting process with $F_1(X)$
and $F(X)/F_1(X)$, then after about $n$ {\sc gcd} computation we
should be able to split $F(X)$ into linear factors.  This would lead
to an $O(n^2 \log q)$ factoring algorithm. 

The technique of the previous paragraph split the zeroes of $F(X)$ up\Marginpar{This needs to be analyzed more carefully.}
into classes depending upon whether they were quadratic residues or
non-residues.  We can scramble the quadratic character of the zeroes
by adding an arbitrary constant to each zero.  That is, by working
with $F_1(X-b)$ instead $F(X)$.  Intuitively we are assuming that if
$\{ a_i \}$ is a set of quadratic residues modulo $q$.  Then, on
average, only half of the elements of the set $\{ a_i + b\}$ will be
quadratic residues.

Thus we can split $F_1(X)$ further by computing the {\sc gcd} of $F_1(X- b)$
and $X^{(q-1)/2} - 1$.  Further splits can be obtained by choosing
other values for $b$.  It turns out that it is slightly more efficient
to take the {\sc gcd} of $F_1(X)$ and $(X+b)^{(q-1)/2} - 1$ so that we can
avoid the cost of shifting the $F_1(X)$ (we always need to raise $X$
to the $(q-1)/2$ power).

\begindsacode
PolyLinearFactors ($F$) := $\{$ \\
\> declare $F \in \F_q[X]$; \\
\> if $\deg F = 1$ then $\{ F \}$; \\
\> else $\{$\\
\> \> $F_1 \leftarrow 
   \mbox{PolyGCD}(F, \mbox{PolyExptMod}(X+\mbox{random}(\F_q), (q-1)/2, F) -1)$;\\
\>\> if \=$0 < \deg F_1 < \deg F$ then \\
\>\>\> $\mbox{PolyLinearFactors}(F_1) \cup \mbox{PolyLinearFactors}(F/F_1)$;\\
\>\> else $\mbox{PolyLinearFactors}(F)$;\\
\>\> $\}$\\
\> $\}$
\enddsacode

\noindent
The routine \keyw{random} chooses a uniformly distributed random
element from its argument, which is assumed to be a finite set.

We delay the analysis of this algorithm to the end of the next section
where a more powerful version is discussed.

%This algorithm requires $O(n)$ {\sc gcd} computations to separate
%the roots of $F(X)$.  Each one requires one call to
%\keyw{PolyExptMod}, which requires $O(n^2 \log p)$ coefficient
%operations, a small polynomial {\sc gcd} of about $O(n^3)$ operations
%and a division of no more than $O(n^2)$ operations.  Thus each
%separation requires about $O(n^2 \log p)$, and the whole process will
%require $O(n^3 \log p)$ operations to compute the linear factors of a
%polynomial.

\section{Cantor-Zassenhaus Algorithm}
\label{PolyFF:Cantor:Sec}

The technique developed in the last section for factoring a polynomial
$F(X)$ finds a sequence of polynomials such that the {\sc gcd} of
$F(X)$ and each polynomial gives a proper factor of $F(X)$.  We showed
that the sequence of polynomials $(X+a)^{(q-1)/2} - 1$, where $a$ is a
random element of $\F_q$, works when $F(X)$ consists only of linear
factors.  In this section, we construct other polynomials that work
when $F(X)$ has higher degree factors.

{\Rabin} suggested a direct generalization of the approach used in the
previous section.  Let $F(X)$ be a polynomial of degree $d$ over a finite 
field $\F_q$ with factors $F_1, \ldots, F_r$, each of which has degree 
$s$, $d = rs$.  Although $F(X)$ has no linear factors as a polynomial over
$\F_q$, it splits into linear factors over $\F_{q^s}$ since 
\[
\F_q[X]/(F_i(X)) \simeq \F_q[X]/(F_j(X)).
\]
To construct $\F_{q^s}$ we need to find a polynomial of degree $s$,
irreducible over $\F_q$, say $P(X)$.  Then $\F_{q^s} =
\F_q[\alpha]/(P(\alpha))$.  We can find the linear factors of $F(X)$
in $\F_{q^s}$ using \keyw{PolyLinearFactors}.  To convert this
factorization over $\F_{q^s}$ to one over $\F_q$ we choose a linear
factor of $F(X)$, say $X + a(\alpha)$ and take its norm\index{norm}
from $\F_{q^s}[X]$ to $\F_q[X]$, \ie, compute the \key{resultant}
\[
G(X) = \res_Y(X + a(Y), P(Y)).
\]
$G(X)$ will be a polynomial of degree $s$ over $\F_q$ that has a
factor in common with $F(X)$.  Therefore, $G(X)$ is one of the
irreducible factors of $F(X)$.  We then delete all factors of $F(X)$
that divide $G(X)$ and continue.

The major problem with this approach is the need to compute in an
algebraic extension of $\F_q$.  The factoring algorithm of
{\CantorD} and {\Zassenhaus} \cite{Cantor81} eliminates this problem. 

Let $R$ denote the ring $\F_q[X]/(F(X))$, so
\[
R = \F_q[X]/(F_1(X)) \oplus \F_q[X]/(F_2(X)) \oplus \cdots \oplus
\F_q[X]/(F_r(X)).
\]
By the \key{Chinese remainder theorem} there exist polynomials $e_1, \ldots,
e_r$ of degree less than $d$ such that
\begin{equation} \label{Wedd:Basis:Eq}
e_i(X) = 
\begin{cases}
1 & \text{$\pmod{F_i}$} \\
0 & \text{$\pmod{F_j}$, $j\not=i$.}
\end{cases}
\end{equation}
The value of an expression involving the $e_i$ modulo $F(X)$ can be
determined by looking at its image modulo each of the $F_i$.  For
instance, 
\[
e_1(X) + e_2(X) + \cdots + e_r(X) = 1 \pmod{F(X)}
\]
since the sum is equal to $1$ modulo each of the $F_i$.  Furthermore,
the degree of each of the $e_i$ is less than $d = \deg F$ so, we have
\begin{equation} \label{Wedd:Split:Eq}
e_1(X) + e_2(X) + \cdots + e_r(X) = 1
\end{equation}
Similarly, we have
\[
e_i(X) e_j(X) = 
\begin{cases}
1 & when $i = j$ \\
0 & otherwise
\end{cases} \pmod{F_i}
\]
Notice that the $e_i(X)$ are idempotents of the ring $\F[X]/(F(X))$.

Let $S \cup T$ be a disjoint partition of the set $\{1, 2, \ldots, r
\}$.  Define $A_S(X)$ to be 
\[
A_S(X) = a_1 e_1(X) + a_2 e_2(X) + \cdots + a_r e_r(X),
\]
where $a_i = 0$ if $i \in S$ and $a_i = 1$ if $i \in T$.  Let $G_S(X)
= \gcd(F(X), A_S(X))$ be the {\sc gcd} of $A_S(X)$ and $F(X)$.  The
only possible divisors of $G_S(X)$ are the $F_i(X)$. But by
\eqnref{Wedd:Basis:Eq} the remainder of $A_S(X)$ when divided by
$F_i(X)$ is $1$ if and only if $i$ is an element of $T$.  Therefore,
\[
\gcd(F(X), A_S(X)) = \prod_{i \in S} F_i(X).
\]

Thus, a single {\sc gcd} with $A_S(X)$ is likely to split $F(X)$ into
two factors.  If we can produce polynomials like $A_S(X)$ easily, for
different sets $S$, then $F(X)$ can be factored by repeated {\sc
gcd}'s.

Let $m = (q^s -1)/2$ and let $B(X)$ be a randomly chosen element of
$R$, of degree $s$.  By the \key{Chinese remainder theorem}
\[
B(X) = B_1(X) e_1(X) + B_2(X) e_2(X) +\cdots + B_r(X) e_r(X),
\]
where $B_i(X) = B(X) \mod{F_i(X)}$.  Modulo $F_i(X)$, $B^m(X) =
B_i^m(X)$.  Now consider $B_i^m(X)$ as an element of $\F_q[X]/(F_j(X))
\cong \F_{q^s}$.  If $B_i(X)$ is not zero then
\[
B_i^{2m}(X) = \bigl(B_i(X)\bigr)^{q^s - 1} = 1.
\]
So, modulo $F_i$, $B_i^m(X)$ is either $0$ or $\pm 1$.  Letting $b_i =
B_i^m(X) \mod{F_i(X)}$, we can write 
\[
B^m(X) = b_1 e_1(X) + b_2 e_2(X) + \cdots b_r e_r(X),
\]
where $b_i \in \{-1, 0, +1 \}$.

Assume that $B(X)$ is not a constant.  Not all of the $b_i$ are zero
since otherwise $B(X)$ would be a multiple of $F(X)$, but the degree
of $B$ is less than $d$.  If any of the $b_i$ are zero then the {\sc
gcd} of $B(X)$ and $F(X)$ will be non-trivial.  Now we can assume that
none of the $b_i$ are zero.  If some, but not all, of the $b_i$ are
equal to $+1$ then the {\sc gcd} of $B^m(X)-1$ and $F(X)$ will
yield a proper factor of $F(X)$.  

We have the following factorization algorithm, which is very
similar to \keyw{PolyLinearFactors}.  
\begindsacode
CZFactor ($F$, $s$) := $\{$ \\
\> declare $F \in \F_q[X]$; \\
\> $d \leftarrow \deg F$; \\
\> if $d = s$ then $\{ F \}$; \\
\> els\=e $\{$\\
\>\> $B \leftarrow \mbox{RandomPoly}(\F_q, s)$; \\
\>\> $F_1 \leftarrow \mbox{PolyGCD}(B, F)$; \\
\>\> if \=$0 < \deg F_1 <  d$ then return ($\mbox{CZFactor}(F_1) \cup
   \mbox{CZFactor}(F/F_1)$); \\
\>\> else \\
\>\>\> $F_1 \leftarrow 
   \mbox{PolyGCD}(F, \mbox{PolyExptMod}(B, (q^s-1)/2, F) -1)$;\\
\>\>\> if $0 < \deg F_1 < n$ then $\mbox{CZFactor}(F)$;\\
\>\>\> else $\mbox{CZFactor}(F_1) \cup \mbox{CZFactor}(F/F_1)$;\\
\>\>\> \} \\
\>\> $\}$\\
\> $\}$
\enddsacode
\index{CZFactor@\protect\texttt{CZFactor}}
The function
\keyw{RandomPoly}$(L, s)$ returns a monic polynomial of degree $s$
with coefficients chosen randomly from the ring $L$.  Notice that this
routine is very similar to {\tt PolyLinearFactors} when $s = 1$.

\paragraph{Analysis}

The key question in analyzing the behavior of this algorithm is
determining the number of {\sc gcd}'s that will be required to
completely split $F(X)$ into irreducible factors.  

The random polynomial $B(X)$, whose degree is less than $\deg F = d$, will fail to yield a
proper factor of $F$ if and only if either all of the $b_i$ are equal
to $-1$ or all are equal to $+1$.  There are at most $m$ different
polynomials, $B_i(X)$, of degree less than $s$ such that 
\[
B_i^m(X) = 1 \pmod{F_i(X)}
\]
Thus by the \key{Chinese remainder theorem} there are at most $m^r$
polynomials of degree less than $d$ whose residues modulo each of the
$F_i$ are equal to $1$.  Similarly, there are at most $m^r$
polynomials with residues of $-1$.

There are $q^d - q$ non-constant polynomials of degree less than $d$
over $\F_q$.  The number of polynomials we want to avoid is the $2
m^r$ polynomials mentioned above.  Thus, if $B(X)$ is chosen uniformly
from the set of non-constant polynomials, the likelihood that it will
fail to yield a factor of $F(X)$ is
\[
\frac{2m^r}{q^d - q} = 2 \frac{((q^s-1)/2)^r}{q^d -q} 
  < 2 \frac{q^d}{2^r (q^d -q)} \le \frac{1}{2^{r-1}}.
\]
So the expected number of random polynomials required to produce a
single factor of $F(X)$ is no more than $2$.

Now we compute the time required when the {\sc gcd} successfully
produces a non-trivial factor of $F$.  Since $F$ has $r$ factors, $r$
``good'' {\sc gcd}'s will be required.  The cost of the {\sc gcd}
itself is $O(d^2)$ operations, while the call to {\tt PolyExptMod}
cost $O(d^2 \log (q^s - 1)) = O(s d^2 \log q)$ arithmetic operations.
Since $O(r)$ non-trivial {\sc gcd}'s are required, the time of
this algorithm is $O(d^3 \log q)$.

\section*{Notes}

\small

\notesectref{FFac:Distinct:Sec} The distinct degree factorization
method described here was first published by {\Arwin} \cite{Arwin18}.

\notesectref{FFac:Linear:Sec}  There has been significant amount of
research to improve techniques of this section in special cases.  The
first work along these lines is due to {\Moenck} \cite{Moenck77} who 
requires that $q-1 = 2^{\ell} Q$.  This approach was generalized in
\cite{Menezes88,Oorschot89,Menezes92} to efficiently factor
polynomials when the prime divisors of $q-1$ where no larger than
$O(\log q)$. 

All of the root finding algorithms are either deterministic and take
time polynomial in $p$ or are probabilistic and take time polynomial
in $\log p$.  An interesting exception to this is a special case
solved by {\Schoof} \cite{Schoof85}.  Using elliptic 
curves,\index{elliptic curve} he demonstrated an algorithm that finds 
linear factors of $X^2 - a \mod{p}$ in time polynomial in $\log p$ and, 
surprisingly, $|a|$!   

\notesectref{PolyFF:Cantor:Sec} The algorithm presented in this
section has the drawback that it assumes that the characteristic of
the field is odd.  When the characteristic is even, the original
algorithms of {\Berlekamp} \cite{Berlekamp67,Berlekamp70} seem
preferable.

Other results for special cases are given in
\cite{Huang91a,Huang91b,Ronyai89,Niederreiter93a,Niederreiter93b}.\Marginpar{Check
out the last two of these results.} 

\index{factorization!of univariate polynomials over finite fields|)}

\normalsize
