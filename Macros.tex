%% The bulk of this file is abbreviations used in the book.  At the
%% end there is some additional 

% \RequirePackage{mathrsfs}
\RequirePackage{amsmath}
\RequirePackage{amssymb}

%% Commutative diagrams
%\usepackage{amscd}
\usepackage{tikz-cd}
\usepackage{pb-diagram}
%\usetikzlibrary{matrix,arrows,decorations.pathmorphing}

\usepackage{hyperref}
\usepackage{hypcap}

% Hebrew fonts\
\usepackage{cjhebrew}

%%%%%% Font support

\usepackage{mathpazo}
% Lucida
% \usepackage[T1]{lucidabr}
% \usepackage{lucidabr}

%CM Bright
%\usepackage{cmbright}

%% Palantino/Euler
%\usepackage{pxfonts}
%\usepackage{eulervm}

%% Concrete Math and EulerVM
% \usepackage [ T1 ]{ fontenc } % Needed for Type1 Concrete
% \usepackage {concrete} % Loads Concrete + Euler VM

% Kurier
%\usepackage[math]{kurier}

% Palatino
%\usepackage{pxfonts}

% Arev sans and Arev Math
%\usepackage{arev}

% Bitstream Charter with Math Design math
% \usepackage[charter]{mathdesign}

% Utopiawith Math Design math
%\usepackage[utopia]{mathdesign}

% Artemsia with Euler Math
% \usepackage{gfsartemisia-euler}
% \usepackage[T1]{fontenc}

\usepackage{graphicx}
\graphicspath{{Pix/}}

%\usepackage{makeidx}
\usepackage{makeidx,showidx}
\usepackage{xcite}
\usepackage[font=small,totoc=true]{idxlayout}

\setlength{\textwidth}{11.5cm}
\setlength{\textheight}{19cm}
\setlength{\topmargin}{24pt}
\setlength{\headheight}{12pt}
\setlength{\headsep}{20pt}
\setlength{\footskip}{24pt}

\usepackage{losymbol}


\newcommand{\mathify}[1]{\ifmmode{#1}\else\mbox{$#1$}\fi}
\newcommand{\bbth}[1]{\mathify{{}^{{{\textit{#1}}}}}}
\newcommand{\rd}{\bbth{rd}}
\newcommand{\thHigh}{\bbth{th}}
\newcommand{\nd}{\bbth{nd}}
\newcommand{\st}{\bbth{st}}

% Math symbols
\newcommand{\A}{\mathbb{A}}    % Affine space
\newcommand{\C}{\mathbb{C}}
\newcommand{\F}{\mathbb{F}}
%\newcommand{\N}{\mathbb{N}}
\newcommand{\Q}{\mathbb{Q}}
\newcommand{\R}{\mathbb{R}}
\newcommand{\Z}{\mathbb{Z}}

\newcommand{\cont}{\mathop{\rm cont}\nolimits}
\newcommand{\prim}{\mathop{\rm prim}\nolimits}
\newcommand{\Lt}{\mathop{\rm Lt}\nolimits}
\newcommand{\lt}{\mathop{\rm lt}\nolimits}
\newcommand{\lc}{\mathop{\rm lc}\nolimits}
\newcommand{\lexp}{\mathop{\rm le}\nolimits}
\newcommand{\terms}{\mathop{\rm terms}\nolimits}
\newcommand{\coef}{\mathop{\rm coef}\nolimits}
\newcommand{\dens}{\mathop{\rm dens}\nolimits}
\newcommand{\skel}{\mathop{\rm skel}\nolimits}
\newcommand{\sep}{\mathop{\rm sep}\nolimits}
\newcommand{\Char}{\mathop{\rm char}\nolimits}
\newcommand{\Mor}{\mathop{\rm Mor}\nolimits}
\newcommand{\End}{\mathop{\rm End}\nolimits}
\newcommand{\prem}{\mathop{\rm prem}\nolimits}
\newcommand{\lcm}{\mathop{\rm lcm}\nolimits}
\newcommand{\num}{\mathop{\rm num}\nolimits}
\newcommand{\den}{\mathop{\rm den}\nolimits}
\newcommand{\res}{\mathop{\rm res}\nolimits}
\newcommand{\val}{\mathop{\rm val}\nolimits}
\newcommand{\Tr}{\mathop{\rm Tr}\nolimits}
\newcommand{\Ord}{\mathop{\rm ord}\nolimits}
\newcommand{\Res}{\mathop{\rm Res}\nolimits}
%\newcommand{\Mult}{\mathop{\rm Mult}\nolimits}
\newcommand{\Spec}{\mathop{\rm Spec}\nolimits}
\newcommand{\coker}{\mathop{\rm coker}\nolimits}
\newcommand{\Gal}{\mathop{\rm Gal}\nolimits}
\newcommand{\Norm}{\mathop{\mathbf{N}}\nolimits}
\newcommand{\Dscr}{\mathop{\mathbf{D}}\nolimits}
\newcommand{\Logint}{\mathop{\rm Li}\nolimits}
\newcommand{\unifdeg}{\mathop{\rm unifdeg}\nolimits}
\newcommand{\fieldDegree}[2]{[#1\!:\!#2]}
\newcommand{\groupDegree}[2]{(#1\!:\!#2)}
\newcommand{\legendre}{\overwithdelims()}
\newcommand{\dist}{\mathop{\rm dist}\nolimits}
\newcommand{\vol}{\mathop{\rm vol}\nolimits}

\newtheorem{theorem}{Theorem}
\newtheorem{lemma}[theorem]{Lemma}
\newtheorem{proposition}[theorem]{Proposition}
\newtheorem{definition}{Definition}
\newtheorem{corollary}{Corollary}
\newtheorem{conjecture}{Conjecture}
\newtheorem{fact}{Fact}
\newtheorem{problem}{Problem}
\newenvironment{proof}{\noindent{\bf Proof:}}{\unskip~~$\QEDbox$\medskip}
\def\QEDbox{\fbox{\rule{0ex}{1ex}}}

% Cross referencing commands
\newcommand{\chapref}[1]{Chapter~\ref{#1}}
\newcommand{\sectref}[1]{Section~\ref{#1}}
\newcommand{\appenref}[1]{Appendix~\ref{#1}}
\newcommand{\thmref}[1]{Theorem~\ref{#1}}
\newcommand{\propref}[1]{Proposition~\ref{#1}}
\newcommand{\conjref}[1]{Conjecture~\ref{#1}}
\newcommand{\corref}[1]{Corollary~\ref{#1}}
\newcommand{\defref}[1]{Definition~\ref{#1}}
\newcommand{\lemref}[1]{Lemma~\ref{#1}}
\newcommand{\figref}[1]{Figure~\ref{#1}}
\newcommand{\tableref}[1]{Table~\ref{#1}}
\newcommand{\eqnref}[1]{(\ref{#1})}

% English things in Latin
\newcommand{\ie}{{\em i.e.}}
\newcommand{\viz}{{\em viz.}}
\newcommand{\cf}{{\em cf.}}
\newcommand{\Eg}{{\em E.g.}}
\newcommand{\eg}{{\em e.g.}}
\newcommand{\etal}{{\em et.al.}}

%% Abbreviations used to simplify creating an index entry while
%% defining a term. 
\def\keyi#1{\textit{#1}\index{#1}}
\def\keyb#1{\textbf{#1}\index{#1}}
\def\keyw#1{\texttt{#1}\index{#1@\protect\texttt{#1}}}
\def\key#1{{#1}\index{#1}}

% The following  were intended to be used in in the index, but there
% are better ways to do this now. I'm commenting these out here to
% make sure they don't get used and have included the better approach
% below. 
%\def\bold#1{{\bf #1}}
% \index{cubic|texbf}
%\def\see#1{{\it see} #1}
%\index{cubic|see{Gauss}}

\def\Altran{{\sc Altran}\index{Altran@\sc Altran\rm}}
\def\Alpak{{\sc Alpak}\index{Alpak@\sc Alpak\rm}}
\def\Axiom{{Axiom}\index{Axiom\rm}}
\def\Camal{{\sc Camal}\index{Camal@\sc Camal\rm}}
\def\CLisp{{\sc Common Lisp}\index{Common Lisp@\sc Common Lisp\rm}}
\def\Formac{{\sc Formac}\index{Formac=\sc Formac\rm}}
\def\Lisp{{\sc Lisp}\index{Lisp=\sc Lisp\rm}}
\def\Macsyma{{\sc Macsyma}\index{Macsyma@\sc Macsyma\rm}}
\def\Maple{{\sc Maple}\index{Maple!@\sc Maple\rm}}
\def\Mathlab{{\sc Mathlab 68}\index{Mathlab@\sc Mathlab 68\rm}}
\def\Mathematica{{\sc Mathematica}\index{Mathematica@\sc Mathematica\rm}}
\def\Reduce{{\sc Reduce}\index{Reduce@\sc Reduce\rm}}
\def\Sac{{\sc Sac}\index{Sac@\sc Sac\rm}}
\def\Saint{{\sc Saint}\index{Saint@\sc Saint\rm}}
\def\Weyl{{Weyl}\index{Weyl}}
\def\pger{{\mathfrak p}}
\def\Pger{{\mathfrak P}}
\def\mger{{\mathfrak m}}

\def\BBoxExp{R_{{\cal B}_P}}

\def\lexord{\ge_{\rm lex}}
\def\totord{\ge_{\rm tot}}
\def\revord{\ge_{\rm rev}}
\def\domord{\ge_{\rm dom}}

%Used to be a \Sloppy here
\def\Marginpar#1{\marginpar{\tiny#1}}  
\marginparwidth 0.75in \marginparsep 4pt 

% Below here is special to Books

\newcommand{\longpropref}[1]{Proposition~\ref{#1} (page~\pageref{#1})}
\newcommand{\notesectref}[1]{\medskip\noindent{\bf \S\ref{#1}}}
\newcommand{\exeref}[1]{Exercise~\ref{#1} (page~\pageref{#1})}
\newcommand{\sign}{\mathop{\rm sign}\nolimits}


\makeatletter
\def\@showidx#1{\insert\indexbox{\tiny 
 \hsize\marginparwidth 
 \hangindent\marginparsep \parindent\z@ 
 \everypar{}\let\par\@@par \parfillskip\@flushglue 
 \lineskip\normallineskip 
 \baselineskip .8\normalbaselineskip\sloppy
 \raggedright \leavevmode 
 \vrule height .7\normalbaselineskip width \z@\relax
 #1\relax\vrule
 height \z@ depth .3\normalbaselineskip width \z@}}

\def\thebibliography#1{\chapter*{Bibliography\@mkboth
 {Bibliography}{Bibliography}} \par
 \addcontentsline{toc}{chapter}{\protect\numberline{Bibliography}}
The pages on which each reference is cited are included in brackets at
the end of each reference. \par\medskip\small\list
 {[\arabic{enumi}]}{\settowidth\labelwidth{[#1]}\leftmargin\labelwidth
 \advance\leftmargin\labelsep
 \usecounter{enumi}}
 \def\newblock{\hskip .11em plus .33em minus -.07em}
 \sloppy\clubpenalty4000\widowpenalty4000
 \sfcode`\.=1000\relax}
\makeatother

\def\mdline#1#2{
  \hbox to\hsize{\hskip\leftmargin\small{\tt (#2)}\hfill$\displaystyle{#1}$\hfill}
  \smallskip}

\newcounter{exercisenum}
\newenvironment{exercise}{ \begin{list}{\arabic{exercisenum}.}\item }%
 {\end{list}\addtocounter{exercisenum}{1}}

\newcounter{algstepnum}
\newenvironment{algorithm}[1]%
  {\begin{list}{\bf #1\arabic{algstepnum}.}%
   {\usecounter{algstepnum}}%
   \setcounter{algstepnum}{0}}%
  {\end{list}}
\newcommand{\stepref}[1]{step~{\bf\ref{#1}}}

\def\begindsacode{%
  \par\begin{center}\begin{minipage}{6in}%
  \begingroup\small\tt\begin{tabbing}12\=34\=\kill}
\def\enddsacode{\end{tabbing}\endgroup\end{minipage}\end{center}}

\usepackage{rzbook}
% \usepackage[bookmarks=false,hyperindex=false]{hyperref}
% \let\WriteBookMarks\relax
\raggedbottom

\makeindex

%\includeonly{front,zerotest,interp,spinterp,pgcd}
%\includeonly{euclids,contfrac,dio-anal,lattice,arithfun,f-fields}

\setcounter{secnumdepth}{3}     % Number subsubsection's as well
\setcounter{tocdepth}{3}        % Also put in contents 

\def\th{\thHigh}


% Names of people
\def\Abel{Abel\oldindex{Abel, Niels Henrik}}
\def\Adleman{Adleman\oldindex{Adleman, Leonard Max}}
\def\Aho{Aho\oldindex{Aho, Alfred Vaino}}
\def\Alagar{Alagar\oldindex{Alagar, Vangalur S.}}
\def\Amthor{Amthor\oldindex{Amthor, Carl Ernst August}}
\def\Apostol{Apostol\oldindex{Apostol, Thomas}}
\def\Arwin{Arwin\oldindex{Arwin, A.}}
\def\Atiyah{Atiyah\oldindex{Atiyah, Michael, Sir}}
\def\Babai{Babai\oldindex{Babai, L{\'a}szl{\'o}}}
\def\Babbage{Babbage\oldindex{Babbage, Charles}}
\def\BachE{Bach\oldindex{Bach, Eric}}
\def\Bahr{Bahr\oldindex{Bahr, Knut}}
\def\BartonDRa{Barton\oldindex{Barton, David R.}}
\def\BartonDRb{Barton\oldindex{Barton, David R?.}}
\def\BatemanPT{Bateman\oldindex{Bateman, Paul T.}}
\def\Beauzamy{Beauzamy\oldindex{Beauzamy, Bernard}}
\def\BenOr{Ben Or\oldindex{Ben Or, Michael}}
\def\Bentley{Bentley\oldindex{Bentley, Jon Louis}}
\def\Berlekamp{Berlekamp\oldindex{Berlekamp, Elwyn Ralph}}
\def\Besicovitch{Besicovitch\oldindex{Besicovitch, Abraham Samoilovitch}}
\def\BloomS{Bloom\oldindex{Bloom, S.}}
\def\Borodin{Borodin\oldindex{Borodin, Allan Bertram}}
\def\Bourne{Bourne\oldindex{Bourne, Steven R.}}
\def\Brent{Brent\oldindex{Brent, Richard Peirce}}
\def\Brezinski{Brezinski\oldindex{Brezinski, Claude}}
\def\BrownWS{Brown\oldindex{Brown, William Stanley}}
\def\Buchberger{Buchberger\oldindex{Buchberger, Bruno}}
\def\Canny{Canny\oldindex{Canny, John Francis}}
\def\CantorD{Cantor\oldindex{Cantor, David Geoffrey}}
\def\CantorG{Cantor\oldindex{Cantor, Georg Ferdinand Louis Philippe}}
\def\Capelli{Capelli\oldindex{Capelli, A.}}
\def\Carlitz{Carlitz\oldindex{Carlitz, Leonard}}
\def\Carmichael{Carmichael\oldindex{Carmichael, Robert Daniel}}
\def\Cassels{Cassels\oldindex{Cassels, John William Scott}}
\def\Cauchy{Cauchy\oldindex{Cauchy, Augustin Louis}}
\def\Caviness{Caviness\oldindex{Caviness, Bobby Forrester}}
\def\Cayley{Cayley\oldindex{Cayley, Arthur}}
\def\Cerlienco{Cerlienco\oldindex{Cerlienco, L.}}
\def\Chandra{Chandra\oldindex{Chandra, Ashok Kumar}}
\def\CharBW{Char\oldindex{Char, Bruce W.}}
\def\Chebyshev{\v{C}ebyshev\oldindex{Cebyshev@\v{C}ebyshev, Pafnouty L'vovich}}
\def\Chebotarev{\v{C}ebotarev\oldindex{Cebotarev@\v{C}ebotarev, Nikola\u{\i} Grigor'evich}}
\def\Chistov{Chistov\oldindex{Chistov, A. L.}}
\def\Chrystal{Chrystal\oldindex{Chrystal, George}}
\def\CohenP{Cohen\oldindex{Cohen, Paul J.}}
\def\CohenS{Cohen\oldindex{Cohen, S. D.}}
\def\CohnJHE{Cohn\oldindex{Cohn, John H. E.}}
\def\Collins{Collins\oldindex{Collins, George Edwin}}
\def\Cooley{Cooley\oldindex{Cooley, James William}}
\def\Coppersmith{Coppersmith\oldindex{Coppersmith, Don}}
\def\Cramer{Cram\'er\oldindex{Cram\'er, Harold}}
\def\DavenportJ{Davenport\oldindex{Davenport, James Harold}}
\def\DavenportH{Davenport\oldindex{Davenport, Harold}}
\def\Dedekind{Dedekind\oldindex{Dedekind, Richard}}
\def\Deligne{Deligne\oldindex{Deligne, Pierre}}
\def\DeMillo{DeMillo\oldindex{Demillo, Richard A.}}
\def\Descartes{Descartes\oldindex{Descartes, Ren\'{e}e}}
\def\Diffie{Diffie\oldindex{Diffie, Bailey Whitfield}}
\def\Dirichlet{Dirichlet\oldindex{Dirichlet, Peter Gustav Lejeune}}
\def\Dorge{D\"{o}rge\oldindex{Doerge@D\protect\"{o}rge, K.}}
\def\Dumas{Dumas\oldindex{Dumas, G.}}
\def\Eisenstein{Eisenstein\oldindex{Eisenstein, Ferdinand Gotthold Max}}
\def\Engleman{Engleman\oldindex{Engleman, Carl}}
\def\Erdos{Erd\H{o}s\oldindex{Erdos@Erd\protect\H{o}s, P\'{a}l}}
\def\Euler{Euler\oldindex{Euler, Leonhard}}
\def\EvansR{Evans\oldindex{Evans, R. J.}}
\def\Fagin{Fagin\oldindex{Fagin, Ronald}}
\def\Faltings{Faltings\oldindex{Faltings, Gerd}}
\def\Fateman{Fateman\oldindex{Fateman, Richard J}}
\def\Feldman{Feldman\oldindex{Feldman, Stuart I.}}
\def\Fermat{Fermat\oldindex{Fermat, Pierre de}}
\def\Fibonacci{Fibonacci\oldindex{Fibonacci, Leonardo Pisano}}
\def\Fitch{Fitch\oldindex{Fitch, John}}
\def\Fried{Fried\oldindex{Fried, Michael D.}}
\def\Gallagher{Gallagher\oldindex{Gallagher, Patrick X.}}
\def\Galois{Galois\oldindex{Galois, \'{E}variste}}
\def\Gathen{von zur Gathen\oldindex{von zur Gathen, Joachim}}
\def\Gauss{Gauss\oldindex{Gauss, Karl Friedrich}}
\def\Geddes{Geddes\oldindex{Geddes, Keith O.}}
\def\Gelfond{Gel'fond\oldindex{Gelfond@Gel'fond, Alexsandr Osipovich}}
\def\Genesereth{Genesereth\oldindex{Genesereth, Michael R.}}
\def\Gentleman{Gentleman\oldindex{Gentleman, W. Morven}}
\def\Gianni{Gianni\oldindex{Gianni, Patrizia}}
\def\Golden{Golden\oldindex{Golden, Jeffrey P.}}
\def\Goldwasser{Goldwasser\oldindex{Goldwasser, Shafrira}}
\def\Gonnet{Gonnet\oldindex{Gonnet, Gaston H.}}
\def\Gordan{Gordan\oldindex{Gordan, Paul Albert}}
\def\Gosper{Gosper\oldindex{Gosper, Ralph William, Jr.}}
\def\Griesmer{Griesmer\oldindex{Griesmer, James}}
\def\Grigoriev{Grigor'ev\oldindex{Grigor'ev, Dima Yu.}}
\def\Grothendieck{Grothendieck\oldindex{Grothendieck, Alexandre}}
\def\Habicht{Habicht\oldindex{Habicht, W.}}
\def\Hadamard{Hadamard\oldindex{Hadamard, Jacques Salomon}}
\def\HallA{Hall\oldindex{Hall, Andrew D.}}
\def\Hardy{Hardy\oldindex{Hardy, Godfrey Harold}}
\def\Hearn{Hearn\oldindex{Hearn, Anthony}}
\def\Heilbronn{Heilbronn\oldindex{Heilbronn, H. A.}}
\def\Heintz{Heintz\oldindex{Heintz, Joos}}
\def\Hellman{Hellman\oldindex{Hellman, Martin Edward}}
\def\Hensel{Hensel\oldindex{Hensel, Kurt Wilhelm Sebastian}}
\def\Hermite{Hermite\oldindex{Hermite, Charles}}
\def\Hilbert{Hilbert\oldindex{Hilbert, David}}
\def\Hironaka{Hironaka\oldindex{Hironaka, Heisuke}}
\def\Holdt{van Holdt\oldindex{van Holdt}}
\def\Holder{H\"{o}lder\oldindex{Hoelder@H\protect\"{o}lder, O.}}
%\def\Holder{H\"{o}lder\oldindex{Holder@H\protect\"{o}lder, Ludwig Otto}}
% Jordan-Holder
\def\Hopcroft{Hopcroft\oldindex{Hopcroft, John Edward}}
\def\Hurwitz{Hurwitz\oldindex{Hurwitz, Adolf}}
\def\Ireland{Ireland\oldindex{Ireland, Kenneth}}
\def\Isaacs{Isaacs\oldindex{Isaacs, I. Martin}}
\def\Jenks{Jenks\oldindex{Jenks, Richard D.}}
\def\Jensen{Jensen\oldindex{Jensen, Johan Ludwig William Valdemar}}
\def\JohnsonS{Johnson\oldindex{Johnson, Steven C.}}
\def\JordanC{Jordan\oldindex{Jordan, Camille}}
\def\Kahrimanian{Kahrimanian\oldindex{Kahrimanian}}
\def\Kaltofen{Kaltofen\oldindex{Kaltofen, Erich}}
\def\Kannan{Kannan\oldindex{Kannan, Ravi}}
\def\Karatsuba{Karatsuba\oldindex{Karatsuba, Anatoli\u\i\ Alekseevich}}
\def\Karpinski{Karpinski\oldindex{Karpinski, Marek}}
\def\Kilian{Kilian\oldindex{Kilian, Joseph}}
\def\Klein{Klein\oldindex{Klein, Felix}}
\def\Knobloch{Knobloch\oldindex{Knobloch, Hans--Wilhelm}}
\def\Knuth{Knuth\oldindex{Knuth, Donald Ervin}}
\def\Koblitz{Koblitz\oldindex{Koblitz, Neal}}
\def\Koepf{Koepf\oldindex{Koepf, Wolfram}}
\def\Kronecker{Kronecker\oldindex{Kronecker, Leopold}}
\def\Krull{Krull\oldindex{Krull, Wolfgang}}
\def\Kulp{Kulp\oldindex{Kulp, John L.}}
\def\Kung{Kung\oldindex{Kung, Hsiang Tsung}}
\def\Kummer{Kummer\oldindex{Kummer, Ernst Eduard}}
\def\Lagarias{Lagarias\oldindex{Lagarias, Jeffrey C.}}
\def\Lagrange{Lagrange\oldindex{Lagrange, Joseph Louis}}
\def\Lang{Lang\oldindex{Lang, Serge}}
\def\Lame{Lam\'e\oldindex{Lam\'e, Gabriel}}
\def\LandauE{Landau\oldindex{Landau, Edmund Georg Hermann}}
\def\LandauS{Landau\oldindex{Landau, Susan Eva}}
\def\LehmerD{Lehmer\oldindex{Lehmer, Derrick Norman}}
\def\LehmerE{Lehmer\oldindex{Lehmer, Emma Markovna Trotskaia}}
\def\LenstraA{Lenstra\oldindex{Lenstra, Arjen Klaas}}
\def\LenstraH{Lenstra\oldindex{Lenstra, Hendrik W., Jr.}}
\def\Liouville{Liouville\oldindex{Lioouville, Joseph}}
\def\Lipton{Lipton\oldindex{Lipton, Richard Jay}}
\def\Lovasz{Lov\'asz\oldindex{Lov\'asz, L\'aszl\'o}}
\def\Loos{Loos\oldindex{Loos, R\protect\"{u}diger Georg Konrad}}
\def\Lovelace{Lovelace\oldindex{Lovelace, Ada Augusta, countess of}}
\def\Ma{Ma\oldindex{Ma, Keju}}
\def\Macaulay{Macaulay\oldindex{Macaulay, Francis Sowerby}}
\def\MacDonald{MacDonald\oldindex{MacDonald, I. G.}}
\def\Mack{Mack\oldindex{Mack, Dieter}}
\def\Mahler{Mahler\oldindex{Mahler, Kurt}}
\def\Manove{Manove\oldindex{Manove, Michael}}
\def\MartinW{Martin\oldindex{Martin, William A.}}
\def\Mason{Mason\oldindex{Mason, R. C.}}
\def\Massey{Massey\oldindex{Massey, James L.}}
\def\Mazur{Mazur\oldindex{Mazur, Barry}}
\def\McCarthy{McCarthy\oldindex{McCarthy, John}}
\def\McIlroy{McIlroy\oldindex{McIlroy, M. Douglas}}
\def\Merkle{Merkle\oldindex{Merkle, Ralph}}
\def\Mertens{Mertens\oldindex{Mertens, F.}}
\def\Mignotte{Mignotte\oldindex{Mignotte, Maurice}}
\def\MillerG{Miller\oldindex{Miller, Gary Lee}}
\def\MillerJCP{Miller\oldindex{Miller, Jeffrey Charles Percy}}
\def\MillerV{Miller\oldindex{Miller, Victor Saul}}
\def\Minkowski{Minkowski\oldindex{Minkowski, Hermann}}
\def\Minsky{Minsky\oldindex{Minsky, Marvin}}
\def\MitchellO{Mitchell\oldindex{Mitchell, O. H.}}
\def\Moenck{Moenck\oldindex{Moenck, Robert T.}}
\def\MosesJ{Moses\oldindex{Moses, Joel}}
\def\MuirT{Muir\oldindex{Muir, Thomas}}
\def\Musser{Musser\oldindex{Musser, David Rea}}
\def\Nathanson{Nathanson\oldindex{Nathanson, Melvyn B.}}
\def\Netto{Netto\oldindex{Netto, Eugen}}
\def\Newton{Newton\oldindex{Newton, Isaac, Sir}}
\def\NoetherE{Noether\oldindex{Noether, Emma}}
\def\NoetherM{Noether\oldindex{Noether, Max}}
\def\Norman{Norman\oldindex{Norman, Arthur C.}}
\def\Odlyzko{Odlyzko\oldindex{Odlyzko, Andrew Michael}}
\def\Ofman{Ofman\oldindex{Ofman, Y}}
\def\Olds{Olds\oldindex{Olds, Carl Douglas}}
\def\Ore{Ore\oldindex{Ore, \protect\"{O}ystein}}
\def\Pan{Pan\oldindex{Pan, Viktor \t{Ia}kovlevich}}
\def\Perron{Perron\oldindex{Perron, Oskar}}
\def\Piras{Piras\oldindex{Piras, F.}}
\def\Pollard{Pollard\oldindex{Pollard, John Michael}}
\def\Polya{P\'olya\oldindex{Polya@P\'olya, George}}
\def\Pratt{Pratt\oldindex{Pratt, Vaugh Ronald}}
\def\Probst{Probst\oldindex{Probst, David K.}}
\def\Rabin{Rabin\oldindex{Rabin, Michael Oser}}
\def\Ramanujan{R\={a}m\={a}nujuan\oldindex{Ramanujan@R\protect\={a}m\protect\={a}nujuan, 
   Sr\protect\={\i}niv\protect\={a}sa Aiya\protect\.{n}g\protect\={a}r}}
\def\Redei{R\'{e}dei\oldindex{Redei@R\'{e}dei, Alfred}}
\def\Riemann{Riemann\oldindex{Riemann, Georg Fiedrich Bernhard}}
\def\Renyi{R\'enyi\oldindex{R\'enyi, Alfr\'ed}}
\def\Richards{Richards\oldindex{Richards, Ian}}
\def\Risch{Risch\oldindex{Risch, Robert H.}}
\def\Risler{Risler\oldindex{Risler, Jean-Jacques}}
\def\Ritchie{Ritchie\oldindex{Ritchie, Dennis M.}}
\def\Ritt{Ritt\oldindex{Ritt, Joseph Fels}}
\def\Rivest{Rivest\oldindex{Rivest, Ronald Linn}}
\def\Ronga{Ronga\oldindex{Ronga, Felice}}
\def\Rosen{Rosen\oldindex{Rosen, Michael}}
\def\Rothschild{Rothschild\oldindex{Rothschild, Linda Preis}}
\def\Rump{Rump\oldindex{Rump, Siegfried M.}}
\def\Runge{Runge\oldindex{Runge, Carl David Tolme}}
\def\SalvyB{Salvy\oldindex{Salvy, Bruno}}
\def\SchatzS{Schatz\oldindex{Schatz, Steven}}
\def\Schinzel{Schinzel\oldindex{Schinzel, Andrej}}
\def\Schnorr{Schnorr\oldindex{Schnorr, Claus-Peter}}
\def\Schoenhage{Sch\"{o}nhage\oldindex{Schoenhage@Sch\protect\"{o}nhage, Arnold}}
\def\Schoof{Schoof\oldindex{Schoof, Ren\'ee}}
\def\SchwartzJ{Schwartz\oldindex{Schwartz, Jacob Theodore}}
\def\SchwartzL{Schwartz\oldindex{Schwartz, Laurent}}
\def\Seidenberg{Seidenberg\oldindex{Seidenberg, Abraham}}
\def\Serre{Serre\oldindex{Serre, Jean Pierre}}
\def\Shackell{Shackell\oldindex{Shakell, John R.}}
\def\Shallit{Shallit\oldindex{Shallit, Jeff}}
\def\Shamir{Shamir\oldindex{Shamir, Adi}}
\def\Silverman{Silverman\oldindex{Silverman, Joseph H.}}
\def\Singer{Singer\oldindex{Singer, Michael F.}}
\def\Slagle{Slagle\oldindex{Slagle, James R.}}
\def\Solovay{Solovay\oldindex{Solovay, Robert Martin}}
\def\Stark{Stark\oldindex{Stark, Harold M.}}
\def\Sterling{Sterling\oldindex{Sterling, James}}
\def\Swan{Swan\oldindex{Swan, R. G.}}
\def\SwinnertonDyer{Swinnerton-Dyer\oldindex{Swinnerton-Dyer, Henry Peter Francis}}
\def\Sylvester{Sylvester\oldindex{Sylvester, James Joseph}}
\def\Szego{Szeg\"o\oldindex{Szego@Szeg\protect\"o, Gabor}}
\def\Tarjan{Tarjan\oldindex{Tarjan, Robert Endre}}
\def\Tarski{Tarski\oldindex{Tarski, Alfred}}
\def\Tate{Tate\oldindex{Tate, John T.}}
\def\Tiwari{Tiwari\oldindex{Tiwari, Prasoon}}
\def\Tompa{Tompa\oldindex{Tompa, Martin}}
\def\TrabbPardo{Trabb Pardo\oldindex{Trabb Pardo, Luis Isidoro}}
\def\Trager{Trager\oldindex{Trager, Barry Marshall}}
\def\Traub{Traub\oldindex{Traub, Joseph Fredrick}}
\def\Tukey{Tukey\oldindex{Tukey, John Wilder}}
\def\Turing{Turing\oldindex{Turing, Alan Mathison}}
\def\Hulzen{van Hulzen\oldindex{van Hulzen, J. A.}}
\def\Ullman{Ullman\oldindex{Ullman, Jeffrey David}}
\def\ValleePoussin{de la Vall\'ee-Poussin\oldindex{Vallee-Poussin@de la
   Vall\protect\'{e}e-Poussin, Charles Jean Gustave Nicolas}}
\def\Vaughn{Vaughn\oldindex{Vaughn, R. C.}}
\def\Waerden{van der Waerden\oldindex{van der Waerden, Bartel Leendert}}
\def\Waldschmidt{Waldschmidt\oldindex{Waldschmidt, Michel}}
\def\WangP{Wang\oldindex{Wang, Paul Shyh-Horng}}
\def\Weierstrass{Weierstrass\oldindex{Weierstrass, Karl Theodor Wilhelm}}
\def\Weinberger{Weinberger\oldindex{Weinberger, Peter J.}}
\def\Weil{Weil\oldindex{Weil, Andr\'e}}
\def\Weyl{Weyl\oldindex{Weyl, Hermann Klaus Hugo}}
\def\WilliamsH{Williams\oldindex{Williams, Hugh Cowie}}
\def\Wolfram{Wolfram\oldindex{Wolfram, Steven}}
\def\Wright{Wright\oldindex{Wright, Edward Maitland}}
\def\Yao{Yao\oldindex{Yao, Andrew Chi-Chih}}
\def\Yun{Yun\oldindex{Yun, David Yuan-Yee}}
\def\Zassenhaus{Zassenhaus\oldindex{Zassenhaus, Hans Julius}}
\def\Zippel{Zippel\oldindex{Zippel, Richard Eliot}}


