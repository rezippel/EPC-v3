%$Id: poly-gcd.tex,v 1.1 1992/05/10 19:41:36 rz Exp rz $
\chapter[Polynomial GCD's: Interpolation Algorithms]{Polynomial GCD's\\Interpolation Algorithms}
\label{Poly:GCD:Chap}

\index{greatest common divisor!of polynomials|(}

We now use the interpolation algorithms of
Chapters~\ref{Interpolation:Chap} and \ref{Sparse:Interp:Chap} to
compute the {\sc gcd} of two polynomials.  This is the first of the
modern algorithms that we discuss.  Although the principles
behind the sparse polynomial {\sc gcd} algorithm are quite simple, the
final algorithm is more complex than any discussed thus far.

For preciseness, assume that we wish to compute the {\sc gcd} of two
multivariate polynomials, $F(X_1, \ldots, X_v)$ and $G(X_1, \ldots,
X_v)$ whose coefficients lie in an integral domain $R$.  Unless
otherwise stated, $R$ is either $\Z$, $\Q$ or $\F_q$.  When $R$ has
positive characteristic, we assume that $R$ has plenty of elements for
the interpolation process. Further assume that the {\sc gcd} of $F$ and
$G$ is $H(X_1, \ldots, X_v)$.  The {\sc gcd} problem for several
polynomials can be reduced to the two polynomial problem by the
technique of \sectref{PRS:Content:Sec}.

Historically, the motivation for the techniques discussed in this
chapter was the empirical observation that the {\sc gcd} of two
polynomials is very often $1$.  This is the worst case for
the classical {\sc gcd} algorithms discussed in \chapref{PRS:Chap},
all of which effectively compute something at least as large as the
resultant of $F$ and $G$.  Some technique needed to be
developed whose intermediate structures were no larger than either the inputs
or the {\sc gcd}.  Thus arose the interpolation ideas.

A naive approach to this problem is to choose random values for
$\vec{x}_i \in R^v$ and compute 
\[
\gcd(F(\vec{x}_i), G(\vec{x}_i)) = h_i \in R.
\]
One could then try to interpolate the $h_i$ to get $H$.
Unfortunately, as stated this algorithm does not work well, since it
assumes that $h_i = H(\vec{x}_i)$, which is not always the case.  A
trivial example is $F = X+1$ and $G = X-1$.  For odd integer values of
$x$, $2$ divides $\gcd(x+1, x-1)$ even though the {\sc gcd} of $F$ and $G$
is $1$.  Nonetheless, a variant of this technique is effective for
many problems over $\Z$ that involve a small number of variables.
This variant is described in \sectref{PGCD:Heuristic:Sec}.

To use the modular interpolation approach successfully, one
does not perform the {\sc gcd}'s in $R$ but rather in $R[X_v]$.  That is,
values are chosen for $X_1, \ldots, X_{v-1}$ and we compute the {\sc gcd} of
$F(\vec{x}_i, X_v)$ and $G(\vec{x}_i, X_v)$ where $\vec{x}_i \in
R^{v-1}$.  This {\sc gcd} is a univariate polynomial.  So rather than have a
single interpolation problem for {\sc gcd}, we have several problems, one
for each of the coefficients of the univariate polynomials.  As we
shall see, the likelihood that
\[
\gcd(F(\vec{x}_i, X_v), G(\vec{x}_i, X_v)) = H(\vec{x}_i, X_v)
\]
is quite high.

With both of these algorithms, the basic approach is {\em
probabilistic\/}.  One uses a technique that produces a polynomial
that is likely to be the {\sc gcd} of the $F$ and $G$, but is not
guaranteed to be the {\sc gcd}.  The candidate can then be verified by
\key{trial division}.  If the candidate is not the {\sc gcd} then a
randomization of some sort is applied and a new candidate is produced.

For the algorithms discussed here, if the candidate divides both $F$
and $G$ then it is the {\sc gcd} of $F$ and $G$.  In practice, this
\keyi{test division} is quite cheap.  An essential part of proving the
validity of the algorithm is to show that the candidates produced
cannot be proper divisors of the {\sc gcd}.

\section{Heuristic GCD}
\label{PGCD:Heuristic:Sec}

The ``Heuristic {\sc gcd}'' algorithm (called ``\key{GCDHeu}'' for short) is
one of the simplest and easiest to implement of the modern algorithms.
Its speed derives from the fact that most computer algebra systems
have carefully coded and efficient large precision integer arithmetic
packages.  Unfortunately, it is difficult to prove that it is
efficient for polynomials involving a large number of variables and
its practical efficiency decreases correspondingly.

\index{radix interpolation|(}
We begin with a new type of interpolation scheme, called \keyi{radix
interpolation}.  Let 
\[
F(X) = f_0 X^m + f_1 X^{m-1} + \cdots + f_m
\]
be a polynomial and assume $f_i \in \Z$.  Recall from
\chapref{PBounds:Chap} that
\[
|F| = \| \langle f_0, \ldots, f_m \rangle \|_{\infty} 
  = \max \{ |f_0|, \ldots, |f_m| \}.
\]

With the Lagrange or Newton interpolation schemes the values of $F$ at
$m+1$ points are needed to compute the coefficients of $F$.  However,
if we know the value of $F$ at a sufficiently large point, $F$ can be 
reconstructed using only a single value.  Let $\xi \ge 2 |F| + 1$ be a
large integer.  Then
\[
\begin {aligned}
  F(\xi) & = f_0 \xi^m + f_1 \xi^{m-1} + \cdots + f_m, \\
      & = \xi (f_0 \xi^{m-1} + \cdots + f_{m-1}) + f_m.
\end{aligned}
\]
The quantity in parentheses is an integer, so the remainder when
$F(\xi)$ is divided by $\xi$ is either $f_m$ or $f_m + \xi$ if $f_m$
is negative.  Obviously, we should modify the remainder calculation so
that it always returns $f_m$.

We call this type of remainder calculation, a \keyi{balanced
remainder}.  (This is similar to the {\em balanced presentation} used
for elements in a finite field as discussed in
\chapref{Finite:Fields:Chap}.)  Define \keyw{BalRem} for rational
integers as 
\begindsacode
Bal\=Rem ($p$, $q$) := $\{$ \\
\> $r \leftarrow \mbox{remainder}(p, q)$; \\
\> if $r < q/2$ then $r$ else $r - q$ \\
\> $\}$ 
\enddsacode

The constant term of $F(X)$ is $f_m = \mbox{\tt BalRem}(F(\xi),
\xi)$.  The linear term of $F(X)$, $f_{m-1}$ can be computed as
\[
\mbox{\tt BalRem}\left( \frac{ F(\xi) - f_m}{\xi}, \xi\right).
\]
This process can be repeated for all the other coefficients of $F(X)$.
Thus all the coefficients of $F(X)$ can be computed from $F(\xi)$ if
$\xi \ge 2 |F| + 1$.  Furthermore, the number of arithmetic operations
required is linear in the degree of $F$.  On average, $F(\xi)$ requires
approximately $m \log |F|$ bits, so the bit complexity of this
interpolation process is quadratic in the size of $F(X)$.

The following routine implements this approach. 
\index{RadixInterp@\protect\texttt{RadixInterp}}

\begindsacode
Rad\=ixInterp ($h$, $\xi$) := $\{$ \\
\> $i \leftarrow 0$; \\
\> $H \leftarrow 0$; \\
\> loo\=p while $h \not= 0$ do \{ \\
\>\> $h_i \leftarrow \mbox{BalRem}(h, \xi)$; \\
\>\> $H \leftarrow X^i h_i + H$; \\
\>\> $h \leftarrow (H - h_i)/\xi$; \\
\>\> $\}$ \\
\> return($H$); \\
\> $\}$
\enddsacode
\index{radix interpolation|)}

\medskip
Let $F(X)$ and $G(X)$ be univariate polynomials over $\Z$, whose {\sc gcd}
is $H(X)$,
\[
\begin{aligned}
F(X) &= f_0 X^m + f_1 X^{m-1} + \cdots + f_m, \\
G(X) &= g_0 X^n + g_1 X^{n-1} + \cdots + g_n.
\end{aligned}
\]
GCDHeu generates {\sc gcd} candidates by computing the integer {\sc gcd} of
$F(\xi)$ and $G(\xi)$, which we denote by $h$, and then reconstructing
$H(X)$ from $h$ using the procedure discussed above.  We denote by
$H_{\xi}(X)$ the candidate polynomial produced by this process, \ie,
\[
H_{\xi}(X) = \mbox{\tt RadixInterp}(\gcd(F(\xi), G(\xi)), \xi).
\]

There are two reasons why $H_{\xi}(X)$ might not be $H(X)$.  First,
the {\sc gcd} of $F(\xi)$ and $G(\xi)$ might not be $H(\xi)$.  If this is
the case then \keyw{RadixInterp} cannot possibly reconstruct $H(X)$.
Second, even if $\gcd(F(\xi), G(\xi)) = H(\xi)$, $\xi$ must be larger
than the coefficients of $H(X)$.  That is, $\xi > 2|H|+1$.  Since $H$
divides both $F$ and $G$, we can use \longpropref{Factor:CBound:Prop}
to bound the height of $H$:
\[
|H| \le (d+1)^{1/2} 2^d \max\{ |F|, |G| \},
\]
where $d = \max \{m, n\}$.  Thus $\xi$ should be chosen to be an
integer such that
\[
\xi > (d+1)^{1/2} 2^{d+1} \max\{ |F|, |G| \}.
\]

The following proposition, gives a necessary and sufficient condition
for $H_{\xi}(X)$ to actually be the {\sc gcd} of $F(X)$ and $G(X)$.

\begin{proposition}
Let $F(X)$ and $G(X)$ be univariate polynomials over the integers, and
$H_{\xi}(X)$ defined as above.  If $H_{\xi}(X)$ divides both $F(X)$
and $G(X)$ then $H_{\xi}(X)$ is the greatest common divisor of $F(X)$
and $G(X)$.
\end{proposition}

\begin{proof}
Let $H(X)$ denote a {\sc gcd} of $F(X)$ and $G(X)$ and let $A(X)$ and
$B(X)$ denote their {\em cofactors},\index{cofactor} \ie,
\begin{equation} \label{PGCD:HeuGCD:Eqa}
F(X) = H(X) A(X) \quad \mbox{and} \quad G(X) = H(X) B(X).
\end{equation}
By assumption $H_{\xi}(X)$ divides both $F(X)$ and $G(X)$, it must
divide $H(X)$,
\begin{equation} \label{PGCD:HeuGCD:Eqb}
H(X) = H_{\xi}(X) L(X).
\end{equation}
If $H_{\xi}(X)$ is not a {\sc gcd} of $F(X)$ and $G(X)$, then $L(X)$ will
not be $\pm 1$.

Letting $X= \xi$ in \eqnref{PGCD:HeuGCD:Eqb} gives
\[
H(\xi) = H_{\xi}(\xi) L(\xi)
\]
Letting $X = \xi$ in \eqnref{PGCD:HeuGCD:Eqa} shows that $H(\xi)$ must
divide $H_{\xi}(\xi) = \gcd(F(\xi), G(\xi))$, so
\[
L(\xi) \times \frac{H_{\xi}(\xi)}{H(\xi)} = 1
\]
where each factor is an integer.  Thus $L(\xi) = \pm 1$.  Since $L(X)$
divides $H(X)$, it must also divide $F$ and $G$ and thus we have the
same height bound for $L(X)$ as for $H(X)$.   Using \keyw{RadixInterp}
we see that $L(X) = \pm 1$.
\end{proof}

\section{Univariate Polynomials over $\Z$}
\label{PGCD:Uni:Sec}

For simplicity we first consider the problem of computing the {\sc gcd} 
of two polynomials 
\[
\begin{aligned}
F(X) & = f_0 X^r + f_1 X^{r-1} + \cdots + f_r, \\
G(X) & = g_0 X^s + g_1 X^{s-1} + \cdots + g_s, 
\end{aligned}
\]
over the rational integers $\Z$,
where we denote the {\sc gcd} by 
\[
\gcd(F, G) = H(X) = h_0 X^t + h_1 X^{t-1} + \cdots + h_t.
\]  
The approach is to pick a sequence of prime numbers $p_i$ and compute the 
{\sc gcd} of the images of $F(X)$ and $G(X)$ over $\Z/(p_i)$.  These 
polynomials are then interpolated using the \key{Chinese remainder theorem} to 
get a polynomial over the integers, which we expect to be $H(X)$. 

Two issues immediatedly arise with this approach.  First, we need a
bound on the size of the coefficients of $H(X)$ in terms of the size
of the coefficients of $F(X)$ and $G(X)$.  This bound indicates how many 
images modulo $p_i$ are needed.\, and is provided by 
\longpropref{Factor:CBound:Prop}.  Since $H(X)$ 
divides both $F(X)$ and $G(X)$, we have
\[
|H| \le \min( 2^r\sqrt{r+1} |F|, 2^s\sqrt{s+1} |G|).
\]

Second, and more importantly, $\Z$ is not a field while $\Z/(p_i)$ is.
Using a polynomial remainder sequence over a field always produces a
monic polynomial, while the {\sc gcd} of $F$ and $G$ might not be
monic.  Assuming the degree of this {\sc gcd}, $H_{p_i}(X)$ is $t$, it
is always the image of the monic version of $H(X)$, \ie,
\[
H_{p_i}(X) = X^t + \frac{h_1}{h_0} X^{t-1} + \cdots + \frac{h_t}{h_0}
    \pmod{p_i}.
\]
Now, assume we have computed the {\sc gcd} of $F(X)$ and $G(X)$ modulo
several primes and interpolated the results to produce a (necessarily) monic
polynomial
\[
H_m(X) = X^t + \frac{h_1}{h_0} X^{t-1} + \cdots + \frac{h_t}{h_0} \pmod{m},
\]
where $m > (2 |H| + 1) \gcd(f_0, g_0)$.  If we knew $h_0$, we could
recover $H(X)$ by simply multiplying $H_m(X)$ by $h_0$ modulo $m$,
converting to a balanced representation and interpreting the
coefficients as rational integers.

Assume for the moment that $F(X)$ and $G(X)$ are primitive, so, $H(X)$
is also primitive.  Although we do not know $h_0$, we do know that
$f_0$, $g_0$ and $\gcd(f_0, g_0)$ are all multiples of $h_0$.
Multiplying $H_m(X)$ by $\gcd(f_0, g_0) = a h_0$, and interpreting the
coefficients as rational integers gives $aH(X)$.  We can now determine
$H(X)$ by computing the principal part of $aH(X)$.

This process is called {\em imposing a leading
coefficient}\index{leading coefficient!imposing a} on a factor.  It is
a very useful technique that is often used for other problems.

The remaining detail is to establish a bound on the height of $aH(X)$.
Since $a$ divides $f_0$ and $g_0$, we have 
\[
|a H| \le |\gcd(f_0, g_0)| \cdot |H| 
  \le |\gcd(f_0, g_0)| \cdot \min( 2^r\sqrt{r+1} |F|, 2^s\sqrt{s+1}
|G|).
\]
Using this bound and the technique given above we get the following
algorithm for computing the {\sc gcd} of two univariate polynomials over the
rational integers.\Marginpar{Need to add a test divide at the end of
the following routine.}

\begindsacode
UnivariateGCD ($F$, $G$) : = \{ \\
\> $c \leftarrow \gcd(\cont F, \cont G)$; \\
\> $F \leftarrow \prim F$; $G \leftarrow \prim G$; \\
\> $h \leftarrow \gcd(\lc F, \lc G)$; \\
\> $\ell \leftarrow \lcm(\lc F, \lc G)$; \\
\> $r \leftarrow \deg F$; $s \leftarrow \deg G$; \\
\> $B \leftarrow 2 |h| \cdot \min( 2^r\sqrt{r+1} |F|, 2^s\sqrt{s+1}
|G|) + 1$; \\
\> $H \leftarrow 0$; \\
\> $m \leftarrow 1$; \\
\> loo\=p $\mbox{\rm choose a random prime $p$ such that $\ell \not=0 \mod{p}$}$ do \{ \\
\>\> $\hat{H} \leftarrow \gcd(F(X), G(X)) \mod{p}$; \\
\>\> if $\deg \hat{H} > \deg H$ $\mbox{\rm throw out $\hat{H}$ and $p$
and try again}$; \\
\>\> if \=$\deg \hat{H} < \deg H$ then \{ \\
\>\>\> $H \leftarrow \hat{H}$; \\
\>\>\> $m \leftarrow p$; \\
\>\>\> \} \\
\>\> if $\deg \hat{H} = \deg H$ then \{ \\
\>\>\> $H \leftarrow \mbox{\rm Apply the Chinese remainder algorithm to $H$
and $\hat{H}$}$; \\
\>\>\> $m \leftarrow m \cdot p$; \\
\>\>\> if $m > B$ then exit; \\
\>\>\> \} \\
\>\>\} \\
\> $H \leftarrow \mbox{\rm Balance $h \cdot H$ and convert to a polynomial
over $\Z$}$; \\
\> return($c \cdot \prim(H)$); \\
\> \}
\enddsacode

The first few steps of {\tt UnivariateGCD} compute the {\sc gcd} of the
content of $F$ and $G$, and replace $F$ and $G$ by  their primitive
parts.  Note that we want $F$ and $G$ to be primitive even if $c= 1$.
Otherwise the leading coefficient fix up technique does not work.  We
then compute $h$, a multiple of the leading coefficient of $\gcd(F,
G)$, and $B$, a bound on the size of the coefficients in $\gcd(F, G)$.

The core of the algorithm is the loop.  Throughout the loop $H =
\gcd(F, G) \mod{m}$.  Random primes $p$ are chosen for which the
degrees of $F$ and $G$ do not change.  $\hat{H}$ is then set to
$\gcd(F, G) \mod{p}$.  If $\deg \hat{H} \not= \deg H$ then the one
with the larger degree includes extraneous divisors and is discarded.

\paragraph{Analysis}

The crucial issue in analyzing \keyw{UnivariateGCD} is determining how
many {\sc gcd}'s modulo $p$ are computed.  Two issues arise here.
First, some of the {\sc gcd}'s will be useless because they produce
polynomials whose degree is too large.  Second, how large should the
primes $p$ be chosen so as to minimize the total amount of computation
in the {\sc gcd}'s modulo $p$.

We consider the second question first, since it is the easiest.
Assume that all of the {\sc gcd}'s yield polynomials of degree $t$ and
that the primes chosen, $p$, are all around $N$.  Then the number of
{\sc gcd}'s needed will be 
\[
k = \left\lceil \frac{\log B}{\log N} \right\rceil.
\]
Each {\sc gcd} will cost $C_{\gcd(F,G)} \cdot C_{M}(\log N)$ bit
operations, where $C_{\gcd(F,G)}$ is the number of arithmetic
operations required to compute the {\sc gcd} of $F$ and $G$, and
$C_{M}(M)$ is the number of bit operations required to multiply two
integers of $M$ bits.  

Increasing the size of the primes decreases the number of {\sc gcd}'s,
but increases the number of bit operations required to compute them.
Since $C_{M}$ grows faster than linearly, increasing the size of $N$
does not improve the total cost.  Decreasing $N$ below the word size
of the computer (the intrinsic parallelism of the machine) also does
not improve the total cost.

\smallskip
Now consider what must happen for $H_p(X)$ to have degree greater than
$t$.  Let $F(X) = \bar{F}(X) \cdot H(X)$ and $G(X) = \bar{G}(X) \cdot
H(X)$, where the cofactors $\bar{F}$ and $\bar{G}$ are relatively prime
over $\Z$.  Denote the degree of $\bar{F}$ by $\bar{r}$ and the degree
of $\bar{G}$ by $\bar{s}$.  If the prime $p$ leads to a false {\sc gcd},
then $\bar{F}$ and $\bar{G}$ are not relatively prime modulo $p$.
Since $\bar{F}$ and $\bar{G}$ have a common factor if and only if
their resultant vanishes, $p$ must divide $M = \res (\bar{F},
\bar{G})$.  \longpropref{Resultant:Bound:Prop} gives a bound on the
size of $M$: 
\[
|M| \le (\bar{r} + \bar{s})!\, |\bar{F}|^{\bar{r}} |\bar{G}|^{\bar{s}}.
\]
Using \longpropref{Factor:CBound:Prop}, we have
\[
\begin{aligned}
|\bar{F}| & \le 2^r \sqrt{r+1} |F|, \\
|\bar{G}| & \le 2^s \sqrt{s+1} |G|.
\end{aligned}
\]
Bounding $\bar{r} \le r$ and $\bar{s} \le s$ we have:
\[
|M| \le (r+s)! 2^{r^2+s^2} (r+1)^{r/2} (s+1)^{s/2} |F| \, |G|.
\]
Thus there are only a finite number of primes $p$ such that $H_p(X)$
does not have degree $t$.

\paragraph{Refinements}

Let the cofactors\index{cofactor} of $F$ and $G$ be $A$ and $B$ respectively, and
denote their images modulo $p$ by $\hat{A}$ and $\hat{B}$.  Given $F$
and $G$, and any one of $A$, $B$ or $H$ the other two can be computed
by two exact polynomial divisions.  So it does not really matter which
of $A$, $B$ or $H$ is computed in the {\sc gcd} algorithm.

The cost of a {\sc gcd} algorithm is dominated by the interpolations
used to combine the different $\hat{H}$ to produce $H$, and this cost
is a function of the size of $H$.  If one cofactor, say $A$, is much
smaller than the {\sc gcd} then it is more efficient to compute
$\hat{A}$ from $\hat{H}$ and use the Chinese remainder
algorithm\index{Chinese remainder theorem} to produce
$A$.\Marginpar{Expand on this}

The benefit of this approach for univariate polynomials is small, but
it is well worth the effort for multivariate {\sc gcd}'s.

\section{Multivariate Polynomials}
\label{PGCD:Multi:Sec}

The multivariate {\sc gcd} algorithm can be viewed as many copies of
the sparse interpolation scheme of \sectref{Interp:PSparse:Sec}.
Assume we wish to compute $H$, the {\sc gcd} of $F$ and $G$, where
\[
\begin{aligned}
F(X_1, \ldots, X_v) & = f_0(X_1, \ldots, X_{v-1}) X_v^m + \cdots + f_m(X_1,
\ldots, X_{v-1}), \\
G(X_1, \ldots, X_v) & = g_0(X_1, \ldots, X_{v-1}) X_v^n + \cdots + g_n(X_1,
\ldots, X_{v-1}), \\
H(X_1, \ldots, X_v) & = h_0(X_1, \ldots, X_{v-1}) X_v^r + \cdots + h_r(X_1,
\ldots, X_{v-1}).
\end{aligned}
\]
In essence we construct black boxes that represent the $h_i$:
\[
{\cal B}_{h_i}(y_1, \ldots, y_{v-1}) 
  = \coef(\gcd(F(y_1, \ldots, y_{v-1}, Z), G(y_1, \ldots, y_{v-1}, Z)), Z^i),
\]
and then use sparse interpolation methods to produce the $h_i(X_1,
\ldots, X_v)$ and thus $H$.  The actual algorithm, however, is
slightly complicated by (1) the desire to minimize the number of
univariate {\sc gcd} calculations, (2) the need to impose the leading
coefficient and (3) other performance considerations.

\begin{figure}
\begindsacode
SparseGCD1 ($\ell(X_1, \ldots, X_k)$, $F(X_1, \ldots, X_k; Z)$, $G(X_1, \ldots, X_k; Z)$) := \{\\
\>if $k= 0$ then return($\ell \times \gcd(F, G)$); \\
\>els\=e \{ \\
\>\> $d \leftarrow \min(\deg_{X_k} F, \,\,\deg_{X_k} G)$; \\
\>\> $x_k \leftarrow \mbox{random}(K)$; \\
\>\> $I \leftarrow \mbox{SparseGCD1}(F(X_1, \ldots, X_{k-1}, x_k; Z),
G(X_1, \ldots, X_{k-1}, x_k; Z))$; \\
\>\> $T_k \leftarrow \mbox{\rm Maximum number of terms in the coefficients of $I$}$; \\
\>\> for \=$1 \le i \le d$ do \{ \\
\>\>\> $y_i \leftarrow \mbox{random}(K)$; \\
\>\>\> $W \leftarrow \mbox{SparseGCD2}($\=$\skel I, \ell(X_1, \ldots, X_{k-1}, y_i),$ \\
\>\>\>\> $F(X_1, \ldots, X_{k-1}, y_i; Z), G(X_1, \ldots, X_{k-1}, y_i; Z), T_k)$;\\
\>\>\> for \=$0 \le j \le D$ do \{ \\
\>\>\>\> ${\cal H}_j \leftarrow {\cal H}_j \cup \{(y_i, \coef(W, Z^j))\}$; \\
\>\>\>\> \} \\
\>\>\> \} \\
\>\> for \=$0 \le j \le D$ do \{ \\
\>\>\> $\mbox{\rm Dense interpolate the coefficients of the monomials in
$X_1, \ldots, X_k$}$\\
\>\>\> $\mbox{\rm of the elements of ${\cal H}_j$ to get $h_j$.}$\\
\>\>\> \} \\
\>\> $H \leftarrow h_d Z^d + h_{d-1} Z^{d-1} + \cdots + h_0$; \\
\>\> $\mbox{\rm If $H$ does not divide both $F$ and $G$ start again
with a new evaluation point.}$\\
\>\> return($H$); \\
\>\> \} \\
\> \}
\enddsacode
\hrule
\begindsacode
SparseGCD2 ($S$, $\ell(X_1, \ldots, X_k)$, $F(X_1, \ldots, X_{k}; Z)$, $G(X_1, \ldots, X_{k}; Z)$, $T_k$) := \{\\
\>if $k=0$ then return($\ell \times \gcd(F, G)$); \\
\>els\=e \{ \\
\>\> $\mbox{\rm set $y_1, \ldots, y_k$ to random non-zero elements of $K$}$ \\
\>\> for \=$1 \le i \le T$ do \{ \\
\>\>\> $W \leftarrow \ell(y_1^i, \ldots, y_k^i) \times \gcd(F(y_1^i, \ldots, y_k^i; Z), G(y_1^i, \ldots, x_k^i; Z))$;\\
\>\>\> for \=$0 \le j \le D$ do \{ \\
\>\>\>\> ${\cal H}_j \leftarrow {\cal H}_j \cup \{(i, \coef(W, Z^j)) \}$;
\\
\>\>\>\> \} \\
\>\>\> \} \\
\>\>\> for \=$0 \le j \le D$ do \{ \\
\>\>\> $\mbox{\rm Use {\tt SolveVandermondeTD} to interpolate the elements of
${\cal H}_j$, in}$\\
\>\>\> $\mbox{\rm accordance with $\coef(S, Z^j)$, to get $h_j$}$\\
\>\>\> \} \\
\>\> return($h_d Z^d + h_{d-1} Z^{d-1} + \cdots + h_0$); \\
\>\> \} \\
\> \}
\enddsacode

\caption{Sparse Modular {\sc gcd} Algorithm \label{SparseGCD:Fig}}
\end{figure}

We single out one variable that will remain throughout the different
evaluations.  For simplicity, assume that the variable is $X_v$ and
throughout replace $X_v$ by $Z$.  We assume that $D$ bounds the degree
of each of $X_1, \ldots, X_{v-1}$ in $F$ and $G$.  The core of the
sparse {\sc gcd} algorithm is contained in the two routines in
\figref{SparseGCD:Fig}.

It is convenient to write
\[
\begin{aligned}
F^{(i)} &= F^{(i)}(X_1, \ldots, X_i; Z) = 
F(X_1, \ldots, X_i, x_{i+1}, \ldots, x_{v-1}, Z), \\
& = f^{(i)}_0(X_1, \ldots, X_i) Z^m + f^{(i)}_1(X_1, \ldots, X_i) Z^{m-1} 
 + \cdots f^{(i)}_m(X_1, \ldots, X_i),
\end{aligned}
\]
and similarly for $G^{(i)}$ and $H^{(i)}$.  The top level routine of
the algorithm, \keyw{SparseGCD1}, is initially invoked on $F^{(v-1)}$
and $G^{(v-1)}$ and recursively calls itself on $F^{(v-2)}$ and
$G^{(v-2)}$ and so on.  The {\sc gcd} is constructed from the bottom
up, producing $H^{(0)}$, then $H^{(1)}$ and so on.  We call the
construction of $H^{(k)}$ from $H^{(k-1)}$, where the variable $X_k$
is introduced into the {\sc gcd}, the $k$-th stage of the
computation.

Initially, we assume that $F$ and $G$ are monic.  In this case
the polynomials $\ell$, which represent leading coefficients, are
equal to $1$ and can be ignored.

In the zeroth stage, when $F$ and $G$ are univariate polynomials
in $Z$, a standard polynomial remainder sequence {\sc gcd} algorithm
can be used to compute
\[
H^{(0)} = \gcd(F^{(0)}, G^{(0)}) = 
\gcd(F(x_1, \ldots, x_{v-1}, Z), G(x_1, \ldots, x_{v-1}, Z)).
\]
All of the references to {\sc gcd} calculations in
\figref{SparseGCD:Fig} are of univariate {\sc gcd}'s that can also use
polynomial remainder sequence algorithms.

Assume we have computed $H^{(k)}$, the {\sc gcd} of $F^{(k)}$ and $G^{(k)}$.
Denote by $T_k$ the maximum number of non-zero monomials in the
coefficients of powers of $Z$ in $H^{(k)}$.  To compute $H^{(k+1)}$ we
need $D+1$ images of $H$ in $X_1, \ldots, X_k$:
\[
\begin{aligned}
H^{(k+1)}_0 & = H^{(k)}(X_1, \ldots, X_{k}, x_{k+1}; Z) = H^{(k)}, \\
H^{(k+1)}_i & = H^{(k)}(X_1, \ldots, X_{k}, x^{(i)}_{k+1}; Z),
\quad\mbox{for $i = 1, \ldots, D$,}
\end{aligned}
\]
where $x^{(i)}_{k+1}$ are chosen randomly.  Using the skeleton of
$H^{(k+1)}_0$, the $H^{(k+1)}_i$ are computed using sparse
interpolation techniques by {\tt SparseGCD2}.

Using a univariate interpolation algorithm we can interpolate the
coefficients of the monomials in $H^{(k+1)}_i$ to yield $H^{(k+1)}$.
We then need to verify that $H^{(k+1)}$ is the image of $H$ by
checking that $H^{(k+1)}$ exactly divides both $F^{(k+1)}$ and
$G^{(k+1)}$.  If this test division fails, then the entire {\sc gcd}
computation should be restarted, not just this stage.

The technique used by \keyw{SparseGCD2} to compute the $H^{(k+1)}_i$
is the same as that used for sparse interpolation of black boxes.  We
compute $T_k$ univariate {\sc gcd}'s
\[
H^{(k+1)}_{i,j}(Z) = 
\gcd(F^{(k+1)}(x_1^j, \ldots, x_k^j, x_{k+1}^{(i)}; Z), 
G^{(k+1)}(x_1^j, \ldots, x_k^j, x_{k+1}^{(i)}; Z)),
\]
where $1 \le j \le T_k$.  Each $H^{(k+1)}_i$ is assumed to have the
same skeleton as $H^{(k+1)}_0$.  To determine the coefficients of the
polynomial, we solve the {\em modified} transposed Vandermonde system
that arises from the coefficients of $Z^{\ell}$ in $H^{(k+1)}_{i,j}$.
These polynomials are then combined to produce $H^{(k+1)}_i$.  This
procedure is then used iteratively to produce $H^{(v)}$, which is the
{\sc gcd} of $F$ and $G$.

\medskip
Two additional issues need to be addressed to produce a correct
algorithm: the leading coefficient may not be $1$ and the test
division at the end of each stage may not succeed.  First, as with the
univariate case, the core of the algorithm needs to work with
primitive polynomials.  As a consequence, the first step of the
algorithm computes the content and primitive part of $F$ and $G$.
The {\sc gcd} of the content of $F$ and the content of $G$ is the
content of the {\sc gcd} of $F$ and $G$ by Gauss's lemma
(\propref{Gauss:Lemma:Prop}). 

Second, if one of the test divisions performed at the end of each
stage fails then the {\sc gcd} being generated is too large.  This is
caused by a bad ``starting point,'' the $x_1, \ldots, x_{v-1}$.  In
this case, a new random starting point must be chosen and the algorithm
restarted.  

\paragraph{Leading Coefficient}

The most complex step is the introduction of the leading coefficient.
The approach used is an adaptation of that used in the univariate
case, which is most easily explained in the passage from $H^{(0)}$ to
$H^{(1)}$.  Assume that the true value of $H^{(1)}$ is
\[
H^{(1)}(X_1; Z) = h^{(1)}_0(X_1) Z^r + h^{(1)}_1(X_1) Z^{r-1} + 
  \cdots + h^{(1)}_r(X_1).
\]
As discussed, univariate {\sc gcd}'s are always monic and thus the
coefficient of $Z^u$ will be  \Marginpar{I don't like the ``always monic''}
\[
\frac{h^{(1)}_{u}(x_1)}{h^{(1)}_0(x_1)}.
\]

To get the leading coefficient correct, we compute $2D$
additional polynomials $H^{(1)}_i$, instead of $D$ polynomials as
described earlier.  Using univariate interpolation on the coefficients
of $Z$, we get a polynomial
\[
\hat{H}^{(1)}(X_1; Z) = Z^r + \hat{h}^{(1)}_1(X_1)Z^{r-1} + \cdots +
\hat{h}^{(1)}_r(X_1),
\]
where the $\hat{h}_i(X_1)$ are degree $2D$ polynomials such that 
\[
\hat{h}^{(1)}_i(X_1) = \frac{h^{(1)}_i(X_1)}{h^{(1)}_0(X_1)} 
  \pmod{(X_1 - x_1)(X_1 - x^{(1)}_1) \cdots (X_1 - x^{(2D)}_1)}
\]
Let $\ell^{(1)} = \gcd(\lc F^{(1)}, \lc G^{(1)})$.  Then $\ell^{(1)}$ is a
multiple of $h_0(X)$ and is a polynomial of degree less than or equal
to $D$.  Multiplying $\hat{H}^{(1)}(X_1; Z)$ by $\ell^{(1)}$ we have
\[
\ell^{(1)} \hat{h}_i(X_1) = \frac{h^{(1)}}{h_0(X_1)} h_i(X_1),
\]
which is a polynomial of degree $\le 2D$.  Therefore the coefficients
of $Z^{\ell}$ of $h^{(1)} \hat{H}^{(1)}(X_1; Z)$ are polynomials and
the primitive part of $h^{(1)} \hat{H}^{(1)}(X_1; Z)$ is $H^{(1)}(X_1;
Z)$.

More generally, let
\[
\ell(X_1, \ldots, X_{v-1}) = \gcd(\lc F, \lc G) = \gcd (f_0, g_0).
\]
At the $k$-th stage of the interpolation, we force the leading
coefficient of $H^{(k)}$ to be $\ell^{(k)}$.  This guarantees that the
coefficients of powers of $Z$ in $H^{(k)}$ will be polynomials.  As
shown in \figref{SparseGCD:Fig}, this is effected by passing the
images of $\ell$ along with those of $F$ and $G$.  At the point where
univariate {\sc gcd}'s are computed, the monic {\sc gcd} is multiplied
by the image of $\ell$.

One defect needs to be remedied.  The test division at the end {\tt
SparseGCD1} will fail if the leading coefficient of the {\sc gcd} of
$F$ and $G$ is not $\ell$.  We can avoid this problem by replacing $F$
and $G$ by $\ell F$ and $\ell G$. 

\begindsacode
SparseGCD ($F$, $G$) := \{ \\
\> $\ell \leftarrow \mbox{SparseGCD}(\lc F, \lc G)$; \\
\> return($\prim \mbox{SparseGCD1}(\ell, \ell F, \ell G)$); \\
\> \}
\enddsacode

\paragraph{Analysis}

There are two crucial issues that need to be addressed in the analysis of
the sparse {\sc gcd} algorithm: (1) What is the likelihood that one of
the test divisions will fail? and (2) What is the maximum size of the
expressions that occur in the computation.  

The first question is the most interesting.  The test divisions will
fail if and only if the result of introducing a variable has produced
a polynomial of larger degree in some variable than the true {\sc
gcd}.  Define $R^{(k)}$ to be
\[
R^{(k)}(X_1, \ldots, X_k) = \res_Z (F^{(k)}, G^{(k)}).
\]
The computed {\sc gcd} can only be too large if one of 
\[
R^{(1)}(x_1), R^{(2)}(x_1, x_2), \ldots, R^{(v-1)}(x_1, \ldots, x_{v-1})
\]
vanishes, or if one of
\[
R^{(1)}(y_1), R^{(2)}(y_1, y_2), \ldots, R^{(v-1)}(y_1, \ldots, y_{v-1})
\]
vanishes.

\paragraph{Refinements}

A number of refinements can be used to substantially speed up the
multivariate {\sc gcd} algorithm.  First, the cost of the
interpolation process is a function of $T_k$, the maximum number of
non-zero terms in the coefficients of $Z^{\ell} = X_v^{\ell}$.  To
minimize this cost it is wise to choose for $Z$ the variable that has
the fewest non-zero coefficients in $H$.  Often this is the variable
of highest degree in $H$.

Second, instead of lifting the {\sc gcd} it is worthwhile to lift the
smaller of the {\sc gcd} and the cofactors.  One effective way to pick
this variable in accordance with these two considerations is as
follows.

We assume that the problem has already been reduced to primitive $F$
and $G$.  For each variable $X_i$ compute the univariate polynomial
\[
\gcd(F(x_1, \ldots, x_{i-1}, X_i, x_{i+1}, \ldots, x_v), 
     G(x_1, \ldots, x_{i-1}, X_i, x_{i+1}, \ldots, x_v)).
\]
The degree of this polynomial is the degree of $X_i$ is $H$.  The
degree of $X_i$ in each cofactor can then be computed.  The smaller of
this triple indicates which quantity should be interpolated.  The
variable with largest minimum is the one that should be $Z$.

A final issue that is often overlooked, especially when comparing
this algorithm to the GCDHeu algorithm of
\sectref{PGCD:Heuristic:Sec}, is the cost of performing the univariate
{\sc gcd}'s.  An efficient univariate {\sc gcd} algorithm needs to be
used.  Properly implemented, it should cost little more than computing
the {\sc gcd} of comparable sized integers.

\section*{Notes}

\small

\notesectref{PGCD:Heuristic:Sec} The Heuristic {\sc gcd} algorithm was
introduced by {\CharBW}, {\Geddes} and {\Gonnet} in 1984 \cite{Char84}.
Results of experimentation with GCDHeu and a partial analysis was
provided in \cite{Davenport85a}, and complete analysis was published
in \cite{Char89}.

\normalsize

\index{greatest common divisor!of polynomials|)}
