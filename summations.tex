%$Id: /usr/u/rz/AMBook/RCS/diffalg.tex,v 1.1 1992/05/10 19:38:47 rz Exp rz $
\chapter{Summations}
\label{Sum:Chap}

This chapter discusses the problem of obtain an expression for
summations in closed forms.  This is a discrete version of the
integration problem, which is discussed in the following sections.
While the integration problem has lead to a wealth of elegant
mathematics, which has its own intrinsic interest, the summation
problem has led to far less.

\section{Definitions}

Throughout this chapter we will always work of a fields of
characteristic zero.  

Let $f(n)$ be a function $n$ involving algebraic, arithmetic,
combinatorial and other types of operations.  The first
\keyi{difference} of $f(x)$ is defined by 
\[
\Delta f(n) = f(n) - f(n-1).
\]
Repeated application of $\Delta$ gives higher order difference
operators, \eg,
\[
\begin {aligned}
\Delta^2 f(n) &= \Delta f(n) - \Delta f(n-1) = f(n) - 2 f(n-1) +
f(n-2), \\
\Delta^3 f(n) & = f(n) - 3 f(n-1) + 3 f(n-2) - f(n-3).
\end{aligned}
\]

The difference operator is the discrete analogue of the derivative,
\eg,  $\Delta c = D_n c = 0$ if $c$ is a constant.  Notice however,
that the $k${\th} difference of a polynomial of degree $r$ is a
polynomial of degree $r-k$.  Define
\[
n^{(k)} = n (n-1) \times \cdots \times (n - k + 1). 
\]
It is easy to show
\[
\Delta n^{(k)} = k n^{(k-1)}\quad\mbox{and}\quad
\Delta 2^n = 2^{n-1}.
\]

The inverse of the difference operator is the \keyi{summation
operator}, which is defined as follows: $\sum f(n) = g(n)$ if and only
if $f(n) = \Delta g(n) = g(n) - g(n-1)$.  As with integration, the
summation of a function is only defined up to a constant.  Observe
that if $\sum f(n) = g(n)$, then
\[
\sum_{i=1}^{N} f(i) = g(N) - g(0).
\]
The problem of summing expressions like the previous is called the
{\em indefinite sumamtion problem} since it is reducible to the
indefinite sum $\sum f(i)$.

The general summation problem is concerned with determining the value
of 
\[
\sum_{i=1}^{n} a(i, n), 
\]
where $a(i, m)$ is some expression involving algebraic, arithmetic,
combinatorial and perhaps transcendental operations.  We call $a(i,n)$
the \keyi{summand}.  When the summand does not involve the limits and
the summation is finite, 
\[
a(n) = S(n) - S(n-1)
\]
where we are again to determine $S(n)$.  

Examples of such problems include
\[
\sum_{i=0}^n \binom{n}{i} = 2^n \quad\mbox{and}\quad 
\sum_{i=1}^n i^2 = \frac{n(n+1)(2n+1)}{6}.
\]
These problems tend to much more amenable to general solution
technqiues when the summand does not involve either of the two
limits.  



\section{Indefinite Summation}

The most important advance in the indefinite summation problem is the
algorithm of {\Gosper} \cite{Gosper78}.  Let 
\[
\sum a(n) = S(n).
\]
Gosper's algorithm is able to determine $S(n)$ if $S(n)/S(n-1)$ is a
rational function.  

The problem considered is that of
indefinite summation.  That is, given $a(i)$ we want to determine
$S(n)$ such that
\[
\sum_{i=1}^n a(i) = S(n).
\]
This is equivalent to solving the following difference equation
\[
a_n = S(n) - S(n-1) = \Delta S(n).
\]


If $S(n)/S(n-1)$ is a rational function, then Gosper's algorithm will
be able to determine $S(n)$ given just $a(i)$.  

\section{Difference Equations}
