%$Id: groebner.tex,v 1.1 1992/05/10 19:43:07 rz Exp rz $
\chapter{Polynomial Elimination}
\label{Grobner:Chap}

The fundamental problem of algebra is to determine the values of $X_1,
\ldots, X_n$ such that 
\begin{equation} \label{Elim:Eq}
\begin{aligned}
F_1(X_1, \ldots, X_n) & = 0, \\
\vdots \\
F_m(X_1, \ldots, X_n) & = 0,
\end{aligned}
\end{equation}
where the $F_i$ are polynomials in the $X_j$.  If $m = n = 1$ then
we have the familiar problem of finding the zeroes of a single polynomial
in one variable.  The more general, multivariate problem has proven to
be quite difficult.  One way to solve \eqnref{Elim:Eq} is to
use the constraints of one equation to eliminate a variable from
the others.  If this is done properly, we can (often) obtain a system of
equations of the form
\begin{equation} \label{Elim:Eqb}
\begin{aligned}
G_1(X_1) & = 0, \\
G_2(X_1, X_2) & = 0, \\
\vdots \\
G_m(X_1, \ldots, X_n) & = 0.
\end{aligned}
\end{equation}
By substituting each zero of the first equation into the others, and
then solving the second and repeating we can obtain all of the
solutions of \eqnref{Elim:Eq}.

The process of going from \eqnref{Elim:Eq} to \eqnref{Elim:Eqb} is
called \keyi{elimination}.  This chapter discusses one approach to
elimination, via polynomial resultants.\index{resultant}  This is the oldest
approach, but is still effective for many problems.  Furthermore, an
understanding of resultants is useful for understanding the more
sophisticated techniques of multivariate polynomial resultants and
Gr\"{o}bner bases.

In \sectref{Symmetric:Sec} we give an introduction to symmetric
functions and some of their properties.  Polynomial resultants are
defined in \sectref{Poly:Resultant:Sec}.  Two different determinant
representations are given, both of which can be used to compute the
resultant of two polynomials.  Subresultants and their relationship to
the
\key{subresultant {\sc gcd} algorithm} are given in
\sectref{Subresultant:Sec}.  Finally, in \sectref{Result:Examp:Sec} we
give a few applications of how resultants can be used in certain
calculations.

\section{Symmetric Functions}
\label{Symmetric:Sec}

Let $F(X)$ be an irreducible polynomial over the field $K$ and
denote its zeroes by $\alpha_1, \ldots, \alpha_n$:
\[
\begin{aligned}
  F(X) &= X^n - f_1 X^{n-1} + f_2 X^{n-2} + \cdots +(-1)^n  f_n,\\
       &= (X - \alpha_1) (X - \alpha_2) \cdots (X - \alpha_n).
\end{aligned}
\]
Even though the $\alpha_i$ lie in an extension of $K$, the
coefficients of $F$
\[
\begin{aligned}
f_1 &= \alpha_1 + \alpha_2 + \cdots + \alpha_n, \\
f_2 &= \alpha_1 \alpha_2 + \alpha_1 \alpha_3 + \cdots 
     + \alpha_{n-1} \alpha_n, \\
  & \vdots \\
f_n &= \alpha_1 \alpha_2 \cdots \alpha_n,
\end{aligned}
\]
lie in the field $K$.  We can view the $f_i$ as elements of the ring
of polynomials in (the symbols) $\alpha_i$ over $K$.  What
distinguishes these polynomials, from other polynomials in
$K[\vec{\alpha}]$ is that they are unchanged when the $\alpha_i$ are
permuted.  An element of $K[\vec{\alpha}]$ with this property is
called a \keyi{symmetric function}.  To be precise, we often indicate
over which variables the polynomial is symmetric.  For instance, the
following polynomial,
\[
x_1^2 + x_1 x_2 + x_2^2 + t_1 + t_2 + t_3,
\]
is symmetric over $ x_1$ and $x_2$ and it is also symmetric over
$\{t_1, t_2, t_3\}$, but it is not symmetric over all five variables.

The following problem illustrates how symmetric functions can be used
to advantage.  Assume we want to obtain the solutions of the system of
equations
\begin{equation}\label{TripleSymm:Eq}
\begin{aligned}
x   + y   + z   &= 6, \\
x^2 + y^2 + z^2 &= 14,\\
x^4 + y^4 + z^4 &= 98.
\end{aligned}
\end{equation}
Using resultants or Gr\"obner bases\index{Groebner@Gr\"{o}bner basis} to triangulate the system is
possible, but a cleaner solution follows from the observation that the
left hand side of these equations are symmetric in $x$, $y$ and $z$.
Instead of directly solving for $x$, $y$ and $z$, we will first create
a single polynomial equation  whose zeroes are $x$, $y$ and $z$.

Such an equation will have the form
\[
\begin{aligned}
F(T) & = (T - x) (T - y) (T - z), \\
  &= T^3 - (x + y + z) T^2 + (xy + xz +yz) T - xyz,\\
  &= T^3 - \sigma_1 T^2 + \sigma_2 T - \sigma_3.
\end{aligned}
\]
The quadratic coefficient of $F$ is already known, $\sigma_1 = x+y+z =
6$.  The linear coefficient can be computed as follows:
\[
\sigma_1^2 -(x^2 + y^2 +z^2) = 2(xy + yz +xz) = 2 \sigma_2.
\]
So, $\sigma_2 = (6^2 - 14)/2 = 11$.  The constant term is a bit more
involved.  
\[
(x^2 + y^2 + z^2)^2 - (x^4 + y^4 + z^4) = 2(x^2 y^2 + y^2 z^2 + z^2x^2),
\]
or
\[
x^2 y^2 + y^2 z^2 + z^2x^2 = \frac{14^2 - 98}{2} = 49.
\]
Squaring $\sigma_2$ gives
\[
\begin{aligned}
(xy + yz +zx)^2 &= x^2 y^2 + y^2 z^2 + z^2 x^2 + 2(xy^2z + x^2 y z + xyz^2),\\
  & = 49 + 2 \times \sigma_3 \times \sigma_1.
\end{aligned}
\]
So, $\sigma_3 = (11^2 - 49)/(2 \times 6) = 6$.

Thus the zeroes of \eqnref{TripleSymm:Eq} are the zeroes of 
\[
\begin{aligned}
 F(T) & = T^3 - 6 T^2 + 11 T - 6,\\
      & = (T - 1) (T - 2) (T - 3).
\end{aligned}
\]

In solving this problem we have shown that the symmetric functions
$\sigma_1$, $\sigma_2$ and $\sigma_3$ can be written in terms of $x^i
+ y^i + z^i$ for $i = 1, 2, 4$.  In fact, all symmetric polynomials in
$n$ variables can be written as a polynomial in $n$ basis functions.
The two most useful basis sets are the primitive symmetric functions,
$\sigma_i$, mentioned above, and the sum of powers, $s_i$.

Over the variables $x_1, \ldots, x_n$, the {\em primitive symmetric
functions} are\index{symmetric function!primitive}
\[
\begin{aligned}
\sigma_1(x_1, \ldots, x_n) &= x_1 + x_2 + \cdots + x_n, \\
\sigma_2(x_1, \ldots, x_n) &= x_1 x_2 + x_1 x_3 + \cdots + x_{n-1} x_n, \\
  & \vdots \\
\sigma_n(x_1, \ldots, x_n) &= x_1 x_2 \cdots x_n.
\end{aligned}
\]
The {\em sum of powers symmetric functions}\index{symmetric
function!sum of powers} are defined by
\[
s_j(x_1, \ldots, x_n) = x_1^j + x_2^j + \cdots + x_n^j.
\]
When clear from the context, we will write $\sigma_i$ in place
$\sigma_i(x_1, \ldots, x_n)$ and $s_j$ in place of $s_j(x_1, \ldots,
x_n)$.

It is not difficult to see that all symmetric polynomials can be
written in terms of the primitive symmetric functions:  Assume we have
a polynomial $P(x_1, \ldots, x_n)$, symmetric in the $x_i$, written
in expanded representation.  Order the monomials of $P$ lexically in
the exponents.  That is the exponent vector $\vec{e}_1$ comes before
$\vec{e}_2$ if and only if the first non-zero component of $\vec{e}_1
- \vec{e}_2$ is positive.

Let $c_1 x_1^{e_1} x_2^{e_2} \cdots x_n^{e_n}$ be the leading term of
$P$,  then $e_1 \ge e_2 \ge \cdots \ge e_n$.  If this were not the
case then permuting the $x_i$'s would give a term with dominating
exponent vector.  The leading term of the polynomial \Marginpar{Not clear}
\[
\sigma_1^{e_1 - e_2} \sigma_2^{e_2 - e_3} \cdots \sigma_n^{e_n}
\]
is $x_1^{e_1} x_2^{e_2} \cdots x_n^{e_n}$.  Therefore the polynomial 
\[
P(x_1, \ldots, x_n) - c_1 \sigma_1^{e_1 - e_2} \sigma_2^{e_2 - e_3}
\cdots \sigma_n^{e_n}
\]
is a symmetric function of smaller leading term.  Repeating this
process eventually leads to a constant polynomial.  This gives the
following proposition.  

\begin{proposition}
All symmetric polynomials can be written as polynomials in the
$\sigma_i$.
\end{proposition}

\noindent
It is not difficult to show that the presentation of $P$ in terms of
the $\sigma_i$ is unique.  We leave this as an exercise for the
reader. 


\medskip
The relationship between the $s_i$ and $\sigma_j$ is explicitly given
by the \keyi{Newton identities}.
\begin{proposition}[Newton's Identities] \label{Newton:Ident:Prop}
For symmetric functions in $n$ variables:
\begin{align}
s_m - \sigma_1 s_{m-1} + \sigma_2 s_{m-2} + \cdots + (-1)^m m \sigma_m
  = 0, \quad m \le n, \label{Newton:Ident:Eqa}\\
s_m - \sigma_1 s_{m-1} + \sigma_2 s_{m-2} + \cdots + (-1)^n
    \sigma_{n} s_{m-n} 
  = 0, \quad m \ge n. \label{Newton:Ident:Eqb}
\end{align}

\end{proposition}

\begin{proof}
For each $x_i$, we have
\[
x_i^m - \sigma_1 x_i^{m-1} + \cdots + (-1)^n \sigma_n x_i^{m -n} = 0.
\]
Summing this expression for $i = 1, \ldots, n$ we have
\eqnref{Newton:Ident:Eqb}.

The first identity is slightly more complex.  Observe that
\[
\begin{aligned}
\frac{f(X)}{X-x_1} = X^{n-1} + (x_1 - \sigma_1)X^{n-2} 
  &+ (x_1^2 - \sigma_1 x_1 + \sigma_2) X^{n-3} + \cdots  \\
  &+ (x_1^{n-1} - \sigma_1 x_1^{n-2} + \cdots + (-1)^n\sigma_{n-1}).
\end{aligned}
\]
From \longpropref{Ratfun:LogDeriv:Prop}, we know that if $F(X)$ is
square free
\[
f'(X) = \frac{f(X)}{X - \alpha_1} + \cdots + \frac{f(X)}{X - \alpha_n},
\]
so,
\[
\begin{aligned}
f'(X) = n X^{n-1} + (s_1 - n \sigma_1)X^{n-2} 
  &+ (s_2 - \sigma_1 s_1 + n \sigma_2) X^{n-3} + \cdots  \\
  &+ (s_{n-1} - \sigma_1 s_{n-2} + \cdots + (-1)^n n \sigma_{n-1}).
\end{aligned}
\]
Equating the coefficients of $X$ we have
\[
s_m - \sigma_1 s_{m-1} + \cdots + (-1)^m n \sigma_m 
  = (-1)^m (n - m) \sigma_m,
\]
from which \eqnref{Newton:Ident:Eqa} follows immediately.
\end{proof}

It is easy to express $s_m$ in terms of the primitive symmetric functions 
by using the recursions of \propref{Newton:Ident:Prop}.  For $m \le n$, 
we can write the $\sigma_m$ as polynomials in $s_i$ by interpreting 
\eqnref{Newton:Ident:Eqa} as a recursion in $\sigma_m$.  For $m > n$ 
the $\sigma_m$ are undefined.

\section{Polynomial Resultants}
\label{Poly:Resultant:Sec}
\index{resultant|(}
The {\em resultant} of the two polynomials
\[
\begin{aligned}
F(X)&= f_0 X^m + f_1 X^{m-1} + \cdots + f_m\\
G(X)&= g_0 X^n + g_1 X^{n-1} + \cdots + g_n
\end{aligned}
\]
is the smallest polynomial $R$ in the $f_i$ and $g_j$ that vanishes if
and only if $F$ and $G$ have a common zero.  Let the zeroes of $F$ be
$\alpha_1, \ldots, \alpha_m$ and those of $G$ be $\beta_1, \ldots,
\beta_n$:
\[
\begin{aligned}
F(X) &= f_{0}(X - \alpha_{1}) (X - \alpha_{2}) \cdots (X - \alpha_{m}) \\
G(X) &= g_{0}(X - \beta_{1}) (X - \beta_{2}) \cdots (X - \beta_{n})
\end{aligned}
\]
$F$ and $G$ have a common root if and only if one of $F(\beta_1),
\ldots, F(\beta_n)$ vanish, or equivalently if and only if the product
$G(\alpha_1)\cdots G(\alpha_n)$ vanishes.  This product is symmetric in
the roots of $G(X)$, so it can be written as a polynomial in the $f_i$
and $g_j$.  However, its value might change sign when $F$ and $G$ are
interchanged.

Consider the following tableau
\[
\begin{matrix}
1 & g_0 & g_0 & \cdots & g_0 \\
f_0 & (\beta_1 - \alpha_1) & (\beta_1 - \alpha_2) & \cdots & (\beta_1 - \alpha_m)\\
f_0 & (\beta_2 - \alpha_1) & (\beta_2 - \alpha_2) & \cdots & (\beta_2 - \alpha_m)\\
    &     \vdots           &                      &        & \vdots \\
f_0 & (\beta_n - \alpha_1) & (\beta_n - \alpha_2) & \cdots & (\beta_n - \alpha_m)
\end{matrix}
\]
Multiplying the rows together and multiplying the columns together we have
\[
\begin{matrix}
1 & g_0 & g_0 & \cdots & g_0 & \rightarrow & g_0^m\\
f_0 & (\beta_1 - \alpha_1) & (\beta_1 - \alpha_2) & \cdots & (\beta_1 - \alpha_m)
& \rightarrow & F(\beta_1) \\
f_0 & (\beta_2 - \alpha_1) & (\beta_2 - \alpha_2) & \cdots & (\beta_2 - \alpha_m)
& \rightarrow & F(\beta_2) \\
    &     \vdots           &                      &        & \vdots \\
f_0 & (\beta_n - \alpha_1) & (\beta_n - \alpha_2) & \cdots & (\beta_n - \alpha_m)
& \rightarrow & F(\beta_n) \\
\downarrow & \downarrow & \downarrow & \cdots & \downarrow \\
f_0^n & (-1)^n G(\alpha_1) & (-1)^n G(\alpha_2) & \cdots & (-1)^n
G(\alpha_m)
\end{matrix}
\]
The product of all the elements on the bottom row has the same value
as the product of all the elements on the right hand most column.
These products are defined to be the {\em resultant} of $F(X)$ and
$G(X)$.  That is,
\begin{equation} \label{Resultant:Def:Eq}
\begin{aligned}
\res_X(F(X), G(X)) & = g_0^m F(\beta_1) F(\beta_2) \cdots F(\beta_n), \\
& = (-1)^{mn} f_0^n G(\alpha_1) G(\alpha_2) \cdots G(\alpha_m).
\end{aligned}
\end{equation}
We often write this as $\res_{X}(F, G)$ or $\res(F, G)$
if the main variable is apparent.
\addsymbol{$\res_X(F, G)$}{The resultant of $F$ and $G$ with respect
to $X$}

\propref{Resultant:Properties:Prop} summarizes some frequently used and easy to
prove properties of resultants.\Marginpar{Prove the last statement and then tie it to the polynomial remainder sequence.}

\begin{proposition} \label{Resultant:Properties:Prop}
Let $F(X)$ and $G(X)$ be polynomials of degree $m$ and $n$
respectively.  Then
\begin{itemize}
\item $\res_X (F, G) = (-1)^{mn} \, \res_X(G, F)$.
\item If $F$ does not depend on $X$ then $\res_X(F, G) = F^n$.
\item If $F(X) = A(X) B(X)$ then $\res_X(F, G) = 
\res_X(A, G) \res(B, G)$.
\item If $F(X) = Q(X) G(X) + R(X)$ then
$\res_X(F, G) = g_0^{n-\deg R} \res_X(R, G)$.
\end{itemize}
\end{proposition}

The following proposition allows us to explicitly write the resultant of
two binomials.
\begin{proposition} \label{Binomial:Result:Prop}
Let $m = M k$ and $n = N k$ be positive integers where $M$ and $N$ are
relatively prime.  Then 
\[
\res_X(aX^m + b, cX^n + d) = (b^N c^M - (-1)^M a^N d^M)^k.
\]
\end{proposition}

\begin{proof}
Assume that $a$ and $c$ are equal to $1$, so 
\[
\res_X(X^m + b, X^n +d) 
   = \prod_{0 \le i < n} \left((\zeta_n^i (-d)^{1/n})^n + b\right),
\]
where $\zeta_n$ is a primitive $n$-th root of unity.
Since $\zeta_n^m$ is a primitive $N$-th root of unity, we have
\[
\begin{aligned}
\res_X&(X^m + b, X^n +d) \\
 & = \prod_{0 \le i < n} \left(\zeta_N (-d)^{M/N} + b\right)
   = \prod_{0 \le i < N} \left(\zeta_N (-d)^{M/N} + b\right)^k, \\
 & = (b^N - (-1)^M d^M)^k.
\end{aligned}
\]
Using \propref{Resultant:Properties:Prop} we have
\[
\begin{aligned}
\res_X(aX^m + b, cX^n +d) 
  & = a^n c^m \res_X (X^m + \frac{b}{a}, X^n + \frac{d}{c}), \\
  & = a^{dN} c^{d M} 
     \left(\left(\frac{b}{a}\right)^N 
         - (-1)^M \left(\frac{d}{c}\right)^M\right)^k, \\
  & = \left(b^N c^M - (-1)^M a^N d^M\right)^k
\end{aligned}
\]
\end{proof}

\medskip
A useful concept for reasoning about resultants is that of a
\keyi{homogeneous polynomial}.
Let $R$ be a ring and
\[
m = r f_{1}^{e_{1}} \cdots f_{n}^{e_{n}} \in R[f_1, f_2, \ldots, f_n]
\]
a monomial in the $f_{i}$.  The {\em weight} of the
monomial\index{monomial!weight of} $m$ is defined to be $e_{1} +
\cdots + e_{n}$.  A polynomial, $p \in R[\vec f]$ is {\em
homogeneous} if every monomial of $p$ has the same weight.  If the
weight of each term of $p$ is $k$ then $p$ is said to be a {\em
homogeneous polynomial of weight} $k$.\index{homogeneous
polynomial!weight of} The following examples should illustrate this:
\smallskip
\begin{center}
\begin{tabular}{|c|c|c|}
  \multicolumn{1}{c}{$p$} & 
    \multicolumn{1}{c}{homogeneous?} & 
    \multicolumn{1}{c}{weight} \\ \hline
  $f_0 + f_1 + f_2$ & yes & 1 \\ \hline
  $f_0 + f_1 f_2 + f_2$ & no & \\ \hline
  $f_0 f_1 + f_2^2$ & yes & 2 \\ \hline
\end{tabular}
\end{center}
\smallskip
$F(X)$ is homogeneous in the $f_i$, but not in the $f_i$ and $X$.  In
the $f_i$ $F(X)$ has weight $1$.  The sum of two homogeneous
polynomials of the same degree is again homogeneous, while the product
of two homogeneous polynomials of weight $r$ and $s$ is a homogeneous
polynomial of weight $r+s$.  We thus have the following easy
proposition.

\begin{proposition} \label{Resultant:Weight:Prop}
The resultant of two polynomials $F(X)$ and $G(X)$ of degrees $m$ and
$n$ respectively is a homogeneous polynomial of weight $m+n$ in the
coefficients of $F$ and $G$.
\end{proposition}

\begin{proof}
As a polynomial in the $f_{i}$ the $F(\beta_{j})$ are homogeneous of
weight $1$.  So the product $F(\beta_{1}) \cdots F(\beta_{n})$ is a
homogeneous polynomial of weight $n$.  Furthermore,
\[
g_{0}^{m} F(\beta_{1}) \cdots F(\beta_{n})
\]
is also homogeneous of weight $n$ in the $f_{i}$.  Similarly, the
product 
\[
f_{0}^{n} G(\alpha_{1}) \cdots G(\alpha_{m})
\]
 is a homogeneous polynomial of weight $m$ in the $g_{i}$.  Viewed as
a homogeneous polynomial in $f_{i}$ and $g_{i}$ jointly, their weight
must be $m+n$.
\end{proof}

Using \eqnref{Resultant:Def:Eq}, if $F_1(X) = G_1(X) G_2(X)$, then the
resultant of $F_1(X)$ and $F_2(X)$ 
is the product of two smaller resultants:
\[
\res(G_1(X) \, G_2(X), F_2(X)) = \res(G_1(X), F_2) \cdot \res(G_2(X), F_2).
\]
We also have
\[
\res(F_1^r(X), F_2(X)) = \left( \res(F_1(X), F_2(X)) \right)^r.
\]

In the following paragraphs we give two different representations of
the resultant in terms of matrices whose entries are the coefficients
of $F$ and $G$.  \propref{Resultant:Weight:Prop} is used to prove that
the determinant of the matrix is actually the resultant and not a
multiple of the resultant.

\paragraph{Sylvester Matrix}
\index{Sylvester matrix|(}
{\Euler} produced the first matrix whose determinant was the resultant. 
This matrix is now called the \keyi{Sylvester matrix}. We follow 
{\Sylvester}'s derivation which is fairly simple. 

As before, assume we want to compute the resultant of 
\[
\begin{aligned}
F(X)&= f_0 X^m + f_1 X^{m-1} + \cdots + f_m,\\
G(X)&= g_0 X^n + g_1 X^{n-1} + \cdots + g_n,
\end{aligned}
\]
where we assume that $m \ge n$.  We construct a system of $N = m+n$
homogeneous linear equations in $N$ unknowns.\index{homogeneous linear
equations} This system can have a non-trivial solution only if its
determinant vanishes.  This determinant is Sylvester's matrix.

Consider the following two linear equations:
\[
\begin{aligned}
f_0 X_m + f_1 X_{m-1} + \cdots + f_{m-1} X_1 + f_m X_0 & = 0,\\
g_0 X_n + g_1 X_{n-1} + \cdots + g_{n-1} X_1 + g_n X_0 & = 0.
\end{aligned}
\]
These two equations vanish when $X_j = \alpha^j$, where $\alpha$ is a common
zero of $F$ and $G$.  There are other values of $X_{i}$ that satisfy these two
linear equations, but by adding additional equations we can eliminate them.
These additional relations among the $X_{i}$ come from the equations $X F(X) =
0$, $X^2 F(X) = 0$ and so on.  For instance, the relations $X F(X) = 0 = X
G(X)$ gives
\[
\begin{aligned}
f_0 X_{m+1} + f_1 X_{m} + \cdots + f_{m-1} X_2 + f_m X_1 & = 0,\\
g_0 X_{n+1} + g_1 X_{n} + \cdots + g_{n-1} X_2 + g_n X_1 & = 0.
\end{aligned}
\]
To get the additional needed equations we use the equations 
\[
\begin{array}{r@{,\quad}r}
  F(X)=0 & G(X) = 0,\\
  XF(X)=0 & XG(X) = 0,\\
  \vdots & \vdots \\
  X^{m-1}F(X)=0 & X^{n-1} G(X) = 0.
\end{array}
\]
These equations give a total of $m+n$ linear relations among the
variables $X_{0}, \ldots, X_{m+n-1}$:
\begin{equation}
\vcenter{\halign to\displaywidth{
\hfil{$\null#$}& \hfil{$\null#$}& \hfil{$\null#$}& \hfil{$\null#$}& \hfil{$\null#$}& {$\null#$}\hfil\cr 
f_0 X_{N-1} &+ f_1 X_{N-2} + \cdots &+ f_m X_{N-m-1}& & & = 0\cr
&f_0 X_{N-2}  + \cdots &+ f_{m-1} X_{N-m-1}& + f_m X_{N-m-2} & & = 0\cr
&&&\vdots&\cr
&&f_0 X_{m+1} + f_1 X_{m} + \cdots  &+ f_m X_1& & = 0\cr
&&f_0 X_m + \cdots &+ f_{m-1} X_1 &+ f_m X_0 & = 0\cr
g_0 X_{N-1} &+ g_1 X_{N-2} + \cdots &+ g_n X_{N-n-1}& & & = 0\cr
&g_0 X_{N-2}  + \cdots &+ g_{n-1} X_{N-n-1}& + g_n X_{N-n-2} & & = 0\cr
&&&\vdots&\cr
&&g_0 X_{n+1} + g_1 X_{n} + \cdots  &+ g_n X_1& & = 0\cr
&&g_0 X_n + \cdots &+ g_{n-1} X_1 &+ g_n X_0 & = 0\cr
}}
\label{Sylvest:Lin:Eq}
\end{equation}

This is a system of $N = m+n$ linear equations in $N$
unknowns.  If $F$ and $G$ have a  common factor then
\eqnref{Sylvest:Lin:Eq} has a non-trivial solution.  A linear system
of equations $\mathbf{A}\cdot \vec{x} = 0$ can have a non-trivial solution
if and only if $\mathbf{A}$ is non-singular.  Thus a necessary condition
for $F$ and $G$ to have a common factor is for the \key{Sylvester
matrix} to vanish:
\begin{equation}
\begin{pmatrix}
f_0 & f_1 & f_2 &\cdots & 0 & 0 & 0 \\
0 & f_0 & f_1 & \cdots & 0 & 0 & 0 \\
\vdots& & \vdots & & \vdots & & \vdots \\
0 & 0 & 0 & \cdots & f_{m-1} & f_m & 0 \\
0 & 0 & 0 & \cdots & f_{m-2} & f_{m-1} & f_m \\
g_0 & g_1 & g_2 &\cdots & 0 & 0 & 0 \\
0 & g_0 & g_1 & \cdots & 0 & 0 & 0 \\
\vdots& & \vdots & & \vdots & & \vdots \\
0 & 0 & 0 & \cdots & g_{n-1} & g_n & 0 \\
0 & 0 & 0 & \cdots & g_{n-2} & g_{n-1} & g_n \end{pmatrix}. 
\label{Sylvester:Matrix:Eq}
\end{equation}
Thus the resultant of $F$ and $G$ divides the determinant of the
Sylvester matrix.  But the determinant of the Sylvester matrix is a
homogeneous polynomial of weight $m+n$.  Therefore it is the resultant
up to a sign by \propref{Resultant:Weight:Prop}.

A few examples illustrate this construction.  Let $F(X) = f_0 X + f_1$
and $G(X) = g_0 X + g_1$.  Their resultant is
\[
\left|
  \begin{array}{cc} 
    f_{0} & f_{1} \\
    g_{0} & g_{1} 
  \end{array}
\right| = f_{0} g_{1} - f_{1} g_{0}.
\]
When $F(X)$ and $G(X)$ have a common zero, the resultant is zero and thus
$f_{0}/f_{1} = g_{0}/g_{1}$ and the equations only differ by a
multiplicative constant.  If $m = 3$ and $n= 2$ the resultant takes the
form:
\[
\left|
  \begin{array}{ccccc} 
    f_{0} & f_{1} & f_{2} & f_{3} & 0 \\
    0 & f_{0} & f_{1} & f_{2} & f_{3} \\
    g_{0} & g_{1} & g_{2} & 0 & 0 \\
    0 & g_{0} & g_{1} & g_{2} & 0 \\
    0 & 0 & g_{0} & g_{1} & g_{2} \end{array} \right|.
\]

A polynomial $F(X)$ is square free if and only if $F(X)$ and $F'(X)$
are relatively prime.  For quadratic equations this means that 
\[
R_q = \res_X(aX^2+bX+c, 2aX+b)
\]
must be non-zero.  That is,
\[
\left|\begin{array}{ccc}
a & b & c \\ 2a & b & 0 \\ 0 & 2a & b 
\end{array}
\right| = a(4ac - b^2)
\]
must be different from zero, as expected.

\medskip
One useful result that can be deduced from the Sylvester matrix is the
relationship between the size of the coefficients of a resultant and
the size of the original polynomials.

\begin{proposition} \label{Resultant:Bound:Prop}
Let $F(X, Y)$ and $G(X, Y)$ be elements of $K[X, Y]$, where $K$ is a
field with a valuation, $|\,|$.  Denote the degree of $F$ in $X$ by
$m$, that of $G$ by $n$ and $H(Y) = \res_X(F, G)$.  Denote the largest
term in $f$ by $|f|$ and similarly for $g$ and $h$.  Then
\[
|H| \le (m+n)!\, |F|^n |G|^m.
\]
\end{proposition}

\begin{proof}
This follows immediately from the determinant of Sylvester
matrix.\Marginpar{The $(m+n)!$ term can be eliminated by a trick of
Hadamard.} The determinant contains $(m+n)!$ terms where each term is
the product of $n$ coefficients of $f$ and $m$ coefficients of $g$.
\end{proof}
\index{Sylvester matrix|)}
\paragraph{Bezout Matrix}
\index{Bezout matrix|(}

\index{Bezout matrix} A smaller determinant can be produced by using a
slightly different procedure first suggested by Bezout.  In this
section we use a presentation originally due to {\Cauchy}.  The
formulas that result are fairly simple, but somewhat messy.  The
following notation simplifies them significantly,
\[
f_{[i}g_{j]} = f_i g_j - f_j g_i.
\]
It is essentially a ``commutator'' on the subscripts.  An undefined $f_i$,
$f_j$, $g_i$ or $g_j$ is assumed to be zero.

The two polynomials $F(X)$ and $X^{m-n}G(X)$ have the same degree.  If we
separate the first $k$ terms of each, we have the following two equations:
\[
\begin{aligned}
f_0 X^m + \cdots + f_{k-1} X^{m-k+1} & = - f_k X^{m- k} - \cdots - f_m\\
g_0 X^m + \cdots + g_{k-1} X^{m-k+1} & = - g_k X^{m - k} - \cdots - g_m
X^{m-n}
\end{aligned}
\]
Taking the ratio of these two equations, and canceling common factors of
$X$ between the numerator and denominator yields
\[
\frac{f_0 X^{k-1} + \cdots f_{k-1}}{g_0 X^{k-1} + \cdots + g_{k-1}} =
\frac{f_k X^{n - k} + \cdots + f_n}{g_k X^{n-k} + \cdots + g_m X^{m-n}}.
\]

Cross multiplying and subtracting we have a polynomial that vanishes when
$X$ is a zero of the {\sc gcd} of $F$ and $G$:
\begin{multline}
(f_0 X^{k-1} + \cdots f_{k-1}) \times (g_k X^{n-k} + \cdots + g_m
 X^{m-n})\\
 - (g_0 X^{k-1} + \cdots + g_{k-1}) \times (f_k X^{n - k} + \cdots +
 f_n)
\end{multline}
For $k = 1, 2$ this gives
\[
\begin{aligned}
f_{[0} g_{1]} X_{n-1} + f_{[0} g_{2]} X_{n -2} + \cdots + f_{[0} g_{n]}
X_0& = 0 \\ f_{[0} g_{2]} X_{n-1} + (f_{[1} g_{2]} + f_{[0} g_{3]}) X_{n-2}
+ \cdots + f_{[1} g_{n]} X_0& = 0
\end{aligned}
\]
Since $k$ can vary from $1$ to $n$, we can generate $n$ linear
relations among the $X_i$ from a single pair of equations $F(X) = 0 =
G(X)$, where the Sylvester construction yields only $2$.  If
additional equations are needed, we can use the first $m-n$ of
\eqnref{Sylvest:Lin:Eq}.  The resulting matrix is called the
\keyi{Bezout matrix}.  While the dimensions of the Sylvester matrix
are $(m+n)\times(m+n)$ the Bezout is $m \times m$, but its entries are
more complicated.  This reduction of the size of the determinant can
dramatically reduce the time required to compute the resultant.

We give a few examples to illustrate this.  Consider the resultant of
two general quadratics, \ie, $m=n=2$.  The Sylvester matrix is
\[
\begin{pmatrix}
  f_0& f_1& f_2& 0 \\
  0& f_0& f_1 & f_2\\
 g_0& g_1& g_2& 0 \\
0& g_0& g_1 & g_2\end{pmatrix},
\]
while the Bezout matrix is
\[
\begin{pmatrix}f_{[0}g_{1]} &f_{[0}g_{2]} \\ f_{[0}g_{2]} &f_{[1}g_{2]} \end{pmatrix}=
\begin{pmatrix}f_0g_1 - f_1 g_0 & f_0 g_2 - f_2 g_0 \\ f_0 g_2 - f_2 g_0 & f_1
g_2 - f_2 g_1\end{pmatrix}.
\]

For the case of two cubics we have
\begin{equation}
\begin{pmatrix}
  f_{[0}g_{1]} & f_{[0}g_{2]} & f_{[0} g_{3]} \\
  f_{[0}g_{2]} & f_{[0}g_{3]} + f_{[1}g_{2]} & f_{[1} g_{3]} \\
  f_{[0}g_{3]} & f_{[1}g_{3]} & f_{[2} g_{3]} \\
  \end{pmatrix}.
\label{Cubic:Bezout:Eq}
\end{equation}
Had a Sylvester matrix be generated, its rank would have been $6$
rather then $3$ as in the Bezout case.  Furthermore, notice that the
Bezout matrix is symmetric and only six different commutators were
used to generate it.

With two quintics, the Bezout matrix is the sum of the three following
matrices:
\[
\begin{pmatrix}
   f_{[0}g_{1]} & f_{[0}g_{2]} & f_{[0} g_{3]} & f_{[0}g_{4]} & f_{[0}g_{5]} \\
 f_{[0}g_{2]} & f_{[0}g_{3]} & f_{[0} g_{4]} & f_{[0}g_{5]} & f_{[1}g_{5]} \\
 f_{[0}g_{3]} & f_{[0}g_{4]} & f_{[0} g_{5]} &f_{[1}g_{5]} & f_{[2}g_{5]} \\
 f_{[0}g_{4]} & f_{[0}g_{5]} & f_{[1} g_{5]}& f_{[2}g_{5]} & f_{[3}g_{5]} \\
 f_{[0}g_{5]} & f_{[1}g_{5]} & f_{[2}g_{5]} & f_{[3}g_{5]} & f_{[4}g_{5]} 
\end{pmatrix}
\]
and
\[
\begin{pmatrix}
  0 & 0 & 0 & 0 & 0 \\
  0 & f_{[1}g_{2]} & f_{[1} g_{3]} & f_{[1}g_{4]} & 0 \\
  0 & f_{[1}g_{3]} & f_{[1} g_{4]} & f_{[2}g_{4]} & 0 \\
  0 & f_{[1}g_{4]} & f_{[2} g_{4]} & f_{[3}g_{4]} & 0 \\
  0 & 0 & 0 & 0 & 0 
 \end{pmatrix},\qquad
\begin{pmatrix}
  0 & 0 & 0 & 0 & 0 \\
  0 & 0 & 0 & 0 & 0 \\
  0 & 0 & f_{[2} g_{3]} & 0 & 0 \\
  0 & 0 & 0 & 0 & 0 \\
  0 & 0 & 0 & 0 & 0
  \end{pmatrix}
\]

As with the cubic and all other Sylvester matrices, the quintic matrix is
symmetric.  A slightly more complex case is the resultant of a cubic and a
quadratic, that is when $m= 3$ and $n= 2$.  We can rewrite $G(X)$ as
\[
G(X) = g_0 X^3 + g_1 X^2 + g_2 X 
\]
with $g_3 = 0$ and use the cubic Bezout matrix of \eqnref{Cubic:Bezout:Eq}.
Replacing the commutators that involve $g_3$ by their values we have:
\[
\begin{pmatrix}
  f_{[0}g_{1]} & f_{[0}g_{2]} & - f_3 g_0 \\
  f_{[0}g_{2]} & -f_3 g_0 + f_{[1}g_{2]} & -f_3 g_1 \\
  -f_3g_0 & -f_3g_1 & -f_3 g_2 
  \end{pmatrix}
\]
The determinant of this matrix is a homogeneous polynomial of weight
$6$, and clearly is a multiple of the resultant.  Since the resultant
is of weight $5$ there is a homogeneous polynomial of weight $1$ that
divides this determinant.  Looking at the bottom row, we see that
$-f_3$ can be divided out to leave
\[
\begin{pmatrix}
  f_{[0}g_{1]} & f_{[0}g_{2]} & - f_3 g_0 \\
  f_{[0}g_{2]} & -f_3 g_0 + f_{[1}g_{2]} & -f_3 g_1 \\
  g_0 & g_1 & g_2 
  \end{pmatrix},
\]
as desired.  

For problems requiring the computation of resultants, the direct
evaluation of the Bezout matrix can be the most effective approach,
especially when the answer involves many variables.  The techniques of
\chapref{Interpolation:Chap}  can be very useful in this
case.\Marginpar{Dixon resultants? defined in Proc. London Math
  Soc. 1908, vol. 6, 468--478}
\index{Bezout matrix|)}
\index{resultant|)}

\section{Subresultants} 
\label{Subresultant:Sec}

Thus far we have not seen much of a relationship between the resultant,
polynomial remainder sequences and polynomial {\sc gcd}'s.  This
section details these relationships.  For these discussions it is
more convenient to work with the transpose of the Sylvester matrix.  Since
$\det A = \det A^{T}$, the resultant of the two polynomials
\[
\begin{aligned}
  F(X)&= f_0 X^m + f_1 X^{m-1} + \cdots + f_m\\
  G(X)&= g_0 X^n + g_1 X^{n-1} + \cdots + g_n 
\end{aligned}
\]
is also the determinant of
\[
\left(
\begin{array}{cccccccccccc}
  f_0 & 0 & 0 & \cdots & 0 & 0 & g_0 & 0 & 0 & \cdots & 0 & 0 \\
  f_1 & f_0 & 0 & \cdots & 0 & 0 & g_1 & g_0 & 0 & \cdots & 0 & 0 \\
  f_2 & f_1 & f_0 & \cdots & 0 & 0 & g_2 & g_1 & g_0 & \cdots & 0 & 0 \\
  \vdots & \vdots & \vdots & & \vdots & \vdots & \vdots & & \vdots & & \vdots & \vdots \\
  0 & 0 & 0 & \cdots & f_m & f_{m-1} & 0 & 0 & 0 & \cdots & g_n & g_{n-1} \\
  0 & 0 & 0 & \cdots & 0 & f_m & 0 & 0 & 0 & \cdots & 0 & g_n 
\end{array}
\right). 
\]

Without changing the value of the determinant, we can multiply the next to
last row of the matrix by $X$ and add it to the last row.  If we multiply
the third row from the bottom by $X^2$, the fourth by $X^3$ and so on, and
then add all of them to the last row we obtain for the resultant of $F$ and
$G$:
\begin{equation}
\begin{vmatrix}
   f_0 & \cdots & 0 & 0 & g_0 & \cdots & 0 & 0 \\
   f_1 & \cdots & 0 & 0 & g_1 & \cdots & 0 & 0 \\
   f_2 & \cdots & 0 & 0 & g_2 & \cdots & 0 & 0 \\
   \vdots & & \vdots & & \vdots & & & \vdots \\
   0 & \cdots & f_m & f_{m-1} & 0 & \cdots & g_n & g_{n-1} \\
   X^{n-1} F(X) & \cdots & X F(X) & F(X)& 
     X^{m-1} G(X) & \cdots & X G(X) & G(X)
\end{vmatrix}. 
\label{Res:Deta:Eq} 
\end{equation}

Expanding the matrix by minors along the last row we see that 
\[
\res(F(X), G(X)) = A(X) F(X) + B(X) G(X)
\]
where $A(X)$ and $B(X)$ are polynomials of degrees less than $n$ and
$m$ respectively.  If $F(X)$ and $G(X)$ were primitive polynomials,
any common divisor they possess must also divide the resultant.  Since
the resultant is free of $X$, if $F(X)$ and $G(X)$ have a non-trivial
common divisor the resultant must be zero.

\smallskip
If one uses the subresultant {\sc prs} on $F$ and $G$, the final
remainder will be the resultant of $F$ and $G$.  The other terms in
the {\sc prs} are called {\em subresultants\/}.\index{subresultant}
Just as the resultant can be defined as the determinant of a matrix,
so can the subresultant of $F$ and $G$.  

The $j$-th {\em subresultant}\index{subresultant!$j$-th} of $F$
and $G$, $S_j(F, G)$ is the determinant of the $(m+n - 2j)\times (n+m
-2j)$ matrix
\[
\begin{pmatrix}
f_0 & \cdots &  0  & g_0 & \cdots &  0 \\
f_1 & \cdots &  0  & g_1 & \cdots &  0 \\
\vdots &     & \vdots &  &     & \vdots \\
0 & \cdots &  f_{n-j-1} & 0 & \cdots  & g_{m-j-1} \\
X^{n-j-1} F(X) & \cdots & F(X)& X^{m-j-1} G(X) & \cdots & G(X)\end{pmatrix}.
\]

This matrix is obtained from the one in \eqnref{Res:Deta:Eq} by removing
the first $j$ columns from the matrix and the first $j$ columns that
include coefficients of $G$.  The top $j$ rows now contain only zeroes so
they are removed.  Then the next to last row and the $j-1$ rows above it
are removed.  This leaves the matrix given above.

Notice that the zeroth subresultant $S_0(F, G)$ is the resultant of $F$ and
$G$.  An example should clarify this.  Consider the case, where $F(X)$ is a
cubic polynomial and $G(X)$ is quadratic.  The zeroth subresultant,
$S_{0}(F, G)$ is
\[
\begin{aligned}
\det \left|
 \begin{array}{ccccc}
    f_{0} & 0 & g_{0} & 0 & 0 \\
    f_{1} & f_{0} & g_{1} & g_{0} & 0 \\
    f_{2} & f_{1} & g_{2} & g_{1} & g_{0} \\
    f_{3} & f_{2} & 0 & g_{2} & g_{1} \\
    X F(X) & F(X) & X^{2} G(X) & X G(X) & G(X)
 \end{array} \right| \\
  \qquad\qquad\qquad\qquad\qquad\qquad\qquad\qquad
     = A_{0}(X) F(X) + B_{0}(X) G(X).
\end{aligned}
\]
where $\deg A_{0} \le 1$ and $\deg B_{0} \le 2$.  The first subresultant,
$S_{1}(F, G)$ is 
\[
\det \left|
  \begin{array}{ccc}
    f_{0} & g_{0} & 0 \\
    f_{1} & g_{1} & g_{0} \\
    F(X) & X G(X) & G(X) 
  \end{array} \right|= A_{1}(X) F(X) + B_{1}(X) G(X).
\]
Unlike $S_{0}(F, G)$, $S_{1}(F, G)$ is a polynomial in $X$.  The
reason subresultants are so important in analyzing polynomial
remainder sequences is that, up to a multiplicative factor, the
polynomial remainder sequence for $F$ and $G$, as given will be: $F,
G, S_{1}, S_{0}$. By specifying that $\beta_i$ be chosen so that
remainders are subresultants, we will derive the algorithm discussed
in the previous section.  

As we show below, the polynomials in the subresultant {\sc prs} are
actually the subresultants of $F$ and $G$.  A simple modification to
the subresultant {\sc gcd} algorithm gives the following algorithm for
computing the resultant of two polynomials.
\begindsacode
PolyResultant ($U$, $V$) := $\{$\\
\> $h \leftarrow 1$; \\
\> loo\=p while $V \not= 0$  do \{ \\
\> \> $\delta \leftarrow \deg U - \deg V$; \\
\> \> $\beta \leftarrow (-1)^{\delta+1} \times \lc U \times h^{\delta}$; \\
\> \> $h \leftarrow h \times ((\lc V)/h)^{\delta}$; \\
\> \> $(U, V) \Leftarrow (V, \prem (U, V)/\beta)$; \\
\> \> $\}$ \\
\> if $\deg_X(U) = 0$ then return ($U$); \\
\>\> else return($0$); \\
\> $\}$
\enddsacode 

\noindent
Recall that when $F$ and $G$ have a common factor their resultant is zero.

\section{Elimination Examples}
\label{Result:Examp:Sec}

In this chapter we have given a number of different ways of computing
and interpreting the resultant of two polynomials.  In this section we
first summarize the different properties of the resultant and then
give a few examples of how resultants can be used to perform some
interesting calculations.

Let $F(X, Y)$ and $G(X,Y)$ be two polynomials and let $R(Y)$ be the
resultant of $F$ and $G$ with respect to $X$.  There are three basic
interpretations of $R(Y)$.
\begin{itemize}
\item $R(Y)$ is the product of $F(\beta, Y)$ evaluated at each of the
zeroes of $G$, $G(\beta, Y) = 0$, and equivalently the product of
$G(\alpha, Y)$ evaluated at each of the zeroes of $F$, $F(\alpha, Y) =
0$:
\[
\begin{aligned}
R(Y) & = g_0^m F(\beta_1, Y) F(\beta_2, Y) \cdots F(\beta_n, Y), \\
  & = (-1)^{mn} f_0^n G(\alpha_1, Y) G(\alpha_2, Y) \cdots G(\alpha_m, Y).
\end{aligned}
\]
\item The vanishing of $R(Y)$ is a necessary and sufficient condition
for $F$ and $G$ to have a {\sc gcd} involving $X$:
\[
R(Y) = A(X, Y) F(X, Y) + B(X, Y) G(X, Y).
\]
\item The zeroes of $R(Y)$ are the $Y$ coordinates of the common zeroes
of $F(X, Y)$ and $G(X, Y)$.
\end{itemize}
All three of these interpretations of the resultant are useful in
applications. 

\paragraph{Minimal polynomials}

\index{algebraic extension|(} Let $\alpha$ be an algebraic element
over a field $K$.  The \keyi{minimal polynomial} of $\alpha$ is the
monic polynomial $P(X) \in K[X]$ of lowest degree such that $P(\alpha)
= 0$.  The minimal polynomial of an algebraic element must be
irreducible.  The algebraic extension $K[\alpha]$ is isomorphic to
$K[x]/(P(x))$, the ring of polynomials over $K$ modulo the polynomial
$P(x)$. The isomorphism between these two fields sends $x$ to
$\alpha$.

Arithmetic in $K[\alpha]$ can be implemented in much the same manner
as polynomial arithmetic in $K[x]$, except that the remainder with
respect to $P(\alpha)$ is returned after each arithmetic operation.
In this fashion it is very similar to the arithmetic in
$\F_p$.

One of the useful operations with elements of fields is to compute their
norm.\index{norm!of an algebraic element}  Let $\alpha = \alpha_1,
\alpha_2, \ldots, \alpha_d$ be the conjugates of
$\alpha$\index{conjugate} (the other zeroes of $P(x)$).  Let
$q(\alpha)$ be an element of $L = K[\alpha]$.  The {\em norm} of
$\alpha$ is defined to be:
\[
\Norm_{L/K} q(\alpha) = q(\alpha_1) q(\alpha_2) \cdots q(\alpha_d).
\]
The norm of an element of $L$ is clearly a symmetric function in the
$\alpha_i$ and thus can be written in terms of the primitive symmetric
functions $\sigma_i$ which are just the coefficients of $P(x)$.  Thus,
$\Norm_{L/K} q(\alpha)$ is an element of $K$.  By
\eqnref{Resultant:Def:Eq} we see that
\[
\Norm_{L/K} q(\alpha) = \res_t (q(t), P(t)).
\]

The minimal polynomial of $q(\alpha)$ is a divisor of 
\[
Q(X) = (X - q(\alpha_1))(X - q(\alpha_2)) \cdots (X - q(\alpha_d)).
\]
$Q(X)$ can also be computed by a resultant:
\begin{equation} \label{AlgElt:Min:Eq}
Q(X) = \res_t (X - q(t), P(t)).
\end{equation}
The following proposition due to {\Trager} \cite{Trager1976-ri} is
extremely useful in generating minimal polynomials like the above.

\begin{proposition}[\Trager]
If $f(x, \alpha)$ is an irreducible polynomial over $K = k[\alpha]$
then $F(X) = \Norm_{K/k} f$ is a power of an irreducible polynomial over $k$.
\end{proposition}

\begin{proof}
$F(X)$ can be written as
\[
F(X) = \Norm_{K/k} f(X, \alpha) = f(X, \alpha_1) \cdots f(X, \alpha_n),
\]
where $\alpha_1 = \alpha$ and $\alpha_2, \ldots, \alpha_n$ are its
conjugates.\index{conjugate} Assume $F(X)$ were reducible, $F(X)= C(X)
D(X)$.  We can assume that $C(X)$ and $D(X)$ are relatively prime.
The polynomial $f(x, \alpha_1)$ divides $F(X)$, and since $f(x,
\alpha_1)$ is irreducible, it divides either $C(X)$ and $D(X)$.
Without loss of generality we can assume $C(X) = f(X, \alpha_1) g(X,
\alpha_1)$.  The fields $k[\alpha_1]$ and $k[\alpha_j]$ are
canonically isomorphic under a mapping $\sigma_j$ that sends
$\alpha_1$ to $\alpha_j$ and is the identity on $k$.  $\sigma_j$ can
be extended to the ring of polynomials in $x$ over those fields and
still remain an isomorphism.  Since all the coefficients of $C(X)$ are
in $k$ it is invariant under $\phi_j$, $f_1$ and $g_1$ are mapped to
$f_j$ and $g_j$ respectively.  Consequently, each of the $f_j$ divides
$C(X)$.  Since $C(X)$ and $D(X)$ are relatively prime, none of the
$f_j$ can divide $D(X)$.  Since $D(X)$ divides the norm of $f$, it
must be equal to one. Thus the norm of $f$ is the power of an
irreducible polynomial.
\end{proof}

Since $X - q(\alpha)$ is linear and thus irreducible, the $Q(X)$
computed in \eqnref{AlgElt:Min:Eq} is either irreducible or not square
free.  Thus we do not need to factor $Q(X)$ to find the minimal
polynomial of $q(\alpha)$, just computing the square free
decomposition suffices.

\medskip
If $\alpha$ and $\beta$ have minimal polynomials $P_1(X)$ and $P_2(X)$
respectively, another computation that is often needed is determining
the minimal polynomial of $\alpha + \beta$.  If the
conjugates\index{conjugate} of $\alpha$ are denoted by $\alpha_i$ and
those of $\beta$ by $\beta_j$ then the minimal polynomial of
$\alpha+\beta$ divides
\begin{equation} \label{AlgSum:Min:Eq}
Q_2(X) = \prod_{i,j} (X - \alpha_i - \beta_j).
\end{equation}
This polynomial can be computed as 
\[
\begin{aligned}
Q_2(X) & = \res_t ( \res_s ((X- s - t), P_1(s)), P_2(t)), \\
  & = \res_t (P_1(X - t), P_2(t)).
\end{aligned}
\]

\begin{figure}
\begin{center}
\begin{tabular}{|c|c|} \hline
$\alpha + \beta$ & $\res_t(P_2(Z - t), P_1(t))$ \\[3pt] \hline
$\alpha - \beta$ & $\res_t(P_2(Z + t), P_1(t))$ \\[3pt] \hline
$\alpha \times \beta$ & $\res_t(t^{d_2} \cdot P_2(Z/t), P_1(t))$ \\[3pt] \hline
$\alpha / \beta$ & $\res_t(P_2(t Z), P_1(t))$ \\[3pt] \hline
$\root r \of{\alpha}$ & $\res_t(Z^r - t, P_1(t))$ \\[3pt] \hline
$a \alpha + b$ & $a^{n_1} P_1((Z - b)/a)$ \\[3pt] \hline
\end{tabular}
\end{center}
\caption{Common Minimal Polynomials \label{Minimal:Polynomial:Fig}}
\end{figure}

For example, the minimal polynomial of $\sqrt{2} + \sqrt{3}$ is 
\[
\res_t((X -t)^2 - 2, t^2 - 3) = X^4 - 10X^2 + 1,
\]
which is irreducible.  Similar formulas for the minimal polynomials of 
other expressions in $\alpha$ and $\beta$ are given in 
\figref{Minimal:Polynomial:Fig}. 

\index{algebraic extension|)}

\paragraph{Powers of Zeroes}

Let $P(X)$ be a monic polynomial with zeroes $\alpha_1, \ldots, \alpha_d$
\[
P(X) =  \prod_{1 \le i \le d}(X - \alpha_i).
\]
How do we find a polynomial whose roots are $\alpha_i^2$?  This is
still a polynomial of degree $d$.  

Intuitively, the ``polynomial''
\[
\bar{P}(Y) =  \prod_{1 \le i \le d}\left(Y^{\frac{1}{2}} - \alpha_i\right)
\]
has $\alpha_i^2$ as its zeroes, but it is not a polynomial.  Instead,
we have to make the substitution a bit more carefully, by using a
resultant.  One approach would be to use the next to last formula in
\figref{Minimal:Polynomial:Fig}, but replacing $r$ by $1/2$, \eg, 
$\res_X(Y - X^2, P(X))$.

We can get an even more succinct formula as follows.  Write $P(X)$ in
the form
\[
P(X) = p_e(X^2) + X p_o(X^2).
\]
To square the zeroes of $P(X)$ we replace $X^2$ by $Y$ in $p_o$ and
$p_e$ and then use a resultant to eliminate $X$:
\[
\begin{aligned}
P_2(Y) & = \res_X (p_e(Y) + X p_o(Y), X^2 - Y), \\
    & = Y p_o(Y)^2 - p_e(Y)^2.
\end{aligned}
\]

More generally, if the zeroes of $P_k(Y)$ are $\alpha_i^k$ then $P_k$
can be computed as follows.  Write $P(X)$ in the form:
\[
P(X) = p_0(X^k) + X p_1(X^k) + \cdots + X^{k-1} p_{k-1}(X^k),
\]
where $X^ip_{k-i}(X^k)$ containts the terms of $P(X)$ whose degree is equal to $i$ modulo $k$.  Then 
\[
P_k(Y) = \res_X(P_0(Y) + Xp_1(Y) + \cdots + X^{k-1}p_{k-1}(Y), X^k -
Y).
\]

The table below tabulates these values for small $k$.  For succinctness
we write $p_i = p_i(X)$.
\[
\begin{aligned}
P_2(X) = p_1^2 X&- p_0^2, \\
P_3(X) = p_2^3 X^2 &- 3 p_0 p_1 p_2 X + p_1^3 X + p_0^3, \\
P_4(X) = p_3^4 X^3 &+ (4 p_1 p_2^2 p_3 - p_2^4 - 4p_0 p_2 p_3^2 -
2p_1^2 p_3^2) X^2 \\
  & + (2 p_0^2 p_2^2 + p_1^4 - 4 p_0 p_1^2 p_2 + 4 p_0^2
p_1 p_3) X - p_0^4.
\end{aligned}
\]

\paragraph{Algebraic Dependence of Polynomials}

Let $X(t)$ and $Y(t)$ be polynomials in $t$ that parameterize some
algebraic curve.  Thus there must be some polynomial $P(X, Y)$ such
that $P(X(t), Y(t)) = 0$.  An interesting question is how to determine
$P$.  Since the answer to this question is a polynomial in two
variables, but we are given polynomials in one variable it seems
unlikely that resultants would be of much use.

Nonetheless, this problem can be solved using resultants using the
third interpretation given at the beginning of this section.  Consider
the two polynomial equations:
\[
\begin{aligned}
X(t) -x & = 0, \\
Y(t) - y & = 0.
\end{aligned}
\]
Each is a surface in three dimensions.  Their common intersection is
a line in three space and the projection to the $x-y$ plane is the
curve $P(X(t), Y(t))$.  So,
\[
P(x, y) = \res_{t}(X(t) - x, Y(t) - y)
\]

By similar techniques we can find a three variable polynomial $P(x, y,
z)$ that is satisfied by three bivariate polynomials $X(s, t)$, $Y(s,
t)$ and $Z(s, t)$.  In general, given three univariate polynomials
this cannot be done.  This can also be understood from geometric
considerations. 

\medskip
Another interesting question is: Given polynomials 
\[
F_1(X_1, \ldots, X_n), \ldots, F_m(X_1, \ldots, X_n)
\]
and a polynomial $P(X_1, \ldots, X_n)$, can $P$ be expressed as a
polynomial (or rational function) in the $F_i$.

Essentially the same technique is used for this problem as for the last
one.  Consider the system of equations:
\[
\begin{aligned}
f_1 - F_1(X_1, \ldots, X_n) & = 0, \\
\vdots \\
f_m - F_m(X_1, \ldots, X_n) & = 0, \\
p - P(X_1, \ldots, X_n) & = 0,
\end{aligned}
\]
which are interpreted as polynomials in the variables $X_1,
\ldots, X_n$, $f_1, \ldots, f_m$ and $p$.  The projection of this
$m+n+1$ dimensional geometric structure onto the $f_1, \ldots, f_m, p$
subspace will give the relationship:
\[
Q(f_1, \ldots, f_m, p) = 0.
\]
This polynomial should be factored into irreducible factors.  Assume
it has the linear factor
\[
a(f_1, \ldots, f_m) p - b(f_1, \ldots, f_m).
\]
Then we have 
\[
P(X_1, \ldots, X_n) = \frac{b(F_1, \ldots, F_m)}{a(F_1, \ldots, F_m)}.
\]
If there are no linear factors then the problem has no solutions.  If
there are no monic linear factors then there are no polynomial
solutions.  

\section*{Notes}

\small

Two major topics in elimination theory have been omitted in this 
discussion: multivariate resultants and 
Gr\"{o}bner bases\index{Groebner@Gr\"{o}bner basis}.  Whereas
the resultant eliminates one variable from two polynomials, the
multivariate resultant eliminates several variables from several
polynomials in a single step.  The multivariate resultant was invented by
{\Macaulay} \cite{Macaulay1902-ez,Macaulay1916-rw,Macaulay1921-gz} and has recently
been used and developed by {\Canny} \cite{Canny1987-sc,Manocha1991-iq,Manocha1993-re}.

Gr\"{o}bner bases are another approach to \key{elimination} that is
best described in terms of commutative algebra and ideal theory.  The
basic ideas behind Gr\"{o}bner bases were developed in
{\Buchberger}'s thesis \cite{Buchberger1970-pa} and independently in the
work of {\Hironaka} on desingularization of algebraic varieties
\cite{Hironaka1964-du}.  Since the late 1970's it has been an area of
intense activity in computer algebra.  An introduction to the basic
principles involved in Gr\"{o}bner basis computations is contained in
\cite{Buchberger1985-sz}.  A collection of important papers on
Gr\"{o}bner bases and their applications is contained in
\cite{Robbiano1989-jx}.

\notesectref{Symmetric:Sec}  {\Newton}'s formulas for symmetric
functions were first published in 1707 \cite{Newton1707-yd}.  An accessible
summary of the basic results on symmetric functions and their
application to Galois theory is contained in {\Netto}
\cite{Netto1908-mu} and {\JordanC} \cite{Jordan1870-vz}.

\notesectref{Poly:Resultant:Sec} The determinant form of the resultant
was first stated by {\Sylvester} in \cite{Sylvester1853-vw}.

\notesectref{Subresultant:Sec} The subresultant {\sc gcd} algorithm
was the culmination of a large body of research by {\BrownWS} and
{\Collins} on {\sc gcd} algorithms.  Only later was it realized that the
basic ideas behind the subresultants and the subresultant {\sc gcd}
algorithm had already appeared in papers by {\Habicht}
\cite{Habicht1948-tn,Habicht1948-oi}.   A thorough discussion of subresultants
and the subresultant {\sc prs} is given in {\Loos}'s paper
\cite{Loos1982-up}.

\notesectref{Result:Examp:Sec}  An alternative approach to the last 
problem in this section using Gr\"{o}bner bases is given in
\cite{Shannon1988-gh}. 

\normalsize


