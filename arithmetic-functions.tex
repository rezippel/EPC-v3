$Id: anal-nt.tex,v 1.1 1992/02/24 01:01:16 rz Exp rz $
\chapter{Arithmetic Functions}
\label{Analytic:NT:Chap}

When analyzing symbolic computing algorithms one often needs to know
some property about the density of prime numbers, the size of $n!$ or
the average number divisors of a randomly chosen integer.  Tied up
with these questions are certain functions whose fundamental character
is arithmetic not analytic.  Examples of these functions are $d(n)$
the number of divisors of the integer $n$, $\pi(x)$ the number of
prime numbers less than $x$ and $n!$ the product of all positive
integers less than or equal to $x$.  The questions that arise can be
quite subtle, like the size of the largest gap between consecutive
prime numbers.  These questions can often only be answered by a
combination of combinatorial and analytic techniques

This chapter contains an introduction to some of the tools used for
these types of problems as well as a survey of some of the results
that we will need later.   Additional references for further study are
mentioned throughout the chapter and in the notes at the end of the
chapter.

\section{Arithmetic Functions}
\label{Arithmetic:Functions:Sec}

Functions that are only defined for integer arguments are called
\keyi{arithmetical functions}.  Examples of arithmetic functions are
$d(n)$, the number of positive divisors of $n$, $\pi(n)$ the number of
prime numbers less than $n$ and the \keyi{partition function} $p(n)$,
which is the number of ways $n$ can be expressed as the sum of positive
integers ignoring the order of the parts.  These and other arithmetic
functions often arise when analyzing algorithms in symbolic
computation.  This section introduces a number of arithmetic functions
and discusses their properties.

An arithmetical function $f(n)$ that satisfies the equation $f(mn) =
f(m) f(n)$, where $m$ and $n$ are relatively prime is said to be {\em
multiplicative}\index{multiplicative function}.  The divisor function
is multiplicative, while the partition function is not.  If $f(n)$ is
a multiplicative function then $f(1 \cdot m) = f(1) f(m)$.  Thus, if
$f(n)$ is multiplicative then $f(1) = 1$.

Let $d_1, \ldots, d_k$ be the divisors of $n$, including both $1$ and
$n$.  Then a sum over the $d_i$ will be denoted as
\[
\sum_{1 \le i \le k} F(d_i) = \sum_{d\mid n} F(d).
\]
For instance, the sum of the divisors of an integer $n$ is defined to be
\[
\sigma_1(n) = \sum_{d \mid n}d
\]
and $\sigma_1(6) = 1+2 +3+6 = 12$.

We call such a summation a {\em multiplicative
sum}.\index{multiplicative sum} One of the nice features of this type
of summation over divisors is that it is a multiplicative operation.
The proof of this is a simple calculation.

\begin{proposition} \label{MultiplicativeSum:Prop}      
If $f(n)$ is a multiplicative function and $F(n)$
is defined by
\[
F(n) = \sum_{d\mid n} f(d) 
\]
then $F(n)$ is multiplicative.
\end{proposition}

\begin{proof}
\[
\begin{aligned}
  F(a) F(b) &= 
      \biggl(\sum_{d_1\mid a} f(d_1) \biggr) \cdot
      \biggl(\sum_{d_2\mid b} f(d_2) \biggr)\\
    & = \sum_{\substack{d_1\mid a \\ \scriptstyle d_2\mid b}}
      f(d_1 d_2) 
     =  \sum_{d\mid ab} f(d) = F(ab)
\end{aligned}
\]
\end{proof}

Similar to the regular convolution of two functions, we have the
multiplicative convolution of two multiplicative functions $f(n)$ and
$g(n)$:
\[
F(n) = \sum_{d \mid n} f(d) g(\frac{n}{d}).
\]
By \propref{MultiplicativeSum:Prop}, $F(n)$ is also multiplicative. 

\index{Moebius function@M\"obius function} One of the most useful
arithmetical functions is the {\em M\"obius} function, which is
defined by 
\[
\mu(n) =
  \begin{cases}
    1& \text{if $n = 1$,}\\
    0& \text{if $n$ is not square free,} \\
    (-1)^\ell& \text{if $n = p_1 \cdots p_\ell.$}
  \end{cases}
\]
The M\"obius function is clearly multiplicative.
\addsymbol{$\mu(n)$}{M\"obius function}

\begin{proposition} \label{Inclusion:Exclusion:Prop}
If $f$ is a multiplicative function then 
\[
\sum_{d\mid n} \mu(d) f(d) = \prod_{p\mid n}(1 - f(p)),
\]
where the product is over all prime divisors of $n$.
\end{proposition}

\begin{proof}
Let $n = p_1^{e_1} \cdots p_k^{e_k}$.  Since $\mu(d) = 0$ if $d$ is not
square free, we can replace $n$ by $N = p_1 \cdots p_k$.  Thus
\[
\begin{aligned}
  \sum_{d\mid n} \mu(d) f(d) & = 1 + \sum_{p_1\mid N} \mu(p_1) f(p_1) + 
  \sum_{\substack{p_1, p_2\mid N \\ p_1 \not=p_2}} \mu(p_1 p_2) f(p_1 p_2) +
      \cdots,\\
    & = 1 - \sum_{p_1\mid N} f(p_1) 
       + \sum_{\substack{p_1, p_2\mid N \\ p_1 \not=p_2}} f(p_1) f(p_2)
       + \cdots,\\
    & = \prod_{p\mid N} \left(1 - f(p)\right).
\end{aligned}
\]
Since the prime divisors of $N$ are the same as those of $n$, we are
finished. \end{proof}

The following proposition is an immediate corollary of
\propref{Inclusion:Exclusion:Prop}. 

\begin{proposition} \label{MobiusSum:Prop}
\[
\sum_{d\mid n} \mu(d) = 
  \begin{cases}
    1 & \text{if $n = 1$} \\
    0& \text{if $n > 1$}
  \end{cases}
\]
\end{proposition}

The following proposition is one of the most useful formulas
involving arithmetic functions.  It is called the {\em M\"obius
inversion formula}\index{Moebius Inversion@ M\"obius inversion
formula} and allows us to revert a function defined as a sum over divisors.

\begin{proposition}
\label{Mobius:Prop}
Assume that $f(n)$ is a multiplicative function, and define
\begin{equation}\label{Moebius:Eq:a}
F(n) = \sum_{d\mid n} f(d).
\end{equation}
$F(n)$ is also multiplicative and 
\begin{equation}\label{Moebius:Eq:b}
f(n) = \sum_{d\mid n} \mu(d) F(\frac{n}{d}).
\end{equation}
\end{proposition}

\begin{proof}
This is simply a matter of evaluating the right hand side of the
conclusion carefully.  Using the definition of $F(n)$ we have
\[
\begin{aligned}
  \sum_{d\mid n} \mu(d) F(\frac{n}{d}) &=
      \sum_{d_1\mid n} \biggl( \mu(d_1) 
         \sum_{d_2\mid \frac{\scriptstyle n}{\scriptstyle d_1}} f(d_2) \biggr)
      = \sum_{d_1 d_2 \mid n} \mu(d_1) f(d_2) \\
    & = \sum_{d_2 \mid  n} \biggl[ f(d_2) 
      \sum_{d_1\mid \frac{\scriptstyle }{\scriptstyle d_2}} \mu(d_1)\biggr].
\end{aligned}
\]
By \propref{MobiusSum:Prop}, the last sum is zero unless $d_2 = n$.
When $d_2 = n$, this sum is equal to $1$, so the entire sum is just
$f(n)$.
\end{proof}

While \propref{Mobius:Prop} shows that \eqnref{Moebius:Eq:a} implies
\eqnref{Moebius:Eq:b}, the two equations are actually equivalent.
That is, if $F(n)$ is a multiplicative function and $f(n)$ is defined
by \eqnref{Moebius:Eq:b} then
\[
f(n) = \sum_{d\mid n} \mu(d) F(\frac{n}{d}).
\]

The M\"obius function is used in another inversion formula that is useful
when studying the distribution of primes numbers.

\begin{proposition}\label{Mobius:Prop:b}
Let 
\[
f(x) = F(x) + \frac{1}{2} F(x^{1/2}) + \cdots = 
\sum_{n=1}^{\infty} \frac{1}{n}F(x^{1/n}),
\]
then 
\[
F(x) = \sum_{n=1}^{\infty}\frac{\mu(n)}{n}f(x^{1/n}).
\]
\end{proposition}

\noindent
This proposition can be proved in a manner similar to
\propref{Mobius:Prop}.  We leave it as an exercise.




\medskip
\index{Euler $\phi$-function} \index{totient function}
For $n$ greater than $2$ the {\em Euler $\phi$-function}, or {\em
totient} function, $\phi(n)$ is defined to be the number of positive
integers less than $n$ that are relatively prime to $n$.  In addition,
$\phi(1) = 1$.  Clearly $\phi(p) = p -1$ if $p$ is prime.
\addsymbol{$\phi(n)$}{Euler totient function}

We use the following easy proposition as a stepping stone to a formula
for computing $\phi(n)$.

\begin{proposition}
\label{SumTotient:Prop}
\[
\sum_{d\mid n} \phi(d) = n
\]
\end{proposition}

\begin{proof}
Consider the sequence of rational numbers
\[
\frac{1}{n}, \frac{2}{n}, \frac{3}{n}, \ldots,
\frac{n - 1}{n}, \frac{n}{n},
\]
reduced to lowest terms.  The denominators of these reduced fractions
divide $n$.  Group together the fractions with denominator $d$.  There
will be $\phi(d)$ of these.  Thus each of these groups corresponds to
one term of the summation.  Since the total number of rational numbers
considered is $n$, the proposition is proved \end{proof}

The following proposition follows immediately by using the M\"obius
inversion formula of \propref{Mobius:Prop}.

\begin{proposition} \label{TotientSum:Prop}
\[
\phi(n) = \sum_{d\mid n} \mu(d) \frac{n}{d}
\]
\end{proposition}

Propositions \ref{TotientSum:Prop} and \ref{MultiplicativeSum:Prop} show that
$\phi(n)$ is multiplicative.  Let $N$ be the square free part of $n$.
Then we have
\[
\begin{aligned}
  \phi(n) &= \sum_{d\mid n} \mu(d) \frac{n}{d} =
      n \sum_{d\mid N} \frac{\mu(d)}{d},\\
    &= n \prod_{p\mid N} \left(1 - \frac{1}{p}\right) = 
      n \prod_{p\mid n} \left(1 - \frac{1}{p}\right).
\end{aligned}
\]
If $n = p_1^{e_1} \cdots p_m^{e_m}$ then this formula can also be
written as 
\[
\phi(n) = p_1^{e_1-1} (p_1 - 1) \cdots p_m^{e_m-1} (p_m - 1).
\]


\medskip
There is a wide variety of other arithmetic functions.  Among those that are
particularly useful when analyzing algorithms are $d(n)$, the
number of divisors of $n$; $\sigma_r(n)$, the sum of the $r$-th
powers of divisors of $n$ and the $\Lambda$ function which is defined
below.
\addsymbol{$d(n)$}{Number of divisors of $n$}
\addsymbol{$\sigma_r(n)$}{Sum of the $r$-th powers of divisors of $n$}

We can write the definitions of $d(n)$ and $\sigma_r(n)$ algebraically
as 
\[
d(n) = \sum_{d\mid n} 1,
\]
and
\[
\sigma_r(n) = \sum_{d\mid n} d^r.
\]
Clearly, $d(n) = \sigma_0(n)$.

By \propref{MultiplicativeSum:Prop}, $\sigma_r(n)$ is a multiplicative
function. So we can determine the value of $\sigma_r(n)$ from
$\sigma_r(p^e)$ where $p$ is a prime, which is
\[
\begin{aligned}
\sigma_r(p^e) & = 1^r + p^r + (p^2)^r + \cdots + (p^e)^r, \\
 & = \displaystyle \frac{p^{er} - 1}{p^r - 1},
\end{aligned}
\]
for $r \not= 0$.  For $r=0$ we have
\[
\sigma_0(p^e) = d(p^e) = e + 1.
\]
So, if
\[
n = p_1^{e_1} \cdots p_k^{e_k}, 
\] 
where the $p_i$ are prime numbers, then 
\[
d(n) = (e_1 + 1) (e_2 + 1) \cdots (e_k + 1)
\]
and
\[
\sigma_r(n) = \frac{p_1^{e_1 r} - 1}{p_1^{r} - 1} \, 
\frac{p_2^{e_2 r} - 1}{p_2^{r} - 1} \cdots
\frac{p_k^{e_k r} - 1}{p_k^{r} - 1}.
\]


The $\Lambda$ function is very useful when studying the distribution
of prime numbers.  It is defined as
\[
\Lambda(n) = 
  \begin{cases}
    \log p & \text{if $n = p^{e}$,} \\
    0 & \text{otherwise}
  \end{cases}
\]
This function satisfies the following useful identities:
\addsymbol{$\Lambda(n)$}{If $n = p^e$ then $\log p$ otherwise $0$}

\begin{proposition}
\label{Lambda:Fn:Prop}
\[
 \log n  = \sum_{d \mid n} \Lambda (d), \quad\mbox{and}\quad
 \Lambda(n) = \sum_{d\mid n} \mu(\frac{n}{d}) \log d
\]
\end{proposition}
The first equality can be easily proven using the factorization of $n$.
The second follows from the first using the M\"obius inversion formula.

\section{The Gamma Function}
\label{GammaFunction:Sec}

One arithmetic function that arises frequently in our work is the
\keyi{factorial function}, which is recursively defined by 
\[
\begin{aligned}
0! & =  1, \\
n! &= n \times (n-1)!.
\end{aligned}
\]
Euler discovered that there is a continuous function that takes on the
value of the factorial function at integer points.  This function is
called the Gamma function and for technical reasons it is more
convenient to define it as $\Gamma(n) = (n-1)!$.

Euler's development of the $\Gamma$-function is based on the
observation that the $n$-th derivative of $X^n$ is $n!$ and the use
of integration by parts to create a recursive formula.  Notice that
\[
\begin{aligned}
F(n) & = \int^b_a f(t) t^n \, dt  = 
\left.f(t) t^n \right|^b_a - n \int_a^b f(t) t^{n-1}\, dt, \\
& = \left.f(t) t^n \right|^b_a - n F(n-1). 
\end{aligned}
\]
So if $f'(t) = 0 f(t)$ and $f(t) t^n$ vanishes at $t=a$ and $t=b$ then
\[
F(n) = n F(n-1).
\]

The first assumption is satisfied by $f(t) = e^{-t}$ and the second by
choosing $a=0$ and $b=\infty$.  So,
\[
F(n) = \int_0^{\infty} e^{-t} t^n \, dt
\]
satisfies $F(n) = n F(n-1)$.  A simple computation shows $F(0) = F(1)
= 1$, so $F(n) = n!$ for integer values of $n$.  Thus the simplest
definition of the $\Gamma$ function is
\begin{equation} \label{GammaIntegralDef:Eq}
\Gamma(z) = \int_0^{\infty} e^{-t} t^{z-1} \, dt.
\end{equation}

There are two other definitions of the $\Gamma$ function that are
often useful for investigating its properties.  The first is due to
{\Weierstrass}
\cite{Weierstrass56} and
takes the form:
\begin{equation} \label{GammaWeierDef:Eq}
\frac{1}{\Gamma(z)} = z e^{\gamma z} \prod_{n=1}^{\infty} \left\{\left(1+
\frac{z}{n}\right)e^{-z/n}\right\},
\end{equation}
where $\gamma$ is {\em Euler's} or {\em Mascheroni's} constant.
\[
\begin{aligned}
\gamma & = \displaystyle \lim_{m\rightarrow\infty} \left\{1 + \frac{1}{2} +
\frac{1}{3} + \cdots + \frac{1}{m} - \log m\right\}, \\
 & = 0.577215664901532860606512.
\end{aligned}
\]
This definition is particularly useful since th product in
\eqnref{GammaWeierDef:Eq} is convergent for all values of $z$.  This
is not the case for the integral in \eqnref{GammaIntegralDef:Eq} which
is only convergent for values of $z$ with positive real parts.

Recognizing that the integral in \eqnref{GammaIntegralDef:Eq} is not
convergent for the entire complex plane, {\Euler} developed the
following infinite product representation of the $\Gamma$
function:\Marginpar{Published in a letter to Goldbach in
\cite{Fuss43}.} 
\begin{equation} \label{GammaEulerDef:Eq}
\Gamma(z) = \frac{1}{z} \prod_{n=1}^{\infty} \left\{ \left( 1 +
\frac{1}{n}\right)^z \left(1 +  \frac{z}{n}\right)^{-1}\right\}.
\end{equation}

From this definition, it is also easy to show that $\Gamma(z)$
satisfies the difference equation
\[
\Gamma(z+1) = z \Gamma(z).
\]
Computing the ratio of $\Gamma(z+1)$ and $\Gamma(z)$ we have
\[
\begin{aligned}
\displaystyle \frac{\Gamma(z+1)}{\Gamma(z)} &
\displaystyle = \frac{z}{z+1} \prod_{m=1}^{\infty}
\left(1 + \frac{1}{n}\right) \left(1 + \frac{z}{n}\right)
\left(1 + \frac{z+1}{n}\right)^{-1}, \\
& \displaystyle = 
\prod_{m=1}^{\infty} \frac{n+1}{n} \frac{n+z}{n+z+1}.
\end{aligned}
\]
Reording the terms slightly, we have
\[
\frac{\Gamma(z+1)}{\Gamma(z)} = 
\left( \frac{z}{z+1} \frac{z+1}{z+2} \frac{z+2}{z+3} \cdots\right)
\cdot
\left(\frac{2}{1} \frac{3}{2} \frac{4}{3} \cdots \right) = z.
\]

The relationship between the two product formulas for $\Gamma(z)$ can
be revealed by looking at the formulas for $\log \Gamma(z)$.  Taking
the logarithm of {\Weierstrass}'s formula \eqnref{GammaWeierDef:Eq}
gives
\[
\log \Gamma(z) = 
- \log z - \gamma z + \sum_{n=1}^{\infty} \frac{z}{n} - \log \left(1 + \frac{z}{n}\right).
\]
Taking the logarithm of Euler's formula \eqnref{GammaEulerDef:Eq}
gives
\[
\log \Gamma(z) = - \log z + \sum_{n=1}^{\infty} z \log \left(1+
\frac{1}{n}\right) - \log \left( 1 + \frac{n}{z}\right).
\]
Taking the second derivative of either formula yields
\[
\frac{d^2}{dz^2} \log \Gamma(z) = \sum+{n=1}^{\infty}
\frac{1}{(n+z)^2}.
\]

Using asymptotic analysis techniques it can be shown that
\begin{equation} \label{LogGammaStirling:Eq}
\log \Gamma(z) =
\left(z - \frac{1}{2}\right) \log z - z + \frac{1}{2}\log 2\pi
+ \sum+{n=1}^{\infty} \frac{(-1)^{r-1} B_r}{2r(2r-1) z^{2r-1}},
\end{equation}
where $B_r$ are the Bernoulli numbers.  Exponentiating this expression
and computing the first few values we have:
\begin{equation} \label{GammaStirling:Eq}
\Gamma(z) =
e^{-z} z^{z-1/2} \sqrt{2\pi} \left(1 + \frac{1}{12 z} + \frac{1}{288
z^2}
- \frac{139}{51840 z^3} - \frac{571}{2488320 z^4} + \cdots \right).
\end{equation}

When performing asymptotic analysis it is sometimes more useful to use
factorials rather than $\Gamma$ function.  In this case, we have $z! =
\Gamma(z+1) = z \Gamma(z)$ so
\begin{equation} \label{FactStirling:Eq}
z!
   = e^{-z} z^{z+1/2} \sqrt{2\pi} \left(1 + \frac{1}{12 z} + \frac{1}{288
z^2}
- \frac{139}{51840 z^3} - \frac{571}{2488320 z^4} + \cdots \right).
\end{equation}
and 
\begin{equation} \label{logFactStirling:Eq}
\log z! =
\left(z + \frac{1}{2}\right) \log z - z + \frac{1}{2}\log 2\pi
+ \sum+{n=1}^{\infty} \frac{(-1)^{r-1} B_r}{2r(2r-1) z^{2r-1}},
\end{equation}

\medskip
One quantity that is worth estimating in this fashion is the binomial
coefficients.  On the one hand we know that 
\[
\binom{n}{0} + \binom{n}{1} + \cdots + \binom{n}{n} = 2^n,
\]
so we might expet the binomial coefficients to be smaller than $2^n$,
but since there are only $n+1$ terms in the sum we must have
\[
\binom{n}{i} < \frac{2^n}{n+1}.
\]
The largest of the binomial coefficient is the middle one, so it is
most interesting to bound 
\[
\binom{n}{n/2} = \frac{n!}{(n/2)!^2} =
\frac{\Gamma(n+1)}{\Gamma(n/2+1)^2} = \frac{2}{n}
\frac{\Gamma(n)}{\Gamma(n/2)^2}.
\]
¯Taking just the first term of \eqnref{GammaStirling:Eq} we have
\[
\frac{\Gamma(n)}{\Gamma(n/2)^2} = 
\frac{e^{-n} n^{n-1/2} \sqrt{\pi}}{e^{-n} (n/2)^{n-1} 2\pi} =
\frac{2^{2n-1/2} \sqrt{n}}{\sqrt{2\pi}}.
\]
Thus we have
\begin{equation} \label{BinomialStirling:Eq}
\binom{n}{n/2} \approx 2^n \sqrt{\frac{2}{\pi n}}.
\end{equation}

\section{The Riemann Zeta Function}
\label{RiemannZeta:Sec}

An essential tool in the study of arithmetic functions is the {\em Riemann
zeta function}, which is defined by 
\[
\zeta(s) = \sum_{n=1}^{\infty} \frac{1}{n^s},
\]
when the real part of $s$ is greater than $1$.  The function can then
continued analytically over the rest of the complex plane, other than
the point $s = 1$.
The source of the relationship between the $\zeta(s)$ and arithmetic
functions is remarkable product formula for the $\zeta(s)$.  Consider
the product
\[
\left(1 + \frac{1}{2^s} + \frac{1}{2^{2s}} + \cdots \right) \cdot
\left(1 + \frac{1}{3^s} + \frac{1}{3^{2s}} + \cdots \right) \cdots
=
\prod_p \left(1 + \frac{1}{p^s} + \frac{1}{p^{2s}} + \cdots \right),
\]
where the product is over the primes, $p$.  Ignoring convergence
issues, we can reorder the terms of the product to get a series of the
form:
\[
\prod_p \left(1 + \frac{1}{p^s} + \frac{1}{p^{2s}} + \cdots \right)
 = \sum_{n=1}^{\infty} \frac{a_n}{n^s}.
\]
Since rational integers have a unique representation as a product of
primes, $a_n = 1$.  Thus we have
\begin{equation}\label{ZetaProduct:Eq}
\zeta(s) = \prod_p \left(1 - p^{-s}\right)^{-1}.
\end{equation}
This remarkable formula relates a relatively normal function,
$\zeta(s)$, with a product over primes numbers.

By taking logarithms of \eqnref{ZetaProduct:Eq} we have
\[
\log \zeta(s) = - \sum_p \log (1 - p^{-s}) 
  = \sum_p \sum_{n=1}^{\infty}\frac{1}{n p^{ns}}.
\]
Rearranging this slightly we have
\[
\log \zeta(s) = \sum_p \frac{1}{p^s} + \sum_p \sum_{n=2}^{\infty}
\frac{1}{n p^{ns}}.
\]
The second term on the right hand side can be bounded as follows:
\[
\sum_p \sum_{n=2}^{\infty} \frac{1}{n p^{ns}}
 < \sum_p \sum_{n=2}^{\infty} \frac{1}{p^{ns}} = \sum_p
\frac{1}{p^s(p^s-1)} 
< \sum_{p=2}^{\infty} \frac{1}{p^{2s-\epsilon}}.
\]
This final sum is convergent and finite at least for all values of $s
\ge 1$.  However, $\zeta(s)$ is 
infinite when $s=1$, so must have that 
\[
\sum_p \frac{1}{p^s}
\]
diverges as $s$ approaches $1$.  Thus we have two results.  There are
an infinite number of primes and the sum of the reciprocals of the
prime numbers diverges.  The second statement is interesting in light
of the fact that the sum of the reciprocals of the prime twins converges.

\medskip
A series of the form
\[
F(s) = \sum_{n=1}^{\infty} \frac{f(n)}{n^s}
\]
is called a {\em Dirichlet series}.  The function $F(s)$ is called
{\em Dirichlet generating function} for the numbers $f(n)$.  If $F(s)$
and $G(s)$ are Dirichlet generating functions for the numbers $f(n)$
and $g(n)$ the product $F(s)G(s)$ has the form:
\[
F(s) G(s)
= \sum_{n=1}^{\infty}\sum_{m=1}^{\infty} \frac{f(n) g(m)}{(n \cdot m)^s}
= \sum_{n=1}^{\infty} 
    \biggl( \sum_{d \mid n} f(d) g(\frac{n}{d})\biggr) \frac{1}{n^s}.
\]

Taking the reciprocal of \eqnref{ZetaProduct:Eq}, we have
\[
\zeta(s)^{-1} = \prod_p\left(1 - p^{-s}\right) =
\sum_{n-1}^{\infty} \frac{\mu(n)}{n^s}.
\]

So,
\[
\zeta(s) \sum_{n=1}^{\infty} \frac{a_n}{n^s} = 
\sum_{n=1}^{\infty} \biggl( \sum_{d \mid n} \mu(d)\biggr) \frac{1}{n^s}.
= 1,
\]
by \propref{MobiusSum:Prop}.

Since,
\[
\zeta(s-1) = \sum_{n=1}^{\infty}\frac{n}{n^s},
\]
we have
\[
\frac{\zeta(s-1)}{\zeta(s)} = 
\sum_{n=1}^{\infty} \biggl( \sum_{d \mid n} \mu(d) \frac{n}{d}\biggr)
\frac{1}{n^s}
= \sum_{n=1}^{\infty} \frac{\phi(n)}{n^s},
\]
by \propref{TotientSum:Prop}.

Dirichlet generating functions for sums of divisors can be easily
computed also.  For instance, note that
\[
\zeta(s)^2 = \sum_{n=1}^{\infty} \biggl( \sum_{d \mid n} 1\biggr) \frac{1}{n^s}
 = \sum_{n=1}^{\infty} \frac{d(n)}{n^s}.
\]
Since,
\[
\zeta(s -r ) = \sum_{n=1}^{\infty} \frac{n^r}{n^s},
\]
we have
\[
\zeta(s) \zeta(s-r) 
 = \sum_{n=1}^{\infty} \biggl( \sum_{d \mid n} d^r\biggr) \frac{1}{n^s}
 = \sum_{n=1}^{\infty} \frac{\sigma_r(n)}{n^s}.
\]


\section{Asymptotic Behavior of Arithmetic Functions}
\label{NT:Asymp:Arith:Sec}

Arithmetic functions often occur in the analysis of various
algorithms.  For instance, in some algorithms certain loops occur
$d(n)$ times.  Thus it is useful to be able to estimate the size of
$d(n)$.  More generally, it is useful to be able to quantify the
behavior of arithmetic functions for large values of their arguments.
In this section we derive some simple results that are needed in later
sections, beginning with a study of the divisor function $d(n)$.

Clearly, $d(n) < n$.  One might expect $d(n)$ to grow like
$O(\log^c n)$ for some constant $c$, but this is not the case.  Let $n
= 2^{\ell}$ for some integer $\ell$, then
\[
d(n) = \ell + 1 = O(\frac{\log n}{\log 2}).
\]
If $n = (2\cdot 3)^{\ell}$ we have
\[
d(n) = (\ell + 1)^2 = O(\frac{\log^2 n}{\log 6}).
\]
So for any value of $c$, $d(n)$ grows faster than $O(\log^c n)$.
However, $d(n)$ also grows slower than $O(n^{\epsilon})$ for any
positive value of $\epsilon$.  This is shown by the following
proposition. 

\begin{proposition} \label{NT:Divisor:Est:Prop}
For all positive $\epsilon$
\[
d(n) = o(n^{\epsilon}).
\]
\end{proposition}

\begin{proof}
Assume $n = p_1^{e_1} p_2^{e_2} \cdots p_k^{e_k}$, then
\begin{equation}\label{Div:Est:Eqa}
\frac{d(n)}{n^{\epsilon}} = \frac{e_1 +1}{p_1^{e_1 \epsilon}} \cdot 
\frac{e_2 +1}{p_2^{e_2 \epsilon}} \cdots \frac{e_k +1}{p_k^{e_k
\epsilon}}.
\end{equation}
Next we bound each of the factors of \eqnref{Div:Est:Eqa} from above.
For sufficiently large $p_i$
\[
\frac{e_i +1}{p_i^{e_i \epsilon}} \le 1 \quad\mbox{\ie, when} \quad
e_i + 1 \le (p_i^{\epsilon})^{e_i}.
\]
In particular, if $p_i^{\epsilon} \ge 2$ then for all non-zero
integers $e_i$, $e_i + 1 \le 2^{e_i}$.  Thus each factor for which
$p_i$ is greater than $2^{1/\epsilon}$ is less than $1$.

For $p_i \le 2^{1/\epsilon}$ we estimate as follows.  First, 
\[
p_i^{e_i \epsilon} \ge 2^{e_i \epsilon} = e^{e_i \epsilon \log 2}
\ge e_i \epsilon \log 2.
\]
Applying this to the $i$-th factor gives
\begin{equation}\label{Div:Est:Eqb}
\frac{e_i +1}{p_2^{e_i \epsilon}} \le 1 + \frac{e_i}{p_i^{e_i
\epsilon}}
\le 1 + \frac{1}{\epsilon \log 2} \le \exp (\frac{1}{\epsilon \log
2}).
\end{equation}
Applying \eqnref{Div:Est:Eqb} to \eqnref{Div:Est:Eqa} gives
\begin{equation} \label{DivisorFn:Bound:Eq}
\frac{d(n)}{n^{\epsilon}} =
 \prod_{p \le 2^{1/\epsilon}} \exp \left( \frac{1}{\epsilon \log 2}\right)
 \le \exp \left( \frac{2^{1/\epsilon}}{\epsilon \log 2}\right),
\end{equation}
which is a constant independent of $n$.
\end{proof}

On the other hand, observe what happens if we let 
\[
\epsilon = \frac{(1+ \delta/2) \log 2}{\log \log n}
\]
in \eqnref{DivisorFn:Bound:Eq}.  First observe that
\[
2^{1/\epsilon} = \exp \left( \frac{1}{\epsilon} \log 2 \right)
  = \frac{\log \log n}{1 + \delta/2},
\]
so
\[
2^{1/\epsilon} = (\log n)^{1/(1+ \delta/2)}.
\]
Taking the logarithm of both sides of \eqnref{DivisorFn:Bound:Eq}
gives
\[
\log d(n) \le \frac{(\log n) (\log 2) (1 + \delta/2)}{\log \log n}
+ \frac{(\log n)^{1/(1+ \delta/2)} \log \log n}{(1+ \delta/2)(\log
2)^2}.
\]
The second term is bounded by
\[
\frac{\delta}{2}\,\, \frac{(\log 2) (\log n)}{\log \log n},
\]
so
\begin{equation} \label{DivisorUBound:Eq}
\log d(n) \le (1 + \delta) \frac{(\log 2)(\log n)}{\log \log n}.
\end{equation}

The fluctuations in arithmetic functions make it somewhat difficult to
succinctly describe their growth.  As a consequence, we use an {\em
average estimate} for their size.  For instance, despite
\propref{NT:Divisor:Est:Prop}, for most $n$, $d(n)$ is quite a bit
smaller than $O(n^{\epsilon})$. 

The source of the problem here is that our use of asymptotic analysis
focused on the value of $d(n)$ for a single value of $n$.  So instead
of analyzing the asymptotic behavior of $d(n)$ it may be preferable to
study the behavior of 
\[
\frac{d(n) + d(n-1) + d(n-2) + \cdots d(n-k)}{k+1}.
\]
This way, if the especially bad cases of $d(n)$ are isolated, their
impact on the asymptotics will be smoothed.  The approach to this that
has been the most successful is to use the concept of \keyi{average
order}.  The average order of $f(n)$ is defined to be $g(n)$ if
\[
f(1)+ f(2) +\cdots + f(n) \sim g(1)+ g(2) + \cdots + g(n).
\]

When applied to the divisor function this smoothing is remarkably
effective.  The average order of $d(n)$ is actually $\log n$, \ie,
\[
\begin{aligned}
  d(1) + d(2) +\cdots + d(n) 
   &\sim \log(1) + \log 2 +\cdots + \log n = \log n!,\\
   & \sim n \log n,
\end{aligned}
\]
by {\em Stirling's approximation}\index{Stirling's approximation}
\begin{equation} \label{Stirling:Eqc}
\log n! = n \log n - n + O(\log n).
\end{equation}
The precise result is 

\begin{proposition}
\[
d(1) + d(2) +\cdots + d(n) = n\log n + (2\gamma -1) + O(n^{\beta}),
\]
where $\gamma$ is \key{Euler's constant} and $\frac{1}{4} \le \beta \le
\frac{1}{3}$.
\end{proposition}

\medskip
Similar results exist for the other arithmetic functions: \Marginpar{Need to include the analysis for the following proposition since it is used in section 18.2.}

\begin{proposition}
\[
\mathop{\limsup}\limits_{n\rightarrow\infty} \frac{\sigma_1(n)}{n \log \log n}
   = e^{\gamma}.
\]
\end{proposition}

Detailed and accessible discussions of these results can be found in
\cite{Hardy:Wright} or, in more detail, \cite{Landau:Primzahlen}.

\medskip
We begin with a result of {\Mertens} \cite{Mertens74} on the average
order of $\phi(n)$.

\begin{proposition}[Mertens]
\label{Mertens:phi:Prop}
\[
\Phi(n) = \phi(1) + \phi(2) + \cdots + \phi(n) =
\frac{3 n^2}{\pi^2} + O(n \log n)
\]
\end{proposition}

\begin{proof}
Using the formula for $\phi(n)$ in \propref{TotientSum:Prop} we have
\[
\Phi(n) = \sum_{m=1}^{n} m \sum_{d\mid m}\frac{\mu(d)}{d} =
\sum_{d d' \le n} d' \mu(d).
\]
For each $d \le n$ in the last sum, the applicable $d'$ lie
between $1$ and $\lfloor n/d\rfloor$.  Thus we can rewrite the
last sum as
\[
\Phi(n) = \sum_{d=1}^{n} \mu(d)
\left(\sum_{d'=1}^{\lfloor\frac{n}{d}\rfloor} d'\right) =
\frac{1}{2}\sum_{d=1}^{n} \mu(d)
\left(\left\lfloor\frac{n}{d}\right\rfloor^2 +
    \left\lfloor\frac{n}{d}\right\rfloor\right).
\]
The deviation of $\lfloor x\rfloor^2$ from $x^2$ is $O(x)$ so we now have
\[
\begin{aligned}
  \Phi(n) &= \frac{n^2}{2}\sum_{d=1}^{n}\frac{\mu(d)}{d^2} + 
    \frac{1}{2} \sum_{d=1}^{n} \mu(d) O(\frac{n}{d}) \\
  & = \frac{n^2}{2}\sum_{d=1}^{\infty}\frac{\mu(d)}{d^2} +
    O\left(n^2 \sum_{d=n+1}^{\infty} \frac{1}{d}\right) +
    O\left(n \sum_{d=1}^{n} \frac{1}{d}\right).
\end{aligned}
\]
The first sum is just $\zeta(2)^{-1} = 6/\pi^2$.  The second and
third sums can be approximated by the replacing summations by
integrals.  This gives $O(n^{-1})$ for the second sum and $O(\log n)$
for the third.\Marginpar{This needs to be improved}  Putting these results together gives
\[
\Phi(n) = \frac{3 n^2}{\pi^2} + O(n) + O(n \log n)
\]
\end{proof}

Since the sum $1 + 2 + \cdots + n \approx \frac{1}{2} n^2$ we have the
following corollary.

\begin{proposition}\label{Totient:Average:Prop}
The average order of $\phi(n)$ is 
\[
\phi(n) \approx \frac{6n}{\pi^2}.
\]
\end{proposition}

We can use this result to determine the likelihood that $2$ randomly
chosen integers are relatively prime.  

Now choose a positive integer $1 \le r \le n$.  Of the integers less than
$r$, $\phi(r)$ will be relatively prime.  Summing over all $r$ less
than $n$, we see that there are 
\[
\phi(n) + \phi(n-1) + \cdots + \phi(1) = \Phi(n)
\]
pairs of integers $n \ge r > s \ge 1$ that are relatively prime.  Since
there are $n(n+1)/2$ such pairs, the probability that a randomly chosen
pair of distinct integers less than or equal to $n$ will be relatively
prime is
\[
\Phi(n) \cdot \frac{2}{n(n+1)} 
  = \frac{3 n^2}{\pi^2} \cdot \frac{2}{n^2+n} 
      + O\left(\frac{\log n}{n+1}\right)
  = \frac{6}{\pi^2}+ O\left(\frac{\log n}{n}\right),
\]
by \propref{Mertens:phi:Prop}.
This proves the following proposition
\begin{proposition}
\label{RelPrime:Pairs:Prop}
Let $r$ and $s$ be two distinct uniformly chosen integers in the
interval $[1, n]$.  The probability that $r$ and $s$ will be
relatively prime is
\[
\frac{6}{\pi^2}+ O\left(\frac{\log n}{n}\right).
\]
\end{proposition}

\section{Distribution of Primes}
\label{Prime:Dist:Sec}

Define $\pi(x)$ to be the number of prime numbers less than or equal
to $x$.  The
rate at which $\pi(x)$ grows indicates the density of primes in $\Z$.
In this section we show that 
\begin{equation} \label{Prime:Asymp:Eq}
\pi(x) = O\left(\frac{x}{\log x}\right).
\end{equation}
In fact, a much stronger result is known, the famous \keyi{prime number
theorem} of {\Hadamard} and {\ValleePoussin},\Marginpar{Need to
include references here. Hardy and Wright? Newman?}
\[
\pi(x) \sim \frac{x}{\log x}.
\]

The easiest approach to get an estimate like \eqnref{Prime:Asymp:Eq}
of the distribution of primes is due to {\Chebyshev}.  It is based on
the first identity of \propref{Lambda:Fn:Prop}:
\[
\log n  = \sum_{d \mid n} \Lambda (d).
\]
Summing both sides for $n \le x$, gives
\[
T(x) = \sum_{1\le n \le x}\log n = \log \lfloor x \rfloor !
= \sum_{1\le n \le x}\sum_{d\mid n} \Lambda(d)
= \sum_{d \le n}\Lambda(d) \left\lfloor\frac{x}{d}\right\rfloor.
\]

Stirling's approximation \eqnref{Stirling:Eqc} gives
\begin{equation} \label{TStirlingApprox:Eq}
T(x) = \log \lfloor
x\rfloor! = x \log x - x + O(\log x) \le x \log x.
\end{equation}
By choosing appropriate combinations of $T(x)$ we can develop upper
and lower bounds on $\pi(x)$.  {\Chebyshev}'s paper used
\[
T(x) - T(\frac{x}{2}) - T(\frac{x}{3}) - T(\frac{x}{5})
 + T(\frac{x}{30})
\]
which leads to the bounds:
\[
0.92 \, \frac{x}{\log x} < \pi(x) < 1.105 \, \frac{x}{\log x}.
\]
A detailed version of this proof is given in {\LandauE}
\cite{Landau:Primzahlen}.  

Following {\DavenportH} \cite{Davenport:Multiplicative}, we will use the
simpler expression $T(x) - 2 T(\frac{x}{2}) \approx x \log 2$:
\[
T(x) - 2 T(\frac{x}{2}) = 
  \sum_{d\le x} \Lambda(d) \left( \left\lfloor \frac{x}{d}\right\rfloor 
     - 2 \left\lfloor \frac{x}{2d}\right\rfloor \right).
\]
The parenthesized expression in the sum on the right varies between $0$ and $1$
so
\[
T(x) - 2 T(\frac{x}{2}) \le \sum_{d\le x} \Lambda(x)
= \sum_{p \le x} \left\lfloor \frac{\log x}{\log p}\right\rfloor \, \log p,
\]
where the second sum is over all primes less than $x$.  Replacing the
integer part of the fraction by the fraction itself, we can bound the
sum on the right as
\[
\sum_{p \le x} \left\lfloor \frac{\log x}{\log p}\right\rfloor \, \log p
\le \sum_{p \le x} \log x = \pi(x) \log x.
\]
Using  the approximation \eqnref{TStirlingApprox:Eq} for $T(x) - 2
T(\frac{x}{2})$ we have 
\[
x \log 2 \le \pi(x) \log x.
\]
To get an upper bound on $\pi(x)$ we note that 
\[
\left\lfloor \frac{x}{d}\right\rfloor - \left\lfloor \frac{x}{2d}\right\rfloor 
\]
is always $1$ for $d$ between $x/2$ and $x$.  Ignoring the terms that arise
from $d < x/2$ gives
\[
T(x) - 2 T(\frac{x}{2}) \ge \sum_{x/2 \le d \le x} \Lambda(d) 
 \ge \sum_{x/2 \le p \le x} \log p \ge \log(x/2) \left( \pi(x) - \pi(x/2)
\right), 
\]
which gives
\[
\begin{aligned}
\pi(x) - \pi(x/2) & \le \log 2 \, \frac{x}{\log x/2}
\frac{x \log 2}{\log x} \left[ 1 - \frac{\log 2}{\log x}\right]^{-1},\\
& \le \log 2 \frac{x}{\log x}.
\end{aligned}
\]

Telescoping this result we have
\[
\begin{aligned}
\left[\pi(x) - \pi(x/2)\right] &+ 
\left[\pi(x/2) - \pi(x/4)\right] + \cdots \\
& \displaystyle \le \log 2 \frac{x}{\log x} + 
\log2 \frac{x/2}{\log x/2} + \log2 \frac{x/4}{\log x/4} + \cdots, \\
& \le \log 2 \frac{x}{\log x} 
     \left[ 1 + \frac{1}{2} + \frac{1}{4} + \cdots\right], \\
& \le 2 \log 2 \frac{x}{\log x}.
\end{aligned}
\]
So,
\[
\log 2 \frac{x}{\log x} \le \pi(x) \le 2 \log 2 \frac{x}{\log x}.
\]

{\em von Mangoldt's formula and the Riemann hypothesis should be
inserted here also.}

\section{Bertrand's Postulate}
\label{Bertrands:Sec}
\index{Bertrand's postulate|(}

Another type of result we will need is an estimate the size of the
gaps between consecutive prime numbers.  We prove that
there is always a prime between $x$ and $2x$, although much more is
conjectured to be true.

Bertrand's postulate is that there is always a prime number between
$x$ and $2x$.  Bertrand assumed this result was true and verified it
for $x < 3000000$, but is was first proven by {\Chebyshev}.  The
approach used here is due to {\Erdos} \cite{Erdos32}.

A function that we will find useful is
\[
\vartheta(x) = \sum_{p \le x} \log p.
\]
The first step is to bound the size of $\vartheta(x)$.

\begin{proposition} \label{Theta:Bound:Prop}
\[
\vartheta(n) \le 2 n \log 2 \approx 1.386 n
\]
\end{proposition}

\begin{proof}
\[
M = \binom{2m+1}{m} = \frac{(2m+1) \cdots (m+2)}{m!}
\]
is an integer.  Since $M$ occurs twice in the expansion
\[
2^{2m+1} = (1+1)^{2m+1} = \binom{2m + 1}{0} + \binom{2m + 1}{1} +  \cdots
+ \binom{2m + 1}{2m+1}
\]
$M < 2^{2m}$.  Every prime $m+1 < p \le 2m+1$ divides the numerator of
$M$ but not the denominator, so every such $p$ must divide $M$.  Thus
the logarithm of the product of the primes between $m+1$ and $2m+1$
can be bounded as follows:
\begin{equation}\label{VarTheta:Induct:Eq}
\sum_{m+1< p \le 2m+1} p = \vartheta(2m+1) - \vartheta(m+1) \le \log M
< 2m \log 2.
\end{equation}

We now proceed by induction.  The proposition is true for $n = 1, 2$.
Assume the proposition is true for all $n < n_0$.  We need to show
that it is satisfied for $\vartheta(n_0)$.  If $n_0$ is even then
$\vartheta(n_0) = \vartheta(n_0-1)$ since $n_0$ is not a prime.  Thus
\[
\vartheta(n_0) = \vartheta(n_0-1) \le 2 (n_0-1) \log 2 < 2 n_0 \log 2.
\]

If $n_0$ is odd, then we can use \eqnref{VarTheta:Induct:Eq}.  Assume
$n_0 = 2m+1$,
\[
\begin{aligned}
\vartheta(n_0) &= \vartheta(2m+1) - \vartheta(m+1) + \vartheta(m+1), \\
  &\le 2m \log 2 + 2(m+1) \log 2, \\
  &\le 2 n_0 \log 2.
\end{aligned}
\]
\end{proof}

Let $p$ be a divisor of $n!$ and $k_p$ be the number of times $p$
divides $n!$, \ie, $p^{k_p} \mid n!$ but $p^{k_p +1} \nmid n!$.  So,
\[
n! = \prod_{p \mid  n!} p^{k_p},
\]
where $p$ ranges over primes.
Let ${\cal N}$ be the set $\{1, 2, 3, \ldots, n\}$.  $p^k$ divides
$\lfloor n/p^k \rfloor$ of the elements of ${\cal N}$ so
\begin{equation}\label{Factor:Div:Eq}
k_p = \sum_{1 \le k} \left\lfloor \frac{n}{p^k} \right\rfloor.
\end{equation}
The sum above is finite.

\begin{proposition}[Bertrand's Postulate]\label{Bertrands:Post:Prop}
If $n \ge 1$ then there exists at least one prime $p$ such that $n < p
\le 2n$, or equivalently, $p_{r+1} < 2 p_r$ for every $r$.
\end{proposition}

\begin{proof}
We prove this for $n > 468$ and then empirically check the smaller
values of $n$.  Assume the proposition is false, for some $n$, \ie,
there is no prime number between $n$ and $2n$.  Define $N =
(2n)!/n!^2$ and let $p$ be a prime divisor of $N$ of order $k_p > 0$.
By the hypothesis, $p \le n$.  If $p$ is large, \ie, $\frac{2}{3}n < p \le
n$, then 
\[
p^2 > \frac{4 n^2} 9 > 2n
\]
since $n > 469$.  Thus
\[
k_p = \sum_{1 \le m} \left( \left\lfloor \frac{2n}{p^m}\right\rfloor
 - 2 \left\lfloor \frac{n}{p^m}\right\rfloor \right)
 = \left\lfloor \frac{2n}{p}\right\rfloor
    - 2 \left\lfloor \frac{n}{p}\right\rfloor = 2 - 2 = 0.
\]
So, $p \le \frac{2}{3} n$ for every prime factor of $N$.

Now,
\[
\sum_{p\mid n} \log p \le \sum_{p \le \frac{2}{3}n} \log p =
\vartheta(\frac{2}{3} n) \le \frac{4}{3} n \log 2.
\]
By \eqnref{Factor:Div:Eq}, $k_p \le \lfloor (\log 2n)/\log p \rfloor$
so\Marginpar{Why is $k_p \ge 2$ and not $1$ here?}
\[
2\log p \le k_p \log p \le \log 2n.
\]
So, $p \le \sqrt{2n}$. Hence,
\[
\sum_{2 \le k_p} k_p \log p \le \sqrt{2n} \log 2n.
\]
Apply this result to $\log N$ we have
\[
\begin{aligned}
\log N &\le \sum_{k_p = 1} \log p + \sum_{k_p \ge 2} k_p \log p 
  \le \sum_{p\mid N} \log p + \sqrt{2n} \log 2n, \\
  &\le \frac{4}{3} n \log 2 + \sqrt{2n} \log 2n.
\end{aligned}
\]
$N$ is the largest term in
\[
\begin{aligned}
2^{2n} & = (1 + 1)^{2n} = \binom{2n}{0} + \binom{2n}{1} + \cdots + 
  \binom{2n}{2n}, \\
 \le 2n N.
\end{aligned}
\]
Taking logarithms,
\[
2n \log 2 \le \log 2n + \log N \le \frac{4}{3} n \log 2 + (1 +
\sqrt{2n})\log 2n,
\]
or
\[
\frac{2}{3} n \log 2 \le (1 + \sqrt{2n}) \log 2n.
\]
Both the left and right hand sides of this equation are monotonically
increasing functions.  Since the left hand side increases faster than
the right, for sufficiently large $n$ we have a contradiction.  By
numerical calculation, we see that for $n = 468$ the left hand side is
equal to $216.26$ and the right hand side is $216.15$.  So the
proposition is true for all $n > 468$.  The sequence of primes
\[
2, 3, 5, 7, 13, 23, 43, 83, 163, 317, 613
\]
shows the proposition is true for $n < 613$.
\end{proof}

\index{Bertrand's postulate|)}

\section*{Notes}

\small

\notesectref{NT:Asymp:Arith:Sec}
The estimates on the size of the divisor function, $d(n)$ in
\Marginpar{This needs to be checked.}
\eqnref{DivisorUBound:Eq} are due to \cite{Wigert07},
\cite[pp. 1--9]{Landau:Primzahlen}, \cite[pp. 85--86]{Ramanujuan27}.


\normalsize


