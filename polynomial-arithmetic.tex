% $Id: polyarith.tex,v 1.1 1992/05/10 19:43:45 rz Exp rz $
\chapter{Polynomial Arithmetic}
\label{Poly:Arith:Chap}

For a number of good reasons, the core of most algebraic manipulation
systems is a polynomial arithmetic package.  First, a large number of
problems in pure and applied mathematics can be expressed as problems
solely involving polynomials.  Second, polynomials provide a natural
foundation on which to build more complex structures like rational
functions, algebraic functions, power series and rings of
transcendental functions.  And third, the algorithms for polynomial
arithmetic are well understood, efficient and relatively easy to
implement.

This chapter discusses the classical arithmetic algorithms for
polynomials and analyzes their behavior.
\sectref{Poly:Generalities:Sec} defines the basic notation and 
terminology used to discuss algorithms for polynomials.  We introduce
the algorithmic notation with polynomial addition in
\sectref{Poly:Add:Sec}.  The classical algorithms for multiplication
of polynomials is discussed in some detail in \sectref{Poly:Mult:Sec}
and asymptotically faster algorithms are discussed in
\sectref{Poly:Fast:Sec}.  Exponentiation of polynomials is dealt with
in \sectref{Poly:Expt:Sec}.  Finally, the problem of replacing
variables of a polynomial by values is discussed in \sectref{Poly:Subs:Sec}.

\section{Generalities}
\label{Poly:Generalities:Sec}

Assume $R$ is a ring and $a_{0}, \ldots, a_{d}$ are elements of $R$.
The polynomial
\[
P(X) = a_{0}X^{d}+ a_{1}X^{d-1} + \cdots + a_{d}
\]
is a {\em univariate} polynomial in $X$.\index{polynomial! univariate} The
$a_{i}$ are the {\em coefficients} of $P(X)$ and the expressions $a_{i}
X^{i}$ are the {\em monomials} or {\em terms} of $P(X)$.\index{polynomial!
monomials}\index{polynomial! terms} $R$ is called the {\em coefficient
domain} of $P(X)$.\index{coefficient domain!of a polynomial} The set of
all polynomials in $X$ with coefficients in $R$ is denoted by $R[X]$.
$R[X]$ is a ring. If, as is usually the case, $R$ is a commutative
ring with unit, then so is $R[X]$.
\addsymbol{$R[X]$}{Ring of polynomials in $X$ with coefficients in $R$}

It is necessary to distinguish symbolic literals (such as $X$, $Y$ and
$Z$ in $R[X,Y,Z]$) from unknown quantities that enter into the
computation.  Symbolic literals are only represented by capital
letters.  Lower case letters are used to represent elements of the
coefficient ring.

If $a_{0}$ is non-zero then the {\em degree} of $P(X)$ is $d$, which
we denote by $\deg P(X)$.\index{degree!of a polynomial} The \keyi{leading coefficient} of $P(X)$ is denoted by
$\lc (P(X)) = a_{0}$.\index{polynomial! leading coefficient} If the
leading coefficient of $P(X)$ is $1$ then $P(X)$ is said to be {\em
monic}.\index{polynomial! monic} The \keyi{leading term} of $P(X)$ is
$\lc(P) \cdot X^{\deg P}$.\index{polynomial!leading term}
\addsymbol{$\deg_X P$}{Maximal degree $X$ in the polynomial $P$}
\addsymbol{$\lc P$}{Leading coefficient of the polynomial $P$}

These terms can be extended to multivariate polynomials by considering
them to be univariate polynomials in one variable, with coefficients
that are polynomials in the remaining variables.\index{polynomial!
multivariate} We denote these values by $\deg_{X_{i}} P$ and
$\lc_{X_{i}} P$.

Let $P$ be a multivariate polynomial in the variables $X_{1}, \ldots,
X_{n}$ with coefficients in $R$
\[
P(X_{1}, \ldots, X_{n}) = 
a_{t} X_{1}^{e_{1t}} X_{2}^{e_{2t}} \cdots X_{n}^{e_{nt}}
+
\cdots
+
a_{0} X_{1}^{e_{10}} X_{2}^{e_{20}} \cdots X_{n}^{e_{n0}}.
\]
$P(X_{1}, \ldots, X_{n})$ is an element of the ring $R[X_{1}, \ldots,
X_{n}]$.  The monomials of $P$ are the expressions $a_{i}
X_{1}^{e_{1i}} X_{2}^{e_{2i}} \cdots X_{n}^{e_{ni}}$.  The total
degree of such a monomial is $e_{1i} + \cdots + e_{ni}$.  The {\em
total degree} of $P$ is the maximum of the total degrees of its
monomials.\index{degree! total, of a polynomial}

To minimize the number of subscripts in formulas involving
multivariate polynomials, we often use a ``vectorized subscript''
notation.  Let $\vec X = (X_1, X_2, \ldots, X_n)$ and $\vec e = (e_1,
e_2, \ldots, e_n)$ be two vectors.  Then we write the usual 
dot product as
\[
\vec e \cdot \vec X = e_1 X_1 + e_2 X_2 + \cdots + e_n X_n.
\]
We extend this notation to exponentiation as follows 
\[
X^{\vec e} = (X^{e_1}, X^{e_2},\ldots, X^{e_n})
\qquad\hbox{and}\qquad 
{\vec X}^{\vec e} = X_1^{e_1} X_2^{e_2} \cdots X_n^{e_n}.
\]
Thus the multivariate polynomial
\[
a_1 X_1^{e_{11}} X_2^{e_{12}} \cdots X_n^{e_{1n}}+ a_2 X_1^{e_{21}} X_2^{e_{22}} \cdots X_n^{e_{2n}} + 
\cdots + a_t X_1^{e_{t1}} X_2^{e_{t2}} \cdots X_n^{e_{tn}}
\]
would be written
\[
a_1 \vec X^{\vec e_1} + a_2 \vec X^{\vec e_2} 
  + \cdots + a_t \vec X^{\vec e_t}.
\]
In addition, we write $P(\vec X) \in A[\vec X]$.  The vector accent is
always given when using this notation.

Multivariate polynomials can be represented in a wide variety of
different manners, each appropriate for different classes of problems.
In this chapter, we discuss three different decision points.  The
first is whether the polynomial uses an expanded representation or a
recursive representation.  These two options are illustrated by the
two polynomials:
\[
\begin{aligned}
P_{1} &= X^{2} Y^{3} + X^{2} Z + YZ^{3} + YZ^{2} +Z \\
P_{2} & = X^{2}(Y^{3} + Z) + Y(Z^{3} + Z^{2}) + Z
\end{aligned}
\]
$P_{1}$ is presented using an {\em expanded} representation and can be
viewed as an element of $\Z[X,Y,Z]$.  Expanded representations of
polynomials can be viewed as a set of exponent vector/coefficient
pairs, where each exponent vector has a component for each variable.
This representation is used when we want to treat the variables
equally, without giving preference to any particular one.
\index{representation! expanded} \index{representation! recursive}

$P_{2}$ uses a {\em recursive} representation.  It can be viewed as a
univariate polynomial in $X$ whose coefficients are themselves
polynomials.  Thus, it can be thought of as an element of
$((\Z[Z])[Y])[X]$, where we have added the parentheses for precision.

\index{representation! variable dense}
The second decision point is whether or not the zeroth power of
variables is indicated in the representation.  If the zeroth, and thus
all powers of each variable, is indicated we call the representation
{\em dense\/}.  Otherwise, the representation is called {\em
variable sparse\/}.  $P_{3}$ and $P_{4}$ below are variable dense
versions of $P_{1}$ and $P_{2}$ respectively.  We call $P_{1}$ and
$P_{2}$ {\em variable sparse} representations.
\[
\begin{aligned}
P_{3} &= X^{2} Y^{3}Z^{0} + X^{2} Y^{0} Z + X^{0} Y Z^{3} + X^{0} Y
Z^{2} +X^{0} Y^{0} Z \\
P_{4} & = ((Z^{0}) Y^{3} + Z Y^{0}) X^{2} 
    + ((Z^{3} + Z^{2}) Y + Z Y^{0}) X^{0}
\end{aligned}
\]
Variable sparse representations are most often used when there are a
large number of variables in the computation and/or the number of
variables can change during the computation itself.  The advantage of
being variable dense is that there is no need to include code to
decide whether the variables of a polynomials match up in an
algorithm.  Variable sparse representations are not often used in
conjunction with an expanded representation because of the cost of
introducing variable identifiers into the exponent vectors.
Recursive, variable sparse representations have the problem that one
does not know, {\em a priori\/}, the domain of each of the
coefficients.  An indicator must be present at run time and a check
must be introduced into the code.

The final decision point is whether to use a {\em degree dense}
representation or not.\index{polynomial! degree dense representation} In a
degree dense representation, monomials are included even if their
coefficients are zero.  Thus $P_{4}$ would look like
\begin{multline}
   X^{2}  (Y^{3} + 0 \cdot Y^{2} + 0 \cdot Y^{1} + Z Y^{0}) 
      + 0 \cdot X^{1}\\
  \hfill{}+ \left((Z^{3} + Z^{2} + 0 \cdot Z^{1} + 0 \cdot Z^{0}) Y
    + (Z^{1} + 0 \cdot Z^{0}) Y^{0}\right) X^{0}
\end{multline}
in a degree dense representation.  Because there is a coefficient for
every exponent in a degree dense representation, the exponents need
not be stored in the polynomial.  Instead the coefficients can be
arranged in a vector.  This gives the degree dense representation two
potential advantages over the degree sparse representation.  First,
when most of the coefficients are non-zero, the degree sparse
representation can require substantially more storage than the degree
dense representation.  Second, it may require many exponent
comparisons to access (or modify) a particular coefficient in the
degree sparse representation, while the degree dense representation
can access the desired coefficient by an indexing operation. \index{dense 
polynomial}\index{sparse polynomial} 

However, in many real world computation a large number of variables
are involved, and in this case the polynomials are usually quite
sparse.  In this case, the degree dense representations waste a large
amount of storage representing the terms whose coefficients are
zero.  Consequently, degree dense representations are rarely used for
general multivariate problems.  However, they are often appropriate
for univariate problems where the efficiencies inherent in using
vectors of coefficients to represent a polynomial outweigh the cost in
space of the representation.

By way of example, \Macsyma's primary fast polynomial representation
is recursive, variable sparse and degree sparse.  The use of a
variable and degree sparse representation ensures that large storage
costs are not incurred when polynomials with large numbers of
variables or of large degree are used.  In addition a special degree
dense representation is used for certain univariate polynomial
operations.

\index{variable! main} \index{variable! ordering}
We call the outermost variable of a recursive representation of a
polynomial its {\em main variable\/}.  The sequence of main variables
in a polynomial (descending towards the coefficient domains), is called
the {\em variable ordering\/}.  The degree of $P$ with respect to the
variable $X_{i}$ is the degree of $P$ when viewed as an element of
$R[X_1, \ldots,X_{i-1}, X_{i+1}, X_n][X_i]$, which we denote by
$\deg_{X_{i}} P$.

\index{polynomials! sparse}\index{polynomials! dense}
When analyzing algorithms for polynomials it is useful to know the
maximum number of terms in a polynomial with different degree
characteristics.  The simplest case is univariate polynomials, where a
polynomial of degree $d$ can have as many as $d+1$ terms.  For
multivariate polynomials the problem is slightly more complex.  If the
highest degree to which the variable $x_i$ appears is $d_i$ then the
polynomial can have as many as
\[
(d_1 + 1) (d_2 + 1) \cdots (d_n + 1)
\]
terms or $(d+1)^n$ if the degree of each variable is bounded by $d$.

Occasionally, we instead have a bound on the total degree of each
term.  Here the number of terms is not so apparent.  Let $S(n, d)$ be
the maximum number of terms in a polynomial in $n$ variables of total
degree $d$.  When $n$ is equal to $1$ the polynomial can have only
$1$ term, $x_1^d$.  For $n=2$, we have $S(2, d)= d+1$.  For higher
values of $n$ we can proceed as follows.  Using the recursive
representation, observe that the coefficient of $x_n^i$ is a
polynomial of total degree $d-i$, and thus has no more than $S(n-1,
d-i)$ terms.  Thus,
\[
S(n, d) = \sum_{i=0}^d S(n-1, d-i) = \sum_{i=0}^d S(n-1, i),
\]
where the second sum just reverse the terms of the first.  Thus,
\[
\begin{aligned}
S(3, d) & \displaystyle = \sum_{i=0}^d S(2, i) = \sum_{i=0}^d i+1 = \frac{d(d+1)}{2} +
d + 1, \\
 &\displaystyle = \frac{(d+2)(d+3)}{2}.
\end{aligned}
\]
A similar computation produces
\[
S(4, n) = \frac{(d+2) (d+3) (d+4)}{6}.
\]
More generally we might conjecture that
\[
S(n, d) = \binom{d + n - 1}{n}.
\]
To prove this we need to show that 
\[
\sum_{i=0}^d \binom{i + n - 1}{n - 1} = \binom{n + d}{n}.
\]
Instead we prove a more general identity that is needed later.

\begin{proposition}\label{CombinSum:Prop}
If $n$ and $k$ are non-negative integers then 
\[
\sum_{i=0}^k \binom{ n + i}{n} = \binom{n + k + 1}{n + 1}.
\]
\end{proposition}

\begin{proof}
This identity is easily shown using induction.  For $k = 0$ the
proposition is trivially true.  Assuming the proposition is true for
$k = \ell$, we proceed as follows:
\[
\begin{aligned}
\displaystyle \sum_{i=0}^{\ell+1} \binom{n + i}{n} & = 
\displaystyle \sum_{i=0}^{\ell} \binom{n + i}{n} 
    + \binom{n + \ell + 1}{n}, \\
 & \displaystyle = \binom{n + \ell + 1}{n + 1} + \binom{n + k + 1}{n}
= \binom{n + k + 2}{n + 1}
\end{aligned}
\]
\end{proof}

With some set of degree bounds, there is a maximum number of non-zero
terms which a polynomial might have.   
If nearly all of the terms have non-zero coefficients then the polynomial
is said to be a {\em dense polynomial\/}.  If nearly all of the
coefficients are zero then $P$ is a {\em sparse polynomial\/}.  These
terms are not precise but are meant to characterize a qualitative
difference in the behavior of algorithms using polynomials of these
two different types.  

\section{Polynomial Addition}
\label{Poly:Add:Sec}
\newcommand{\cons}{\mathop{\tt cons}\nolimits}
\newcommand{\first}{\mathop{\tt first}\nolimits}
\newcommand{\rest}{\mathop{\tt rest}\nolimits}
\newcommand{\emptyp}{\mathop{\tt empty?}\nolimits}
\newcommand{\coefp}{\mathop{\tt coef?}\nolimits}
%\newcommand{\terms}{\mathop{\tt terms}\nolimits}
\newcommand{\var}{\mathop{\tt var}\nolimits}


The basic algorithms for polynomial arithmetic are relatively simple.
However, the details that arise when using different data structures
can become quite involved.  In this chapter the implementation of
polynomial addition is described in some detail, but later chapters
are increasingly more abstract and leave more of the details to the
reader.

Through this section we use a recursive, variable sparse, degree
sparse representation for polynomials.  Consider the univariate
polynomial
\[
F(X) = X^7 + 3X^5 - 13X^2 +3
\]
We represent terms of this polynomial as a list of
exponent/coefficient pairs, headed by the variable, \ie, 
\[
F(X) \approx \mbox{\tt ($X$ (7 1) (5 3) (2 -13) (0 3))} = \mbox{\tt F}.
\]
Notice that the term list is sorted by the exponents of each term.
The routines $\var$ and $\terms$ are used to separate the variable of
the polynomial from the term list.  So,
\[
\begin{aligned}
\var(\mbox{\tt F})& \Rightarrow X, \\
\terms(\mbox{\tt F}) & \Rightarrow \mbox{\tt ((7 1) (5 3) (2 -13) (0 3))}.
\end{aligned}
\]

We use the routines $\first$ and $\rest$ to take apart lists, \viz,
\[
\begin{aligned}
\first(\terms(\mbox{\tt F}))& \Rightarrow \mbox{\tt (7 1)},\\
\first(\rest(\terms(\mbox{\tt F})))& \Rightarrow \mbox{\tt (5 3)}. \\
\end{aligned}
\]
We use the predicate $\emptyp$ to determine if a list contains no
elements.

To construct lists, we use the operators {\tt ( , )} and {\tt ( @@ )}.
The comma operator is used to create a list, while the {\tt @@} is used
add elements to the front of a list.  This is best illustrated by
examples.
\[
\begin{aligned}
\mbox{\tt ($1$, ($2$, $3$), $4$, ($5$, $6$))} 
   & \Rightarrow \mbox{\tt (1 (2 3) 4 (5 6))},\\
\mbox{\tt ($1$ @@ (($2$, $3$), $4$, ($5$, $6$)))} 
   & \Rightarrow \mbox{\tt (1 (2 3) 4 (5 6))},\\
\mbox{\tt ($1$,  ($2$, $3$), $4$ @@ ($5$, $6$))} 
   & \Rightarrow \mbox{\tt (1 (2 3) 4 5 6)},\\
\mbox{\tt ($1$ @@  ($2$, $3$), $4$, ($5$, $6$))} 
   & \Rightarrow \mbox{\tt (1 2 3 4 (5 6))},\\
\mbox{\tt ($1$ @@  ($2$, $3$), $4$ @@ ($5$, $6$))} 
   & \Rightarrow \mbox{\tt (1 2 3 4 5 6)}.
\end{aligned}
\]

We call $\rest(\mbox{\tt F})$ the {\em term list}\index{polynomial!
term list}.  Here it is represented as a list.  The term list is a set
of exponent/coefficient pairs minus the identifying variable.  The
term list is the most commonly used structure when implementing
polynomial algorithms on recursively represented polynomials.  To
manipulate term lists, we use the functions, $\lt$ to get the
\key{leading term}, $\lc$ the \key{leading coefficient} and $\lexp$
the \key{leading exponent}.
\[
\begin{aligned}
\lt(\terms(\mbox{\tt F}))& \Rightarrow \mbox{\tt (7 1)}, \\
\lexp(\terms(\mbox{\tt F}))& \Rightarrow \mbox{\tt 7}, \\
\lc(\terms(\mbox{\tt F}))& \Rightarrow \mbox{\tt 1}.
\end{aligned}
\]

With these basic tools, we can implement the classical polynomial
addition algorithm for term lists:
\label{TermsPlus:Def}   
\begindsacode
 1 \=Ter\=msPlus (Fterms, Gterms) := $\{$ \\
 2\>\>if $\emptyp(\mbox{Fterms})$ then Gterms; \\
 3\>\>elif $\emptyp(\mbox{Gterms})$ then Fterms; \\
 4\>\>elif \=$\lexp(\mbox{Fterms}) > \lexp(\mbox{Gterms})$\\
 5\>\>\>then ($\lt(\mbox{Fterms})$ @@ $\mbox{TermsPlus}(\rest(\mbox{Fterms}), \mbox{Gterms})$);\\
 6\>\>elif \=$\lexp(\mbox{Fterms}) < \lexp(\mbox{Gterms})$\\
 7\>\>\>then ($\lt(\mbox{Gterms})$ @@ $\mbox{TermsPlus}(\mbox{Fterms}, \rest(\mbox{Gterms}))$);\\
 8\>\>else $\{$ \=$\mbox{tempc} \leftarrow
\mbox{PolyPlus}(\lc(\mbox{Fterms}), \lc(\mbox{Gterms}))$;\\
 9\>\>\>if $\mbox{tempc} = 0$ then $\mbox{TermsPlus}(\rest(\mbox{Fterms}), \rest(\mbox{Gterms}))$;\\
10\>\>\>else (\=($\lexp(\mbox{Fterms})$, $\mbox{tempc}$)\\
11\>\>\>\> @@ $\mbox{TermsPlus}(\rest(\mbox{Fterms}), \rest(\mbox{Gterms}))$);\\
12\>\>\>$\}$\\
13\>\> $\}$
\enddsacode

For simplicity we have implemented this routine using recursion.
Lines {\tt 2} and {\tt 3}, are base cases that are used when we run
out of terms in either {\tt Fterms} or {\tt Gterms}.  If the degree of
the leading term of {\tt Fterms} is greater than that of {\tt Gterms}
(line {\tt 4}) or vice versa (line {\tt 6}) then a term is added
without any further computation.  If the degree of the leading terms
of {\tt Fterms} and {\tt Gterms} are the same, then the leading
coefficients are added (line {\tt 9}), and, if non-zero, the sum is added
to the answer.

Notice that on line {\tt 8} the leading coefficients of {\tt Fterms}
and {\tt Gterms} are added using {\tt PolyPlus}, which is defined in
the next paragraph.  This is necessary because we have used a
recursive representation, and thus the coefficients of a polynomial
may themselves be polynomials.

To implement a multivariate polynomial algorithm, we must account for
the different variables.  The routine \keyw{PolyPlus} is entry
point for general polynomial addition.  If its arguments have the same
main variable then \keyw{TermsPlus} is used to do the addition.
Otherwise, the polynomial with the lesser main variable is interpreted
as a polynomial with only a constant term in the greater main
variable.  That is if $X$ is more main than $Y$ and we wish to add
$X+1$ and $Y$, we instead add $X+1$ and $X^0 Y$.  This is done in the
following routine.  Notice, that we also need to worry about
constants, \ie, expressions free of any variables.

\begindsacode
Pol\=yPlus (F, G) := $\{$ \\
\>if $\coefp(\mbox{F})$ then \=if $\coefp(\mbox{G})$ then $\mbox{F}+\mbox{G}$; \\
\>\>else $\mbox{PolySimp}(\var(\mbox{G}), \mbox{TermsPlus}(\mbox{((0, F))}, \terms(\mbox{G})))$; \\
\>eli\=f $\coefp(\mbox{G})$\\
\>\> then $\mbox{PolySimp}(\var(\mbox{F}), \mbox{TermsPlus}(\mbox{((0, G))}, \terms(\mbox{F})))$; \\
\>elif $\var(\mbox{F}) > \var(\mbox{G})$ \\
\>\>then $\mbox{PolySimp}(\var(\mbox{F}), \mbox{TermsPlus}(\mbox{((0, G))}, \terms(\mbox{F})))$; \\
\>elif $\var(\mbox{F}) < \var(\mbox{G})$ \\
\>\>then $\mbox{PolySimp}(\var(\mbox{G}), \mbox{TermsPlus}(\mbox{((0, \mbox{F}))}, \terms(\mbox{G})))$; \\
\>else $\mbox{PolySimp}(\var(\mbox{F}), \mbox{TermsPlus}(\terms(\mbox{F}), \terms(\mbox{G})))$;\\
\>$\}$
\enddsacode

In \keyw{PolyPlus} the function \keyw{PolySimp} is used to create
polynomials from term lists and a variable.  This is done instead of
just using a constructor since the terms list could consist of only a
constant term, and thus the variable needs to be elided to maintain
the variable sparse representation.  The definition of \keyw{PolySimp}
is as follows

\begindsacode
Pol\=ySimp (var, terms) := $\{$\\
\>if $\emptyp(\mbox{terms})$ then $0$ \\
\>elif $0 = \lexp(\mbox{terms})$ then $\lc(\mbox{terms})$\\
\>else (var @@ terms)\\
\> $\}$
\enddsacode

The basic idea behind polynomial addition is quite simple.  However,
it was necessary to introduce a bit of complexity in order to maintain
all the details of a recursive, variable sparse, degree sparse
representation.  This complexity often makes some of the highly
efficient algorithms discussed in later sections less attractive.
Nonetheless, the flexibility provided by the variable sparse, degree
sparse representation makes it an ideal representation that can be
used in a wide variety of applications.

Also notice that we have not paid much attention to the coefficient
domain.  In particular, notice that we have assumed that the
coefficient domain has characteristic zero.  In modern systems like
{\Axiom} \cite{Jenks1992-cu} or {\Weyl} \cite{Zippel1993-ef} the coefficient
domain could be more general, and could itself be a polynomial ring.
This generality is extremely valuable when implementing algorithms
like integration of algebraic and transcendental functions or certain
computations in algebraic geometry.  This additional power requires
more sophisticated programming techniques that are beyond the scope of
this book.

Our implementation of this polynomial representation used recursion
extensively, both over the variables in \keyw{PolyPlus} and over the
terms in \keyw{TermsPlus}.  In a practical implementation an iterative
(or tail recursive) structure would be used over the terms since the
number of terms in a polynomial could be large.  The recursion over
the variables, however, is quite convenient and effective.

\section{Polynomial Multiplication}
\label{Poly:Mult:Sec}

For polynomial multiplication we start with the basic high school
algorithm.  Let $F$ and $G$ be two polynomials, 
\[
\begin{aligned}
F(X) &= a_{0} X^{m} + \cdots + a_{m}, \\
G(X) &= b_{0} X^{n} + \cdots + b_{n}.
\end{aligned}
\]
Their product $H(X) = F(X) G(X)$ has coefficients
\begin{equation} \label{PolyCauchyProduct:Eq}
\mbox{coef}(H, X^{m+n-\ell}) \leftarrow a_{\ell} b_{0} + a_{\ell-1} b_{1} + \cdots
 a_{1} b_{\ell-1} + a_{0} b_{\ell}.
\end{equation}

The following program creates $H$ by precisely following this prescription.
\begindsacode
Pol\=yTimes1($F(X), G(X)$) := $\{$\\
\> $H \leftarrow 0$; \\
\> for\=each $a_{i}X^{e_{i}}$ in $F(x)$ \\
\> \> for\=each $b_{j} X^{f_{j}}$ in $G(X)$ \\
\> \> \> $\mbox{coef}(H, X^{e_{i}+f_{j}}) \leftarrow
    \mbox{coef}(H, X^{e_{i}+f_{j}}) + a_{i} b_{j}$; \\
\> return($H$); \\
\> $\}$
\enddsacode
\label{PolyTimes1:Alg}
\noindent
This algorithm assumes that polynomials are represented as sets of
terms, where each term is a coefficient/exponent pair.  The {\tt
foreach} forms are used to loop over the coefficient/exponent pairs of
each polynomial.  The assignment statement is used to update a
coefficient pair with a new value.  This assignment should remove the
term if the resulting coefficient is zero and should not do any
assignment if $a_{i} b_{j}$ is zero.

If $F(X)$ and $G(X)$ have $r$ and $s$ non-zero terms respectively then
the routine \keyw{PolyTimes1} will require $rs$ multiplications.  If
the terms are organized as vectors of coefficients indexed by
exponents, then the assignment statement could be done in unit time.
However, this is only practical when the polynomials are known to be
relatively dense.  When sparse data structures are used, the cost of
modifying terms in $H$ may dominate.

To make these issues a bit clearer, we consider an explicit
implementation of univariate polynomial multiplication, building on the
ideas used to implement polynomial addition in the previous section.
We use the same data structures and operators as before. 

As with polynomial addition, we start with routines for dealing with
term lists.  The base routine for multiplication is a bit more complex
than \keyw{TermsPlus}.  It is best to start with \keyw{TermsMonTimes},
which multiplies a terms list by a monomial and adds it to (merges it
with) another terms list.

\begindsacode
Ter\=msMonTimes (terms, e, c, sum) := $\{$ \\
\>if \=$\lexp(\mbox{terms}) + \mbox{e} > \lexp(\mbox{sum})$ \\
\>\>then (\=($\lexp(\mbox{terms}) + \mbox{e}$, $\lc(\mbox{terms})\cdot\mbox{c}$)\\
\>\>\>@@ $\mbox{TermsMonTimes}(\rest(\mbox{terms}), \mbox{e}, \mbox{c}, \mbox{sum})$;\\
\>elif $\lexp(\mbox{terms}) + \mbox{e} = \lexp(\mbox{sum})$ \\
\>\>then $\{$ \=$\mbox{tempc} \leftarrow
\mbox{PolyPlus}(\mbox{PolyTimes}(\lc(\mbox{terms}), \mbox{c}), \lc(\mbox{sum}))$; \\
\>\>\>if \=$\mbox{tempc} = 0$ \\
\>\>\>\>then $\mbox{TermsMonTimes}(\rest(\mbox{terms}), \mbox{e}, \mbox{c}, \rest(\mbox{sum}))$;\\
\>\>\>else (\=($\lexp(\mbox{sum})$, tempc) \\
\>\>\>\>@@ $\mbox{TermsMonTimes}(\rest(\mbox{terms}), \mbox{e}, \mbox{c}, \rest(\mbox{sum}))$);\\
\>\>\>$\}$\\
\>else (\=($\lexp(\mbox{sum})$, $\lc(\mbox{sum})$)\\
\>\>@@ $\mbox{TermsMonTimes}(\mbox{terms}, \mbox{e}, \mbox{c}, \rest(\mbox{sum}))$);\\
\>$\}$
\enddsacode

\noindent
This routine corresponds to the inner loop in \keyw{PolyTimes1}.  It
takes advantage of the fact that the terms in a terms list are sorted
by exponents.  Notice that this implementation assumes that the
coefficient field has characteristic zero.

With \keyw{TermsMonTimes} it is easy to multiply terms lists.  The
routine \keyw{TermsTimes} multiplies {\tt Gterms} by each term in
{\tt Fterms}, accumulating the result.  It corresponds to the outer
loop in \keyw{PolyTimes1}. 
\begindsacode
Ter\=msTimes (Fterms, Gterms) := $\{$ \\
\>if $\emptyp(\mbox{Fterms})$ then (); \\
\>else TermsMonTimes$($\=$\mbox{Gterms}, \lexp(\mbox{Fterms}), \lc(\mbox{Fterms}),$\\
\>\>$\mbox{TermsTimes}(\rest(\mbox{Fterms}), \mbox{Gterms}))$;\\
\> $\}$
\enddsacode

Using the \keyw{TermsTimes} and \keyw{TermsMonTimes} routines, we can construct
the routine that multiplies real polynomials. It has much the same
structure as \keyw{PolyPlus}, making sure that the polynomials have
the same main variable before calling \keyw{TermsTimes} and using
\keyw{TermsMonTimes} when they do not.
\begindsacode
Pol\=yTimes2 (F, G) := $\{$ \\
\>if $\coefp(\mbox{F})$ then \=if $\coefp(\mbox{G})$ then $\mbox{F}\times\mbox{G}$; \\
\>\>else ($\var(\mbox{G})$, $\mbox{TermsMonTimes}(\terms(\mbox{G}), 0, \mbox{F}, ())$); \\
\>eli\=f $\coefp(\mbox{G}) \vee \var(\mbox{F}) > \var(\mbox{G})$\\
\>\> then ($\var(\mbox{F})$, $\mbox{TermsMonTimes}(\terms(\mbox{F}), 0, \mbox{G}, ())$);\\
\>elif $\var(\mbox{F}) < \var(\mbox{G})$ \\
\>\>then ($\var(\mbox{G})$, $\mbox{TermsMonTimes}(\terms(\mbox{G}), 0, \mbox{F}, ())$);\\
\>else $\{$ \=$\mbox{terms} \leftarrow \mbox{()}$;\\
\>\>loop \=for $(e, c) \in \terms(\mbox{F})$\\
\>\>\>$\mbox{terms} \leftarrow \mbox{TermsMonTimes}(\terms(\mbox{G}),
      e, c, \mbox{terms})$;\\
\>\> ($\var(\mbox{F})$ @@ terms);\\
\>\>$\}$\\
\>$\}$
\enddsacode
These routines are an effective implementation of polynomial
multiplication, except for the fact that recursion was used over
degrees in \keyw{TermsMonTimes} where iteration or tail recursion would
be preferable.  Except for this issue, \keyw{PolyTimes2} is essentially
identical to the polynomial multiplication routines used in
\Macsyma{} and \Axiom.


We now return to the question of the number of exponent comparisons
involved in polynomial multiplication.  Consider the two extreme
cases, multiplying two totally sparse polynomials and multiplying two
dense polynomials, all of which consist of $t$ non-zero terms. In the
dense case, we are multiplying two polynomials of the form
\[
\begin{aligned}
F_{\rm dense}(X) &= a_{0} X^{t-1} + \cdots + a_{t-1}, \\
G_{\rm dense}(X) &= b_{0} X^{t-1} + \cdots + b_{t-1},
\end{aligned}
\]
where all of the $a_i$ and $b_j$ are non-zero.  At the end of the
$i$-th pass through the outer loop of \keyw{PolyTimes1}, \ie, the
$i$-th recursion in \keyw{TermsTimes}, $H$ will have $t+i-1$ terms.
During the $i+1$-th invocation of \keyw{TermsMonTimes} by {\tt
TermsTimes} about $t+i-1$ exponent comparisons are required since {\tt
sum} will have $t+i-1$ terms.  Thus, the total number of exponent
comparisons is approximately
\[
\sum_{i=2}^{t} t + i - 1 = t^2 + \frac{t(t-1)}{2} = O(t^2).
\]
Thus the exponent comparisons have the same order of complexity as the
coefficient arithmetic.  

For the sparse case, however, at the end of the $i$-th loop $H$ may
have as many as $i\cdot t$ non-zero terms.  This is the definition of
being ``totally sparse.''\index{polynomial! totally sparse} Thus the
total number of exponent comparisons can be as large as
\[
\sum_{i=1}^{t} i\cdot t = \frac{t^2\,(t+1)}{2} = O(t^3).
\]
For sufficiently large and sparse problems, the exponent comparisons
dominate the cost of the high school algorithm, not the coefficient
operations.  In practice, the cost of an exponent comparison is
usually much less than that of a coefficient operation, so this growth
rate is generally masked.

\section{Fast Polynomial Algorithms}
\label{Poly:Fast:Sec}

There are asymptotically faster methods for multiplying polynomials
than those discussed in \sectref{Poly:Mult:Sec}.  The fast methods for
polynomial arithmetic can be divided into classes depending upon
whether they reduce the number of coefficient operations or reduce the
number of exponent comparisons.  Those of the former class are most
appropriate for dense (usually univariate) problems, while those that
reduce exponent comparisons may be appropriate for sparse problems.

This section discusses two basic techniques.  First, we demonstrate
how more efficient data structures for term lists can significantly
reduce the number of exponent comparisons required for multiplication
(while increasing the cost of addition).  Second, we give a ``divide
and conquer'' algorithm that only requires $O(n^{1.56})$ coefficient
operations.  However, this technique places significant demands on the
data structure used for the term lists and thus may not be effective
for small problems.

Fast polynomial multiplication algorithms are most beneficial when the
number of terms in the polynomial is very large.  Problems with large
polynomials are often multivariate problems.  However, multivariate
problems frequently use recursive, variable sparse representations.
Thus the polynomials are represented as multiple layers of relatively
small polynomials.  Consequently, the fast algorithms may not provide
as much benefit as one might expect.  Expanded, variable sparse
representations might be able to take advantage of these optimized
algorithms, but then the exponent comparisons can be quite expensive.
Expanded, variable dense representations can be used effectively, but
they require that the number of variables that occur in the problem be
known ahead of time.  This is not practical for many problems that
involve transcendental functions, such as integration or
simplification of trigonometric identities.

\sectref{Poly:FFT:Sec} discusses an even faster technique for
multiplying polynomials, one that only requires $O(n \log n)$
coefficient operations.  Unfortunately, this technique intrinsically
assumes that the polynomials are dense and thus is not often of much
practical use.

\paragraph{Sorting Techniques}

The algorithms discussed in \sectref{Poly:Mult:Sec} use the simplest
possible data structure for the terms list of the polynomial, a linear
list.  Unfortunately, the time to find a particular element in a
linear list of length $O(n)$ is approximately $O(n)$.  The basic
polynomial multiplication routine \keyw{PolyTimes1} inserts $O(n^2)$
elements into the terms list of $H$, which on the average has $O(n)$
elements in it.  Thus, it requires $O(n^3)$ exponent comparisons when
a linear list is used.  By using a more efficient data structure, the
exponent comparison cost can be reduced to $O(n^2 \log n)$ worst case
or $O(n^2)$ average case.\rightmarginnote{This needs to be expanded a bit
  and references included.}

In place of the list structure used for the \key{terms list} of the
polynomial, a balanced structure like a \key{heap} or {\sc avl}
tree\index{AVL tree@{\sc avl} tree} can be used.  In order to maintain
a level of abstraction, we describe these algorithms using the
abstract operations given in \figref{TermList:Fig}.

\begin{figure}
\begin{center}
\begin{tabular}{lp{2.25in}}
$\mbox{\tt newTermList}()$ & Creates a new, empty term list using some
efficient structure.\\[4pt] 
$\mbox{\tt insert}((e : c), \mbox{\em terms})$& Inserts a new term in
the terms list {\em terms} using the key $e$ and the value $c$. \\[4pt]
$\mbox{\tt delete}(e, \mbox{\em terms})$& Delete the term is key $e$ from
{\em terms} \\[4pt]
{\tt foreach $(e : c) \in \mbox{\em terms}$} & Binds $e$ and $c$ to
each key and value (exponent and coefficient) in {\em terms} and then
evaluates its body. 
\end{tabular}
\end{center}
\caption{Term list operations\label{TermList:Fig}}
\end{figure}

Using these abstract operations a version of \keyw{TermsPlus} can be
written as follows. 
\begindsacode
Ter\=msPlus3 (Fterms, Gterms) := $\{$ \\
\> $\mbox{Hterms} \leftarrow \mbox{copy}(\mbox{Fterms})$; \\
\> for\=each $(e : c) \in \mbox{Gterms}$ $\{$ \\
\>\> $\mbox{new-c} \leftarrow \mbox{lookup}(\mbox{Hterms}, e)$; \\
\>\> if $\mbox{new-c} = \phi$ then $\mbox{insert}((e : c), \mbox{Hterms})$;\\
\>\> else $\{$ \= $\mbox{delete}(e, \mbox{Hterms})$;\\
\>\>\> $\mbox{insert}((e : c + \mbox{new-c}), \mbox{Hterms})$; \\
\>\>\> $\}$\\
\>\> $\}$ \\
\> return Hterms; \\
\> $\}$
\enddsacode

The analysis of this routine is quite simple.  Assume that number of
non-zero terms in {\tt Fterms} and {\tt Gterms} is bounded by $n$.
{\tt Hterms}, which is used to accumulate the sum, begins with $n$
elements and ends with as few as $0$ terms or as many as $2n$ terms.
The body of the \keyw{foreach} loop is executed $n$ times.  Let $C(k)$
denotes the maximum number of comparisons required when
inserting/deleting/looking-up an element in a set of size $k$ and
assume that $C$ is monotonic with $k$.  The total number of
comparisons in \keyw{TermsPlus3} is bounded by
\[
\sum_{1 \le i \le n} C(m+i) \le n C(m+n).
\]
The number of additions is always bounded by $n$.  

Three different term list representations are worth mentioning, a
linear list, a balanced tree or heap of some sort and a hash table.
This simplest, most easily implemented and most common is the linear
list.  The other two approaches lead to better asymptotic results and
for sufficiently large problems exhibit better performance than the
linear list structure.  The asymptotic characteristics of each of
these three term list representations is given in
\figref{OperationCount:Add:Fig}.  However, the complexity of efficiently
implementing these structures and their increased cost for small
polynomials often discourages their use.

For the linear case, we use the sorted structure of {\tt
TermsPlus}.  
\begin{figure}
\begin{center}
\begin{tabular}{|l|c|c|c|} 
\multicolumn{1}{l}{}& \multicolumn{1}{c}{$C(k)$} & 
  \multicolumn{1}{c}{Comparisons} & \multicolumn{1}{c}{Arith Ops} \\ \hline
Linear list & $k$ & $O(n)$ & $O(n)$ \\ \hline
Balanced tree & $\log k$ & $O(n \log n)$ & $O(n)$ \\ \hline
Hash table & $O(1)$ & $O(n)$ & $O(n)$ \\ \hline
\end{tabular} 
\end{center}
\caption{Operation counts for addition of
polynomials\label{OperationCount:Add:Fig}} 
\end{figure}

Unless used more cleverly, the balanced tree data structure slows down
the asymptotic behavior of the polynomial addition.  The linear list
structure requires only linear time because we know that the inserts are in
order.

\medskip
Polynomial multiplication proceeds in a similar fashion.  Below we
have coded a simple version of \keyw{TermsTimes}, using the term list
operators discussed earlier.

\begindsacode
Ter\=msTimes3 (Fterms, Gterms) := $\{$ \\
\> $\mbox{Hterms} \leftarrow \mbox{newTermList}()$; \\
\> for\=each $(e_f : c_f) \in \mbox{Fterms}$ $\{$ \\
\>\> for\=each $(e_g : c_g) \in \mbox{Gterms}$ $\{$ \\
\>\>\> $\mbox{new-c} \leftarrow \mbox{lookup}(e_f + e_g, \mbox{Hterms})$;\\
\>\>\> if $\mbox{new-c} = \phi$ then $\mbox{insert}((e_f + e_g : c_f \times c_g), \mbox{Hterms})$;\\
\>\>\> else $\{$ \= $\mbox{delete}(e_f + e_g, \mbox{Hterms})$; \\
\>\>\>\> $\mbox{c-new} \leftarrow \mbox{c-new} + c_f \times c_g$; \\
\>\>\>\> unl\=ess $\mbox{c-new} = 0$ $\{$ \\
\>\>\>\>\> $\mbox{insert}((e_f + e_g : \mbox{c-new}), \mbox{Hterms})$;\\
\>\>\>\>\> $\}$\\
\>\>\>\> $\}$\\
\>\>\> $\}$\\
\>\> $\}$ \\
\> return Hterms; \\
\> $\}$
\enddsacode

This time the number of passes through the inner loop is $O(n^2)$.
$O(n^2)$ arithmetic operations and $O(n^2)$ tree operations are
performed.  Taking into account the size of the trees, we get the
table of operation counts in \figref{OperationCount:Mult:Fig}.

\begin{figure}
\begin{center}
\begin{tabular}{|l|c|c|c|} 
\multicolumn{1}{l}{}& \multicolumn{1}{c}{$C(k)$} & 
  \multicolumn{1}{c}{Comparisons} & \multicolumn{1}{c}{Arith Ops} \\ \hline
Linear list & $k$ & $O(n^3)$ & $O(n^2)$ \\ \hline
Balanced tree & $\log k$ & $O(n^2 \log n)$ & $O(n^2)$ \\ \hline
Hash table & $O(1)$ & $O(n^2)$ & $O(n^2)$ \\ \hline
\end{tabular} \\
\end{center}
\caption{Operation counts for multiplication of
polynomials\label{OperationCount:Mult:Fig}} 
\end{figure}

As can be seen by these tables, the asymptotic data structure costs of
polynomial arithmetic can be reduced with better data structures.  By
using mergeable heaps or similar data structures, the additional cost
incurred can be eliminated so that only $O(n)$ exponent comparisons
are required.  During the seventies there was a significant amount of
work on efficient data structures that would also be effective for
moderate size problems \cite{Horowitz1975-xl,Klip1979-pd} but these techniques have not been incorporated in subsequent systems.

The reason for this is that linear lists can be searched and
manipulated very efficiently on current computer architectures.  For
moderate size polynomials the performance benefits of the more
efficient algorithms is not realized.

\paragraph{Divide and Conquer}

An idea of {\Karatsuba} and {\Ofman} \cite{Karatsuba1963-xj} leads to a
simple algorithm that only requires $O(n^{1.56})$ multiplications.  If
a proper representation is chosen, the overhead of this algorithm is
small enough for it to be quite effective.  The algorithm is based on
the following identity.  Let $F$ and $G$ are polynomials degree $n$,
where $n = 2^k$.  We can rewrite $F(X)$ and $G(X)$ as follows:
\[
\begin{aligned}
F(X)&= f_0(X) X^{2^{k-1}} + f_1(X),\\
G(X)&= g_0(X) X^{2^{k-1}} + g_1(X),
\end{aligned}
\]
where $f_i$ and $g_i$ are polynomials of degree $\le 2^{k-1}$.  Then
the product $F(X) G(X)$ can be written as:
\[
\begin{aligned}
  F(X) G(X) &= f_0 g_0 X^{2^k} + (f_1 g_0 + f_0 g_1) X^{2^{k-1}} + f_1 g_1\\
    &= f_0 g_0 X^{2^k} 
       + ((f_1 + f_0) (g_1 + g_0) - f_0 g_0 - f_1 g_1) X^{2^{k-1}}
       + f_1 g_1
\end{aligned}
\]

Notice that the terms $f_0 g_0$ and $f_1 g_1$ appear twice in this formula.
Thus only three polynomial multiplications are required.  If $M(n)$ denotes
the number of coefficient multiplications required to multiply two
polynomials of degree $n$, then using the above multiplication scheme we
have $M(n) = 3 M(n/2)$.  Thus
\[
M(n) = 3 ^{\log_2 n} = n^{\log_2 3} \approx n^{1.5649625},
\]
for dense polynomials.  

This approach can also be applied to sparse polynomials.  The
essential idea is to find a way to split $F(X)$ and $G(X)$ into two
pieces that have equal numbers of terms.  That is, assume that
\[
\begin{aligned}
F(X)&= f_0(X) X^{M} + f_1(X),\\
G(X)&= g_0(X) X^{M} + g_1(X),
\end{aligned}
\]
where $f_0$ and $f_1$ have about the same number of non-zero terms and
$g_0$ and $g_1$ have the same number of non-zero terms.  The product
can be written as
\[
  F(X) G(X) = f_0 g_0 X^{2M} 
       + ((f_1 + f_0) (g_1 + g_0) - f_0 g_0 - f_1 g_1) X^{M}
       + f_1 g_1.
\]
Notice that $F(X)$ and $G(X)$ must be split at the same degree, which
may force non-optimal splits of  $F(X)$ and $G(X)$.  It also
introduces substantially more overhead than is required by the
classical algorithm.  

Nonetheless, this divide and conquer algorithm is easy to implement
for univariate polynomials using the dense representation and can be a
good choice for moderate degree problems.  

This is apparently better than the $O(s^2)$ complexity that the
classical algorithm requires.  There are some problems though.
Splitting the polynomials in half, as is required, introduces a fair
amount of overhead.  For some implementations, this algorithm is
faster than the classical one only for polynomials of degree greater
than about 40 \cite{Fateman1974-xv}.


\section{Polynomial Exponentiation}
\label{Poly:Expt:Sec}

Polynomial exponentiation is significantly more subtle than
exponentiating floating point numbers.  The simplest technique for
computing $P(X)^s$ is to multiply $P(X)$ by itself repeatedly, \ie,
\begindsacode
Pol\=yExptRm ($p$, $s$) := $\{$\\
\> $q \leftarrow 1$; \\
\> loo\=p for $1 \le i \le s$ do \{\\
\> \> $q \leftarrow q \times p$; \\
\>\> $\}$ \\
\> return ($q$); \\
\>$\}$
\enddsacode

\noindent
This is called the {\em repeated multiplication algorithm\/}.  For numerical
computations we know that the {\em repeated squaring} algorithm given
below uses fewer multiplications.
\begindsacode
Pol\=yExptSq ($p$, $s$) := $\{$\\
\> $q \leftarrow 1$;  $m \leftarrow p$; \\
\> loo\=p while $s > 0$ do \{\\
\> \> if oddp($s$) then $q \leftarrow q \times m$; \\
\> \> $m \leftarrow m \times m$; \\
\> \> $s \leftarrow \lfloor s/2 \rfloor$; \\
\>\> $\}$ \\
\> return ($q$); \\
\> $\}$
\enddsacode

\noindent
With floating point numbers each of the three multiplications in these
routines has the same cost, so counting the number of multiplies is an
accurate way of comparing the relative costs of the two algorithms.
However, for polynomials, powers of $P(X)$ can be much larger than
$P(X)$, so the cost of computing $P(X)^{s/2} \times P(X)^{s/2}$ can be
much larger than the cost of multiplying $P(X)$ by $P(X)$, or even
multiplying $P(X)$ by $P(X)^{s-1}$.  Thus a more careful analysis is
required.  In fact repeated squaring is more expensive for polynomials
than repeated multiplication---just the opposite of the situation with
floating point numbers or elements of $\F_p$.

A result of {\Gentleman} \cite{Gentleman1972-ti} quantifies this
behavior.  The size of $P(X)^s$ is largest relative to the size of
$P(X)$ when $P(X)$ is completely sparse.  For a worst case analysis,
let $P$ be a polynomial with $n+1$ independent terms:
\[
P = 1 + t_1 + t_2 + \cdots + t_n,
\]
and
\[
P_i = (1 + t_1 + t_2 + \cdots + t_n)^i.
\]
In the following we ignore the cost of exponent comparisons and we assume a
classical $O(n^{2})$ multiplication algorithm.  Thus the cost of
multiplying two polynomials is the product of the number of terms in each.
If we let $L(P_i)$ denote the number of terms in $P_i$, then
\[
L(P_i) = \binom{n + i}{n},
\]
as noted in \sectref{Poly:Generalities:Sec}.

\newcommand{\Csq}{C_{\rm Sq}}
\newcommand{\Cmul}{C_{\rm Mul}}
Now consider the problem of computing $P_{r+s}$ given $P_r$ and $P_s$.
Multiplying $P_r$ by $P_s$ corresponds to using the repeated squaring
algorithm, while repeated multiplying by $P_1$ corresponds to the repeated
multiplication algorithm.  The cost of multiplying $P_r P_s$ is
\[
\Csq(r, s) = L(P_r) L(P_s) = \binom{n + r}{n} \binom{n + s}{n}.
\]
Repeated multiplication costs
\[
\begin{aligned}
  \Cmul(r, s) = 
  \sum_{j=r}^{r+s-1} L(P_1) L(P_j) &=
    (n+1) \sum_{j=r}^{r+s-1} \binom{n + j}{n}\\
     &= (r + s)\, \binom{n + r + s}{r+s} - r\, \binom{n + r}{r}
\end{aligned}
\]
where we have used \propref{CombinSum:Prop} to evaluate the sum.


We estimate the cost of producing $P_{2r}$.  For the repeated squaring
case, we assume that we already know $P_{r}$ and merely have to
multiply them.  Thus, the cost of using the repeated squaring
algorithm is
\[
\begin{aligned}
 \Csq(r,r) &= \binom{n + r}{r}^{2} \\
     & = \frac{\left(n^{r} + \frac{1}{2}r (r+1) n^{r-1}
                 + O(n^{r-2})\right)^{2}}{(r!)^{2}} \\
     & = \frac{n^{2r}}{(r!)^{2}}\left(1 + \frac{r(r+1)}{n}\right) 
                + O(n^{2r-2}).
\end{aligned}
\]
For repeated multiplication, we multiply by $P_{1}$ $2r$ times.  Thus, the
cost is
\[
\begin{aligned}
  \Cmul(2r, 1) & = 2r \binom{n + 2r}{2r} - n-1 \\
     & 2r \frac{n^{2r} + \frac{1}{2}(2r)(2r+1) n^{2r-1} 
        + O(n^{2r-2})}{(2r)!} \\
     & = \frac{2r}{(2r)!}n^{2r}\left(1 + \frac{r(2r+1)}{n}\right) 
        + O(n^{2r-2}).
\end{aligned}
\]
The ratio of these two costs is 
\[
\frac{\Csq(r,r)}{\Cmul(2r,1)} =
\frac{1}{2r} \binom{2r}{r} \left(1 - \frac{r^{2}}{n}\right) 
  + O(n^{-2}).
\]
For sufficiently sparse polynomials (large values of $n$), the right
hand side of this expression is greater than $1$ for all $r \ge 2$.
Thus, the final multiplication in computing $P^{2r}$ by repeated
squaring can dominate the cost of computing
\[
\overbrace{P \times P \times \cdots \times P}^{2r},
\]
which many find counterintuitive.

\medskip
An algorithm that has been used successfully in \Macsyma, and seems to have
somewhat better behavior than repeated multiplication is based on the
binomial theorem.  The idea is based on the binomial formula:
\[
(a + b)^n = a^n + n a^{n-1} b + \frac{n (n - 1)}{2} a^{n-2} b^2 +
\cdots + b^n.
\]
The leading term of the polynomial is used for $a$ and the rest of the
polynomial takes the place of $b$ in the formula.  Notice that each of the
powers of $b$ can be computed by a single multiplication.  The following
program implements the binomial theorem version of the exponentiation
algorithm.  The variable $c$ is successively bound to binomial
coefficients. 

\begindsacode
PolyExpt ($p$, $n$) := $\{$ \\
\> if $n=0$ then return($1$); \\
\> elif $n=1$ then return($p$); \\
\> else $\{$ \= $b \leftarrow \mbox{\rm rest}(p)$; \\
\> \> $m \leftarrow \lt p$; \\
\> \> $l \leftarrow (b^{n-1}, \ldots, b)$; \\
\> \> $c \leftarrow n$; \\
\> \> $r \leftarrow b \times \mbox{\rm first}(l)$; \\
\> \> $M \leftarrow m$; \\
\> \> unl\=ess null($l$) $\{$ \\
\> \> \> $r \leftarrow r + c \times M \times \mbox{\rm first}(l)$; \\
\> \> \> $l \leftarrow \mbox{\rm rest}(l)$; \\
\> \> \> $k \leftarrow k+1$; \\
\> \> \> $c \leftarrow (c \times (n - k))/(k+1)$; \\
\> \> \> $M \leftarrow m \times M$; \\
\> \> \> $\}$ \\
\> \> $\}$ \\
\> \> return($r$); \\
\> $\}$
\enddsacode

A version of this algorithm that uses the multinomial expansion
formula has been described by {\Alagar} and {\Probst}
\cite{Alagar1987-nq}.  Their algorithm behaves better with purely
dense polynomials, but not so well on sparse polynomials.  The one
given above seems to be a reasonably good, general purpose algorithm
for general multivariate polynomials.  For univariate dense
polynomials, the FFT techniques described in \sectref{Poly:FFT:Sec}
are both asymptotically better and, for sufficiently large
polynomials, superior in practice.

\section{Polynomial Substitution}
\label{Poly:Subs:Sec}

By polynomial substitution we mean the problem of computing $F(G)$
when given $F(X) \in R[X]$ and $G$ an element of an $R$-module
(including $R[X]$).

A well known method of evaluating a polynomial is \keyi{Horner's
rule}.  That is, if
\[
F(X) = f_0 X^d + f_1 X^{d-1} + \cdots + f_d,
\]
we evaluate $F(G)$ by
\[
F(G) = (((f_0 \cdot G + f_1) \cdot G + f_2) \cdot G + \cdots.
\]
Since repeated multiplication is not a bad way to exponentiate a
polynomial, Horner's rule is an especially good way to do
substitution.  

The following routine implements Horner's rule for substituting a
polynomial $G$ into a term list.  Some care is taken to deal with the
use of a degree sparse representation. 

\begindsacode
Ter\=msHorner(Fterms, $G$) := $\{$ \\
\> $H \leftarrow \lc(\mbox{Fterms})$; \\
\> $f \leftarrow \lexp(\mbox{Fterms})$; \\
\> for\=each $(e : c) \in \mbox{Fterms}$ \{ \\
\>\> $H \leftarrow G^{f - e} \times H + c$; \\
\>\> $f \leftarrow e$; \\
\>\> \} \\
\> return($H \times G^{e_{old}}$); \\
\> \}
\enddsacode

For dense polynomials of degree $d$, Horner's rule requires $O(d^3)$
coefficient multiplies.  Asymptotically faster algorithms for
substitution exist.  In particular, the multiplication cost can be
reduced to $O(d^2 \log d)$ by using fast Fourier transforms and
interpolation.  As with other asymptotic results, this improvement is
only realized for dense univariate polynomials.

\section*{Notes}

\small

A survey of asymptotically fast algorithms for polynomials is
contained in \cite{Pan1992-dl}.  Some experiences with parallel
implementations of polynomial arithmetic algorithms are given in
\cite{Ponder1991-zd, Silverman1990-hp}. 

\notesectref{Poly:Generalities:Sec} The ``vectorized subscript''
notation is fairly commonly used now, although some authors use
capital letters in the subscript to indicate the vector, \ie, instead
of $\vec{X}^{\vec{e}}$ they write $X_I^{e_I}$.  I believe this
notation was first introduced in Laurent {\SchwartzL}'s work on
distributions.

\notesectref{Poly:Fast:Sec} {\Altran} was the first computer algebra
system to use heap structures to represent polynomials
\cite{Brown1973-gv}.  Despite the performance improvement demonstrated
in {\Altran} \cite{Sundblad1973-wi}, most other systems have not chosen to
follow {\Altran}'s lead.  Largely, this is because of the increased
complexity of the algorithms and the infrequency with which these
methods provide significant performance improvements.  However,
similar approaches have proven useful for use in large, well contained
computations such as Gr\"{o}bner bases.

\notesectref{Poly:Expt:Sec} When raising a polynomial to a large
power {\em modulo another polynomial} a variant of the ``repeated
squaring technique'' can be used effectively.  This technique is
described in \sectref{FFac:Distinct:Sec} on page
\pageref{FFac:Distinct:Sec}.

\normalsize


