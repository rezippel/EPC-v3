%$Id: poly-factor.tex,v 1.1 1992/05/10 19:41:26 rz Exp rz $
\chapter{Irreducibility of Polynomials}
\label{Irred:Chap}

Let $F$ be a polynomial over an integral domain $R$, $F \in
R[\vec{X}]$.  As with rational integers, we say that $F$ is {\em
reducible}\index{reducible!polynomial} if there exist polynomials $G,
H \in R[\vec{X}]$, neither of which is in $R$, such that $F = G\cdot
H$.  Otherwise, $P$ is said to be {\em
irreducible}\index{irreducible!polynomial} or {\em
prime}.\index{prime!polynomial}

This chapter considers two types of questions about irreducibility:
First, When is a given polynomial irreducible? and, How does one test 
a polynomial for irreducibility?  This is similar to the
primality question for rational integers.  When $R$ is a finite field,
the factoring algorithms of the previous chapter allow us to quickly
determine when a univariate polynomial is irreducible.  The polynomial
primality question is most interesting for polynomials over $\Q$.

The second question is unique to multivariate polynomials.  Assume
$F(X_1, \ldots, X_n)$ is an irreducible polynomial.  For which values
of $t_1, \ldots, t_{n-1}$ of $R$ is $F(t_1, \ldots, t_{n-1}, X_n)$
irreducible?  As expected, $F(t_1, \ldots, t_{n-1}, X_n)$ is
irreducible for almost all values of $t_i$, but the proof is subtle.
This result is fundamental to the factoring algorithm discussed in
\chapref{Factoring:Chap}.

One way to determine if a polynomial is irreducible is to factor it,
but this can be exceedingly time consuming.  Thus it is quite useful
to have a quick test for determining when a polynomial is irreducible.
Deterministic irreducibility tests, including the well known {\em
Eisenstein criteria\/}, are discussed in \sectref{Irred:Test:Sec}.  In
addition, we present a generalization of the Eisenstein criteria that
provides some insight into its underlying principles and may aid in
developing multivariate generalizations.

Unfortunately, not all irreducible polynomials can be identified by
the Eisenstein criteria or its generalizations. An alternative is a
randomized test based on the {\em {\Chebotarev} density theorem},
which is discussed in \sectref{Cebotarev:Sec}.  This technique, which
is only applicable to univariate polynomials, actually identifies the
number of irreducible factors, without determining the factors
themselves.

There are two approaches to maintaining irreducibility under
substitutions.  The first, called the \key{Hilbert irreducibility
theorem}, maps an irreducible multivariate polynomial to an
irreducible univariate polynomial.  This approach is discussed in
\sectref{HIT:Sec}.  The second approach reduces to an irreducible
bivariate polynomial and is called \key{Bertini's theorem}.  It is
discussed in \sectref{HIT:Bertini:Sec}.

\section{Deterministic Irreducibility Testing}
\label{Irred:Test:Sec}

Although there is a polynomial time algorithm for factoring univariate
polynomials over the integers, it does not appear to be practical.
Thus efficient primality testing algorithms for polynomials are still
of interest.  This section presents  deterministic algorithms that
certify that a polynomial is irreducible.  However, these algorithms
cannot certify that  a polynomial is reducible.  In the next section,
we present a probabilistic algorithm that returns the number of
irreducible factors of a polynomial, for any polynomial. 

\paragraph{Modular Reduction}

The simplest approach to irreducibility testing takes advantage of the 
efficient factoring algorithms for polynomials over finite fields.  Let
$F(X) \in \Q[X]$.  If the image of $F$ in $\F_p[X]$ is irreducible
then $F(X)$ must also be irreducible.  Unfortunately, there are
irreducible polynomials over $\Q$ whose images modulo every prime are
reducible.  

For instance, consider the irreducible polynomial $X^4+1$, which has
the following factorizations in algebraic extensions of $\Q$.
\[
\begin{aligned}
X^4 + 1 &= (X^2 + \sqrt{-1})(X^2 - \sqrt{-1}), 
   &\mbox{over $\Q[\sqrt{-1}]$}\\
&= (X^2 + \sqrt{2} X + 1) (X^2 - \sqrt{2} X + 1), 
   & \mbox{over $\Q[\sqrt{2}]$}\\
&= (X^2 + \sqrt{-2} X - 1) (X^2 - \sqrt{-2} X - 1), 
   & \mbox{over $\Q[\sqrt{-2}]$.}\\
\end{aligned}
\]
If any of $-1$, $2$ or $-2$ have square roots in $\F_p$, then $X^4+1$
factors over $\F_p$, \eg,
\[
\begin{aligned}
X^4 +1 & = (X^2 + 2) (X^2 - 2) \pmod{5}\\
 & = (X^2 + 3X + 1) (X^2 - 3X + 1) \pmod{7}\\
 & = (X^2 + 3X - 1) (X^2 - 3X - 1) \pmod{11}
\end{aligned}
\]
So $X^4+1$ is irreducible if and only if $-1$, $2$ and $-2$ are not
quadratic residues modulo $p$.  This is not possible since
\[
\left(\frac{-1}{p}\right) \left(\frac{2}{p}\right) =
\left(\frac{-2}{p}\right).
\]
Thus $X^4+1$ has non-trivial factors modulo every prime number. 

More generally, consider the irreducible polynomial of degree
$2^n$\Marginpar{Why is this irreducible?}
\[
F_n(X) = \prod \left(X \pm \sqrt{2} \pm \sqrt{3} \pm \cdots \pm
\sqrt{p_n}\right),
\] 
where the product is over all combinations of the $\pm$ signs.
When factored over $\F_{p^2}$, $F_n(X)$ has only linear factors since
all quadratic extensions of $\F_p$ are isomorphic to $\F_{p^2}$.
Since $\F_{p^2}$ is a degree two extension of $\F_p$, $F_n(X)$ cannot
have any factors of degree greater than $2$ when factored over $\F_p$.
These polynomials are commonly called {\em {\SwinnertonDyer}
polynomials},\index{polynomial!Swinnerton-Dyer} although this behavior
was known to {\Hilbert}.

\paragraph{Eisenstein Criterion}

An alternative approach, that often works when the modular reduction
fails, is called the {\em {\Eisenstein} criterion}\index{Eisenstein
criterion} and is given in the following proposition.

\begin{proposition}[Eisenstein] \label{Eisenstein:Prop}
Let
\[
F(X) = f_0 X^n + f_1 X^{n-1} + \cdots + f_n
\]
be a polynomial over an integral domain $R$ and let $p$ be a prime in
$R$.  If $f_0 \not= 0 \pmod{p}$, $f_i = 0 \pmod{p}$ for $i = 1, 2,
\ldots, n$ but $f_n \not= 0 \pmod{p^2}$ then $F(X)$ is irreducible.
\end{proposition}

\begin{proof}
Assume $F$ is reducible, $F(X) = G(X) H(X)$ with
\[
\begin{aligned}
G(X) & = g_0 X^r + g_1 X^{r-1} + \cdots + g_r, \\
H(X) & = h_0 X^s + h_1 X^{s-1} + \cdots + h_s, \\
\end{aligned}
\]
and $r+s = n$.  Since $f_n = g_r \cdot h_s$ is divisible by $p$
exactly once, $p$ must divide either $g_r$ or $h_s$, but not both.
Without loss of generality, assume that $p\mid g_r$.  If $p$ divided
all of the coefficients of $g(X)$ then $p$ would divide $f_0$ which is
not possible.  Let $k$ be the largest integer such that $g_k$ is not
divisible by $p$, \ie, the coefficient of the lowest degree term that
is not divisible by $p$.  Then
\[
f_{k+s} = g_k h_s + g_{k+1}h_{s-1} + \cdots + g_r h_{s+k-r}.
\]
By assumption, $p$ divides $g_{k+1}, \ldots, g_r$ but
divides neither $g_k$ nor $h_s$.  Therefore, $p$ cannot divide
$f_{k+s}$ contrary to the assumptions.
\end{proof}

One useful application of \propref{Eisenstein:Prop} is to the
polynomial $(X^p - 1)/(X-1)$ where $p$ is a prime number.  Replacing
$X$ by $X+1$ we have
\[
\frac{(X+1)^p - 1}{(X+1)-1} = X^{p-1} + p X^{p-2} + \frac{p(p-1)}{2}
X^{p-3} + \cdots + p.
\]
Since $p$ divides $\binom{p}{i}$ for all $i$, the polynomial is
irreducible by the Eisenstein irreducibility criterion.  We state this
result in the following proposition.

\begin{proposition}[\Gauss] \label{Cyclotomic:Irred:Prop}
If $p$ is a prime number then
\[
x^{p-1} + x^{p-2} + \cdots + 1
\]
is irreducible over $\Q$.
\end{proposition}

\smallskip
There have been a number of extensions of the Eistenstein criterion
over the years.  In one direction, we can expand $F(X)$ in a series in
another polynomial $\phi(X)$, instead of just powers of $X$.  The
cyclotomic case was an example of expanding in powers of $X-1$.  The
full proposition is given below.

\begin{proposition}\label{Eisensetein:Propb}
Let $p$ be a prime of an integral domain $R$ and $\phi(X)$ a polynomial over
$R$, irreducible over $R/(p)$.  Let
\[
F(X) = f_0(X) \phi^n(X) + f_1(X) \phi^{n-1}(X) + \cdots + f_n(X)
\]
be a polynomial where $\deg f_i < \deg \phi$.  If
\[
F(X) = f_0(X) \phi^n(X) \pmod{p}
\]
but $f_n(X) \not= 0 \mod{p^2}$ then $F(X)$ is irreducible.
\end{proposition}

This proposition is proven in essentially the same manner as
\propref{Eisenstein:Prop}.  The requirement that $\phi(X)$ be
irreducible modulo $p$ is necessary.  Otherwise, $g_r$ and $h_s$ need
not be multiples of $p$ for $g_r h_s = f_n$ to be.

\smallskip
As an example of \propref{Eisenstein:Prop} consider the polynomial 
\[
\begin{aligned} 
F(X, Y) &= X^3 + X^3 Y^2 +X^2 Y + X^2 Y^3 + XY^2 + XY^3 + Y, \\
 & = X^3 (1 + Y^2) + X^2 (Y + Y^3) + X (Y^2 + Y^3) + Y.
\end{aligned} 
\]
Modulo $Y$, $F(X,Y) = X^3$ so $F(X, Y)$ is irreducible.

\begin{figure}
\begin{center} 
\setlength{\unitlength}{0.0100in}%
\begin{picture}(369,160)(86,430)
\thicklines
\put(360,520){\circle*{10}} \put(370,530){$X^3 Y^2$}
\put(360,440){\circle*{10}} \put(370,450){$X^3$}
\put(280,480){\circle*{10}} \put(290,490){$X^2 Y$}
\put(280,560){\circle*{10}} \put(290,570){$X^2 Y^3$}
\put(200,520){\circle*{10}} \put(210,520){$X Y^2$}
\put(200,560){\circle*{10}} \put(210,570){$X Y^3$}
\put(120,480){\circle*{10}} \put(130,490){$Y$}
\put( 86,487){\line( 6,-1){324}}
\put(120,590){\line( 0,-1){160}}
\put(110,440){\line( 1, 0){345}}
\end{picture}
\end{center}
\caption{Newton Polygon for $X^3 (1 + Y^2) + X^2 (Y + Y^3) + X (Y^2
Y^3) + Y$\label{Ore:NewtonPoly:Fig}} 
\end{figure}

\medskip
We can represent the monomials that arise in $F(X, Y)$ as points in 
a lattice of exponent pairs as shown in \figref{Ore:NewtonPoly:Fig}.
The diagonal line drawn is an edge of the \keyi{Newton polygon} of the
polynomial.  The Newton polygon is constructed by beginning with the set of
points corresponding to the polynomial and adjoining the points $(+\infty, 0)$ 
and $(0, +\infty)$.  This set is denoted by ${\cal N}_F$.  Thus for $F(X, 
Y)$ we have
\[
{\cal N}_F = \{(0, 1), (0, \infty), (1, 2) (1, 3), (2, 1) (2, 3), (3, 0), (3, 2),
(\infty, 0) \}.
\]
The \key{convex hull} of ${\cal N}_F$ is the \keyi{Newton polygon} of $F$.

In essence, Eisenstein's criterion almost says that if the Newton
polygon of a polynomial has a single edge and the edge has a certain
slope then the polynomial is irreducible.  Consider the two
polynomials $F_1(X) = X^2 - 2$ and $F_2(X) = X^2 - 4$.  In both cases,
the Newton polygon has a single slope, but $F_1$ is irreducible and
$F_2$ is not.  The difference is that there are no lattice points on
the diagonal edge of the Newton polygon of $F_1$, while $(1, 1)$ lies
on the diagonal of $F_2$.

The following proposition, due to {\Ore} \cite{Ore1923-tf}, makes this
observation precise.

\begin{proposition}[\Ore]
Let $p$ be a prime of the ring $R$, $\phi(X)$ an irreducible
polynomial over $R/(p)$.  Let $F(X)$ be a polynomial over $R$, such
that
\[
F(X) = f_0(X) \phi^n(X) + f_1(X) \phi^{n-1}(X) + \cdots + f_n(X),
\]
$\deg f_i < \deg \phi$ and
\[
F(X) = f_0(X) \phi^n(X) \pmod{p}
\]
Further, assume that the Newton polygon of $F(X)$, corresponding to $(p,
\phi(X))$, has precisely one diagonal edge.  Let $m$ denote the number
of lattice points on this diagonal edge.  The number of irreducible
factors of $F(X)$ is no more than $m -1$.
\end{proposition}

\section{Counting Prime Factors}
\label{Cebotarev:Sec}

A probabilistic approach to primality testing of univariate
polynomials was suggested by {\Weinberger} \cite{Weinberger1984-cx}.  The
mathematics underlying this approach is rather deep, so we begin with a
statement of the algorithm and discuss the rationale behind it before
looking at its behavior rigorously.

Assume we want to determine the number of irreducible factors of
$F(X)$ over a $\Q$.  Furthermore, assume that $F(X)$ is square free.
The routine \keyw{CountFactors} returns the number of irreducible
factors of $F$ over $\Q$ for sufficiently large $N$.

\begindsacode
CountFactors ($F(X)$, $N$) := \{ \\
\> $L \leftarrow 0$; $n \leftarrow N$; \\
\> loo\=p  for $p \le N$ and prime do \{ \\
\>\> wh\=en $\gcd(F(X), F'(X)) = 1 \mod{p}$ do \{ \\
\>\>\> $L \leftarrow L + \deg(\gcd(F(X), X^p - X) \mod{p})$; \\
\>\>\> $n \leftarrow n + 1$; \\
\>\>\> \} \\
\>\> \} \\
\> return($L/n$); \\
\> \}
\enddsacode

This routine counts the total number of linear factors when $F(X)$ is
factored modulo a set of primes.  The average number of linear factors
per prime is the number of irreducible factors of $F(X)$ over $\Q$.

As an example consider the polynomial $X^4 + 1$ discussed earlier.
The factorization of $X^4 + 1 $ modulo $p$ depends on the quadratic
residuacity of $-1$ and $2$.  There are four cases.  If both $-1$ and
$2$ are quadratic residues modulo $p$ then $X^4+1$ will have four
linear factors:
\[
X^4 +1 = \prod \left(X \pm \frac{\sqrt{2}}{2}(1 \pm \sqrt{-1})\right).
\]
Otherwise the $X^4+1$ will have $2$ quadratic factors modulo $p$.

If we assume that the residuacity of $-1$ and $2$ are uncorrelated,
then each of the four cases will be equally likely and the average
number of linear factors per prime will be $1$ as expected.  Similar
reasoning can be applied to the {\SwinnertonDyer}
polynomials.\index{polynomial!Swinnerton-Dyer}  

\medskip
A detailed analysis of this algorithm would involve a long detour into
algebraic and analytic number theory.  In the following brief
paragraphs we quote some of the relevant facts in order to provide
some intuition about what is happening, but prove nothing
rigorously.  

Assume $F(X)$ is a monic irreducible polynomial with coefficients in
$\Z$.  Denote by $K = \Q[\alpha]/(F(\alpha))$ the algebraic number field
generated by a zero of $F(X)$.  Let $A$ be the \keyi{ring of integers}
of $K$, \ie, the set of elements of $K$ whose minimal polynomial over
$\Z$ is monic.  Further assume that $A = \Z[\alpha]$.  So we have the
following rings
\[
\begin{tikzcd}[column sep=small]
K =\Q[\alpha] \arrow{r}\arrow{d}   & A =\Z[\alpha]\arrow{d}\\
\Q\arrow{r} & \Z
\end{tikzcd}
\]
Now let $\mathfrak{p} = (p)$ be a prime ideal of $\Z$.  As an ideal of
$A$, $\mathfrak{p}A$ may no longer be prime and may factor into several
primes, \ie, 
\[
\mathfrak{p}A = \mathfrak{P}_1^{e_1} \mathfrak{P}_2^{e_2} \cdots \mathfrak{P}_r^{e_r}.
\]
In each case $\mathfrak{P}_i \cap \Z = \mathfrak{p}$.  The $e_i$ are called
the {\em ramification indices}\index{ramification index} of $\mathfrak{P}_i$
over $\mathfrak{p}$.  The ramification index can be greater than $1$ only
if $p$ divides the discriminant of $K$, \ie, $\res_X(F(X), F'(X))$.

Again for simplicity, assume that the $e_i$ are all equal to $1$.
$A/\mathfrak{P}_i$ is a finite field of characteristic $p$ and the number
of elements in it is called the {\em norm} of $\mathfrak{P_i}$,
\index{norm!of an ideal} which is written ${\bf N} \mathfrak{P}_i$.  If
${\bf N} \mathfrak{P}_i = p^{f_i}$ then $f_i$ is called the {\em degree}
of $\mathfrak{P_i}$.\index{degree!of an ideal}

The key relationship to factoring is a proposition due to {\Dedekind}
\cite{Dedekind1878-xs} that relates the factorization of the prime $(p)$ to the
factorization of $F(X)$ modulo $p$.  In particular, if
\[
F(X) = F_1(X) F_2(X) \cdots F_r(X) \pmod{p}
\]
then
\[
\mathfrak{p}A = \mathfrak{P}_1 \mathfrak{P}_2 \cdots \mathfrak{P}_r,
\quad\mbox{where $\mathfrak{P}_i = (p, F_i(\alpha))$.}
\]
The degree of $\mathfrak{P}_i$ is the degree of $F_i$.

The result underlying the primality testing algorithm is that the
number of linear primes in $A$ with norms less than a bound $B$ is
about the same as the number of primes in $\Z$ less than $B$.  This
result follows from a comparison of the \keyi{prime number theorem}
for integers with the prime number theorem for algebraic number
fields.  A good source of additional details on this result is
{\Heilbronn}'s paper \cite{Heilbronn1967-uq}.

\section{Hilbert Irreducibility Theorem}
\label{HIT:Sec}
\index{Hilbert irreducibility theorem|(}

Let $F(X_1, \ldots, X_r, Y_1, \ldots, Y_s)$ be an irreducible
polynomial in $r+s$ variables over a field $k$.  A set of $t_i \in k$
such that $F(X_1, \ldots, X_r, t_1, \ldots, t_s)$ is irreducible is said
to be a \keyi{Hilbertian point}.  The classical {\em Hilbert
irreducibility theorem} asserts that when $k$ is $\Q$ or a finite
algebraic extension of $\Q$ almost all sets of $t_i$ are Hilbertian.

This is not the case when $k$ is a finite field.  Consider, for
instance, $F(X, Y) = X^2 - Y$.  If $k$ is a finite field then $F(X,
t)$ is reducible if $t$ is a quadratic residue, \ie, for half of all
possible candidates.  When $k = \Q$, however, the number of $t$, with 
that $|t| < N$, such that $F(X, t)$ is reducible is approximately
$O(\sqrt{N})$.  In the finite field case the density of Hilbertian
points is $1/2$, while when $k = \Q$ the density of non-Hilbertian
points goes to zero.

The finite field case can be saved however by looking for
substitutions for the $Y_i$ that are linear in a new variable $T$.
This approach gives sharp bounds on the density of ``Hilbertian
points'' and the result is called \keyi{Bertini's theorem}.  The
Hilbert irreducibility theorem is a number theoretic statement.  The
functional version, Bertini's theorem, is quite different.  The
difference between the integer and polynomial form of the {\ABCconj}
discussed in \sectref{FLT:Sec} is relevant here.  Bertini's theorem is
discussed in \sectref{HIT:Bertini:Sec}.

\medskip
In symbolic computing, the Hilbert irreducibility theorem can be used
in several applications.    Let $F(X_1, \ldots, X_n)$ be a multivariate
polynomial whose complete factorization is
\[
F(X_1, \ldots, X_n) = F_1 \cdots F_{\ell}.
\]
By the Hilbert irreducibility theorem, if we choose random values for
$X_1, \ldots, X_{n-1}$ the images of the $F_i$ will each remain
irreducible.  Thus we can count the number of irreducible factors of a
multivariate polynomial by using the Hilbert irreducibility theorem in
tandem with the techniques of \sectref{Cebotarev:Sec}.  As we shall
see in \chapref{Factoring:Chap}, this same type of approach is also
used in all modern multivariate factoring algorithms.

Let $F(X_1, \ldots, X_n)$ be a monic irreducible polynomial of degree
$d$ in $X_1$ over a Hilbertian field $K$.\Marginpar{Define what a
Hilbertian field is.}  Let $G$ and $H$ be monic
polynomials in $X_1$ of degrees $r$ and $s$ respectively, $r+s = d$,
with coefficients $g_i$ and $h_i$.  Assume $F = G \cdot H$, so the
$g_i$ and $h_i$ are irrational algebraic functions of $X_2, \ldots,
X_n$ satisfying the following equations.
\begin{equation} \label{HIT:Algebraic:Eq}
\begin{aligned}
f_1(X_2, \ldots, X_n) & = g_1 + h_1, \\
f_2(X_2, \ldots, X_n) & = g_2 + g_1  h_1 + h_2, \\
& \vdots
\end{aligned}
\end{equation}

Let $x_2, \ldots, x_n$ be elements of $K$ and assume that $F(X_1, x_2,
\ldots, x_n)$ has factors of degree $r$ and $s$.  The coefficients
of the two factors are the values of the algebraic functions $g_i$ and
$h_i$ at integer values of $X_2, \ldots, X_n$.  The classical form of
the Hilbert irreducibility theorem says that the density of integral
points at which an irrational algebraic function takes on integral
values has density zero.

Alternatively, we can view \eqnref{HIT:Algebraic:Eq} as a system of
diophantine equations in $X_2, \ldots, X_n$, the $g_i$ and the $h_i$.
Any integer value of all of these variables induces a univariate
factorization of the irreducible polynomial $F$.  The classical form
of the Hilbert irreducibility theorem indicates how many solutions
there can be to \eqnref{HIT:Algebraic:Eq}, subject to the condition
that all the $X_i$, $g_i$ and $h_i$ have absolute value less than $N$.

Restricting ourselves to a single diophantine equation in two
variables we can gain insight into what occurs.  Linear diophantine
equations are not really appropriate since then $F(X_1, \ldots, X_n)$
would be reducible.  For quadratic equations we have two choices.  If
the equation is linear in one variable and quadratic in the other,
\eg, $X^2 - Y$, then the number of solutions grows like $O(\sqrt{N})$.
If both variables appear quadratically, \eg, $X^2 - D Y^2 = 1$, then
the number of solutions grows like $O(\log N)$ (recall
\sectref{Pell:Equation:Sec}). 

For cubic equations of the form 
\[
Y^2 = 4X^3 + a X + b,
\]
\ie, for {\em elliptic curves},\index{elliptic curve} the density of
solutions grows like $O(\log \log N)$.  Higher order equations have
only a finite number of solutions by {\Faltings}' theorem
\cite{Faltings1984-fj,Faltings1983-fm,Faltings1986-wb}.

This discussions suggests that the irreducible bivariate polynomial
that is reducible at the most points is $F(X_1, X_2) = X_1^2 - X_2$,
which has $O(\sqrt{N})$ non-Hilbertian points of magnitude less than
$N$.  In fact this is also the upper bound on the number of
non-Hilbertian points as the following theorem of {\Fried} shows
\cite{Jarden1982-wn,Fried1974-yu}:

\begin{proposition}[Fried]
Let $f(X, Y)$ be an irreducible polynomial over $\Q$ and let $R(N)$
denote the number of integers $x \in \Z$ with $|x| < N$ such that
$f(x, Y)$ is reducible.  Then
\[
R(N) < c \cdot N^{1/2}
\]
where $c$ depends only on $f$. 
\end{proposition}


The following proposition, due to {\CohenS} \cite{Cohen1981-ub}, is the
sharpest version of the Hilbert irreducibility theorem known to the
author.  Its proof is rather involved and well beyond the scope of this
text.

\begin{proposition}[Cohen] 
Suppose that $K/k$ is a finite extension of number fields and let $A$
denote the ring of integers of $K$ and $B$ the ring of integers of
$k$.  Let $f(X_1, \ldots, X_r, t_1, \ldots, t_s)$, $r, s \ge 1$ be an
irreducible polynomial of total degree not greater than $d$ over $A$
and let $g(t_1, \ldots, t_s)$ be a non-zero polynomial of total degree
not greater than $d$ in $A$.  Then with at most $c(d, r, s, K) \gamma
N^{s - \frac{1}{2}} \log N$ exceptions, all $f(X_1, \ldots, X_r,
\alpha_1, \ldots, \alpha_s)$ are irreducible and $g(\alpha_1, \ldots,
\alpha_s) \not= 0$, where $\alpha_i \in B$.  This holds
\begin{enumerate}
\item with $\gamma = |f|^{c/3}$ provided $N > (|f| |g|)^c$, or
\item with $\gamma = 1$ provided $N > e^{c(\|f\|^2 + \|g\|^2)}$, or
\item with $\gamma = 1$ provided the generalized Riemann hypothesis
holds and $N > \|f\|^4 + \|g\|^4$
\end{enumerate}
\end{proposition}

We do not really need the full generality of this theorem, and instead
use the following simpler proposition.

\begin{proposition} \label{HIT:Cohen:Prop}
  Let $F(X_1, \ldots, X_n, Y)$ be an irreducible polynomial of total
  degree $d$ over $\Q$ and let $R(N)$ denote the number of $n$-tuples
  over $\Z$ with $|x_i| < N$ such that $F(x_1, \ldots, x_n, Y)$ is
  reducible.  Then for $N > |F|^{c_1}$
\[
R(N) < c_3(n, d) |F|^{c_3/3} \cdot N^{n - 1/2} \log N,
\]
where $c_3$ depends only on the degree of $F$ and $c_1$ and $c_2$ are
effectively computable.
\end{proposition}

For applications, it would be very useful to determine the constants
$c_i$\Marginpar{Need a proof of the above proposition, and a response
  to {\tt sptung@cchp01.cc.cycu.edu.tw}}
in the above theorem precisely.  For most applications, it would
suffice if the following conjecture were true.

\begin{conjecture} \label{HIT:Conj}
In \propref{HIT:Cohen:Prop}, there exist absolute constants $c_1$,
$c_2$ such that $c(d) < c_1 d^{c_2}$, \ie
\[
R(d,n,N) < c_1 d^{c_2} \cdot D \cdot  N^{n-\frac{1}{2}} \log N.
\]
\end{conjecture}

We also note the following intriguing result of Sprind\v{z}uk
\cite{Sprindzuk1981-wc,Fried1985-jt}.

\begin{proposition}
Let $F(X, Y)$ be an irreducible polynomial over $\Q$ and 
let $H$ denote the set of integers:
\[
H = \{ \left\lfloor e^{\sqrt{\log \log m}}\right\rfloor + m! \cdot
2^{m^2} \mid m = 1, 2, \ldots \}.
\]
Then $F(h, Y)$ is reducible for only a finite number of elements of
$H$.
\end{proposition}

\index{Hilbert irreducibility theorem|)}

\section{Bertini's Theorem}
\label{HIT:Bertini:Sec}

\index{Bertini's theorem|(}

Geometrically, Bertini's theorem \cite{Bertini1882-gp} says that the
intersection of an irreducible algebraic variety with a hypersurface,
not necessary a plane, is almost always an irreducible subvariety,
\eg, the intersection of sphere with a plane is always circle and
never consists of more than one component.  The general form of
Bertini's theorem can be found in most books on algebraic geometry
\cite{Jouanolou1983-om,Shafarevich2012-mf,Hartshorne2014-cr,Lang2013-vg}.

In this section we prove a simple version of Bertini's theorem, due to
{\Kaltofen} \cite{Kaltofen1985-ji} that is in a form most useful for
factoring polynomials.  The ideas behind this version are quite
simple.  We begin with an overview using an irreducible, square free
polynomial in three variables $F(X, Y, Z)$ over a field $K$.  We want
to find $a_1, b_1, a_2, b_2$, from a finite subset ${\cal S}
\subsetneq L$, such that $F(a_1+b_1 T, a_2+b_2 T, Z)$ is an
irreducible bivariate polynomial.

This is done in two steps. First, we assume that $F(0, 0, Z)$ is
square free.  If this is not the case, then find $a_1, a_2$ such that
$F(a_1, a_2, Z)$ is square free and replace $F(X_1, \ldots, X_v)$ by
$F(X + a_1, Y + a_2, Z)$.  Such $a_i$ can be found by looking for
$a_i$ that do not make the discriminant of $F$ vanish.

Now, each factorization  of $F(0, 0, Z)$ over $R$
\[
F(0, 0, Z) = G(Z) \cdot H(Z),
\]
can be lifted by Hensel's lemma to a factorization over $R_{(X,Y)}$:
\[
F(X, Y, Z) = \hat{G}(Z) \hat{H}(Z).
\]
At least one coefficient of $\hat{G}(Z)$ must be an infinite series in
the $X, Y$.  Let that coefficient be 
\[
c(X, Y) = c_0 + (c_{10} X + c_{01} Y) + 
(c_{20} X^2 + c_{11} XY + c_{01} Y^2) + \cdots
\]
If we can find values for $b_1$ and $b_2$ such that $c(b_1 T, b_2 T)$ is
still an infinite series, then $F(b_1 T, b_2 T, Z)$ will still be
irreducible.  

This again is easy.  Replace $X$ by $R T$ and $Y$ by $S T$, we get
\[
c(RT, S T) = c_0 + c_1(R, S) T + c_2(R, S) T^2 + c_3(R, S) T^3 + \cdots,
\]
where the $c_i$ are polynomials in $R$ and $S$.  Assume that $c_k(R,
S)$ is not identically zero.  If $b_1$ and $b_2$ are chosen such that
$c_k(b_1, b_2)$ does not vanish then $G(Z) H(Z)$ does not lift to a
factorization of $F(b_1T, b_2T, Z)$.  If this is true for all
factorizations of $F(0, 0, Z)$ then $F(b_1T, b_2T, Z)$ is irreducible.
The necessary condition is thus that $b_1$, $b_2$ not be the zeroes of
a certain polynomial.

All that remains is to work out the details, and in particular, make
precise how large ${\cal S}$ needs to be.  We begin with the estimate
for the $a_i$.


\begin{proposition} \label{SquareFree:HIT:Prop}
Let $F(X_1, \ldots, X_v, Z)$ be a square free irreducible polynomial
over an integral domain $A$, of degree $d$ in $Z$ and total degree $D$
in the $X_i$.  Let $f_0(X_1, \ldots, X_v)$ be the leading coefficient
of $F$ with respect to $Z$.  Assume that
\[
\frac{\partial F(X_1, \ldots, X_v, Z)}{\partial Z} \not= 0.
\]
If ${\cal S}$ is a subset of $A$ of cardinality $B$
then 
\[
\begin{aligned}
{\cal P}(\mbox{$F(a_1, \ldots, a_v, z)$ is square free} 
\wedge f_0(a_1, \ldots, a_v) \not= 0\mid a_i \in {\cal S}) \\
\qquad \le 1 -\frac{(2d - 1)D}{B}.
\end{aligned}
\]
\end{proposition}

\begin{proof}
Define 
\[
R(X_1, \ldots X_v) = \res_Z(F(X_1, \ldots, X_v, Z), \frac{\partial
F}{\partial Z}(X_1, \ldots, X_v, Z)).
\]
If $F(a_1, \ldots a_v, Z)$ is not square free then $R(a_1, \ldots,
a_v) = 0$.  Furthermore, the leading coefficient of $F$, $f_0(X_1,
\ldots, X_v)$, divides $R(X_1, \ldots, X_v)$.  So, if $f_0(a_1, \ldots,
a_v) = 0$ then $R$ also vanishes.  Thus
\[
\begin{aligned}
{\cal P}(\mbox{$F(a_1, \ldots, a_v, z)$ is square free} 
\wedge f_0(a_1, \ldots, a_v) \not= 0\mid a_i \in {\cal S}) \\
\qquad \le 1 - {\cal P}(R(a_1, \ldots, a_v) = 0 \mid a_i \in {\cal S}).
\end{aligned}
\]

By \propref{Resultant:Weight:Prop}, $R$ is a form of weight $d+d-1$ in
the coefficients of $F$.  Therefore the total degree of $R$ in the
$X_i$ is bounded by $(2d-1)D$.  By \propref{Prob:Total:Zero:Prop}
\[
{\cal P}(R(a_1, \ldots, a_v) = 0 \mid a_i \in {\cal S}) \le 
\frac{(2d-1)D}{B}.
\]
\end{proof}

\begin{proposition} \label{Monic:KIT:Prop}
Let $F(X_1, \ldots, X_v, Z)$ be a monic polynomial of $R[Z]$, where $R
= A[X_1, \ldots, X_v]$ and $A$ is an integral domain.  Let the degree
of $Z$ in $F$ be $d$ and the total degree of the $X_1, \ldots, X_v$ in
$f$ be $D$.  Further assume that $F$ is irreducible and that $F(0,
\ldots, 0, Z)$ is square free with respect to $Z$. Let ${\cal S}$ be a
subset of $A$ of cardinality $B$.  Then
\[
{\cal P}(\mbox{$F(b_1 T, \ldots, b_v T, Z)$ is irreducible over $A[T, Z]$}
   \mid b_i \in {\cal S}) \ge 1 - \frac{2D 2^{d}}{B}.
\]
\end{proposition}

\begin{proof}
If $F(0, \ldots, 0, Z)$ is irreducible over $A$ then for every choice
of $b_i$, $F(b_1 T, \ldots, b_v T, Z)$ will also be irreducible.  So,
we can assume that $F(0, \ldots, 0, Z)$ is reducible.  Let $G(Z)$ be a
factor of $F(0, \ldots, 0, Z)$ of degree $r$ and $H(Z)$ be its
corresponding cofactor of degree $s$, $r+s=d$:
\[
\begin{aligned}
G(Z) & = Z^r + g_1 Z^{r-1} + \cdots + g_r, \\
H(Z) & = Z^s + h_1 Z^{s-1} + \cdots + h_s. 
\end{aligned}
\]
Since $G$ and $H$ are relatively prime we can use
\propref{Hensel:Lemma:Prop} to lift $G$ and $H$ to factors of $F(X_1
T, \ldots, X_v T, Z)$ over $R_{(T)}$, $\hat{G}$ and $\hat{H}$
\[
\begin{aligned}
\hat{G}(Z) & = Z^r + \hat{g}_1 Z^{r-1} + \cdots + \hat{g}_r, \\
\hat{H}(Z) & = Z^s + \hat{h}_1 Z^{s-1} + \cdots + \hat{h}_s. 
\end{aligned}
\]
Each of $\hat{g}_i$ and $\hat{h}_i$ can be written in the form
\[
\hat{g}_i = g_i + g_{i,1}(X_1, \ldots, X_v)T + g_{i,2}(X_1, \ldots, X_v) T^2
 + \cdots,
\]
where the $g_{i,j}$ and $h_{i,j}$ are homogeneous polynomials of total
degree $j$ in the $X_i$.

Since $F$ is irreducible, at least one of the $\hat{g}_i$ must
represent an algebraic function and thus must be an infinite power
series.  Assume that $\hat{g}_i$ is such an infinite series.  We
claim that one of
\[
g_{i,D+1}, g_{i,D+2}, \ldots, g_{i,2D}
\]
is not identically zero.  

To see this, assume that this is not the case, \ie, that none of the
$\hat{g}_i$ or $\hat{h}_i$ has terms of total degree $k$, $D < k \le
2D$.  So we can write
\[
\begin{aligned}
\hat{G}(Z) & = G^{[D]}(Z) + T^{2D+1} \bar{G}(Z), \\
\hat{H}(Z) & = H^{[D]}(Z) + T^{2D+1} \bar{H}(Z),
\end{aligned}
\]
where $G^{[D]}$ and $H^{[D]}$ represent the terms of $\hat{G}$ and
$\hat{H}$ of total degree less than or equal to $D$.

Since $F(X_1, \ldots, X_v, Z) = \hat{G} \hat{H}$,
\[
0 = \left[ G^{[D]} H^{[D]} - F \right]
+ T^{2D+1} \left[ G^{[D]} \bar{H} +
H^{[D]} \bar{G} + T^{2D+1} \bar{G}\bar{H}\right].
\]
Since there can be no cancellation between the two bracketed
expressions $f$ must equal $G^{[D]}H^{[D]}$, contradicting the
assumption that $F$ is irreducible. 

Therefore, for each factor $G(Z)$ of $F(0, \ldots, 0, Z)$ there exists
a polynomial $W_g(X_1, \ldots, X_v)$ that must vanish at $X_1  = b_1, \ldots,
X_v = b_v$ for $F(b_1 T, \ldots, b_v T, Z)$ to be divisible by a polynomial
whose image modulo $T$ is $G(Z)$.  The total degree of $W_g$
is $\le 2D$.  By \propref{Prob:Total:Zero:Prop} we have
\[
{\cal P}(W_g(b_1, \ldots, b_v) = 0 \mid b_i \in {\cal S}) \le
\frac{2D}{B}.
\]

Since there are at most 
\[
d + \binom{d}{2} + \cdots + \binom{d}{d-1} + 1 =2^d -1
\]
divisors of $F(0,\ldots, 0, Z)$ we have the proposition.
\end{proof}

For applications we need to strengthen \propref{Monic:KIT:Prop} in two
ways.  We need to deal with non-monic polynomials and we need to deal
with the case when the coefficient domain has positive characteristic.
The non-monic case can be dealt with very easily.  

If $F$ is not monic, then we can write
\[
\begin{aligned}
F(X_1, \ldots, X_v; Z)\\
 \qquad = f_0(X_1, \ldots, X_v) Z^d + f_1(X_1, \ldots, X_v) Z^{d-1} + 
  \cdots + f_d(X_1, \ldots, X_v).
\end{aligned}
\]
Define the polynomial $\bar{F}$ by
\[
\bar{F}(X_1, \ldots, X_v; Z)
  = Z^d + f_0 f_1 Z^{d-1} + f_0^2 f_1 Z^{d-2} + \cdots + f_0^{d-1} f_d.
\]
$\bar{F}$ has been defined such that
\[
\bar{F}(X_1, \ldots, X_v; f_0 Z) = f_0^{d-1} F(X_1, \ldots, X_v; Z).
\]
By construction, $\bar{F}(b_1 T, \ldots, b_v T; Z)$ is reducible if
and only if the polynomial $F(b_1 T, \ldots, b_v T; Z)$ is also.
Since $\bar{F}$ is square free we apply \propref{Monic:KIT:Prop}.  The
total degree of the $X_i$ in $\bar{F}$ is bounded by $dD$.  This gives
the following proposition

\begin{proposition} \label{NonMonic:KIT:Prop}
Let $F(X_1, \ldots, X_v, Z)$ be an irreducible polynomial of $R[Z]$,
where $R = A[X_1, \ldots, X_v]$ and $A$ is an integral domain.  Let
the degree of $Z$ in $F$ be $d$,
\[
\frac{\partial F}{\partial Z} \not= 0
\]
and the total degree of the $X_1, \ldots, X_v$ in $F$ be $D$.  Let
${\cal S}$ be a subset of $A$ of cardinality $B$.  Then
\[
\begin{aligned}
{\cal P}(\mbox{$F(a_1 + b_1 T, \ldots, a_v + b_v T, Z)$ is irreducible over $A[T, Z]$}
   \mid b_i \in {\cal S})\\
 \qquad \ge 1 - \frac{4dD 2^{d}}{B}.
\end{aligned}
\]
\end{proposition}

\begin{proof}
The probability that $F(a_1, \ldots, a_v, Z)$ is square free and
$f_0(a_1, \ldots, a_v)$ is different from zero is greater than or
equal to
\begin{equation}\label{HIT:Sqfr:Eq}
1 - \frac{(2d -1) D}{B}.
\end{equation}
Define $\bar{F}$ by
\[
\begin{aligned}
\bar{F}(X_1, \ldots, X_v, f_0(X_1 + a_1, \ldots, X_v + a_v) Z) \\
  \qquad\qquad= f_0^{d-1}(X_1 + a_1, \ldots, X_v + a_v) 
   F(a_1 + X_1, \ldots, a_v + X_v, Z).
\end{aligned}
\]
$\bar{F}(X_1, \ldots, X_v, Z)$ is monic, $\bar{F}(0, \ldots, 0, Z)$ is
square free  with probability greater than or equal to
\eqnref{HIT:Sqfr:Eq}.\Marginpar{I don't like this type of reference.}
The total degree of $X_i$ in $\bar{F}$ is bounded by $dD$.  Thus the
probability that $\bar{F}$ is irreducible is greater than or equal to
\[
\left[1 - \frac{(2d-1)D}{B}\right] \cdot
\left[1 - \frac{2 d D 2^d}{B} \right] \ge 1 - \frac{4dD2^d}{B}.
\]
\end{proof}

The positive characteristic case is a bit delicate.  We begin with a
useful proposition.

\begin{proposition}\label{KIT:Lift:Prop}
Let
\[
F(Z) = f_0 Z^d + f_1 Z^{d-1} + \cdots + f_d \in K[Z]
\]
be an irreducible polynomial over an integral domain $K$ of
characteristic $p$.  Assume that for some $i$, $f_i$ is not a perfect
$p$-th power.  Then $F(Z^{p^{\lambda}})$ is irreducible for all
$\lambda$.
\end{proposition}

\begin{proof}
We use induction on $\lambda$.  The proposition is true for $\lambda =
0$ by assumption.  Assume that $F(Z^{p^{\mu-1}})$ is irreducible but
$F(Z^{p^{\mu}})$ is reducible.  Then there exist relatively prime
polynomials $P$ and $Q$ such that
\[
F(Z^{p^{\mu}}) = P^r(Z) \cdot Q(Z),
\]
where $r\ge 2$ if $Q$ is equal to $1$.  Differentiating this equation
and dividing out $P^{r-1}$ gives
\[
0 = r P' \cdot Q + P \cdot Q'.
\]
Since $P$ and $Q$ are relatively prime, each term on the right hand
side of this equation must vanish.  Since $r P' = 0$, either $r$ is a
multiple of $p$ or $P' = 0$.  If $r$ is equal to $p\ell$ the we can
write $P^r(Z) = P^{\ell}(Z^p)$.  If $P'$ is equal to zero then we can
write $P^r(Z) = \bar{P}^r(Z^p)$.  In either case there exists a
polynomial $\bar{P}$ and an integer $\bar{r}$ such that
\[
P^r(Z) = \bar{P}^{\bar{r}}(Z^p).
\]
Since $P \not=0$, $Q' = 0$.  Thus there exists a polynomial $\bar{Q}$
such that $Q(Z) = \bar{Q}(Z^p)$.  (This holds even if $Q = 1$.)

Letting $\bar{Z} = Z^p$ gives
\[
F(\bar{Z}^{p^{\mu-1}}) = \bar{P}^{\bar{r}}(\bar{Z}) \cdot
\bar{Q}(\bar{Z}).
\]
By the induction hypothesis, $\bar{Q} = 1$ and $\bar{r} = 1$.  That
is,
\[
F(Z^{p^{\mu}}) = P^p(Z).
\]
But this means that each of the $f_j$ is a perfect $p$-th power.
\end{proof}

Now we prove the unconditional version of the proposition. 

\begin{proposition}[Kaltofen] \label{KIT:Prop}
Let $F(X_1, \ldots, X_v, Z)$ be an irreducible polynomial of $R[Z]$, where $R =
A[X_1, \ldots, X_v]$ and $A$ is a perfect field.  Let the degree of
$Z$ in $F$ be $d$ and the total degree of the $X_1, \ldots, X_v$ in
$F$ be $D$.  Then
\[
\begin{aligned}
{\cal P}(\mbox{$F(a_1+ b_1T, \ldots, a_v+b_vT, Z)$ is reducible over $A[T, Z]$}
   \mid a_i, b_i \in {\cal S}) \\
 \quad \ge 1 - \frac{4 dD 2^{d}}{B}.
\end{aligned}
\]
\end{proposition}

\begin{proof}
If the characteristic of $A$ is zero or the derivative of $F$ with respect
to $Z$ does not vanish, then the proposition is just a
restatement of \propref{NonMonic:KIT:Prop}.  Thus we can assume that 
characteristic of $A$ is $p$, and we can write $F(X_1, \ldots, X_v; Z)
= \bar{F}(X_1, \ldots, X_v; Z^{p^{\mu}})$ and
\[
\bar{F}(X_1, \ldots, X_v; Z) = f_0 Z^{\delta} 
    + f_1 Z^{\delta-1} + \cdots + f_{\delta},
\]
$d = \delta p^{\mu}$.  By \propref{NonMonic:KIT:Prop}
the probability that $\bar{F}(a_1+b_1T, \ldots, a_v+b_vT; Z)$ is
irreducible is greater than or equal to
\[
1 - \frac{4 \delta D 2^{\delta}}{B} \ge 1 - \frac{4d D 2^{d}}{pB}
\ge 1 - \frac{2dD2^d}{B},
\]
since $p \ge 2$.  In order to apply \propref{KIT:Lift:Prop} we also
require that one of
\[
f_i(a_1 + b_1T, \ldots, a_v + b_vT)
\]
not be a perfect $p$-th power.  Since $F$ is irreducible we can
assume that $f_r(X_1, \ldots, X_v)$ is not a perfect $p$-th power.
We now introduce the additional requirement that
\[
\frac{\partial f_i(a_1 + b_1T, \ldots, a_v + b_vT)}{\partial T} \not =
0.
\]
To ensure this we only need to ensure that the coefficient of some
power of $T$ does not vanish.  That coefficient is a polynomial in the
$a_i$ and $b_i$ whose total degree is bounded by $D$.  So, by
\propref{Prob:Total:Zero:Prop} the probability that it is non-zero is
greater than $1 - D/B$.  Thus the probability that $F$ is irreducible
is greater than
\[
\left[1 - \frac{2dD2^d}{B}\right] \cdot
\left[1 - \frac{D}{B} \right] \ge
 1 - \frac{4dD2^d}{B}.
\]
\end{proof}

\index{Bertini's theorem|)}

\section*{Notes}

\small

\notesectref{Irred:Test:Sec}
{\Musser} \cite{Musser1978-ez} tried to extend the modular irreducibility
test by combining the results of factorizations modulo several primes.
Thus if a polynomial factored into $3$ quadratic factors modulo one
prime and $2$ cubic factors modulo a second it must be irreducible
over $\Q$.  The Swinnerton-Dyer
polynomials\index{polynomial!Swinnerton-Dyer} show that even this
extension is not sufficient.  The density results discussed in
\sectref{Cebotarev:Sec} are a deeper and more successful way to combine
the factorizations at different primes.

Another approach to primality testing of polynomials, somewhat akin to
the GCDHeu technique of \sectref{PGCD:Heuristic:Sec} has been studied
by {\Adleman} and {\Odlyzko} \cite{Adelman1983-ya}.  Practical aspects of
algorithms of this sort are discussed in \cite{Monagan1992-og}.

{\Hilbert} \cite{Hilbert1887-ki} observed that 
\[
t^4 + 13 t^2 + 81 = \prod \left(t \pm \frac{\sqrt{5} \pm \sqrt{-31}}{2}\right),
\]
while irreducible over the integers, is the product of two
quadratic polynomials modulo every prime.  Polynomials of this form
appear to have been well known and are mentioned in {\Polya} and
{\Szego} \cite{Polya1978-hx}.   {\SwinnertonDyer} \cite{Berlekamp1970-jw}
mentioned to {\Berlekamp} that the polynomials $F_n(X)$ would be
particularly difficult to factor for algorithms that make use of
finite field factorization algorithms.  {\Berlekamp} actually used the
polynomials 
\[
\prod \left( x \pm \sqrt{-1} \pm \sqrt{2} \pm \cdots \pm \sqrt{p_n} \right).
\]

\propref{Cyclotomic:Irred:Prop} was first shown by {\Gauss}
\cite[Art.~341]{Gauss1966-gm} by very different means. 

\notesectref{HIT:Sec} {\Runge} \cite{Runge1886-po} used the Puiseux
expansions to determine when a bivariate polynomial has an infinite
number of integer solutions.  This idea was used subsequently by
{\Hilbert} to produce the irreducibility theorem quoted.

The Hilbert irreducibility theorem is exceedingly important, not just
for symbolic computing, but also in other areas of mathematics.
Hilbert's original motivation for the irreducibility theorem was the
study of the ``inverse problem in Galois theory,'' \ie, given a group
$G$ and a field $k$, how does one determine if $k$ has an algebraic
extension $K$ such that the Galois group of $K/k$ is $G$.

Hilbert's idea was to construct an extension $E$ of the ring $k[t_1,
\ldots, t_s]$ which has Galois group $G$.  By the irreducibility
theorem, for almost all specializations of $t_i$ to elements of $k$
the structure of the image of $E/k[t_1, \ldots, t_s]$ remains. This
approach only seems to work for the case of $G = \mathfrak{S}_n$, the full
symmetric group. 

An alternative approach, suggested by Emma {\NoetherE}, is somewhat
more successful.  Assume the degree of $G$ is $r$.  Construct the ring
$K = k[X_1, \ldots, X_r]$ and the subring $E \subsetneq k[T_1, \ldots,
T_s]$ such that $K/E$ has the Galois group $G$.  This is easy since we
can define $E$ to be $k[T_1, \ldots, T_s]$ where the $T_i$ are the
invariants of $G$'s action on $K$.  For instance, if $G = \mathfrak{S}_r$, then
the $T_i$ can be chosen to be the primitive symmetric functions of the
$X_i$, $T_i = \sigma_i(X_1, \ldots, X_r)$.  In this case, $K/E$ is the
splitting field of the polynomial
\[
X^d - T_1 X^{d-1} + T_2 X^{d-2} + \cdots + (-1)^r T_d,
\]
which has Galois group $\mathfrak{S}_n$.  The Hilbert irreducibility theorem says
that for almost all specializations of the $T_i$ to elements in $k$,
this polynomial stays irreducible and has the same Galois group.  Thus
we get {\Waerden}'s theorem: Almost all polynomials of degree $r$ over
$\Q$ have $\mathfrak{S}_n$ as their Galois group \cite{Van_der_Waerden1936-hu}.  

This result can be quantified this as follows.  Let $E_d(N)$ be the
number of monic polynomials of degree $d$ over $\Z$ that have $\mathfrak{S}_n$ as
their Galois group.  A line of research including {\Waerden} 
\cite{Van_der_Waerden1936-hu} and {\Knobloch} \cite{Knobloch1955-pd,Knobloch1956-id} culminated 
with the following result of {\Gallagher} \cite{Gallagher1973-bj}:
\[
E_d(N) = O(N^{d -\frac{1}{2}} \log N).
\]

Th result of {\CohenS} cited in \propref{HIT:Cohen:Prop} is a 
generalization of {\Gallagher}'s work. 

\notesectref{HIT:Bertini:Sec}  {\Gathen} gives a more direct derivation of
a version of the functional form of the Hilbert irreducibility theorem
from Bertini's theorem in \cite{Zur_Gathen1985-ni}.

\normalsize
