%$Id: dio-anal.tex,v 1.1 1992/05/10 19:39:51 rz Exp rz $
\chapter{Diophantine Equations}
\label{Diophan:Eq:Chap}

\index{diophantine equation|(}
Problems where only integral (or sometimes rational) solutions are of
interest are called \keyi{diophantine problems} and the equations they
involve are called {\em diophantine equations} after Diophantus of
Alexandria,\index{Diophantus of Alexandria} an early Greek
mathematician who wrote a famous book that posed many such problems.
Since one thinks of the real world as being continuous, one might
think that diophantine equations are just mathematical curiosities.
However, due to the quantization introduced by finite precision
arithmetic, quantum mechanics and the discretization introduced by
synchronous computers, they arise in a number of practical situations.
This chapter discusses several different classes of diophantine
equations that frequently arise and techniques for their solution.

The simplest class of diophantine equations consists of linear
equations in two variables, which has the form $aX - bY = c$, where
$a$, $b$ and $c$ are known and integral values for $X$ and $Y$ are to
be determined.  As we shall see in \sectref{Linear:Dio:Sec} this
problem is essentially equivalent 
to the computing the {\sc gcd} of $a$ and $b$.  

\sectref{General:Linear:Dio:Sec} discusses the generalization to
linear equations in more than two variables, and to more than one
equation, as long as the linear system is under-determined (otherwise
there would be a single solution, which could be found by Gaussian
elimination).  The number of solutions of a linear diophantine
equation is either zero or infinite, and if it is infinite the
solutions are uniformly spaced.

The next most complicated diophantine equations are the quadratic
equations.  These equations are also relatively easy to solve.  Some
equations like $X^2 + Y^2 = 5$ have only finite number of integer
solutions.  If $(x,y)$ is a solution of this equation then $|x|, |y|
\le \sqrt{5}$ and thus all possible solutions can be determined by
exhaustion.  The more interesting quadratic equations, like $X^2 -
5Y^2 = 1$, have an infinite number of solutions.  In
\sectref{Pell:Equation:Sec} we show how continued fractions
can be used to solve these equations.  When a quadratic equation has
an infinite number of solutions, the solutions are spaced out
exponentially.

Equations of the form $Y^2 = P(X)$, where $P(X)$ is a polynomial of
degree three or four, are called {\em elliptic diophantine
equations}.\index{diophantine equation!elliptic} The properties of
elliptic equations are more complex than those of quadratic and
linear equations.  For instance, irreducible elliptic equations have
only a finite number of solutions and these solutions form a group.
Among the most famous such equations is $Y^2 = X^3 -2$, whose only
solution is $(3, 5)$, even though there is no {\em a priori} reason
why there should be a finite number of solutions.

Elliptic equations often arise in the most unexpected places.  For
instance, the question of Pythagorean triangles with integral area
discussed in \sectref{Euclid:DE:Sec} rests on the zeroes of the
equation $U^2 = V^4 + 4n^2$.  The properties of elliptic equations
have also been used to factor integers and prove integers are prime.

The Pythagorean problem, $X^2 + Y^2 = Z^2$ has led to one of the most
famous diophantine problems, ``{\Fermat}'s last
theorem,''\index{Fermat's last theorem} which states that for each
integer $n$ greater than $2$, the diophantine equation
\[
X^n + Y^n = Z^n
\]
has no integral solutions.  The current proof of this theorem
\cite{Wiles:1995:FLT,taylor_wiles:1995:hecke} is very complex and uses deep mathematical results that
are far the scope of this book.  However, the question of polynomial
solutions to this equation can be resolved quite easily, as is shown
in \sectref{FLT:Sec}.  This is a nice demonstration of the power that
a symbolic parameter in a problem can provide.

\section{Two Variable Linear Diophantine Equations}
\label{Linear:Dio:Sec}

\index{diophantine equation!linear|(}

The general linear diophantine equation in two variables has the form:
\begin{equation}\label{Dio:2Linear:Eq}
a X - b Y = c.  
\end{equation}
If the {\sc gcd} of $a$ and $b$, $(a, b)$, does not divide $c$ then
\eqnref{Dio:2Linear:Eq} has no solutions.   If $(a, c)$ does not
divide $b$ then $(a,c)$ must divide $Y$.  So \eqnref{Dio:2Linear:Eq}
can be replaced by 
\[
\frac{a}{(a, c)} X - b Y' = \frac{c}{(a, c)},
\]
where $Y = (a, c) Y'$.  So, without loss of generality, we can assume
that $a$, $b$ and $c$ are pairwise relatively prime.

Equation \eqnref{Dio:2Linear:Eq} has an infinite number of integer
solutions if it has any at all.  Assume $x_0$ and $y_0$ is a solution
of \eqnref{Dio:2Linear:Eq}.  Then $x_0 + bt$ and $y_0 + at$ are also
zeroes for any integer value of $t$.  In fact, every solution of this
equation is of this form.  Let $x_1$ and $y_1$ be another zero of
equation \eqnref{Dio:2Linear:Eq}:
\[
\begin{aligned}
a x_0 - b y_0 &= c,\\ a x_1 - b y_1 &= c.
\end{aligned}
\]
Subtracting one equation from the other yields
\[
a (x_0 - x_1) = b (y_0 - y_1).
\]
Since $a$ and $b$ are relatively prime, $a$ must divide $y_0 - y_1$
and $b$ divides $x_0 - x_1$.  This gives the following proposition.

\begin{proposition}\label{Dio:2Linear:Prop}
Let $a$, $b$ and $c$ be pairwise relatively prime rational integers.
If the equation 
\[
a X - b Y = c
\]
has any rational integer solutions, then it has an infinite number of
them.  Furthermore, each solution is of the form
\[
\begin{aligned}
X &= x_0 + bt,\\ Y &= y_0 + at
\end{aligned}
\]
where $t$ is a rational integer and $(x_0, y_0)$ is a primitive
solution of the equation.
\end{proposition}

The pair $(x_0, y_0)$ is called a \keyi{particular solution} of the
diophantine equation $a X -b Y = c$.  By \propref{Dio:2Linear:Prop}
all of the solutions of a linear diophantine equation can be derived
from a particular solution.

Assume we know $(x', y')$ such that $ax' - by' = 1$ and we want to
find the solutions of $a X - b Y=c$.  Clearly, $(cx', cy')$ is a
particular solution, so the general solution is
\[
\begin{aligned}
X &= cx' + bt,\\ Y &= cy' + at.
\end{aligned}
\]
Furthermore, if $x_0$ and $y_0$ are the smallest positive solutions to
$a X - b Y = -1$ then
\[
a\,(b-x_0) - b\,(a-y_0) = 1.
\]
Thus $b-x_0$ and $a-y_0$ are the smallest positive solutions to $a X
-b Y = 1$.

These techniques reduce the problem of solving $a X - b Y = c$ to solving
$a X-bY=1$, where $a$ and $b$ are relatively prime. This problem can be
easily solved using the continued fraction techniques developed in
\chapref{CF:Chap}.  If $(x, y)$ is a solution of this equation, then
we can write
\[
\frac{a}{b} - \frac{y}{x} = \frac{1}{bx}.
\]
Thus $y/x$ is a good rational approximation of $a/b$.  Recall from
\eqnref{CFUnitIdentity:Eq} (on page~ \pageref{CFUnitIdentity:Eq}) that
two consecutive convergents of a continued fractions satisfy
\[
P_n Q_{n-1} - P_{n-1} Q_n = (-1)^{n-1}.
\]
Letting $x/y$ be the next to last convergent of the continued
fraction expansion of $a/b$ we have
\[
ax - by = \pm 1.
\]
If the negative sign occurs, then we can use $b-x$ and $a-y$ as the
particular solutions of $a X - b Y=1$.

To illustrate these techniques we solve the following linear
diophantine equation
\[
15 X - 28 Y = 9.
\]
Since $(15, 9) = 3$ and $3$ does not divide $28$, the $Y$ component of
any solution of this equation must be a multiple of $3$. Thus we can
replace $Y$ by $3 Y'$:
\[
5 X - 28 Y' = 3.
\]
If $x$ and $y'$ are solutions of this equation, then 
\[
x = 3 u + 28t, \qquad y' = 3 v - 15t
\]
where $u$ and $v$ is a solution of 
\begin{equation} \label{Dio:2Linear:Eqa}
5 U - 28 V = 1.
\end{equation}
To solve \eqnref{Dio:2Linear:Eqa}, we compute the continued fraction
expansion of $28/5$:
\[
\begin{aligned}
\cfrac{28}{5} & = 5 + \cfrac{3}{5} = 5 + \cfrac{1}{1 + \cfrac{2}{3}}
=  5 + \cfrac{1}{1 + \cfrac{1}{1 + \frac{1}{2}}}\\
& = [5, 1, 1,  2]
\end{aligned}
\]
Using the recurrence relation \eqnref{CFRecurrence:Eq}, the convergents
can be computed in the following tableau
\begin{center}
\begin{tabular}{|c|c|c|c|c|c|c|}
\multicolumn{2}{c}{} & \multicolumn{1}{c}{5} & \multicolumn{1}{c}{1} 
 & \multicolumn{1}{c}{1} & \multicolumn{1}{c}{2} \\ \hline
0 & 1 & 5 & 6 & 11 & 28 \\ \hline
1 & 0 & 1 & 1 &  2 &  5 \\ \hline
\end{tabular}
\end{center}
Since the number of convergents is even, $u = 28 -11 = 17$ and 
$v = 5 - 2 = 3$, and we have 
\[
\begin{aligned}
  x & =  3 \times 17 + 28t =  51 + 28 t\\
  y & =  3 \times (3 \times 3 - 5t) = 27 - 15t
\end{aligned}
\]

\medskip
The next section discusses a more general approach to solving linear
diophantine equations that handles $ax +by=c$ as a special case.
Below, \keyw{SolveLinear} implements an algorithm for finding a
particular solution of $ax-by=1$ where $a$ and $b$ are assumed to be
relatively prime.  This special case is a useful tool that is
used in several other problems.

\begindsacode
SolveLinear(a, b) := $\{$\\
\> $p_k \leftarrow 1$; $p_{k-1} \leftarrow 0$; $q_k \leftarrow 0$; $q_{k-1} \leftarrow 1$; \\
\>  $a' \leftarrow a$; $b' \leftarrow b$; \\
\> $\mbox{FlipSign} \leftarrow \mbox{False}$; \\
\> loo\=p while $p_k \not = \mbox{a}$ do \{ \\
\> \> $r \leftarrow \mbox{iquo}(a', b')$; \\
\> \>  $(a', b') \leftarrow (b', \mbox{rem}(a', b'))$; \\
\> \> $(p_k, p_{k-1}) \leftarrow (r \cdot p_k + p_{k-1}, p_k)$; \\
\> \> $(q_k, q_{k-1}) \leftarrow (r \cdot q_k + q_{k-1}, q_k)$; \\
\> \> $\mbox{FlipSign} \leftarrow \neg \mbox{FlipSign}$; \\
\> \>$\}$ \\
\> if FlipSign = True then return($q_{k-1}$, $p_{k-1}$)  \\
\> else return ($\mbox{b} - q_{k-1}$, $\mbox{a} - p_{k-1}$); \\
\> $\}$
\enddsacode

\noindent
As before, subscripting is used in the variable names for clarity and
does not imply that they are fronting for an array.

\section{General Linear Diophantine Equations}
\label{General:Linear:Dio:Sec}

This section discusses the problem of solving linear diophantine
equations with more than two unknowns and systems of linear
diophantine equations.  Recall that we solved \eqnref{Dio:2Linear:Eqa}
using the continued fractions expansion of $28/5$.  This approach will
need to be generalized for the more general problems.  

To see how this is done, rewrite the linear equation as an
approximation problem:
\[
\frac{u}{v} - \frac{28}{5} = \frac{1}{5v}.
\]
Expanding the first partial quotient of $28/5$ gives:
\[
\frac{u}{v} - \left(5 + \frac{3}{5}\right) = \frac{1}{5v},
\quad\mbox{or}\quad
\frac{u - 5v}{v} - \frac{3}{5} = \frac{1}{5v}.
\]
We view this reduction step as a change of variables so the problem is
now
\[
\frac{u_1}{v_1} - \frac{3}{5} = \frac{1}{5 v_1}.
\]

Multiplying by  $5/3$ and $v_1/u_1$ gives
\[
\frac{5}{3} - \frac{v_1}{u_1} = \frac{1}{3u_1}.
\]
Writing $5/3$ as $1 + 2/3$ and performing the same sort of
transformations gives
\[
\frac{2}{3} - \frac{v_1-u_1}{u_1} = \frac{1}{3u_1}
\quad\mbox{or}\quad
\frac{u_2}{v_2} - \frac{3}{2} = \frac{1}{2v_2}.
\]
Notice that at each step in this process the change of variables is
both reversible and has reduced the magnitude of the numerator and
denominator of the number we are trying to approximate.

We could continue this process in this fashion, but it is more
convenient to keep everything as a linear equation.  
\begin{equation}\label{Dio:2D:Steps:Eq}
\begin{array}{rlcrl}
5u - 28v&=1 & \quad\Longleftrightarrow\quad & 5u_0 - 28v_0 & =1\\
5(u_0 - 5v_0) - 3v_0 &=1 & \quad\Longleftrightarrow\quad & 5u_1 - 3v_1 &=1\\ 
2u_1 - 3(v_1 - u_1) &=1 & \quad\Longleftrightarrow\quad & 2u_2 - 3v_2 &=1\\ 
2(u_2 - v_2) - v_2 &=1 & \quad\Longleftrightarrow\quad & 2u_3 - v_3 &=1 \\
0\cdot u_3 - (v_3 -2 u_3) &=1 & \quad\Longleftrightarrow\quad 
   & 0\cdot u_4 - v_4 &=1
\end{array}
\end{equation}
At this point, the solution is evident.  $v_4$ must take on the value
$-1$, while any integer value can be chosen for $u_4$, \eg, $u_4 = t$.
$u$ and $v$ can now be determined by unwinding the steps above:
\[
\begin{aligned}
u_0 &= 28u_4 + 11v_4 = -11 + 28t, \\
v_0 &= 5u_4 + 2v_4 = -2 + 5t
\end{aligned}
\]

This approach of choosing invertible linear changes of variables that
reduce coefficients of the diophantine equation is the key to the more
general linear diophantine problem.  Recall that such transformations
can be viewed as {\em unimodular matrices}.\index{unimodular matrix}
The wonderful feature of the continued fraction algorithm is that it
``factors'' a rational number into a product of unimodular matrices.

The sequence of subtractions in \eqnref{Dio:2D:Steps:Eq} and the
computation required in the unwinding can be combined into performing
matrix operations.  To do this we rewrite the \eqnref{Dio:2Linear:Eqa}
as
\[
\begin{pmatrix}5 & -28 \\ 1 & 0 \\ 0 & 1\end{pmatrix} \cdot \begin{pmatrix}u\\ v\end{pmatrix} = 
A \cdot \begin{pmatrix}u\\ v\end{pmatrix} = 
\begin{pmatrix}1\\ u \\ v\end{pmatrix}.
\]
The sequence of steps in \eqnref{Dio:2D:Steps:Eq} are column operations
on $A$.  In particular the first step is
\[
\begin{pmatrix}5 & -28 \\ 1 & 0 \\ 0 & 1\end{pmatrix} \cdot \begin{pmatrix}u_0\\ v_0\end{pmatrix} = 
\begin{pmatrix}1\\ u \\ v\end{pmatrix}
\Longrightarrow
\begin{pmatrix}5 & -3 \\ 1 & 5 \\ 0 & 1\end{pmatrix} \cdot \begin{pmatrix}u_1\\ v_1\end{pmatrix} = 
\begin{pmatrix}1\\ u \\ v\end{pmatrix}.
\]

Continuing, using the Euclidean algorithm to reduce the elements in
the top row, gives
\[
\begin{aligned}
\begin{pmatrix}5 & -28 \\ 1 & 0 \\ 0 & 1\end{pmatrix}
&\Longrightarrow
\begin{pmatrix}5 & -3 \\ 1 & 5 \\ 0 & 1\end{pmatrix} 
\Longrightarrow
\begin{pmatrix}2 & -3 \\ 6 & 5 \\ 1 & 1\end{pmatrix} 
\Longrightarrow \\
&\Longrightarrow 
\begin{pmatrix}2 & -1 \\ 6 & 11 \\ 1 & 2\end{pmatrix} 
\Longrightarrow
\begin{pmatrix}0 & -1 \\ 28 & 11 \\ 5 & 2\end{pmatrix} 
\Longrightarrow
\begin{pmatrix}1 &0 &\\ -11 & 28 \\ -2 & 5\end{pmatrix}.
\end{aligned}
\]
Notice that the last two rows of the matrix are precisely the
convergents of the continued fraction expansion of $28/5$.
In matrix form the diophantine equation is now
\[
\begin{pmatrix}1 &0 \\ -11 & 28 \\ -2 & 5\end{pmatrix} \cdot \begin{pmatrix}1 \\ t\end{pmatrix} = 
\begin{pmatrix}1 \\ -11 + 28t \\ -2 + 5t\end{pmatrix} =
\begin{pmatrix}1 \\ u \\ v\end{pmatrix}.
\]

This same approach can be applied to equations in more than one
variable, \eg 
\[
7x + 3y + 13 z = 1.
\]
The sequence of matrix reductions in this case is
\begin{equation}\label{Dio:3D:Steps:Eq}
\begin{pmatrix}7 & 3 & 13 \\ 1 & 0 & 0 \\ 0 & 1 & 0 \\ 0 & 0 & 1 \end{pmatrix}
\Longrightarrow
\begin{pmatrix}3 & 1 & 1 \\ 0 & 1 & 0 \\ 1 & -2 & -4 \\ 0 & 0 & 1 \end{pmatrix}
\Longrightarrow
\begin{pmatrix}1 & 0 & 0 \\ 1 & -3 & -1 \\ -2 & 7 & -2 \\ 0 & 0 & 1 \end{pmatrix}.
\end{equation}
Thus,
\[
\begin{pmatrix}1 & 0 & 0 \\ 1 & -3 & -1 \\ -2 & 7 & -2 \\ 0 & 0 & 1 \end{pmatrix}
\cdot \begin{pmatrix}1 \\ r \\ s\end{pmatrix} = \begin{pmatrix}1 \\ x \\ y \\ z\end{pmatrix},
\]
or
\[
\begin{aligned}
x & = 1 - 3 r - s, \\
y & = -2 + 7r - 2s,\\
z & = s.
\end{aligned}
\]

Notice that the steps in \eqnref{Dio:3D:Steps:Eq} only involved column
operations.  Thus there is a \key{unimodular matrix} that relates the first
and last matrices
\[
\begin{pmatrix}7 & 3 & 13 \\ 1 & 0 & 0 \\ 0 & 1 & 0 \\ 0 & 0 & 1 \end{pmatrix}
= 
\begin{pmatrix}1 & 0 & 0 \\ 1 & -3 & -1 \\ -2 & 7 & -2 \\ 0 & 0 & 1 \end{pmatrix}
 \cdot
\begin{pmatrix}7 & 3 & 13\\ 2 & 1 & 4 \\ 0 & 0 & 1 \end{pmatrix}.
\]

The basic idea, illustrated above, is to use the entries of smallest
magnitude  to reduce all the others.  In order for each of the
transformations to be unimodular, we restrict ourselves to the following
set of operations.
\begin{enumerate}
\item Permuting rows/columns
\item Multiplying all the elements of a row/column by $-1$ (a unit).
\item Adding an integral multiple of a row/column to another.
\end{enumerate}

For example, consider the following system of linear equations
\[
\begin{aligned}
  7x + 6y + 13z & = 3,\\
  6x + 8y + 12z & = 2.
\end{aligned}
\]
Since we permit row operations as well as column operations it is
necessary to keep track of the left hand side of the equations as well
as the right hand side.  Thus we use the following tableau: 
\[
\begin{array}{rrrr}
7 & 6 & 13 & 3 \\
6 & 8 & 12 & 2 \\
1 & 0 & 0 &  \\
0 & 1 & 0 &  \\
0 & 0 & 1 &  
\end{array}
\]
Notice that we have added an identity matrix below the two rows
corresponding to the diophantine equations.

Subtracting one copy of the second column from the first column,
two from the third column and permuting the columns gives the
first tableau below.  Using the first row to reduce the second row,
gives the second tableau:
\[
\begin{array}{rrrr}
6 & 1 & 1 & 3 \\
8 & -2 & -4 & 2 \\
0 & 1 & 0 &  \\
1 & -1 & -2 &  \\
0 & 0 & 1 &  
\end{array}
\Longrightarrow
\begin{array}{rrrr}
6 & 1 & 1 & 3 \\
2 & -3 & -5 & -1 \\
0 & 1 & 0 &  \\
1 & -1 & -2 &  \\
0 & 0 & 1 &  
\end{array}
\]

The second column is used to reduce the first and third columns, and
is then interchanged with the first column.  In the second tableau we
have reduced the second row by the first row. 
\[
\begin{array}{rrrr}
1 & 0 & 0 & 3 \\
-3 & 20 & -2 & -1 \\
1 & -6 & -1 &  \\
-1 & 7 & -1 &  \\
0 & 0 & 1 &  
\end{array}
\Longrightarrow
\begin{array}{rrrr}
1 & 0 & 0 & 3 \\
0 & 20 & -2 & 8 \\
1 & -6 & -1 &  \\
-1 & 7 & -1 &  \\
0 & 0 & 1 &  
\end{array}
\]

Finally, we use the third column to reduce the second column and
interchange them:
\[
\begin{array}{rrrr}
1 & 0 & 0 & 3 \\
0 & 2 & 0 & -8 \\
1 & -1 & -16 &  \\
-1 & -1 & -3 &  \\
0 & 1 & 10 &  
\end{array}
\]
Interpreting the tableau as a system of equations we have
\[
\begin{pmatrix}3 \\ -8 \\ x \\ y \\ z \end{pmatrix} =
\begin{pmatrix}1 & 0 & 0 \\ 0 & 2 & 0 \\
1 & -1 & -16 \\ -1 & -1 & -3 \\ 0 & 1 & 10 \end{pmatrix} \cdot
\begin{pmatrix}u \\ v \\ w \end{pmatrix}
=
\begin{pmatrix}u \\ 2 v \\ 
u - v -16 w \\
-u -v -3 w \\
v + 10w \end{pmatrix}.
\]
The first two rows give $u = 3$ and $v = -4$, while the last three
equations give the general solution to system
\[
\begin{aligned}
x &= -1 - w \\
y &= -7 - 3 w \\
z & = -4 + 10 w
\end{aligned}
\]

There are a couple of points worth noticing about this process.
First, our original $2\times 3$ matrix of coefficients was converted
to diagonal matrix.  In general, we can always convert a matrix of
integer entries into this form.  Second, the second diagonal entry of
the matrix was $2$, which happened to divide $-8$, the right hand side
of the second equation.  If the right hand side had been odd, then the
diophantine system, could not have possessed any integer solutions
(since all the transformations have been unimodular).\index{unimodular
matrix} More generally, we have the following proposition.

\begin{proposition} Given $m< n$ and a system of linear diophantine
equations
\begin{equation} \label{Dio:Gen:Lin:Eq}
\begin{aligned}
a_{11} x_1 + a_{12} x_2 + \cdots + a_{1n} x_n & = c_1 \\
a_{21} x_1 + a_{22} x_2 + \cdots + a_{2n} x_n & = c_2 \\
 \vdots \\
a_{m1} x_1 + a_{m2} x_2 + \cdots + a_{mn} x_n & = c_m 
\end{aligned}
\end{equation}
There exists an effectively computable unimodular change of variables
that converts this system to a diagonal system\index{unimodular matrix}
\[
\begin{aligned}
\delta_{1} u_1 & = d_1 \\
\delta_{2} u_2 & = d_2 \\
 \cdots \\
\delta_{r} u_2 & = d_r 
\end{aligned}
\]
where $r$ is the rank of the matrix $a_{ij}$.  Furthermore
\eqnref{Dio:Gen:Lin:Eq} has a solution in integers if and only if
$\delta_i$ divides $d_i$ for $1 \le i \le r$.
\end{proposition}

The reduction we have informally described here converts the general
matrix into its diagonal or \keyi{Smith normal
form}.\footnote{Actually, the Smith normal form also requires that
$\delta_i$ divide $\delta_{i+1}$.} This is somewhat more than is
actually needed to solve linear diophantine equations.  Instead of
reducing the equation to diagonal form, we only need to make it
triangular, which is called the \keyi{Hermite normal form}.  The
Hermite normal form is substantially easier to compute than the
\key{Smith normal form}.

\index{diophantine equation!linear|)}

\section{Pell's Equation}
\label{Pell:Equation:Sec}

To find the solutions of
\begin{equation}\label{Dio:Pell:Eq}
x^2 - D y^2 = 1,
\end{equation}
we use the maxim of interpreting diophantine problems as approximation
problems.  Notice that the better $p/q$ approximates $\sqrt{D}$ the
smaller $p^2 - Dq^2$ is.  Since we are looking for $p$ and $q$ such
that $p^2 -Dq^2$ is equal to $1$, $p/q$ must be a very good
approximation to $\sqrt{D}$.

Let $p/q$ be an approximation of $\sqrt{D}$ that satisfies
\eqnref{Dio:Pell:Eq}.  Then
\[
\frac{p}{q} - \sqrt{D} = \frac{p^2 - D q^2}{q(p+q \sqrt{D})} 
  = \frac{1}{q(p+q\sqrt{D})},
\]
using \eqnref{Dio:Pell:Eq} to reduce the numerator.  So,
\[
\left| \frac{p}{q} - \sqrt{D}\right| \le \frac{1}{2q^2}.
\]
By \longpropref{RationalCF:Prop} an approximation this good must be a
convergent of the continued fraction of $\sqrt{D}$. 

We know that $\sqrt{D} = [a_0, \overline{a_1, \ldots, a_{k-1},
2a_0}]$.  The best approximations of a continued fraction are those
that come just before the large partial quotients.  The largest
partial quotient is $2a_0$, so $P_{k-1}/Q_{k-1}$ looks like a good
candidate.

To illustrate this consider the following table of convergents for
\[
\sqrt{22} = [4, \overline{1, 2, 4, 2, 1, 8}].
\]

\[
\def\mR#1{\multicolumn{1}{r}{#1}}
\def\mC#1{\multicolumn{1}{c}{#1}}
\begin{array}{c|r|r|r|r|r|r|r|}
\mC{a_i}         & \mR4 & \mR1& \mR2 & \mR4 & \mR2 & \mR1 & \mR8 \\ \cline{2-8}
P_i              &  4 & 5 & 14 & 61 & 136 & 197 & 1312 \\ \cline{2-8}
Q_i              &  1 & 1 &  3 & 13 &  29 &  42 &  365 \\ \cline{2-8}
P_i^2 - 22 Q_i^2 & -6 & 3 & -2 &  3 &  -6 &   1 &   -6 \\ \cline{2-8}
\end{array} 
\]
One zero of \eqnref{Dio:Pell:Eq} must be $x = 197$, $y = 42$.

To see this more generally, recall that
\[
\begin{aligned}
 \sqrt{D} & = [a_0, a_1, \ldots, a_{k-1}, a_0 + \sqrt{D}], \\
    & \displaystyle = \frac{P_{k-1} (a_0 + \sqrt{D}) + P_{k-2}}{Q_{k-1} (a_0 + \sqrt{D}) + Q_{k-2}}.
\end{aligned}
\]
where we have used \longpropref{CF:Bilinear:Subst:Prop}.  Clearing
fractions gives:
\[
Q_{k-1} ( a_0 \sqrt{D} + D) + Q_{k-2} \sqrt{D} = P_{k-1} \sqrt{D} +
(a_0 P_{k-1} + P_{k-2}).
\]
Since $a_0$, $P_i$ and $Q_i$ are integers, we can equate the
coefficients $(\sqrt{D})^1$ and $(\sqrt{D})^0$
separately.\footnote{This is a powerful technique that will be used
over and over again.}  When the resulting equations are solved for
$P_{k-2}$ and $Q_{k-2}$, we get
\[
\begin{aligned}
 P_{k-2} & = D Q_{k-1} - a_0 P_{k-1}, \\
 Q_{k-2} & = P_{k-1} - a_0 Q_{k-1}.
\end{aligned}
\]
Substituting these relations into $P_{k-1} Q_{k-2} - P_{k-2} Q_{k-1} =
(-1)^k$ gives a relationship involving only $P_{k-1}$ and $Q_{k-1}$:
\[
\begin{aligned}
P_{k-1} (P_{k-1} - a_0 Q_{k-1}) - Q_{k-1}(D Q_{k-1} - a_0 P_{k-1})
  & = (-1)^k, \\
P_{k-1}^2 - D Q_{k-1}^2 & = (-1)^k.
\end{aligned}
\]
So if the length of period of $\sqrt{D}$ is even, $P_{k-1}, Q_{k-1}$
is a solution of \eqnref{Dio:Pell:Eq}.  If the period is odd, then
$P_{k-1}, Q_{k-1}$ is a zero of $x^2 - D y^2 = -1$, and $P_{2k-1},
Q_{2k-1}$ is a zero of \eqnref{Dio:Pell:Eq}.

\section{Elliptic Curves}
\label{EllipticCurve:Sec}

\section{Fermat's Last Theorem}
\label{FLT:Sec}
\index{Fermat's last theorem|(}

Perhaps the most famous diophantine equation is that of Fermat's
last theorem\index{Fermat's last theorem}
\begin{equation}\label{FLT:Eq}
x^n + y^n = z^n.
\end{equation}
The case $n=2$ arises in geometry as the \key{Pythagorean theorem}:
the sum of the squares of the two sides of a right triangle is the
square of the hypotenuse. The parameterization of its solution was
discussed briefly in \sectref{Euclid:DE:Sec}.

A solution of \eqnref{FLT:Eq} is said to be {\em trivial} if $xyz =
0$.  {\Fermat} claimed that \eqnref{FLT:Eq} has {\em no} non-trivial
solutions for $n > 2$.  To date this conjecture is still open.
Recently, however, it was discovered that when $x$, $y$ and $z$ are
polynomials in one variable instead of integers, then it is easy to
show that \eqnref{FLT:Eq} does not have any non-trivial solutions if
$n > 2$.

As we shall repeatedly see in later sections, the ability to take
derivatives makes many polynomial problems much easier than their
corresponding integer problems.  Nontheless, the approach used in a
polynomial solution occasionally gives insight into the  integer
cases.  

The fundamental breakthroughs of this approach are due to {\Mason}
\cite{Mason:1984} who has developed effective techniques for solving
diophantine problems over function fields.  Here, we only need the
simplest of his results.  This presentation follows {\Lang}'s article
\cite{Lang90}.

We begin with a simple polynomial identity.

\begin{proposition}\label{Ratfun:LogDeriv:Prop}
Any rational function $R(t)$ can be written as 
\[
R(t) = \prod_i(t - \alpha_i)^{\ell_i}
\]
where $\ell_i$ are positive or negative integers.  Then
\[
\frac{R'(t)}{R(t)} = \sum_i \frac{\ell_i}{t - \alpha_i}.
\]
\end{proposition}

\begin{proof}
This is just a statement about ``logarithmic derivatives:''
\[
\begin{aligned}
  \frac{d}{dt} \log R(t) = \frac{R'(t)}{R(t)} 
    & = \frac{d}{dt} \log \prod_i(t - \alpha_i)^{\ell_i} \\
    & = \sum_i \frac{\ell_i}{t - \alpha_i}.
\end{aligned}
\]
\end{proof}


Each pole and zero of $R$ is a pole of $R'/R$, but while the poles of
$R(t)$ can have arbitrary order, $R'/R$ only has poles of order $1$.
Defining the {\em radical}\index{radical!of a polynomial} $R$ to be
\[
N_0(R) = (t - \alpha_1)  (t - \alpha_2) \cdots (t - \alpha_{\ell})
\]
the denominator of $R'/R$ is the radical of $R$.  That the radical of
a rational function can be determined efficiently is a direct
consequence of the ability to differentiate a polynomial.  This makes
the polynomial case easy.
\addsymbol{$N_0(P)$}{The radical or square free part of $P$}

Let $n_0(F)$ denote the number of {\em distinct} zeroes of $F(t)$,
\ie, $n_0(F) = \deg N_0(F)$.  The key result needed to prove the
insolubility of \eqnref{FLT:Eq} in polynomials is the following
``$ABC$'' proposition.
\addsymbol{$n_0(P)$}{The number of distinct zeros of the polynomial $P$}

\begin{proposition}[Mason]\label{Mason:ABC:Prop}
Let $a(t)$, $b(t)$ and $c(t)$ be relatively prime polynomials such
that $a+b=c$.  Then
\[
\max \{ \, \deg a, \deg b, \deg c \, \}
  \le n_0(abc) - 1.
\]
\end{proposition}

\begin{proof}
We can assume that each of the polynomials have the following form,
\[
a(t) = \prod_i (t - \alpha_i)^{\ell_i}, \quad b(t) = \prod_i (t -
\beta_i)^{m_i}, \quad c(t) = \prod_i (t - \gamma_i)^{n_i}.
\]
Let $f=a/b$ and $g = c/b$, so that $f - g = 1$.  Differentiating, we
have $f' = g'$ so
\[
\frac{f'}{f} f = \frac{g'}{g} g  \Longrightarrow
\frac{a}{c} = \frac{f}{g} = \frac{g'/g}{f'/f}.
\]
By \propref{Ratfun:LogDeriv:Prop} we have
\begin{equation}\label{Mason:Ratio:Eq}
\frac{a(t)}{c(t)} = 
\frac{\displaystyle\sum_i \frac{n_i}{t - \gamma_i} - \sum_i \frac{m_i}{t - \beta_i}}%
{\displaystyle\sum_i \frac{\ell_i}{t - \alpha_i} - \sum_i \frac{m_i}{t
- \beta_i}}
\end{equation}
Define $N_0$ to be
\[
N_0 = \prod_i(t - \alpha_i) \times
\prod_i(t - \beta_i) \times \prod_i(t - \gamma_i).
\]
Notice that the degree of $N_0$ is $n_0(abc)$.  Both $N_0 f'/f$ and
$N_0 g'/g$ are polynomials.  By \eqnref{Mason:Ratio:Eq} we can write
\begin{equation}\label{Mason:Help:a:Eq}
a \times \left(N_0 \frac{f'}{f}\right) = c \times \left(N_0
\frac{g'}{g}\right),
\end{equation}
where each factor is a polynomial.  Since $a$ and $c$ are relatively
prime, $a(t)$ must divide $N_0 g'/g$, so
\[
\begin {aligned}
\deg a(t) & \le \deg \left(N_0 \frac{g'}{g}\right) 
                = \deg N_0 + \deg g' - \deg g, \\ & \le n_0(abc) - 1.
\end{aligned}
\]
Equation \eqnref{Mason:Help:a:Eq} immediately gives an identical result
for $c(t)$, and interchanging $a$ and $b$ above gives the inequality
for $b(t)$.  Combining all three inequalities gives the proposition.
\end{proof}

This result is proven in more generality by {\Mason}
\cite{Mason:1984} (page 11, Lemma 2).  In particular, he shows that a
generalization of this inequality holds even when $a$, $b$ and $c$ are
allowed to be algebraic functions.  The proof of this result is not
much more difficult than that presented above, but it uses somewhat
more sophisticated concepts. In particular, the statement of the
generalization of \propref{Mason:ABC:Prop} involves the genus of
curve.\index{genus, of a curve}

\propref{Mason:ABC:Prop} gives a very simple proof of Fermat's theorem for
polynomials.

\begin{proposition}\label{Poly:FLT:Prop}
If $x(t)^n + y(t)^n + z(t)^n = 0$ and $x$, $y$ and $z$ are relatively
prime, then $n \le 2$.
\end{proposition}

\begin{proof}
Applying \propref{Mason:ABC:Prop} to $x(t)^n$, $y(t)^n$ and $z(t)^n$
we have
\[
\deg x(t)^n \le \deg x(t) + \deg y(t) + \deg z(t) - 1
\]
or
\[
n \deg x(t) \le \deg x(t) + \deg y(t) + \deg z(t) - 1.
\]
Replacing $x(t)$ on the left hand side with $y(t)$ and $z(t)$ and
summing we have
\[
n (\deg x(t) + \deg y(t) + \deg z(t)) 
   \le 3 (\deg x(t) + \deg y(t) + \deg z(t)) - 3
\]
or
\[
3 \le (3 - n) (\deg x(t) + \deg y(t) + \deg z(t)).
\]
If $n$ not less than $3$, the right hand side is negative.
\end{proof}

\medskip
If we try the same approach in the integer case we are lead to the
``$ABC$'' conjecture.  Let $a$ be an integer and assume
\[
a = p_1^{e_1} p_2^{e_2} \cdots p_k^{e_k}.
\]
As with polynomials, define $N_0(a)$ to be the product of the prime
divisors of $a$, \ie, $N_0(a) = p_1 \cdots p_k$.  Then

\def\ABCconj{{\em ABC} conjecture\index{ABC conjecture@{\em ABC} conjecture}}
\begin{conjecture}[ABC] \label{ABC:Conj}
Let $a$, $b$ and $c$ be relatively prime integers and assume $a+b =
c$.  Then for each positive $\epsilon$ there exists a constant
$C(\epsilon)$ such that
\[
\max\{|a|, |b|, |c| \} \le C(\epsilon) N_0(abc)^{1+\epsilon}.
\]
\end{conjecture}

Notice that the integer form of the {\ABCconj} is written
multiplicatively rather than additively, as is the case for the
polynomial form.  This is because the integer primes have varying
sizes, while the polynomial ``primes,'' the univariate polynomials $(t
- \alpha_i)$, all have the same size.

The need for the $\epsilon$ in exponent of the conjecture also
distinguishes the integer case from the polynomial case.  The example
$(5, 27, 32)$ illustrates its necessity.

If we assume the {\ABCconj} is true, we can prove an asymptotic
version of Fermat's last theorem in much the same way we proved
\propref{Poly:FLT:Prop}.  Assume $x$, $y$ and $z$ are pairwise
relatively prime integers and assume $x^n + y^n = z^n$ for some
integer $n$.  Using \conjref{ABC:Conj}, we have
\[
|x^n| \le C(\epsilon) |xyz|^{1+\epsilon} 
\]
for each positive $\epsilon$.
Replacing $x^n$ by $y^n$ and $z^n$ and multiplying the results gives:
\[
|xyz|^n \le C^3(\epsilon) |xyz|^{3+3\epsilon}.
\]
So the positive integral solutions of $x^n + y^n = z^n$ are
constrainted to satisfy
\[
|xyz|^{\frac{n}{3}-1-\epsilon} \le C(\epsilon).
\]
Thus there can be only a finite number of counterexamples to Fermat's
last theorem.  If $C(\epsilon)$ could be determined explicitly we
could determine if any of the possible counterexamples actually are
counterexamples.  

\index{Fermat's last theorem|)}

\section*{Notes}

\small

\Marginpar{Peth\"{o} \cite{Petho90} seems to have a nice summary of some of the
techniques here.   There is also nice paper in \cite{Williams77}.}

\notesectref{General:Linear:Dio:Sec}  Many of the ideas for efficient
computation of Hermite and Smith normal forms\index{Hermite normal
form}\index{Smith normal form} of matrices are
contained in \cite{Havas79}.  Other work on this problem is contained
in \cite{Bradley71,Kannan79,Chou:Collins82,Domich85} culminating in
the algorithms discussed in \cite{Iliopoulos89a,Iliopoulos89b}.

\notesectref{Pell:Equation:Sec} {\Lagrange} mistakenly attributed the
equation, $x^2 - Dy^2 = 1$ to Pell, who did no work on the equation.
Nonetheless, the name stuck.

\key{Archimedes} proposed the ``cattle
problem''\index{Archimedes!cattle problem} to \key{Eratosthenes}
around 200BCE \cite[pp. 202--205]{Thomas51}.  The problem asks, ``How
many cattle there are in the herds on Helios?''  The herds on Helios
have cattle with four different types of coats: blue, yellow,
white and speckled.  In addition there are male cattle (bulls) and
female cattle (cows).  The facts we know about the cattle are of the
form ``The number of white bulls is one half plus one third the number
blue and white bulls.  Denoting the number of bulls of a particular
color with an upper case letter, and the number of cows with a lower
case letter we have the following equations
\[
\begin{aligned}
  W &= \left(\frac{1}{2} + \frac{1}{3}\right) B + Y, \\
  B &= \left(\frac{1}{4} + \frac{1}{5}\right) Y + S, \\
  S &= \left(\frac{1}{6} + \frac{1}{7}\right) W + Y, \\
\end{aligned}
\qquad
\begin {aligned}
  w &= \left(\frac{1}{3} + \frac{1}{4}\right) (B + b), \\
  b &= \left(\frac{1}{3} + \frac{1}{4}\right) (P + p), \\
  s &= \left(\frac{1}{3} + \frac{1}{4}\right) (Y + y), \\
  y &= \left(\frac{1}{3} + \frac{1}{4}\right) (W + w). 
\end{aligned}
\]

Thus far the problem is relatively, simple.  The solution can be
parameterized in terms of a third variable $n$, as
\[
\begin{aligned}
  W &= 10366482 n, \\
  B &= 7460514n, \\
  S &= 7358060n, \\
  Y &= 4149387n,
\end{aligned}
\qquad
\begin {aligned}
  w &= 7206360n, \\
  b &= 4893246n, \\
  s &= 3515820n, \\
  y &= 5439213n. \\
\end{aligned}
\]

However, in addition, Archimedes indicates that the total number of
white and blue bulls is a perfect square, and the total number of
yellow and speckled bulls is a triangular number.  After a bit of
calculation, the solution to this problem relies on the solution of
the diophantine equation:
\[
t^2 - 4729494 u^2 = 1.
\]
The continued fraction expansion of $\sqrt{4729494}$ is
\[
\begin{aligned}
\sqrt{4729494} = 
[2174,&~1, 2, 1, 5, 2, 25, 3, 1, 1, 1, 1, 1, 1, 15, 1, 2, 16, 1, 2, 1,
1, 8, 6, 1, 21, 1, 1, \\
&3, 1, 1, 1, 2, 2, 6, 1, 1, 5, 1, 17, 1, 1, 47, 3, 1, 1, 6, 1, 1, 3,
  47, 1, 1, 17,1, \\ 
&5, 1, 1, 6, 2, 2, 1, 1, 1, 3, 1, 1, 21, 1, 6, 8, 1, 1, 2, 1, 16, 2,
  1, 15, \\
& 1, 1, 1, 1, 1, 1, 3, 25, 2, 5, 1, 2, 1, 4348]. 
\end{aligned}
\]
The reduction of the cattle problem to this point was first achieved
by {\Amthor} \cite{Amthor80} in 1880.  A modern calculation of the
solution is presented in \cite{Williams65}.

\normalsize

\index{diophantine equation|)}
