%$Header: /usr/u/rz/AMBook/RCS/algebraic-int.tex,v 1.1 1992/05/10 19:34:16 rz Exp rz $
\chapter{Integration of Algebraic Functions}

\section{Integration of Unnested Radicals}

In this section we investigate the problem of integrating algebraic
functions.  We define $K=k(x,y)$ to be an algebraic function field if
it is a finite extension of $k(x)$ the field of rational functions in
$x$.  Any element of $K$ is an algebraic function and satisfies a
polynomial over $k(x)$ whose degree divides that of the extension.

\subsection{An overview of the algebraic case of the Risch Alg}

By {\Liouville}'s Theorem, if $z \in K$ has an elementary integral, it
can be expressed as the sum of an element of $K$ and constant
multiples of $\log$s of such elements.  This permits us to construct a
general form for the integral.  If the derivative of this form cannot
be made to match the integrand, then the original problem was not
integrable. Hardy \cite{Hardy1916-nb} conjectured that there was
no effective general procedure for determining the integrability of
algebraic functions if log terms were present.  In 1968, {\Risch}
\cite{Risch1968-sc} \Marginpar{Wrong reference here} gave an 
algorithmic procedure for determining the integrability of functions
constructed using exponential and logarithmic as well as algebraic
operations.  {\Risch} constructed the pattern from local information
at the singularities of the integrand and was able to reduce the
problem of integration to a problem in algebraic geometry, determining
a bound on the order of the torsion subgroup of the divisor class group,
which he called the \keyi{points of finite order problem}.
In 1970, he outlined a procedure for computing this bound
cite{Risch:Integration:Solution}, 
but an efficient, practical implementation is still lacking.
One problem with {\Risch}'s algorithm is that the need to calculate
power series expansions forces one to operate in a splitting field
of very high degree even when the integral can be expressed in a
field of much lower degree.  

\subsection{Analytic Techniques}

We restrict ourselves here to the class of algebraic functions
which can be expressed using only a single radical.  By this we mean
that the function field can be expressed as $k(x,y)$ where $y^n=p(x)$
for some polynomial $p$ over $k$ whose
roots have multiplicities less than $n$.  For this class of functions
we will present an algorithm for obtaining the algebraic part of the
integral while operating completely in the field of definition for the
integrand.  Although this is a very special case of the general 
problem addressed by {\Risch}, it encompasses elliptic and
hyperelliptic functions and represents the most frequently  encountered
class of algebraic integrals.

The first obstacle to integrating algebraic functions is their
multivalued nature.  Instead of trying to pick a particular branch
of the function, we will consider the entire function on its Riemann
surface.  This surface can be viewed as an $n$-fold covering of the
extended complex plane.  For our restricted class of functions, whenever
$p(x) \not= 0$, the sheets are all distinct over $x$.  Over a simple
root of $p(x)$
all $n$ sheets coalesce to form an $n$ sheeted branch point, \ie,
any path on the surface must circle this point $n$ times before
closing.  If $\gamma$ is a root of $p$ of multiplicity $m$ and $d=\gcd(m,n)$
then over $\gamma$ there are $d$ distinct branch points, each composed of
$n/d$ sheets.  One of the principal advantages of considering functions on
their Riemann surface is that they become single valued and meromorphic
there.  Thus for any particular algebraic funtion
the only singularities we must contend with are poles, and
these are finite in number.

We shall use the local properties of functions
and their integrals as a key to their global behavior.  Our
principal localization technique will be power series computation.
The expansion of an algebraic function at an unramified point on
the surface has integer exponents and is simply a Laurent expansion.
However, if we want to expand at branch points, we must permit
fractional exponents.  In fact if $x_0$ is a $k$-sheeted branch point,
an expansion at $x_0$ will be in terms of $(x-x_0)^{1/k}$.  These
expansions with fractional exponents are known as Puiseux expansions
\cite{Bliss1933-ax}. To get a consistent definition of order for
Puiseux expansions, we will express our series in terms of a
uniformizing parameter.  Instead of expanding in $(x-x_0)^{1/k}$, we
let $x = t_+x_0$ and expand in terms of $t$.  The parameter $t$ is
only determined up to a $k$\th root of unity, $\omega$, so we define
two expansions to be {\em equivalent} if one transforms into the other
by substituting $\omega t$ for $t$.  The order of an algebraic
function at a {\em place} (point on the Riemann surface) $\pger$ is
the degree of the first non-zero term in its power series expansion
(in $t$).  We define $\Ord_{\pger}(y)$ to be the order of $y$ at
$\pger$ and the {\em branch index} to be the number of sheets which
coalesce at $\pger$.  Every neighborhood of a Riemann surface is
locally homeomorphic to a neighborhood in the complex plane. The
uniformizing parameter supplies this mapping and thus a local
coordinate system, but we have no single global coordinate system for
the Riemann surface.  The integral must contain the information
necessary to transform between local coordinate systems along our path
of integration.  This is done by considering the integrand to be a
differential and not a simple algebraic function.  A differential is a
form $f\,dx$ where $f$ is an algebraic function, and the order of the
form at a place $\pger$ with uniformizing parameter $t$ is defined to
be $\Ord_{\pger}(f) + \Ord_{\pger}(dx/dt)$.  Since $x$ is locally a
rational function in $t$, its derivative, $dx/dt$, is also rational
and thus has a well defined order.  At a finite $k$-sheeted branch
point $dx/dt$ is $kt^{k-1}$ and so $\Ord(f\,dx)=\Ord(f)+k-1$.

Algebraic functions on their Riemann surface have
many properties in common with rational functions in the
extended complex plane.  In particular, the following is true
in both domains:

\begin{enumerate}
\item $\Ord_{\pger}(f)$  is nonzero for only a finite set of places. 

\item $\sum \Ord_{\pger}(f) = 0$ if the sum is taken over all the
places on the surface.  This is another way of saying an algebraic
function has the same number of poles and zeroes, counting multiplicities.

\item  $\sum \Res_{\pger}(f\,dx) = 0$ with the sum again taken over
all places on the surface.  $\Res_[\pger](f\,dx)$ is the residue of $f\,dx$
and is defined to be the coefficient of $t^{-1}$ in the series
for $f\,dx/dt$.
\end{enumerate}

The differentials on a Riemann surface are placed in
three categories on the basis of their singularities.  A
differential of the first kind has no singularities on the
entire Riemann surface.  Differentials of the second kind 
have poles, but always with zero residue.  Differentials of
the third kind have only simple poles and thus non-zero
residue.

We define a \keyi{divisor} on a Riemann surface to be an element of
the free abelian group generated by the points of the surface, \ie, a
divisor is a formal sum $\sum n_{\pger}\pger$ where $n_{\pger}$ is a
rational integer and is non-zero for only a finite number of points
$\pger$.  The degree of a divisor is $\sum_{\pger} n_{\pger}$. The
divisor of an algebraic function $f$ is defined to be $\sum_{\pger}
\Ord_{\pger}(f)\pger$ and is denoted by (f).  Since any algebraic
function has the same number of poles and zeros, its divisor is of
degree 0.  There is a bijection between divisors of degree 0 over the
extended plane and rational functions, determined up to constant
multiple.  On the other hand, given a divisor of degree 0 on a Riemann
surface, there may be no algebraic function with precisely those poles
and zeros.  If there is such a function the divisor is called a
\keyi{principal divisor}.

The operation of computing Puiseux expansions at a place provides us
with a differential isomorphism from our function field into the field
of power series over $\C$.  This means that computing power series
commutes with differentiation, \ie, the series for the derivative of
$f$ is the derivative of the series for $f$.  Therefore, to obtain
local information about the integral, we merely compute the power
series expansion of the integrand and integrate the series termwise.
If a power series has negative order, the operation of integration
increases its order by one, or equivalently decreases the order of the
pole by one.  Integration of a power series with non-negative order
produces a series with non-negative order.  Thus we see that
integration will never create a pole, and always decreases the order
of the pole of the integrand by one.

By Liouville's theorem,
\[
\int f\,dx = w_0 + \sum_i c_i \log w_i
\]
where $f$ and  $w_j$ are in $\C(x,y)$, \ie, algebraic functions, and
$c_i$ are complex numbers.
After differentiating we get 
\[
f= w_0' + \sum_i c_i {w_i' \over w_i}.
\]
We
will now examine $w_i'/w_i$ locally to see how $f$ decomposes.  Let
$\pger$ be an arbitrary place on the Riemann surface.  If 
$\Ord_{\pger}(w_i) = n,$
$n$ different from zero,
then $\Ord_{\pger}(w_i') = n-1$ and thus
$\Ord(w_i'/w_i) = -1$.
If $\Ord(w_i) = 0$, then
$\Ord(w_i') \ge 0$ and $\Ord(w_i'/w_i) \ge 0$.
Since $\Ord(f+g) \ge \min(\Ord(f),\Ord(g))$, 
\[
\Ord(\sum_i c_i {w_i' \over w_i}) \ge -1
\]
everywhere on the Riemann surface.

We define the {\em principal part} of a function at a place to be
the sum of all terms with negative degree in the power series expansion
at that place.  The principal part of a differential  at a place
is the sum of
all terms of degree less than $-1$ in its expansion.  Thus termwise
integration of the principal part of a differential produces the
principal part of the integral.

The main result of this section is that for our restricted
class of algebraic functions, we can get a very complete description
of the form of the algebraic part of the integral.  In general the
algebraic part would involve all powers of $y$ less than $n$.  By considering
only function fields defined by $y^n=p(x)$, we were able to prove the
following key theorem.
	
\begin{theorem}
The algebraic part of $\int R(x)y\,dx$, where $y^n=p(x)$ and $R(x)$ is
a rational function in $x$, is of the form $yA(x)$ where $A(x)$ is a
rational function in $x$.
\end{theorem}

Although the statement seems very intuitive, the proof is
fairly long and we will prove it in several stages.  The proof is based
on the relationship between the series expansions for $y$ at
all the places above a given $x$ value.  By examining the formula for
computing $n$\th roots of power series \cite{Zippel1976-yk}, we
see that if 
$p(x_0) \not= 0$ then the $n$ distinct expansions for $y$ can be generated by
taking any one of them and multiplying it successively by the $n$\th roots
of unity. We make the conventions that if $p(x_0) \not= 0$, we call $x_0$ a root
of multiplicity $0$ and define $\gcd(0,n)=n$.
Now in general, if $x_0$ is a root of $p(x)$ of multiplicity $m$ and 
$d=\gcd(m,n)$ then there are $d$ distinct places over $x_0$, each with
branch index $n/d$. 
Two expansions of $y$ over $x_0$ are equivalent if their quotient is an
$(n/d)$\th root of unity.  Again starting with any particular expansion
over $x_0$, we can generate $n$ different expansions by multiplying by
all the $n$\th roots of unity.
These $n$ expansions can be grouped into $d$ sets each with $n/d$ elements,
such that all the elements within a given set are equivalent, but any
two expansions from distinct sets are inequivalent.  
This gives us a coset decomposition of the group of $n$\th roots of unity;
we consider two $n$\th roots equivalent if their quotient is an $(n/d)$\th
root of unity.  If $\omega$ is a primitive $n$\th root of unity, we can let
$\omega^0, \omega^1, \ldots, \omega^{d-1}$ be our canonical representatives of
the equivalence classes.  Now we can
form $d$ inequivalent expansions  of $y$, over $x_0$,
by multiplying any particular
expansion by these $d$ representative elements.
The exponents in the terms of these expansions are all
congruent to $m/d$ modulo $n/d$.  We summarize these results in the
following lemma:

\begin{lemma}
Let $x_0$ be a root of $p(x)$ of multiplicity $m$, $g(t)$ be the power
series expansion of $y$ at some place over $x_0$, $d=\gcd(m,n)$.  Then
$d$ inequivalent series expansions of $y$ over $x_0$ are $g(t), \omega
g(t), \ldots, \omega^{d-1}g(t)$ where $\omega$ is a primitive $n$\th
root of unity.  The exponents in the expansions are all congruent to
$m/d$ modulo $n/d$.
\end{lemma}

In order to be able to compare the expansion of $y$ with that of a
rational function in $x$ and $y$ at a place, we assume the series for
the rational function is computed by formal arithmetic operations on
the series for $y$ and $x$, as in \cite{Zippel1976-yk}.  Since a
rational function of $x$, $R(x)$, has the same expansion at all places
over $x_0$, we can multiply by $R(x)$ without changing the coefficient
relationship among the different sheets.  At an $n/d$ sheeted branch
point, the series for $R(x)$ is a series in $t^{n/d}$, thus
multiplying by $R(x)$ won't change the congruency class, modulo $n/d$,
of the exponents of a power series.  Since the expansion for $y$ is
the $k$\th power of the expansion for $y$, with the same assumptions
as in Lemma 1 above, we get the following analogous result at the
corresponding sheets.

\begin{lemma}
Let $R(x)$ be a rational function in $x$ and $h(t)$ be
the power series expansion for $R(x)y$ at the same place as $g(t)$ in
Lemma 1.  Then $d$ inequivalent series expansions of $R(x)y$ over $x_0$ are
$h(t), \omega_ h(t), \ldots, \omega^{(d-1)k}h(t)$, where
$\omega$ is a primitive $n$\th\ root of unity.  The exponents in
these expansions are all congruent to $km/d$ modulo $n/d$.
\end{lemma}
	
The main step in the proof of Theorem 1 is the following Lemma:

\begin{lemma}
Let $f \in k(x,y)$, 
\[
f = a_0 + a_1 y + \cdots + a_{n-1}y^{n-1}.
\]
If, for every $x$ value, the ratios of the principal parts of the
expansions of $f$ at any two places over $x$ are constants that are
equal to the ratios of the expansions of $y$ at those places, and if
the exponents in the expansions of $f$ are congruent, modulo the
branch index, to the exponents in the expansions of $y$ then $f-a_1 y$
is constant, \ie, $a_2= \cdots =a_{n-1}=0$ and $a_0 \in k$.
\end{lemma}

\begin{proof}
Examine the expansions over a particular point, $x_0$,
where we assume $t$ is a uniformizing parameter.
We will first consider the case when there are no branch points
over $x_0$ and determine what the constraints are on the coefficients
of a particular term, e.g. $t^{-s}$, in the expansions of the $a_j y^j$.
Let $b_j$ be the coefficient of $t^{-s}$ in $a_j y^j$ and let $c$ be its
coefficient in the expansion of $f$ at the first sheet.  Then
their respective coefficients at the $(k+1)$\st sheet will be
$\omega^{jk}b_j$ and $\omega_ c$.  Thus we get a system of $n$ linear
equations in the $b_j$'s.  The coefficients of this system form a
Vandermonde matrix whose determinant is thus 
$\prod_{i \not= j} (\omega^i-\omega^j)$. 
But since $\omega$ is a primitive $n$\th root of unity, each term in
the product is non-zero and so is the determinant.  Therefore
there must be a unique solution.  But by inspection we can
find a solution where $b_1=c$ and all the other $b_j$'s are $0$, and
it must be the only solution.  Since this is true for an
arbitrary term in the principal part of $f$, we see that
the principal parts of $a_j y^j$ for $j \not= 1$ must be zero at
any unramified point.

Now assume there are $d$ distinct branch places above
$x_0$, each with branch index $r=n/d$.  Again examine the
coefficients of the $t^{-s}$ term in all the expansions.  Let $m$
be the multiplicity of $x_0$ as a root of $p(x)$.  Then $d=\gcd(m,n)$,
and, by assumption,
the exponents of all the terms of the principal part of
the expansions of $f$ at $x_0$ are congruent to $m/d$ modulo $r$.
By lemma 2 the exponents in the expansions of $a_j y^j$ are congruent to
$jm/d$.  For a particular $s$ there will be $d$ values of $j$
such that $jm/d$ will be congruent to $s$, thus there are at most
$d$ monomials $a_j y^j$ whose expansions can include  a term in $t^{-s}$.
At each of the $d$ places over $x_0$ we can equate the sum of these $d$
coefficients to the coefficient of $t^{-s}$ in $f$.  This gives us
$d$ linear equations in $d$ of the $b_j$'s.  If we examine the coefficient
matrix and factor out the first element of each row, then the
matrix we have left is Vandermonde with determinant 
$\prod (\omega^{rj}-\omega^{rk})$  with $j,k<d$.  Since $\omega$ is a
primitive $n$\th root 
of unity, $\omega^r$ is a primitive $d$\th root of unity, and just as in 
the previous paragraph, the determinant is non-zero.  If $s$ is congruent
to $m/d$ then we have the unique solution that the coefficient of
$t^{-s}$ in $f$ is $b_1$ and the $d-1$ other $b$'s congruent to 1 modulo
$r$ are all zero.  If $s$ is not congruent to $m/d$ then the obvious unique
solution is to let the $d$ $b_j$'s be zero.  Since this is true for
all terms in the principal part of $f$, we have shown that the
principal parts of $a_j y^j$ for $j \not= 1$ are zero at all branch places.
Together with the previous paragraph, this shows that their principal
parts are zero everywhere, including at infinity.  But this means that they
have no poles, and the only algebraic functions with no poles are
constants.  Thus we have shown that if $f$ is algebraic, it can differ from
$a_1 y$ by at most a constant. 
\end{proof}

With this last lemma under our belt, we are now ready to finish
the proof of theorem 1.

\begin{proof}
We must show that the algebraic part of $\int R(x)y\,dx$ satisfies the
hypotheses of lemma 3.  By lemma 2, the integrand $R(x)y$ has the
required coefficient and exponent properties.  It remains to show that
these properties are preserved after multiplication by $dx$ and
termwise integration of the principal parts of the differential.

Since $dx/dt$ is the same at all sheets above a given $x$ value, 
multiplication by $dx/dt$ does not disturb the coefficient relationships.
The subsequent termwise integration will divide the coefficient
of $t^s$ by $s+1$ on all the sheets over $x_0$, thus again preserving
the ratios between sheets.  If $x_0$ is a branch point of index $k$,
then $dx/dt = kt^{k-1}$ and multiplication by $dx/dt$ will change
the congruency class of the exponents modulo $k$.  But the subsequent
integration will increase all the exponents by 1. Thus we have a net
effect of increasing them by $k$ and leaving their congruency class
unchanged.  Therefore the hypotheses of lemma 3 are satisfied
and the algebraic part of the integral must differ from $yA(x)$ by
at most a constant.
\end{proof}

Note that the principal
part of the integral is generated by the terms of degree less than $-1$ in
the expansion of the integrand while the logs arise from the rest of the
expansion.  Thus by concentrating on the principal parts we were
able to determine the structure of the algebraic part without
the possibility of interference from the log terms.


\begin{corollary}
The derivative of the transcendental part of the $\int R(x)y\,dx$ is
of the form $y B(x)$ where $B(x)$ is a rational function in $x$.
\end{corollary}

\begin{proof}
If the algebraic part is $y A(x)$, then its derivative is
\[
\left({P'(x)A(x) \over nP(x)} + A'(x)\right)y
\]
and thus still of the form $y$ times a rational function in $x$.
Subtracting this from the integrand gives the desired result.
\end{proof}

By theorem 1 and its corollary we have  shown  that if
$\int R(x)y\,dx$ is expressible in elementary functions, then it
can be written as $A(x)y + \int	B(x)y\,dx$, where the second term
has no algebraic part, \ie, integrates completely into logs.
Now we must investigate what the restrictions are on the
rational functions $A(x)$ and $B(x)$.  All the poles of the algebraic
part are derived from poles in the original integrand with
their order increased  by one.  
The differential $y\,dx$ has no finite poles, so all the finite poles
of $R(x)y\,dx$ arise from poles of $R(x)$.  At each of these poles
we have that 
\[
\Ord_{\pger}(A(x)y) = \Ord_{\pger}(R(x)y\,dx) + 1.
\]
Since the order of a product is the sum of the orders, this reduces to 
\[
\Ord_{\pger}(A) = Ord_{\pger}(Rdx) + 1.
\]
Now let $A(x)$ be $S(x)/T(x)$ and $R(x)$ be $U(x)/V(x)$ where $S$, $T$, $U$,
and $V$ are polynomials over $k$.  If $\pger$ is a root of $V(x)$ of
multiplicity $r$, it must be a root of $T(x)$ of multiplicity $r-1$.  Let
$\prod_i V_i^i$ be
a square free decomposition of $V$.  Then $T(x)$ can be expressed
as a similar product with each of the exponents decreased by one, \ie,
$T(x) = \prod_i V_i^{i-1}$.

Next we determine the denominator of $B(x)$.  The order of
the differential $B(x)y\,dx$ is greater than or equal to $-1$ everywhere on
the Riemann surface.  If $\pger$ is not a zero of $P(x)$, then
$\Ord_{\pger}(y\,dx) = 0$. 
Thus at an unramified place $B(x)$ can have at most a simple pole.
If $\pger$ has branch index $r$ and is a zero of $P(x)$ of multiplicity $m$,
then 
\[
\Ord_{\pger}(y\,dx) = \left({m \over n} + 1\right) r -1.
\]
Thus $\Ord(B(x)) \ge -(m/n+1)r$.  Let $B(x)$ be $C(x)/D(x)$ where $C$ and $D$
are polynomials and let $\MULT_{\pger}(D(x))$ be the multiplicity of
$\pger$ as a root of $D(x)$. 
Then
\[
\Ord_{\pger}(B(x)) = r\left(\MULT_{\pger}(C(x)) -
\MULT_{\pger}(D(x))\right).
\]
Since we are interested in the case where $\MULT_{\pger}(D(x)) \not=  0$,
we can assume that $\MULT_{\pger}(C(x))=0$.  Thus we have
$-r(\MULT_{\pger}(D(x))) \ge -(m/n+1)r$ or equivalently
$\MULT_{\pger}(D(x)) \le 1+m/n$.  But $m<n$ and the multiplicity is
always an integer, so again $\pger$ can only be a simple root of $D(x)$.
The differential $B(x)y\,dx$ can only have poles where the original
integrand had poles, thus all the zeros of $D(x)$ were zeros of $T(x)$.
Since $D(x)$ is square free we have that $D(x) = \prod_i V_i(x)$.

At this point we have determined all of the algebraic part
except its numerator, $S(x)$.  Using the integral equation
for the algebraic and transcendental parts, we shall derive a system
of linear equations that the coefficients of the numerator polynomials,
$S(x)$ and $C(x)$, must satisfy.  We start with the equation:
\[
\int {U(x) \over V(x)} y\, dx =
{S(x) \over T(x)} y + \int {C(x) \over D(x)} y \, dx
\]
Then we differentiate both sides yielding:
\[
{U(x) \over V(x)} y = {S'(x) \over T(x)} y - 
{S(x) T'(x) \over T(x)^2} y + 
{1 \over n} {S(x) P'(x) \over T(x) P(x)} y
+ {C(x) \over D(x)} y.
\]

Now we see the power of theorem 1.  The radical, $y$, can be divided out
from both sides of the equation leaving us with only rational functions.
We also multiply through by $V(x) = T(x)D(x)$ and put the right side
over a common denominator.  To simplify the derivation, we
suppress the functional notation for our polynomials.
\[
U = {S'TPD - S T' P D + S T P' D/n \over TP}
+ C T^2 P
\]

Our next goal is to completely eliminate the denominator.  We could
do this by merely multiplying out both sides, but notice that $T(x)$
divides each term on the right side except the second.  We now
examine more closely the structure of $T$, $D$, and $T'$.
\[
T' = \left(\prod_{i=1} V_i^{i-2}\right)
\left(\sum_{i=1} (i -1) V_2 \cdots V_{i-1}V' V_{i+1} \cdots
V_k \right)
\]
\[
D = \prod_{i=1} V_i
\]

Now we see that we can divide $T$ into $T'D$.  We define $W(x)$ to
be the quotient.
\[
W = {T' D \over T} =
\sum_{i=2} (i-1) V_1 V_2 \cdots V_{i-1} V'_i V_{i+1} \cdots V_k
\]

Using this definition, the computation of $W$ appears to require 
both a polynomial multiplication and a polynomial division.  But
we can give another derivation of $W$ which eliminates the
multiplication in favor of a subtraction.  Starting with the definition
$V = TD$ we have:
\[
{V' \over T} = {T' D + D' \over T}
\]
\[
W = {T' D \over T} = {V' \over T} - D'
\]

The other factor of the denominator, $P(x)$, will not, in general,
divide out completely, but we can cancel the $\gcd(P(x),P'(x))$
from the numerator and denominator.  If we let $E = P/\gcd(P,P')$ and
$F = P'/\gcd(P,P')$ then we can multiply both sides of the
equation by $E(x)$ yielding the following polynomial equation:
\[
U(x)E(x) = S'(x)E(x)D(x) + 
S(x)\left({F(x)D(x)\over n} - E(x)W(x)\right) + C(x)T(x)E(x)
\]

The only unknowns in this equation are $S(x)$, $S'(x)$, and $C(x)$.
If we let $S(x)$ and $C(x)$ be polynomials with undetermined coefficients,
then we get a system of linear equations in the coefficients.  All that 
remains is to get degree bounds on $S(x)$ and $C(x)$.  We do this by
examining our original integral equation at a place above $\infty$.
\[
\Ord_{\infty}(<\hbox{trans. part}>') 
= \Ord_{\infty}({C(x) \over D(x)} y\,dx) \ge -1
\]
Which implies
\[
\deg C \le \deg D - {1 \over n} \deg P -1.
\]

\[
\Ord_{\infty}(<\hbox{Algebraic part}>) \ge
\min(\Ord_{\infty}(\hbox{integrand})+1,0)
\]
\[
\Ord_{\infty}({S(x) \over T(x)} y) \ge 
\min(\Ord_{\infty}({U(x) \over V(x)}y\,dx)+1,0)
\]
\[
\deg S \le \max(1+\deg U - \deg D, \deg T - {1 \over n}\deg P)
\]

These degree bounds complete the specification of our integration
algorithm.  If the linear system is inconsistent, then the original
problem was not integrable.  If we happen to know that our integrand
has zero residue everywhere, then we can set the polynomial $D(x)$ to zero
since there will be no log terms.
In this case our linear system is triangular and can be solved very easily.
It is interesting to note that our procedure for integrating algebraic
functions reduces to Horowitz's algorithm for rational function integration
if we set $P(x)$ to one.

\subsection{Algebraic Techniques}

{\Risch}'s landmark paper \cite{Risch1970-na} presented the
first decision 
procedure for the integration of elementary functions.  In that
paper he required that the functions appearing in the integrand be
algebraically independent.  Shortly afterwards in [Risalg]
and \cite{Risch1970-na} he relaxed
that restriction and outlined a complete decision procedure for
the integration of elementary functions in finite terms.
Unfortunately his algorithms for dealing with algebraic functions
required considerably more complex machinery than his earlier
ones for purely transcendental functions.  Moses' implementation
of the earlier approach in MACSYMA \cite{noauthor_1989-as} demonstrated its
practicality, whereas the same has yet to be done for {\Risch}'s 
more recent approach.

This paper will show how {\Risch}'s earlier techniques can be
generalized to deal with unnested radicals.  While this may seem
a severe restriction, perusing an integral table such as [Bois61]
will show that fewer than $1\%$ of the problems are excluded.

We will assume the reader is familiar with the terminology and results
of \cite{Risch1970-na}.  We will use the term field to
mean a differential field of characteristic 0.  If $F$ is a field with
$y$ algebraic over $F$ of degree $n$ such that $y^n \in F$, then we
will call $F(y)$ a simple radical extension of $F$.  Any element of
$F(y)$ can be written as a polynomial in $y$ of degree $n-1$ with
coefficients in $F$.

\section{Structure theorems}

Our first result will be a refinement of Liouville's structure
theorem for simple radical extensions.  Let $F$ be any differential
field with $K$ its field of constants and $y$ radical over $F$. 
{\Risch}'s Strong Liouville Theorem states that $g \in F(y)$ is
integrable if and only if there is a $v \in F(y)$ , $c_i \in K(d)$,
and $w_i$ in $F(y,d) $ with $d$ algebraic over $K$ such that
\begin{equation}
f = v' + \sum c_i {w_i' \over w_i} 
\label{Liouville:Eq}
\end{equation}
We will call $v$ the rational part of the integral and the rest
the transcendental part of the integral.
If we assume that $F$ contains $\omega$ a primitive $n$\th root of
unity then there is a unique differential automorphism of $F(y)$
over $F$ such that $\sigma (y) = \omega y$. Define the operator
\[
T_i = {1\over n} \sum_{j=0}^{n-1} {\sigma^j \over \omega^{ij}}
\]
Note that $T_i (y^j) = y^j$ if $i = j$ else 0.
Thus letting $g = \sum g_i y^i$ and $v = \sum v_i y^i$,
we have $T_i(g) = g_i y^i$.  Also noting that
$T_i$ commutes with the derivation, $T_i (v') = ( v_i 
y^i ) '$.  Since $T_i$ sends the transcendental part of 
the integral to a sum of the same form, by successively applying
$T_j$ to equation (1) for $0 \le j \le n-1$ we deduce
that (1) $g$ is integrable if and only if each $g_i y^i$ is
and (2) the rational part of $\int g_i y^i$ is $v_i y^i$.

Now let $G$ be a compositum of simple radical extensions, \ie,
$G = F(y_1 , \ldots , y_k )$ where $y_i^{ e_i } \in F$
and $[G \,:\, F] = \prod e_i$.  Any $g \in G$ can be written as a polynomial
in the $y_i$'s with coefficients in $F$ where the degree of $y_i$
in $g$ is less than $e_i$.  Then by repeating the previous argument
for each $y_i$, one can show $g$ is integrable if and only if each
term is integrable. The subfield of $G$ generated by a single such term
over $F$ is differentially isomorphic to a simple radical extension of 
$F$ of degree at most the least common multiple of the $e_i$'s.
Thus integrals over compositums of simple radical extensions
can be reduced to integrals over simple extensions 
frequently of much lower degree.

\section{A Generalized Risch algorithm}

Let $F$ be arbitrary differential field and $E=F(\theta)$ where $\theta$
is transcendental over $F$ and $F$ and $E$ have the same constant subfield.
We will additionally assume exactly one of the following is true:

\begin{enumerate}
\item $\theta' = 1$
\item $\theta' = {v'} / v$  for some $v \in F$, \ie, 
$\theta = \log(v)$
\item ${\theta' / \theta} = v'$ for some $v \in F$,
\ie, $\theta = \exp(v)$
\end{enumerate}

\noindent
We will be making use of the fact that $F[ \theta ]$ is a constructive
Euclidean domain.  Thus we can compute gcd's and hence square-free
decompositions.  We are interested in the case where $G=E(y)$ is a 
simple radical extension of degree $n$.  Additionally we will require
that $y$ must depend on $\theta$, \ie, $y^n$ is not in $F$.  By the
previous section we are reduced to considering integrands of the form 
$S y^i$ with $S \in E$.  We will find it convenient to rewrite
this as $R / y^{n-i}$ where $R = S y^n$.  By changing
our choice of $y$ we can assume an integrand of the form $R / y$.
(Note this may involve changing our value of $n$).
Without loss of generality we can finally assume $y^n = P ( \theta ) 
\in F [ \theta ]$ where $P$ has no factors of multiplicity $\ge n$.

Let $R = A / B$ with $A , B \in F[ \theta ]$ and $B$ monic.
After finding a square-free basis for $P$ and $B$ and performing
a partial-fraction decomposition on $A / B$ we can split
our integrands into three cases: $C / {V_ y}$, $C / {W_ y}$,
and $C / y$ where $V$ is relatively prime
to $P$ but $W$ is a square-free factor of $P$.  Unlike the previous section,
integrability of $R / y$ does not guarantee integrability of each
term in the partial fraction decomposition.  However after the splitting
we will be able to apply a variety of reduction formulae to these cases.
There will be a strong similarity between our algorithms for reducing the
integrands and Hermite's algorithm for rational function integration.

Since the case $\theta = \exp(v)$
has additional complications, we will treat it later.  
For the remaining cases a polynomial $Q$ is square-free if and only if
$\gcd(Q,Q')=1$. 

The first problem we encounter is that $y'$ introduces new
denominators, so we will choose a polynomial $f$ such that
$(f/y)' = g / y$ for some $g \in F [ \theta ]$.  In fact we will 
need an $f$ of least possible degree.  If $P = d \prod P_i^{e_i}$ is
our square-free decomposition of $P$ into monic factors with $d \in
F$, then define $f = \prod P_i$.
\[
(f / y)' = f / y \left ( \sum (1-e_i / n) P_i' / P_i
- d' / nd  \right ) = g / y
\]
Clearly $g \in F [ \theta ]$.

\subsection{Case 1 $C / ( V_ y)$ where $\gcd(V,f)=1$}

We want to find a function whose derivative when subtracted from
the integrand will decrease $k$ and not introduce any new denominators.
We want a polynomial $B$ such that
\[
({Bf \over  V^{k-1} y} )' - {C \over V_ y} = {D \over V^{k-1}y}
\]
for some polynomial $D$. Since
\[
({Bf \over V^{k-1} y})' = {(1-k) V' B f \over V_ y} +
{B'f + Bg \over V^{k-1} y}
\]
Thus we must choose $B$ such that $(1-k) V'fB \equiv C $ (modulo V).
Since $\gcd(fV',V)=1$ we can find $B$ as long as $k > 1$.  Thus we can
continue this reduction process until $k=1$.

\subsection{Case 2  $C / { W_ y}$ where $W = P_j$}

Since $W$ divides $f$ we must start with an apparent denominator
of the form $W_ y$. Letting $h = f / W$ we have

\[
( { B f \over W_ y} )' = 
{B g - k W' h \over W_ y} + {B' h \over W^{k-1} y}
\]
Thus we want to choose B such that $Bg-kW' h \equiv C$ (mod W).
$W = P_j$ implies $g \equiv (1-e_j /n) W' h$ (mod W).
Thus we have $B(1-k-e_j /n) W' h \equiv C$ (mod W).
Since $W'h$ is relatively prime to $W$ and $e_j < n$, this
equation is solvable for any $k$.  Thus by repeated applications
of this reduction step we can eliminate all factors of f from the
denominators of our integrands.

\subsection{Case 3 $C / y$}

Here we will try to find a $B$ such that the reduced integrand has
lower degree. Let $B = b_j \theta^j$, with $b_j \in F$.
\begin{equation}
(b_j \theta^j f / y)' = b_j' \theta^j f /y + j 
b_j \theta^{j-1} \theta' f/y + b_j \theta^j g/y
\label{Simple:Int:Alg:Eq:a}
\end{equation}

Letting $m = \deg f$ we have two subcases depending on whether
$d$ is constant or not.  If $d' = 0$ then degree(g) = m-1
else degree(g) = m. Let $C = \sum c_i \theta^i$.

We will first assume that $d' =0$. If $b_j' = 0$ then
\eqnref{Simple:Int:Alg:Eq:a} has degree $j+m-1$ else degree $j+m$.
Thus equating formally highest degree terms we obtain:
\begin{equation}
c_{j+m} = (j+1) b_{j+1} \theta' + b_{j+1} \lc g + b_j'
\label{Simple:Int:Alg:Eq:b}
\end{equation}
where $b_{j+1}$ is a constant.  $\lc g$ is the leading coefficient of $g$.
\[
\lc g = \sum (1-e_i /n) \lc P_i'
\]
If $P_i = \theta_ + a_i \theta_-1 + \ldots$ then $\lc P_i' = k
\theta' + a_i'$. If $\theta' = 1$ then \eqnref{Simple:Int:Alg:Eq:b}
reduces to : 
\begin{equation}
c_j+m = ( j + 1 + \sum \deg(P_i ) (1-e_i /n)) b_j+1
\label{Simple:Int:Alg:Eq:c}
\end{equation}

If $j+1 \ge 0$ then the coefficient of $b_{j+1}$ will always be
nonzero.  Thus we can always reduce $C$ until $\deg(C) < m-1$.

If $\theta = \log(v)$ then equation \eqnref{Simple:Int:Alg:Eq:b} takes the form
\begin{equation}
  c_{j+m} = b_{j+1} ((j+1){v' \over v} 
    + \sum (1-e_i /n) (\deg(P_i ) {v' \over v} + a_i' )) + b_j'
\label{Simple:Int:Alg:Eq:d}
\end{equation}

The coefficient of $b_{j+1} {v'} / v$ in equation
\eqnref{Simple:Int:Alg:Eq:d} is precisely the coefficient of $b_{j+1}$
in equation \eqnref{Simple:Int:Alg:Eq:c} and is therefore nonzero if
$j+1 \ge 0$.  If the original problem is integrable then $c_{j+m}$
must be integrable.  Just as in \cite{Risch1969-pi} $b_{j+1}$ is uniquely
determined since $\theta$ is a monomial over $F$ while $b_j$ is
determined up to an additive constant.  In this case we can reduce $C$
until $\deg(C) < m-1$ only if either the original problem is
integrable or at least the necessary set of coefficients are
integrable.

We will now treat the subcase where $d'$ is nonzero. Note that
we must have $\theta = \log (v)$ for this to hold. Here we must first
assume that there is no constant s such that $sd$ has an $n$\th root
in $F$.  If there is such an $s$ then we can pick a new generator 
$y (sd)^{-1/n}$ for $G$ which puts us back in the previous subcase.
Otherwise we have $\deg(g) = \deg (f)$ and after equating leading
terms we have $c_{k+m} = b_k' - {d' \over n d} b_k$.
Our assumption about $d$ guarantees that this equation can have 
at most one solution in $F$.  Thus we need to be able to solve 
first order linear differential equations for a solution in $F$.
If $F$  is a tower of monomial extensions then \cite{Risch1969-pi} shows how
to do this.  If all the equations we set up have solutions 
we will reduce $C$ so that $\deg(C) < m$.

\subsection{$\theta = e^v$}

The disinguishing characteristic of the exponential function is that
it is a factor of its derivative.  Thus we can no longer claim that
a square-free polynomial must be relatively prime to its derivative.
It will only be necessary to treat factors of the form $\theta$
specially.  We begin by rewriting the square-free decomposition of $P$.
$P = d \theta^j \prod P_i^{e_i}$  where no $P_i$
is divisible by $\theta$.  We will again define $f = \prod P_i$,
noting that now f is not divisible by $\theta$.  We next verify
that $(f/y)'$ still is of the form $g/y$ for some $g \in F[ \theta ]$.

\[
\left( {f \over y} \right)' = {f \over y} \left (
\sum (1-e_i /n) {P_i'
\over P_i} - j/n v' - {d' \over n d} \right ) = {g \over y}
\]


After performing a partial-fraction decomposition of the integrand, we
can deal with all denominators other than $\theta$ 
just as in the previous cases.  Thus we will now assume an integrand
of the form $C / {\theta_ y} $, and we again write $C = \sum 
c_i \theta^i$. We are again trying to decrease $k$ and letting
$B$ be an arbitrary polynomial we compute:
\[
({B f \over \theta_ y})' = {B' f + B g - k v' B f \over \theta_y}
\]

Requiring the numerator to be congruent to $C$ modulo $\theta$ is the
same as equating constant terms
\[
c_0 = b_0' f_0 + b_0 g_0 -k v' b_0 f_0.
\]

$f$ not divisible by $\theta$ implies $f_0 $ is nonzero, thus we can 
divide through by $f_0$.  Again $\theta$ a monomial will force
this equation to have at most one solution in $F$
for $k\ge 0$. As long as the
equation continues to have solutions, we can reduce $k$ to 0.

Finally we must deal with integrands of the form $C / y$ in the
exponential case.  Here $\deg g = ]deg f$ always, and we again assume
a solution of the form ${B f} \over y$ and equate leading terms.
\begin{equation}
c_{k+m} = b_k' + k v' b_k + b_k \lc g
\label{Simple:Int:Alg:Eq:e}
\end{equation}

\[
\lc g = {-j \over n} v' - {d' \over n d} + \sum (1-e_i /n) \deg(P_i ) v'
\]
Equation \eqnref{Simple:Int:Alg:Eq:e} will have at most one solution
as long as the coefficient of $v'$ is nonzero. 
This coefficient is ${k - j \over n} + \sum (1-e_i / n) \deg(P_i )$.
Since the third term is always positive, $k > 0$ or if $j = 0$ then
$k\ge0$ is sufficient to guarantee that $v'$ is present in
equation \eqnref{Simple:Int:Alg:Eq:e}.

\section{Summary and Conclusions}

The reduction formulae in section 3 have enabled us to find the rational
part of our integral if it exists.  If the original problem was
integrable, all the remaining integrands must generate the
transcendental portion of the integral.  Note that cases 1 and 2 will
always reduce any integrand whether it be integrable or not.  
In particular, for the case $\theta' = 1$ we see that any
integral can be reduced to $\int A / {B y} $ where $B$ is square-free
and $\deg(A) - \deg(B) < \deg(f) -1$.

We have shown that the question of computing integrals in
$F(\theta,y)$ can be reduced to the problems of computing integrals in
$F$ and that of solving first order linear differential equations over
$F$.  If $F$ was constructed as a tower of monomial extensions, then
\cite{Risch1969-pi} shows that these problems are solvable.

We claim that the algorithms presented here form a natural extension
to those presented in \cite{Risch1969-pi}.  By restricting
ourselves to this special but very important case, we are able to
generate the integral using nothing more than simple polynomial
arithmetic.  Although our formulas are somewhat more complicated, we
require no additional machinery than those necessary in {\Risch}'s
original approach.  We have not explained how to obtain the
transcendental part of the intgral, and in fact this would require
some new machinery.  However, in a subsequent paper we will show that
by sticking to this special case, much of the algebro-geometric
constructions outlined in \cite{Risch1970-na} can be
avoided. We have presented what we hope are very usable practical
algorithms, and we will in fact implement them in the near future.

\section{The Algebraic Case of the Risch Algorithm}


