%%$Id: poly-remainder.tex,v 1.1 1992/05/10 19:39:31 rz Exp rz $
\chapter[Polynomial GCD's: Classical Algorithms]{Polynomial GCD's\\Classical Algorithms}
\label{PRS:Chap}

\index{greatest common divisor!of polynomials|(}

The computation of polynomial greatest common divisors is required
more frequently than one might at first imagine.  For instance, nearly
all arithmetic operations with \key{rational functions} (quotients of
polynomials) require a {\sc gcd} to reduce the quotients to lowest
terms.  However, computing polynomial {\sc gcd}'s is significantly
more difficult than the arithmetic calculations discussed in
\chapref{Poly:Arith:Chap}.

Polynomial arithmetic algorithms are all polynomial in the size of
their result.  For sparse multivariate polynomials, the simplest {\sc
gcd} algorithms require time that is exponential in the size of the
input polynomials and the size of the {\sc gcd}.  Consequently, a
great deal of effort has been devoted to improving their behavior.

The classical algorithms are based on a generalization of the
remainder sequence used in the \key{Euclidean algorithm} to
polynomials, called a \keyi{polynomial remainder sequence} ({\sc
prs}).  This chapter discusses a number of different schemes for
computing {\sc prs}'s and their application to ``classical''
algorithms for computing polynomial {\sc gcd}'s.
\chapref{Interpolation:Chap} covers more modern approaches that make
use of ``lifting techniques.''

\sectref{PRS:Intro:Sec} defines the greatest common divisor and least
common multiple of polynomials and gives the basic definitions and
assumptions that are used in this chapter.  The probabilistic
technique discussed in \sectref{GCD:Several:Sec} reduces the
computation of the {\sc gcd} of several polynomials to a single {\sc
gcd} of two polynomials.  One of the first steps when computing the
{\sc gcd} of two polynomials is determining the
content\index{content!of a polynomial} of the polynomial, \ie, the
{\sc gcd} of the polynomial's coefficients.  Some properties of
polynomial contents and how they can be effectively computed are
discussed in \sectref{PRS:Content:Sec}.

In \sectref{PRS:Growth:Sec} we define the polynomial remainder
sequence and illustrate some of the issues that make its computation
more complex than arithmetic operations.  One component of computing a
{\sc prs} is computing a ``rationalized'' quotient of two polynomials.
This is discussed in \sectref{PRS:Quotient:Sec}.  The most efficient
classical {\sc prs} algorithm is the \keyi{subresultant remainder
sequence}, which is discussed in \sectref{Subresultant:PRS:Sec}.

\section{Generalities}
\label{PRS:Intro:Sec}

Recall that a \keyi{zero divisor} of a commutative ring $R$ is a
non-zero element of $R$, $a$, such that there exists another non-zero
element of $R$, $b$, and $ab = 0$.  An \keyi{integral domain} is a
commutative ring where $0 \not= 1$ and that does not contain any {\em
zero divisors}.

Let $f$ and $g$ be elements of an integral domain $R$.  If there
exist $a, h \in R$ such that $f = ah$, then $a$ and $h$ are said to
be divisors\index{divisor} of $f$.  If there exist $a, b, h \in R$
such that $f = ah$ and $g = bh$ then $h$ is said to be a \keyi{common
divisor} of $f$ and $g$.  $a$ and $b$ are said to be the {\em
cofactors}\index{cofactor} of $f$ and $g$ with respect to $h$.  If
every common divisor of $f$ and $g$ divides $h$, then $h$ is said to
be a \key{greatest common divisor} ({\sc gcd}) of $f$ and $g$.  The
ratio of two greatest common divisors is a unit.

We let $\gcd(f, g)$ denote a {\sc gcd} of $f$ and $g$.  The
\keyi{least common multiple} ({\sc lcm}) of $f$ and $g$ is defined to be
\[
\lcm(f,g) = \frac{f\cdot g}{\gcd(f, g)}.
\]
These definitions can easily be extended to {\sc gcd}'s and {\sc
lcm}'s of several elements of $R$:
\[
\gcd(f_1, \ldots, f_k) = \gcd(\gcd(f_1,\ldots, f_{k-1}), f_k) 
  = \gcd(f_1,\gcd(f_2,\ldots, f_k))
\]
and similarly for {\sc lcm}'s.
\[
\lcm(f_1, \ldots, f_k) = \lcm(\lcm(f_1,\ldots, f_{k-1}), f_k) 
  = \lcm(f_1,\lcm(f_2,\ldots, f_k)).
\]
\addsymbol{$\gcd(F, G)$}{the greatest common divisor of two elements of ring}
\addsymbol{$\lcm(F, G)$}{the least common multiple of two elements of ring}

Now, let $F$ and $G$ be elements of $R[\vec{X}]$ that are not elements
of $R$, where $R$ is an integral domain.  Among all the common
divisors of $F$ and $G$, let $H_i$ denote one with maximal degree in
$X_i$.  The least common multiple of $H_i$ and $H_j$, $H_{ij} =
\lcm(H_i, H_j)$ is a common divisor with the maximal degree in $X_i$
and $X_j$ among all common divisors.  Thus there exist common divisors
of $F$ and $G$ that have the maximal degree in $X_i$, for all $X_i$,
of any common divisor.  We call any of these polynomials a {\em
greatest common divisor of $F$ and $G$ up to elements of
$R$}.\index{greatest common divisor!of polynomials}

\section{GCD of Several Quantities}
\label{GCD:Several:Sec}

Let $F_i$ be elements of a polynomial ring $A = R[X_1, \ldots, X_v]$,
where $R$ is an integral domain.  Assume we wish to compute $H$, the
{\sc gcd} of the elements of ${\cal F} = \{F_1, \ldots, F_{\ell}\}$.
The straightforward approach computes
\[
H = \gcd(F_1, \gcd(F_2, \ldots)),
\]
which requires $\ell - 1$ {\sc gcd} calculations.  We call this the
{\em iterative multiple {\sc gcd}} algorithm.

For most rings that occur in practice, the following approach is much
more effective.  Choose $k_3, \ldots, k_{\ell} \in R$ and compute
\[
\begin{aligned}
F & = F_1 + k_3 F_3 + k_5 F_5 + \cdots, \\
G & = F_2 + k_2 F_2 + k_4 F_4 + \cdots
\end{aligned}
\]
Let $H' = \gcd(F, G)$.  Clearly, $H$ divides $H'$, but may be a bit
larger.  Now, remove all elements of ${\cal F}$ that are divisible by
$H'$ and then add $H'$ to ${\cal F}$.  If ${\cal F}$ has more than one
element repeat this process, choosing new random values for the $k_i$.
When ${\cal F}$ contains a single element that element is the {\sc
gcd} of the $F_i$.

The expected number of {\sc gcd}'s that need to be computed is very
close to $1$.  Thus this probabilistic approach permits one to compute
the {\sc gcd} of a set of polynomials using only a single {\sc gcd}
computation. 
This is a dramatic improvement as long as the single {\sc gcd} is not
much more difficult than the individual {\sc gcd} calculations used in
the iterative {\sc gcd} algorithm.

Since the $k_i$ are elements of $R$, $F$ and $G$ are each roughly the
same size as the set of input polynomials.  If $R$ is a more complex
ring, say rational functions over some smaller ring, then the $k_i$
should be chosen from the bottom scalar ring.  If this is done, then
the single {\sc gcd} computation used in this method will not be much
more difficult than any of the individual {\sc gcd}'s of the iterative
method.

The remaining issue is determining the expected number of {\sc gcd}'s
used in this method.  A proper analysis requires the
\keyi{resultant} which will be defined in
\sectref{Poly:Resultant:Sec}.  Thus, we leave this analysis to the
reader, although we sketch the basic ideas below.  

The only property of the \key{resultant} needed is that, given two
polynomials $F(X), G(X) \in R[X]$ there is an element of $R$ that is zero
if and only if $F$ and $G$ have a common factor. This element is called 
the \key{resultant} of $F$ and $G$, $\res_X(F(X), G(X))$. Furthermore, the
resultant is a polynomial function of the coefficients of $F$ and $G$.

Now consider the case of three polynomials $F_1$, $F_2$ and
$F_3$ in $R[X]$ and assume they are relatively prime.  We would like
to know for how many values of $k$, $F_1$ and $F_2 + k F_3$ have a
common factor.  Let $G(k) = \res_X (F_1, F_2 + k F_3)$.  The only
values of $k$ for which 
\[
\gcd(F_1, F_2, F_3) \not= \gcd(F_1, F_2 + k F_3)
\]
are the zeroes of $G(k)$.  But there can only be a finite number of
those.  More variables and more polynomials are treated similarly.

\section{Polynomial Contents}
\label{PRS:Content:Sec}



Let $F(X) = f_0 X^n + f_{1} X^{n-1} + \cdots + f_n$ be a polynomial in
$X$ over an integral domain, $f_0\not=0$.  The greatest common divisor
of $f_0, \ldots, f_n$ is defined to be the {\em content} of
$F$,\index{content!of a polynomial} which we denote by $\cont F$.  If
the content of a polynomial is 1, then the polynomial is said to be
\keyi{primitive}\index{primitive polynomial}.  The {\em primitive part} of
$F$\index{primitive part!of a polynomial} is $\prim F = F(X)/(\cont F)$, which
is a primitive polynomial.
\addsymbol{$\prim F$}{the primitive part of a polynomial}
\addsymbol{$\cont F$}{the content of a polynomial}

For multivariate polynomials, the content depends upon which variable
is chosen to be main.  We denote the choice with a subscript,
\eg{} $\cont_{X} F$ will be an element of $R[Y,Z]$ if $F$ is in $R[X,
Y,Z]$.  The {\em scalar content}\index{content! scalar, of a
polynomial} of $F$ is the common factor of the coefficients of
monomials of $F$.  It is an element of $R$ and is denoted by
$\cont_{R} F$.  For polynomials with coefficients in $\Z$, this is
often called the \keyi{integer content}.

Notice that a polynomial that is primitive with respect to one
variable may have a non-trivial content with respect to another.  For
instance,
\[
\begin{aligned}
F(X, Y) & = X^2 + (Y + 1) X + Y, \\
  & = (X+1) Y + X (X+1)
\end{aligned}
\]
is primitive with respect to $X$, but not with respect to $Y$.

If $FG$ is a primitive polynomial then both $F$ and $G$ are
primitive.\index{primitive polynomial}
The following proposition, called {\Gauss}'s Lemma,\index{Gauss's
lemma} gives the converse of this result.

\begin{proposition}[Gauss] \label{Gauss:Lemma:Prop}
If $F$ and $G$ are primitive polynomials over an entire ring $R$ then
$FG$ is also primitive.
\end{proposition}

\begin{proof}
Let 
\[
\begin{aligned}
F(X) &= f_0 X^m + f_{1} X^{m-1} + \cdots + f_m,\\
G(X) &= g_0 X^n + g_{1} X^{n-1} + \cdots + g_n,
\end{aligned}
\]
and let $p \in R$ be a prime that divides the content of $FG$.  Denote the
images of $F$ and $G$ in $R/(p)[X]$ by $\hat F$ and $\hat G$ respectively.  The
images of $F$ and $G$ will each be non-zero since they are primitive, so 
\[
\begin{aligned}
\hat F(X) &= \hat f_{m-r} X^r + \hat f_{m-r-1} X^{r-1} + \cdots + \hat f_m,\\
\hat G(X) &= \hat g_{n-s} X^s + \hat g_{n-s-1} X^{s-1} + \cdots + \hat g_n,
\end{aligned}
\]
where $\hat{f}_{m-r}$ and $\hat{g}_{n-s}$ are non-zero.  Their product is zero
since $p$ divides the content of $FG$, but this means that $\hat f_{m-r}
\hat g_{n-s}$ is zero, which is not possible since $R/(p)$ is an integral
domain.
\end{proof}

\noindent
This result means that when dealing with polynomials over an integral
domain the primitive part\index{primitive part!of a polynomial} of a
reducible polynomial is still reducible.  Thus it is worthwhile
removing the content of polynomials in {\sc gcd} and factoring
calculations if the resulting problem is easier.

A variant of this observation is very useful for multivariate
polynomials.  Assume $F$ and $G$ are polynomials of $R[\vec{X}]$ and
assume their {\sc gcd} is $H$.  If the variable $X_i$ occurs in $G$,
but not in $F$ then $H$ does not involve $X_i$.  So,
\[
H = \gcd(F, G) = \gcd(F, \coef(G, X_i^0), \coef(G, X_i), \coef(G,
X_i^2), \ldots).
\]
This process can be repeated until all of the polynomials of which we
are computing the {\sc gcd} involve the same variables.  Notice that
using the techniques of \sectref{GCD:Several:Sec} the size of the two
polynomials with which the {\sc gcd} is finally computed is no larger
than the size of $F$ and $G$.

If $H$ is not primitive\index{primitive polynomial} then its content is
\[
\cont H = \gcd(\cont F, \cont G) = \gcd(f_0, \ldots, f_m, g_0, \ldots,
g_n).
\]
So, we can determine the content of $H$ with a single {\sc gcd} on
average.  Dividing $F$ and $G$ by $\cont H$, we get two polynomials
whose {\sc gcd} is primitive although they themselves may not be
primitive.

From a theoretical standpoint, this last step is problematic.
Consider the polynomial
\[
F(X_1, X_2) = (X_2^m - 1) X_1^m + (X_2^{m-1} - 1) X_1^{m-1} + 
 \cdots + (X_2 - 1) X_1,
\]
which has $2m$ terms when represented in expanded form.  If the
content of $H$ is $X_2 - 1$ then dividing out $X_2 - 1$ can be very
expensive:
\[
\frac{F(X_1, X_2)}{X_2 -1} = 
(X_2^{m-1} + X_2^{m-2} + \cdots + 1)X_1^m + \cdots + X_1,
\]
which has $m(m+1)/2$ non-zero terms.  However, in practice this type
of problem does not often occur and it is usually safe to factor out
the content of $H$.  The more advanced algorithm of
\chapref{Poly:GCD:Chap} can deal with a polynomial represented as
\[
\frac{F(X_1, X_2)}{X_2 - 1} = 
\frac{X_2^m - 1}{X_2 -1} X_1^m + \frac{X_2^{m-1} - 1}{X_2 - 1} X_1^{m-1} + 
 \cdots + \frac{X_2 - 1}{X_2 -1} X_1,
\]
where the coefficients are represented as fractions that are not
reduced to lowest terms.

\medskip
At this point we assume that the {\sc gcd} problem is reduced to the
{\sc gcd} of two polynomials, both of which involve the same variables
and where the {\sc gcd} is primitive.

\section{Coefficient Growth}
\label{PRS:Growth:Sec}

An analogue of the Euclidean algorithm of
\sectref{Integer:Euclidean:Sec} can be applied to compute the {\sc gcd} of
two polynomials. This technique is simplest for univariate polynomials
over a field, but can easily be extended to polynomials over integral
domains.  For now let $F_1(X)$ and $F_2(X)$ be polynomials over a
field.  $F_1(X)$ can be divided by $F_2(X)$ to obtain a quotient
$q_1(X)$ and a remainder $F_3(X)$ that satisfy
\begin{equation}
F_1(X) = q_1(X)\cdot F_2(X) + F_3(X) \qquad \deg(F_3) < \deg(F_2).
\label{Poly:Remainder:Eq}
\end{equation}
Just as with integers, the {\sc gcd} of $F_2(X)$ and $F_3(X)$ must also be the
{\sc gcd} of $F_1(X)$ and $F_2(X)$. This process can be repeated with $F_2(X)$
and $F_3(X)$ to obtain a quotient $q_2(X)$ and a remainder $F_4(X)$. 
\[
\begin{aligned}
  F_2(X) &= q_2(X)\cdot F_3(X) + F_4(X) \\
     &\qquad\qquad\qquad\vdots\\
  F_{n-3}(X) &= q_{n-3}(X)\cdot F_{n-2}(X) + F_{n-1}(X)\\
  F_{n-2}(X) &= q_{n-2}(X)\cdot F_{n-1}(X) + F_{n}(X)
\end{aligned}
\]
The degrees of the $F_i(X)$ decrease monotonically.  Hence, for some
$i$, the degree of $F_i(X)$ must be zero.  If $F_i(X)$ is a non-zero
constant then the next remainder in the sequence, $F_{i+1}(X)$, is
zero.  Without loss of generality we may assume that $F_{n+1}(X) = 0$.
The {\sc gcd} of $F_1(X)$ and $F_2(X)$ is then $F_n(X)$.  Notice that
the degree of the polynomial is being used in place of the magnitude
metric used for integers in \sectref{Integer:Euclidean:Sec}.  
The source of the exponential cost of computing the {\sc gcd} of two
polynomials can be seen by examining the calculation of the greatest
common divisor of
\[
\begin{aligned}
F_1(X)&= X^8 + X^6 - 3 X^4 - 3 X^3 + 8X^2 + 2X -5 
\qquad \mbox{and}\\
F_2(X)&= 3X^6 + 5X^4 - 4X^2 - 9X + 21.
\end{aligned}
\]
(This is the traditional example used to illustrate polynomial {\sc gcd}
algorithms originally suggested by {\Knuth} \cite{Knuth:II}.)\enspace Using
the Euclidean algorithm we obtain the following \keyi{polynomial remainder sequence}:
\[
\begin{aligned}
  F_{1} & = X^8 + X^6 - 3 X^4 - 3 X^3 + 8X^2 + 2X -5, \\
  F_{2} & = 3X^6 + 5X^4 - 4X^2 - 9X + 21, \\
  F_{3} & = -\frac{5}{9}X^4 + \frac{1}{9}X^2 - \frac{1}{3}, \\
  F_{4} & = -\frac{117}{25}X^2 - 9 X - \frac{441}{25}, \\
  F_{5} & = \frac{233150}{19773}X - \frac{102500}{6591}, \\
  F_{6} & = -\frac{1288744821}{543589225}.
\end{aligned}
\]
Since the last non-zero term in the polynomial remainder sequence is a
unit, the {\sc gcd} of these two polynomials is an element of the coefficient
ring $\Q$.  Since the polynomials were primitive to begin with, as elements
of $\Z[X]$, they are relatively prime.

First, notice that even though the polynomials could be viewed as elements of
$\Z[X]$, the very first division leads to fractional denominators.
This is why $\alpha_i$ should be chosen different from $1$ and we must
use \key{pseudo-division} if we want the {\sc gcd}  algorithm to work over
integral domains instead fields.\Marginpar{Needs work, $\alpha_i$.}

Second, coefficients of the entries in the polynomial remainder
sequence grow until they are substantially larger than the final {\sc gcd},
which is $1$ in this case.  This problem is called \keyi{intermediate
expression swell} and is exacerbated both by using \key{pseudo-division}
and by increasing the number of variables. In the following {\sc prs} we
have used bivariate polynomials of smaller degree than the previous
example, yet the polynomials that result are still quite large.
\[
\begin{aligned}
  F_{1} & = X^4 + X^3 - W\\
  F_{2} & = X^3 +2X^2 +3WX - W +1\\
  F_{3} & = (-3W +2)X^2 + (4W-1)X -2W + 1 \\
  F_{4} & = \frac{27 W^3 -2W^2 -11W +3}{9W^2 -12W +4} X 
            + \frac{9W^3-W^2+4W+1}{9W^2-12W +4}\\
  F_{5} & = \frac{-729 W^7  - 738 W^6 -474W^5 +725W^4
                   + 81W^3 -162W^2 +68W-8}{ 729W^6 -108W^5 
                   - 590W^4 +206W^3 +109W^2-66W+9}
\end{aligned}
\]
Had we used three or more variables the elements of the {\sc prs} could have
filled many pages.

These two problems, eliminating quotients and keeping the intermediate
expressions as small as possible are treated in
\sectref{PRS:Quotient:Sec} and \sectref{Subresultant:PRS:Sec} respectively. 

\section{Pseudo-Quotients}
\label{PRS:Quotient:Sec}

In practice it is not reasonable to use rational coefficients in the
{\sc prs}.  The number of {\sc gcd}'s required to keep these
coefficients reduced to lowest terms is just too great, and not
reducing them leads to horrendous expression growth.  This is
especially true for multivariate polynomials where reducing the
coefficients to lowest terms would require additional polynomial {\sc
gcd}'s.  If we relax \eqnref{Poly:Remainder:Eq} slightly we can obtain
a \keyi{pseudo-quotient} and \keyi{pseudo-remainder} that will always
have integral coefficients.

Let $F_1(X)$ and $F_2(X)$ be primitive polynomials over a ring $R$ and
denote the leading coefficient of $F_i(X)$ by $f_i$.  The difference in
their degrees is denoted by $\delta_{i} = \deg F_{i} - \deg F_{i+1}$.  The
pseudo-remainder $\prem(F_{1}, F_{2}) = R(X)$ and pseudo-quotient $Q(X)$
satisfy
\begin{equation}
f_2^{\delta_1+1} F_1(X) = F_2(X)\cdot Q(X) + R(X). \qquad \deg(R) < \deg F_2
\label{Pseudo:Remainder:Eq}
\end{equation}
The most straightforward way to compute the pseudo-remainder of two
polynomials $F_{1}$ and $F_{2}$ is to multiply $F_{1}$ by
$f_{2}^{\delta_{1}+1}$ and then use the usual long division routine to
compute the remainder.  However, $f_{2}$ could itself be a polynomial
so $f_{2}^{\delta_{1}+1} F_{1}$ may be substantially larger than
$F_{1}$.  The routine \keyw{PolyPseudoRemainder} below computes the
pseudo-remainder but is careful to introduce factors of $f_{2}$ only
when needed.  At the end, additional factors of $f_{2}$ are added to
ensure the result satisfies \eqnref{Pseudo:Remainder:Eq}.
\begindsacode
PolyPseudoRemainder ($U(X)$, $V(X)$) := $\{$ \\
\> if $\deg U < \deg V$ then return($U$);\\
\> $W \leftarrow U$; \\
\> $\delta \leftarrow \deg U - \deg V$; \\
\> loo\=p do \{\\
\> \> $k \leftarrow \deg W - \deg V$; \\
\> \> $W \leftarrow (\lc V) W -  (\lc W)\cdot X^{k} \cdot V$; \\
\> \> if $W = 0$ then return($0$); \\
\> \> elif $\deg W < \deg V$ then return $W (\lc V)^{k}$ \\
\> \> eli\=f $k > \delta + 1$ \\
\> \> \> then $W \leftarrow W (\lc V)^{k - (\delta + 1)}$\\
\>\> $\}$\\
\> $\}$
\enddsacode

Using the remainder computed by \keyw{PolyPseudoRemainder} as the next
term in the polynomial remainder sequence leads to rather severe
expression swell as we see:
\begin{equation}
\begin{array}[15pt]{lr}
  F_1=&1, 0, 1, 0, -3, -3, 8, 2, -5\\
  F_2=&3, 0, 5, 0, -4, -9, 21\\
  F_3=&-15, 0, 3, 0, -9\\
  F_4=&15795, 30375, -59535\\
  F_5=& 1254542875143750, -1654608338437500\\
  F_6=& 12593338795500743100931141992187500
\end{array}
\label{Euclidean:PRS:Eq}
\end{equation}
Here we have only listed the coefficients of $F_i$ for succinctness.

This algorithm is called the \key{Euclidean {\sc gcd}} algorithm since
it is an immediate modification of the Euclidean algorithm for
polynomials over fields to an algorithm that is valid for polynomials
over rings.  The {\sc prs} is called a {\em Euclidean {\sc prs}}.  We
include the code for the Euclidean algorithm here for comparison with
other algorithms.
\begindsacode
PolyEuclideanGCD ($U$, $V$) := $\{$ \\
\> if $V=0$ then $U$ \\
\> else PolyEuclideanGCD($V$, PolyPseudoRemainder($U$, $V$)); \\
\> $\}$
\enddsacode

\noindent
As with the other polynomial remainder sequence algorithms discussed here,
\keyw{PolyEuclideanGCD} assumes that $U$ and $V$ are primitive polynomials
over an entire ring.

The coefficient growth produced by the Euclidean algorithm in
\eqnref{Euclidean:PRS:Eq} is clearly unacceptable.  Most of these
polynomials are not primitive.  Removing their content at each step we
get a sequence of polynomials that is much smaller:
\begin{equation}
\begin{array}[15pt]{lr}
F_1=&1, 0, 1, 0, -3, -3, 8, 2, -5\\
F_2=&3, 0, 5, 0, -4, -9, 21\\
F_3=&-5, 0, 1, 0, -3\\
F_4=&13, 25, -49\\
F_5=& 4663, -6150\\
F_6=& 1
\end{array}
\label{Primitive:PRS:Eq}
\end{equation}

This is known as the \keyi{primitive {\sc prs}} and the algorithm that
generates it is called the {\em primitive} {\sc gcd}
algorithm.\index{GCD! primitive algorithm} It is clear that no {\sc
gcd} algorithm based on {\sc prs}'s can have smaller coefficients.
Unfortunately, a large number of coefficient {\sc gcd}'s are required
to remove the content from each of the terms in the {\sc prs}.  In the
case of multivariate calculations, this is especially costly.
Evidence to this effect is given by {\Collins}
\cite{Collins:Subresultant}.  The primitive polynomial {\sc gcd} 
algorithm is given below.

\begindsacode
PolyPrimitiveGCD ($U$, $V$) := $\{$\\
\> if $V=0$ then $U$ \\
\> else Poly\=EuclideanGCD($V$, \\
\> \>        PolyPrimitivePart(PolyPseudoRemainder($U$, $V$));\\
\> $\}$
\enddsacode

The Euclidean {\sc prs} and the primitive {\sc prs} are two examples
of polynomial remainder sequences where the coefficient domain does
not need to be a field.  In order to compare different polynomial
remainder sequences, we make the following formal definition of a {\sc
prs}.  A sequence of polynomials, $F_1, F_2, F_3, \ldots, F_n \in
R[X]$ that satisfies
\[
\beta_i F_i(X) = \alpha_i F_{i-2}(X) - q_i(X) F_{i-1}(X), 
  \qquad \deg F_i < \deg F_{i-1}
\]
where $\alpha_i$ and $\beta_i$ are elements of $R$ and $q(X)$ is a
polynomial over $R$, is called a \keyi{polynomial remainder sequence}
generated from $F_1$ and $F_2$.  If $R$ is a field then $\alpha_i$ and
$\beta_i$ can be chosen to be $1$.  If $R$ is not a field, but an
integral domain, then by choosing $\alpha_i$ carefully we can
eliminate inexact divisions in the remainder sequence.  By choosing
$\beta_i$ judiciously the coefficients of the $F_i$ can be kept
relatively small.  The process of computing $F_i$ from $F_{i-2}$ and
$F_{i-1}$, when $\alpha_i \not= 1$ is called a \keyi{pseudo-division}.

In keeping with our earlier terminology we denote the leading
coefficient of $F_i$ by $f_i$ and let $\delta_i = \deg F_i- \deg
F_{i+1}$.  In all of the algorithms discussed, we let $\alpha_{i} =
f_{i-1}^{\delta_{i-2}+1}$, so that $\beta_{i} F_{i} = \prem(F_{i-1},
F_{i-2})$. 

Thus polynomial remainder sequences are determined by their choice of
$\beta_{i}$.  The larger the $\beta_{i}$ are, the smaller the terms in
the {\sc prs}.  The Euclidean {\sc prs} is the most conservative and
uses $\beta_i = 1$.  At the other extreme, the Primitive {\sc prs}
uses the largest $\beta_{i}$ possible, $\beta_i = \cont(
\prem(F_{i-1},F_{i-2}))$ so its terms are as small as possible.
However, the cost of computing the $\beta_i$ used in a primitive {\sc
prs} is quite high.  The algorithm considered in
\sectref{Subresultant:PRS:Sec} uses a value of $\beta_i$ that does not
require any coefficient {\sc gcd} calculations and thus is easier to
compute.  Though not quite as large as $\cont(\prem(F_{i-1},
F_{i-2}))$ they are much larger than what is used in the Euclidean
algorithm.

The degrees of the $F_i$, as polynomials in $X$, are independent of
the choice of the $\alpha_i$ and $\beta_i$, since the $\alpha_{i}$ and
$\beta_{i}$ are elements of $R$.  Thus the $\delta_{i}$ also depend
only on $F_{1}$ and $F_{2}$ and not on the particular {\sc prs} used.  We
say that a polynomial remainder sequence is {\em
normal}\index{polynomial remainder sequence!normal} if all of the
$\delta_i$ are equal to $1$.  If a $\delta_i$ is greater than $1$ we
will call the {\sc prs} {\em abnormal\/}.\index{polynomial remainder
sequence!abnormal}

In \eqnref{Euclidean:PRS:Eq} and \eqnref{Primitive:PRS:Eq} the sequence of
$\delta_{i}$ is:
\[
\begin{array}[15pt]{lr}
  F_{1} = X^{8}+X^{6} -3X^{4}-3 X^{3} +8 X^{2} + 2 X -5 & \delta_{1} = 2\\
  F_{2} = X^{6}+5 X^{4} - 4 X^{2} -9 X + 21 & \delta_{2} = 2\\
  F_{3} = -5X^{4} + X^{2} - 3 & \delta_{3} = 2\\
  F_{4} = 13 X^{2} + 25 X - 49 & \delta_{4} = 1\\
  F_{5} = 4663 X - 6150 & \delta_{5} = 1 \\
  F_{6} = 1
\end{array}
\]
Thus the first three steps of this remainder sequence are abnormal.

\section{Subresultant Polynomial Remainder Sequence}
\label{Subresultant:PRS:Sec}

In the late sixties and early seventies, {\Collins} and {\BrownWS}
managed to simplify the formulation of the subresultant {\sc gcd}
algorithm.  This is now the best of the {\sc gcd} algorithms that use
a full polynomial remainder sequence.  However, for sufficiently
sparse polynomials {\BrownWS}'s analysis \cite{Brown78} indicates that
even the subresultant algorithm will require exponential time to
compute the {\sc gcd}, especially if the {\sc prs} is abnormal.

As with the reduced {\sc gcd} algorithm the way to decrease the growth
of the coefficients of the {\sc prs} is to pick a large value for
$\beta_i$.  However, we would like to be able to pick $\beta_i$
without having to compute a {\sc gcd}.  The subresultant {\sc gcd}
algorithm does this, but the proof that the $\beta_i$ chosen actually
leads to a valid {\sc prs} is not easy.  A thorough discussion of the
principles behind the subresultant algorithm and its history is given
in \cite{Loos82}.

Again let $\delta_i = \deg F_{i} - \deg F_{i+1}$.  For the {\em
subresultant} {\sc prs}, $\beta_i$ can be chosen as follows
\[
\begin{aligned}
\beta_3&= (-1)^{\delta_1+1}\\
\beta_i&= (-1)^{\delta_{i-2}+1}f_{i-2}h_{i-2}^{\delta_{i-2}}, 
\qquad i = 4,\ldots, k+1;
\end{aligned}
\]
where
\[
\begin{aligned}
h_2&=f_2^{\delta_1}\\\noalign{\vskip 2pt}
h_i&=f_i^{\delta_{i-1}}h_{i-1}^{1-\delta_{i-1}}, 
\qquad i = 3, \ldots, k.
\end{aligned}
\]
The fact that the $\beta_i$ divide the pseudo-remainder at each step
is rather tricky to prove and we do not do so here.  A careful
presentation of the issues surrounding the subresultant {\sc gcd}
algorithm is given in {\Loos}' paper \cite{Loos82}.

The subresultant {\sc prs} for our standard example is
\[
\begin{array}[15pt]{lr}
  F_{1}=&1, 0, 1, 0, -3, -3, 8, 2, -5\\
  F_{2}=&3, 0, 5, 0, -4, -9, 21\\
  F_{3}=&15, 0, -3, 0, 9\\
  F_{4}=&65, 125, -245\\
  F_{5}=&9326, -12300\\
  F_{6}=&260708
\end{array}
\]

At each normal step of a {\sc prs} the size of the coefficients tends
to grow linearly.  Abnormal steps lead to faster growth.

The subresultant {\sc prs} algorithm does an admirable job of
minimizing intermediate expression swell in the computation of the
polynomial remainder sequence.  If we are interested in the {\sc gcd}
of the two polynomials, only the last term of the {\sc prs} is of
interest.  If the {\sc prs} is relatively short then it will not have
a chance to grow too much.  In this case the {\sc gcd} of the two
polynomials will tend to be a large factor of one of the two
polynomials.  However, if the {\sc gcd} of two polynomials is small,
\eg, the two polynomials are relatively prime, then the {\sc prs}
involved will tend to be rather long.  In this case the swell involved in
the {\sc prs} can be extremely costly.

The following routine implements the subresultant {\sc gcd} algorithm.

\begindsacode
PolySubresultantGCD ($U$, $V$) := $\{$\\
\> $h \leftarrow 1$; \\
\> loo\=p while $V \not= 0$ do \{ \\
\> \> $\delta \leftarrow \deg U - \deg V$; \\
\> \> $\beta \leftarrow (-1)^{\delta+1} \times \lc U \times h^{\delta}$; \\
\> \> $h \leftarrow h \times ((\lc V)/h)^{\delta}$; \\
\> \> $(U, V) \Leftarrow (V, \prem (U, V)/\beta)$; \\
\> \> $\}$ \\
\> if $\deg_X(U) = 0$ then return ($1$); \\
\>\> else return($U$); \\
\> $\}$
\enddsacode 

\noindent
Notice that if the final term in the {\sc prs} is free of the main
variable then $1$ is returned since the polynomials are then
relatively prime.  The polynomials that arise in the subresultant {\sc
prs} are called the \keyi{subresultants} of $F_1$ and $F_2$.  Additional
properties of the subresultants are discussed in the next chapter,
where the {\em resultant} is introduced.

\section*{Notes}

\small

\notesectref{GCD:Several:Sec} The technique for computing {\sc gcd}'s
of several quantities discussed in this section can also be applied to
{\sc gcd}'s of integers, but then the expected number of {\sc gcd}'s
is no longer one.

To analyze this technique we can assume
that $f_0, f_1, f_2 \in R$ have no common factor, so that $g=1$.  Thus
we want to know how often $f_0$ and $f_1 + k f_2$ are relatively
prime.  Assume $k$ is a rational integer and denote the number of
values of $k$ between $1$ and $N$ for which $f_0$ and $f_1+k f_2$ are
relatively prime by $G_R(N)$.  When $R = \Z$, the example $f_0 = 4$,
$f_1 = 3$ and $f_2=5$ shows that
\[
\frac{N}{2} \le G_{\Z}(N) \le N.
\]
\propref{RelPrime:Pairs:Prop} suggests that on average we will have
\[
G_\Z(N) = \frac{6}{\pi^2} N + O(\log N).
\]

\notesectref{PRS:Content:Sec} 
\propref{Gauss:Lemma:Prop}, {\Gauss}'s Lemma, was first shown by
{\Gauss} in {\em Disquisitiones Arithmetic\ae}
\cite{Gauss:Disquisitiones}, \S41. 

\notesectref{PRS:Quotient:Sec}  When dealing with polynomials whose
coefficients are floating point numbers it is not clear what it means
for one polynomial to divide another, or even how to define the {\sc
gcd} of two polynomials.  Some work on this problem has been done by
{\Schoenhage} \cite{Schonhage85}.

{\Ma} and {\Gathen} have analyzed a number of variants of the the
Euclidean algorithm for univariate polynomials over finite fields
\cite{Ma90}. 

\normalsize

\index{greatest common divisor!of polynomials|)}
