%$Header: /usr/u/rz/AMBook/RCS/series.tex,v 1.1 1992/05/10 19:36:06 rz Exp rz $
\chapter{Formal Power Series}
\label{Series:Chap}

This chapter discusses a variety of algorithms for manipulating
\keyi{formal power series}.  Formal power series are infinite power
series where we are not concerned with issues of convergence.  Thus,
both $f(t)$ and $g(t)$
\[
\begin{aligned}
f(t) &= 1 + t + \frac{t^2}{2!} + \frac{t^3}{3!} + \frac{t^4}{4!} +
\cdots, \\
g(t) & = 1 - t + 2! t^2 - 3! t^3 + 4! t^4 + \cdots
\end{aligned}
\]
are formal power series, even though $g(t)$ does not converge for any
non-zero value of $t$. 

Formal power series are to polynomials what the $p$-adic numbers are
to the integers (see \sectref{padic:Arith:Sec}).  A number of
operations can be efficiently performed with formal power series for
which there are no corresponding operations with polynomials.  The ring
of formal power series is a ``completion'' of the ring of polynomials.
\sectref{madic:Arith:Sec} discusses the mathematical foundations of
this concept in sufficient generality to include both $p$-adic numbers
and formal power series as special cases. 

Formal power series are also useful because the algorithms used for
performing arithmetic operations with them work equally well with
convergent power series.  Thus, the techniques discussed here can be
useful tools for problems in perturbation theory, asymptotic
approximations and error analysis.  Computations may be done with
formal power series and the resulting series then checked to see if
they appear to converge.  This is almost always only a plausibility
result since only a finite portion of the series can be examined.

\section{Introduction}
\label{FPS:Intro:Sec}

Let $k$ be a field and $k[t]$ be a ring of polynomials over $k$.  We
define a \keyi{valuation} on $k[t]$ as follows.  Let $f(t)$ be an
element of $k[t]$ and $\alpha$ be a positive number less than $1$.  If
$t^r$ divides $f(t)$, but $t^{r+1}$ does not, then define
\[
\|f(t)\|_{(t)} = \alpha^{r}.
\]
(This is analogous to the $p$-adic valuation described in 
\sectref{padic:Arith:Sec}.)
Under this valuation two polynomials are close if their low order
terms agree.  We can form Cauchy sequences like
\[
1,\quad 1+ t,\quad 1 + t +\frac{t^2}{2},\quad 
 1 + t +\frac{t^2}{2} + \frac{t^3}{6},\quad \ldots
\]
whose limit is denoted by the \keyi{formal power series}:
\[
1 + t +\frac{t^2}{2} + \frac{t^3}{6} + \cdots.
\]
The ring of formal power series in $t$ with coefficients in
$k$ is denoted by $k[[t]]$.

Many elements of $k[[t]]$ have multiplicative inverses even though
their images in $k[t]$ may not.  For instance,
\[
(1 + t) (1 - t + t^2 + \cdots + (-1)^n t^n) = 1 - t^{n+1}
\]
so the reciprocal of $1+t$ is
\[
\frac{1}{1+t} = 1 - t + t^2 - t^3 + \cdots.
\]
There are, however, elements of $k[[t]]$ that do not have inverses,
\eg, $t + t^2$.  Formally, we can construct an inverse for $t+t^2$ as
follows
\[
\frac{1}{t+t^2} = \frac{1}{t} \cdot \frac{1}{1+t} = 
\frac{1}{t} - 1 + t - t^2 + \cdots.
\]
This power series has negative exponents and thus is not an element of
$k[[t]]$.  

Allowing a finite number of negative exponents ensures that every
non-zero element of $k[[t]]$ has an inverse in this larger ring.  The
quotient field of $k[[t]]$ is denoted by $k((t))$ and is called the
ring of \keyi{formal Laurent series} in $t$.  It consists of power
series in $t$ that can have a {\em finite} number of negative
exponents.
\addsymbol{$k[[t]]$}{Ring of power series in $t$ with coefficients in $k$}
\addsymbol{$k((t))$}{Ring of Laurent series in $t$ with coefficients
in $k$}

\smallskip
In addition to containing the reciprocals of elements of $k[t]$ that
are not units, $k[[t]]$ also contains algebraic elements that are not
in $k[t]$.  For instance, if the characteristic of $k$ is not $2$
then
\begin{equation} \label{Sqrt1+x:Eq}
\sqrt{1+t} = 1 + \frac{1}{2} t - \frac{1}{8}t^2 + \frac{1}{16} t^3 -
\frac{5}{128} t^4 + \cdots.
\end{equation}
This can be verified by squaring the right hand side as a polynomial,
\[
\left(1 + \frac{1}{2} t - \frac{1}{8}t^2 + \frac{1}{16} t^3 -
\frac{5}{128} t^4\right)^2 =
1 + t - \frac{7}{128}t^5 + \frac{7}{512}t^6
- \frac{5}{1024} t^7
+ \frac{25}{16384} t^9.
\]
Thus, the right hand side of \eqnref{Sqrt1+x:Eq} is the square root of
$1+t$ modulo $t^5$.  

Again, however, not all algebraic functions of elements of $k[t]$ are
in $k[[t]]$.  In particular, $\sqrt{t+t^2}$ is not an element of
$k[[t]]$.  However, we can extend $k[[t]]$, to a larger field that
does contain an element corresponding to $\sqrt{t+t^2}$:
\[
\begin{aligned}
\sqrt{t+t^2} &= \sqrt{t}\sqrt{1+t} 
  = \sqrt{t}\left(1 + \frac{1}{2} t - \frac{1}{8}t^2 + \frac{1}{16} t^3 -
  \frac{5}{128} t^4 + \cdots\right),\\
& = t^{1/2} + \frac{1}{2} t^{3/2} - \frac{1}{8}t^{5/2} + \frac{1}{16} t^{7/2} -
  \frac{5}{128} t^{9/2} + \cdots.
\end{aligned}
\]

In this case we have passed from the ring $k[[t]]$ to the ring
$k[[t^{1/2}]]$.  Sometimes the coefficient domain needs to be extended
to express a root of a power series, \eg,
\[
\sqrt{2 + t} = \sqrt{2} + \frac{\sqrt{2} t}{4} - \frac{\sqrt{2} t^2}{32}
 + \frac{\sqrt{2} t^3}{128} - \frac{5 \sqrt{2} t^4}{2048} + \cdots.
\]
In general, an algebraic function of an element of $k[[t]]$ is an
element of $K[[t^{1/n}]]$, where $K$ is an \keyi{algebraic extension}
of $k$ and $n$ is some integer.

Notice that
\[
\sqrt[3]{t+t^2}
= t^{1/3} + \frac{t^{4/3}}{3} - \frac{t^{7/3}}{9} + \frac{5
t^{10/3}}{81}
 - \frac{10 t^{13/3}}{243} + \cdots
\]
can be represented as an element of $k[[t^{1/3}]]$.  An extension of
$k[[t]]$ that contains both $\sqrt{t+t^2}$ and $\sqrt[3]{t+t^2}$, must
contain $k[[t^{1/6}]]$.  Thus the algebraic closure of $k[[t]]$ cannot
be presented as $\hat{k}[[t^{1/n}]]$ for any finite $n$, where
$\hat{k}$ is the algebraic closure of $k$.
The issues raised by these types of problems are beyond the scope of
this book.

\smallskip
$k[[t]]$ contains many elements that are transcendental over $k[t]$.
For instance, if the characteristic of $k$ is zero then the following power series
\[
1 + t + \frac{t^2}{2!} + \frac{t^3}{3!} + \cdots + \frac{t^n}{n!} +
\cdots
\]
behaves just like the exponential function $e^t$, \eg, it is its own
derivative. Similar expansions exist for other elementary functions,
such as $\sin t$, $\cos t$, etc.

Despite the fact that all these transcendental functions can be
represented in $k[[t]]$, formal power series do have a \keyi{canonical
form}.  An example of how this can be used to advantage is the
following.  We know that
\[
\begin {aligned}
\sin t & = t - \frac{t^3}{6} + \frac{t^5}{120} - \cdots, \\
\cos t & = 1 - \frac{t^2}{2} + \frac{t^4}{24} + \cdots.
\end{aligned}
\]
Expanding $\sin^2 t + \cos^2 t$ as formal power series in $x$ gives
\[
\begin{aligned}
\sin^2 t + \cos^2 t & = 
   \left(t - \frac{t^3}{6} + \frac{t^5}{120} - \cdots\right)^2
  + 
   \left(1 - \frac{t^2}{2} + \frac{t^4}{24} + \cdots\right)^2, \\
& = \left(t^2 - \frac{t^4}{3} + \cdots\right) +
  \left(1 - t^2 + \frac{t^4}{3} + \cdots\right),\\
& = 1 + \cdots.
\end{aligned}
\]
Thus, $\sin^2 t + \cos^2 t$ is equal to $1$, at least up to order
$t^5$.  This technique can also be used to efficiently obtain the limits
of some analytic expressions.

\smallskip
Finally, we note that there are transcendental operations that can be
performed on formal power series that cannot be performed on
polynomials.  In particular, recall that power series expansion of the
\key{natural logarithm}
\begin{equation} \label{Series:Log:Eq}
\log (1 + t) = t - \frac{t^2}{2} + \frac{t^3}{3} - \frac{t^4}{4} +
\cdots.
\end{equation}
(Notice that this immediately means that $k[[t]]$ has infinite
transcendence degree over $k[t]$.)  The first few terms of $\log
(1+t+t^2)$ can be computed as follows:
\[
\begin{aligned}
\log (1 + t +t^2) & = t + t^2 - \frac{(t+t^2)^2}{2} + \frac{(t+t^2)^3}{3}
  - \frac{(t+t^2)^4}{4} + \cdots,\\
 & = t + \frac{t^2}{2} - \frac{2t^3}{3} + \frac{t^4}{4} + \cdots,
\end{aligned}
\]
where we have substituted $t+t^2$ for $t$ in \eqnref{Series:Log:Eq}.
These are the first few terms of the transcendental
expression $\log (1+t+t^2)$ in $k[[t]]$.  Using similar techniques it is
possible to compute the logarithm of many elements of $k[[t]]$.  Subtle
complications do arise however.  Consider the expansion of 
\[
\log(2+t) = \log 2 + \log (1 + \frac{t}{2}) = \log 2 + \frac{t}{2} -
\frac{t^2}{8} + \frac{t^3}{24} - \frac{t^4}{64} + \cdots.
\]
The power series expansion of $\log(2 + t)$ does not lie in $k[[t]]$,
or even $K[[t]]$, where $K$ is an algebraic extension of $k$, but
rather in $k(\log 2)[[t]]$.  It is necessary to extend $k$
transcendentally.

Even this however, may not be enough.  Notice that we cannot take the
logarithm of $t+t^2$ directly, but must instead proceed as follows:
\[
\begin{aligned}
\log (t + t^2) & = \log t + \log (1 + t), \\
  & = \log t + t - \frac{t^2}{2} + \frac{t^3}{3} - \frac{t^4}{4} +
\cdots,
\end{aligned}
\]
which can be interpreted as an element of $k(\log t)[[t]]$.  The following
example illustrates this:
\[
\frac{1}{\log (t + t^2)} = \frac{1}{\log t} -
\frac{1}{\log^2 t} t + \frac{2 + \log t}{\log^3 t} t^2
- \frac{\log^2 t + 3 \log t + 3}{3 \log^4 t} t^3
+ \cdots.
\]

For more complicated operations with objects of this form, it
is sometimes appropriate to interpret this expression as an element of
$k[[\log t]][[t]]$.  Many issues arise in this case.  Further
work on this sort of problem is discussed in \cite{Shackell1990-ws}.

These examples have noted the need to include two different types of
extensions to formal power series concepts: (1) power series with
negative exponents and (2) power series with fractional exponents.
\sectref{FPS:Arith:Sec} discusses alternative representations for
power series that support these two extensions and the algorithms for
performing the basic arithmetic operations with them.
\sectref{FPS:Expt:Sec} discusses the more complicated problem of
exponentiation of power series to a constant power and, as a special
case, computing the reciprocal of a formal power series and taking
roots of power series.

We have pointed out that substitution of one power series into another
is more complex and restrictive than is the case with polynomials.
Techniques for substituting one power series into another are
discussed in \sectref{FPS:Subst:Sec}. \sectref{FPS:Revert:Sec}
gives an algorithm for reverting a formal power series, another
operation that does not pertain to polynomials.  

\section{Power Series Arithmetic}
\label{FPS:Arith:Sec}

Formal power series are infinite objects, and as such they are
difficult to represent on finite computers.  There are three
approaches that can be taken for dealing with these infinite objects.
First, we can compute with the first $n$ terms of a power series,
which is called a \keyi{truncated power series}.  This is the most
common approach and appears to have the widest applicability.

Second, we can manipulate \keyi{generators} that produce successive
terms of a power series on request.  This approach has been
implemented in a few systems and intellectually is quite elegant.  It
is also an excellent vehicle for explaining the concepts of
closures and co-routines in programming language courses.  However,
the costs of this approach have often outweighed their benefits.

Finally, we can produce general formulas for the $k$-th term of the
power series.  This approach is difficult and is not always possible.
In particular, certain operations with power series may yield results
whose coefficients do not possess closed forms.

This chapter concentrates on the first approach, but many of the
algorithms presented here are on-line algorithms\index{on-line algorithm} 
so they can be
combined with the ``generator'' or ``stream'' style of 
programming\index{programming style, stream or generator}
\cite{Abelson1985-dl} to implement the second approach.

Let $k((t))$ be the field of formal power series with coefficients in
the field $k$.  The 
elements of $k((t))$ are of the form
\begin{equation} \label{PS:Form:Eq}
c_1 t^{e_1} + c_2 t^{e_2} + \cdots + c_n t^{e_n} + o(t^{e_n}),
\end{equation}
where the $c_i$ are elements of $k$ and the $e_i$ are rational
integers, $e_1 < e_2 < \cdots$.  The $o(t^{e_n})$ indicates that we
are only retaining those terms whose order in $t$ is less than or
equal to $e_n$.\footnote{Some computer algebra systems, \eg, \key{Maple},
choose to retain all terms of order less than $e_n$.} Note that the
$e_i$ are not elements of $k$, even if $\Z \subseteq k$.  Even if
the characteristic of $k$ is positive, the characteristic of the
exponent domain is zero.

In order for our algorithms to have the widest generality, and in
light of the discussion in the previous section, we would like to deal
with a larger class of power series. In particular, we need to extend
the set of exponents from just being integers to being rational
numbers,
\ie, we want to be able to manipulate series like
\begin{equation} \label{Sample:Series:Eq}
f = t^{1/3} + t^{1/2} + t + t^{12/5} + o(t^{12/5}).
\end{equation}
Rather than deal with rational exponents directly, which would appear
to be the most general approach, it is better to introduce a
\keyi{uniformizing parameter}, in terms of which $f$ is a formal power
series, \ie,
\[
f = (t^{1/30})^{10} + (t^{1/30})^{15} + (t^{1/30})^{30} +
(t^{1/30})^{72} + o((t^{1/30})^{72}).
\]
Thus the uniformizing parameter for $f$ is $t^{1/30}$.  Furthermore,
notice that the origin is a \keyi{branch point} of $f$ of order precisely
$30$.

Converting the power series to one involving a uniformizing parameter
simplifies the representation of the power series and the algorithms.
It also precisely captures the structure of the underlying
\keyi{Riemann surface} of the function.  For example, the function
with power series expansion
\[
t^{\frac{1}{2}} + t^{1+\frac{1}{3}} + t^{2+\frac{1}{5}} 
   + t^{3+\frac{1}{7}} + t^{4+\frac{1}{9}} + t^{5+\frac{1}{11}} +
\cdots,
\]
where the denominators of the exponents are odd numbers, cannot be
represented with a uniformizing parameter with finite branching order.
Furthermore, notice that the point $t = 0$ is not a simple branch
point of $f$.  

So generally, when given a power series in $t$, 
\[
f  = c_1 t^{e_1} + c_2 t^{e_2} + \cdots + c_n t^{e_n} + \cdots,
\]
where $e_1 < e_2 < e_3 < \cdots$, we assume it has a
uniformizing parameter $u = t^{1/r}$, $r$ a positive integer, and
write it as an element of $k((u))$:
\[
f = c_1 u^{r e_1} + c_2 u^{r e_2} + \cdots + c_n u^{r e_n} + \cdots,
\]
where the $r e_i$ are all rational integers.  $r$ is said to be the
\keyi{branching order} of $f$ at $t=0$.

We define the {\em valence}\index{valence!of a power series} of $f$ to
be $r e_1$.  We write this as $\val f = re_1$.  The {\em order} of $f$
is $r e_n$, which we write as $\Ord f = r e_n$.  For the truncated
power series in \eqnref{Sample:Series:Eq} we have
\[
\begin{aligned}
\val f & = 10, \\
\Ord f & = 72.
\end{aligned}
\]

In an implementation it is probably most efficient to represent the
truncated power series in terms of a uniformizing parameter and
then performing all operations with truncated power series with
integral exponents.  Since power series are often dense it is
convenient to store the coefficients in a vector.  Assume that $f$ is
a truncated power series, with branching order $r$.  Schematically, 
$f$ might be represented by a structure of the form
\begindsacode
TPS := struct \{ \\
\> BranchOrder : Integer; \\
\> Valence : Integer; \\
\> Order : Integer; \\
\> Coeffs : $k$[$\mbox{Order} - \mbox{Valence}$]; \\
\>\}
\enddsacode
The first three components are integers, while the fourth, {\tt
coeffs} is a vector of $\mbox{\tt Order} - \mbox{\tt Valence}$
elements of $k$.  {\tt Coeffs} contains the coefficients of $f$.  The
exponent corresponding to the $i$-th element of {\tt coeffs} is
\[
\frac{i + \val f}{r}.
\]

Addition and multiplication of truncated power series with the same
branching order is relatively straightforward.  When the power series
have different branching orders, say $r$ and $s$, they must each be
converted to a new series with branching order $\lcm (r, s)$.
Although this increases the size of the coefficient arrays by
introducing a large number of zeroes, the result of the binary
operation is typically not as sparse.  From here on, we assume that
truncated power series have the same branching order.

Before considering the special issues that arise in computations with
truncated power series, we quote the formulas for addition and
multiplication of complete, valence zero formal power series.
Assume,
\[
\begin{aligned}
P(t) & \displaystyle= p_0 + p_1 t + p_2 t^2 + \cdots = \sum_{i\ge 0} p_i t^i, \\
Q(t) & \displaystyle= q_0 + q_1 t + q_2 t^2 + \cdots = \sum_{i\ge 0}
q_i t^i.
\end{aligned}
\]
The sum of $P$ and $Q$ is easily computed by rearranging terms
\begin{equation}\label{FPS:Addition:Eq}
\begin{aligned}
P(t) + Q(t) & = p_0 + q_0 + (p_1 + q_1) t + (p_2 + q_2) t^2 + \cdots , \\
 & \displaystyle = \sum_{i\ge0} (p_i + q_i)t^i.
\end{aligned}
\end{equation}
Their product is
\begin{equation} \label{FPS:Multiplication:Eq}
\begin{aligned}
P(t) \cdot Q(t) & = \sum_{i\ge 0} \sum_{j\ge 0} p_i q_j t^{i+j}
 = \sum_{k\ge 0} \left(\sum_{0 \le \ell \le k} p_{\ell}
q_{k-\ell}\right)t^k,\\
& = p_0 q_0 + (p_0 q_1 + p_1 q_0)t + (p_0 q_2 + p_1 q_1 + p_2 q_0) t^2
+ \cdots.
\end{aligned}
\end{equation}
The rearrangements used to derive \eqnref{FPS:Addition:Eq} and
\eqnref{FPS:Multiplication:Eq} need further justification if we
were concerned with convergence issues.  Equation
\eqnref{FPS:Multiplication:Eq} is called the \keyi{Cauchy
multiplication formula}.

These expansions are easily modified for power series with non-zero
valence, \eg, assume $\val P = a$ and $\val Q = b$.  Then
\[
P(t) + Q(t) = \sum_{i\ge \min(a,b)}(p_i + q_i)t^i,
\]
where $p_i = 0$ if $i < a$ and $q_i = 0$ if $i < b$.  For
multiplication we have,
\[
P(t) \cdot Q(t) = \sum_{k\ge 0} \left(\sum_{0 \le \ell \le k}
p_{a+\ell} q_{b+k-\ell}\right) t^{a+b+k}.
\]
Notice that the valence of $P+Q$ is $\min (a, b)$ while $\val P\cdot Q
= a+b$.

With truncated power series we have the additional complication that
the order of a series resulting from a computation may differ
from the order of the constituent formal power series.   To illustrate
this problem, observe that
\[
\left[1 + 2 t + 3t^2 + 4 t^3 + o(t^3) \right] +
\left[ 1 + \frac{t}{2} + \frac {t^2}{3} + o(t^2) \right]
=
2 + \frac{5}{2} t + \frac{10}{3} t^2 + o(t^2).
\]
By \eqnref{FPS:Addition:Eq} the coefficient of $t^3$ in the sum depends
on the coefficients of $t^3$ in both addends.  Since one addend is
only given to order $2$, the best answer that can be given is also of
order $2$.

The same result is true for multiplying truncated power series of zero
valence---the order of a product is the minimum of the order of the
factors.  When the valences are not zero, the formula is slightly
more complex.  These results are summarized in 
\figref{TPS:Valences:Fig}.  The $\ge$ in the valence column indicates
that the valence could be larger if the leading coefficient of the
sum, product or power vanishes.  For products this can only occur if
the coefficient ring has zero divisors.\index{zero divisor}

\begin{figure}
\begin{center}
\begin{tabular}{c|c|c|}
\multicolumn{1}{c}{}& \multicolumn{1}{c}{valence} &
\multicolumn{1}{c}{order} \\ \cline{2-3}
$P(t)$ & $a$ & $m$ \\ \cline{2-3}
$Q(t)$ & $b$ & $n$ \\ \cline{2-3}
$P(t) + Q(t)$ & $\ge \min(a, b)$ & $\min(m,n)$ \\ \cline{2-3}
$P(t) \cdot Q(t)$ & $\ge a + b$ & $a + b + \min(m-a, n-b)$ \\ \cline{2-3}
$P(t)^s$ & $\ge sa$ & $(s-1) a + m$ \\ \cline{2-3}
$P\circ Q$ & $\ge ab$ & $\min(b (\val_2 P) + n - b, bm)$ \\ \cline{2-3}
\end{tabular}
\end{center}
\caption{Parameters for operations with truncated power
series\label{TPS:Valences:Fig}} 
\end{figure}

Notice that the sizes of the arrays used to store the coefficients of
the sum and product can be different from the sizes used for $P$ and
$Q$. For all but the largest truncated power series, adaptations of the
classical polynomial addition and multiplication algorithms can be
used.  On-line versions are also easily implemented due to the explicit 
formulas for the coefficients.

\section{Power Series Exponentiation}
\label{FPS:Expt:Sec}

Exponentiation is actually simpler for power series than for
polynomials.  The following proposition, due to J.~C.~P.~{\MillerJCP},
can be used to effectively compute $P^s(t)$. 
%\Marginpar{According to
%Knuth, see Henrici's article \cite{Henrici56}}

\begin{proposition}[J. C. P. Miller] \label{PS:Expt:PropP}
Let 
\[
P(t) = p_0 + p_1 t + p_2 t^2 + \cdots
\]
be a power series and denote the power series for $P(t)$ raised to the
$s$-th power by
\begin{equation} \label{FPS:Expon:Eq}
P(t)^s = p^{(s)}_0 + p^{(s)}_1 t + p^{(s)}_2 t^2 + \cdots.
\end{equation}
Then the individual coefficients of $P(t)^s$,  $p^{(s)}_i$, can be
expressed as follows: 
\begin{equation}\label{Powering:Formula:Eq}
p^{(s)}_0 = p_0^s, \qquad 
p^{(s)}_k = \frac{1}{p_0 k} 
   \sum_{1 \le \ell \le k} (\ell s - k +\ell) p_\ell p^{(s)}_{k-\ell}
\end{equation}
\end{proposition}

\begin{proof}
We construct a differential equation involving $P^s(t)$, $P(t)$
and their derivatives. Substituting the expansions of $P(t)$ and
$P^s(t)$ into this equation, produces a triangular system of
linear equations that can be solved for the unknown coefficients,
$p^{(s)}_k$.  This technique can be applied more generally and is
used in the next section. 

First, notice that by evaluating at $t = 0$, $p^{(s)}_0 = p_0^s$.  Next, 
let $y(t)$ denote $P^s(t)$.  $y(t)$ satisfies the 
following differential equation:
\begin{equation}\label{Powering:ODE:Eq}
s P^{\prime}(t) y(t) - P(t) y^{\prime}(t) = 0.
\end{equation}
Substituting the expansions for $y(t)$ and $P(t)$ into this equation
gives 
\[
s \left(\sum_{n\ge 0} n p_n t^{n-1}\right)
\left(\sum_{m\ge 0} p^{(s)}_m t^{m}\right)
- 
\left(\sum_{n\ge 0} p_n t^n\right)
\left(\sum_{m\ge 0} m p^{(s)}_m t^{m-1}\right)
= 0
\]
Notice that no $t^{-1}$ terms appear in these expansions although at
first glance it appears that they do.

Multiplying by $t$ and using the Cauchy multiplication formula gives: 
\[
\sum_{k\ge0} 
\left(\sum_{0\le \ell \le k}
   s \ell p_\ell p^{(s)}_{k-\ell} - (k-\ell) p_\ell p^{(s)}_{k-\ell}\right)
 t^k = 0.
\]
Since this formal power series is equal to zero, we can equate each of
the coefficients of $t^k$ to zero giving a system of linear equations
for $p^{(s)}_{\ell}$. 
\[
\begin{aligned}
s p_1 p^{(s)}_0 - p_0 p^{(s)}_1 & = 0, \\
2s p_2 p^{(s)}_0 + (s-1) p_1 p^{(s)}_1 - 2 p_0 p^{(s)}_2  & = 0, \\
3s p_3 p^{(s)}_0 + (2s-1) p_2 p^{(s)}_1 + (s-2) p_1 p^{(s)}_2 - 3 p_0 p^{(s)}_3 & = 0, \\
\vdots \\
ks p_k p^{(s)}_0
 + (2s-k+2) p_2 p^{(s)}_{k-2} + (s-k+1) p_1 p^{(s)}_{k-1} + \cdots
 - k p_0 p^{(s)}_k &= 0\\
\vdots
\end{aligned}
\]
Notice that the equations are triangular in $p_1^{(s)}, p_2^{(s)},
\ldots$.  Given $p_1^{(s)}, \ldots, p_{k-1}^{(s)}$, we have 
\[
\sum_{1\le \ell \le k}(\ell s - k + \ell) p_\ell p^{(s)}_{k-\ell}
= k p_0 p^{(s)}_k.
\]
\end{proof}

The preceding exponentiation formula
\eqnref{Powering:Formula:Eq} allows us to raise a zero valence power
series to any power whatsoever, not just integral powers.  In
particular, given a polynomial
\[
P(X) = p_0 + p_1 X + p_2 X^2 + \cdots + p_{rs} X^{rs},
\]
we can determine if it is a perfect $r$-th power by using
\eqnref{Powering:Formula:Eq} to compute the first $s+1$ terms of
\[
\left( p_0 + p_1 t + p_2 t^2 + \cdots + p_{rs} t^{rs} + o(t^{rs})\right)^{1/r}
\]
and raising the resulting {\em polynomial} to the $r$-th power.

Formal power series with non-zero valence are a bit more delicate.
Let $P(t)$ have valence $a$ and order $m$, so 
\[
\begin{aligned}
P(t) &= p_a t^a + p_{a+1} t^{a+1} + \cdots + p_m t^m + o(t^m), \\
  & = t^a \left[p_a + p_{a+1} t^{1} + \cdots + p_m t^{m-a} + o(t^{m-a})\right].
\end{aligned}
\]
Using \eqnref{Powering:Formula:Eq}, we can compute $P^s(t)$
\[
P^s(t) = t^{sa}
\left[p^{(s)}_a + p^{(s)}_{a+1} t^{1} + \cdots + p^{(s)}_m t^{m-a} + o(t^{m-a})\right].
\]
No problems arises if $sa$ is an integer (in particular, if $a = 0$).
Expanding this expression gives the parameters in
\figref{TPS:Valences:Fig}. 

However, if $sa$ is a rational number a new uniformizing parameter
will need to be introduced.  If $sa$ is neither an integer or a
rational number, then the result cannot be represented as a formal
power series. 

\section{Composition of Formal Power Series}
\label{FPS:Subst:Sec}

Assume we wish to compute the composition of two formal power series,
$P(t)$ and $Q(t)$.  For simplicity assume that $P$ has valence $0$.
Consider what occurs if $Q$ has valence $0$ and we simply replace $t$ in
$P(t)$ by $Q(t)$:
\[
(P\circ Q)(t) = \sum_{i\ge 0} p_i \left(q_0 + q_1 t + q_2 t^2 +
\cdots\right)^i.
\]
The constant term of $P\circ Q$ is itself an infinite series!  Thus if
we are using truncated power series, we will not be able to compute
$P\circ Q$.

On the other hand, assume that the valence of $Q$ is $1$ and write
\[
Q^k(t) = q^{(k)}_1 t + q^{(k)}_2 t^2 + q^{(k)}_3 t^3 + \cdots.
\]
Clearly, $q^{(j)}_i = 0$ if $i < j$ and $q^{(k)}_k = q_1^k$.  By 
\eqnref{Powering:Formula:Eq} $q^{(k)}_1, \ldots, q^{(k)}_n$ depend
only on $q_1, \ldots, q_n$. 

Replacing $t$ by $Q$ in the power series expansion of $P(t)$ and
rearranging the terms gives:
\[
\begin{aligned}
R(t) = (P\circ Q)(t)  = 
 p_0 &+ p_1 Q(t) + p_2 Q^2(t) + p_3 Q^3(t) + \cdots, \\
 = 
 p_0 &+ p_1 q^{(1)}_1 t + (p_1 q^{(1)}_2 + p_2 q^{(2)}_2)t^2 \\
   &+ (p_1 q^{(1)}_3 + p_2 q^{(2)}_3 + p_3 q^{(3)}_3)t^3 + \cdots.
\end{aligned}
\]
Generally, we have
\begin{equation} \label{FPS:Composition:Eq}
r_k = p_1 q^{(1)}_k + p_2 q^{(2)}_k + p_3 q^{(3)}_k + \cdots + p_k q^{(k)}_k.
\end{equation}
Thus, in this case, the order of $P\circ Q$ is the minimum of the
orders of $P$ and $Q$.  The valence of $P\circ Q$ is zero.

The question of valence in the general case is quite simple, $\val
P\circ Q = (\val P) \cdot (\val Q)$, unless the leading coefficient of
$Q$ is nilpotent.  However, the question of order is more subtle.
First, assume $P$ has zero valence and has the form
\[
P(t) = p_0 + p_{a_2} t^{a_2} + \cdots + p_m t^m + o(t^m).
\]
$a_2$ is the second smallest exponent of $P(t)$ that has a non-zero
coefficient, \ie, $a_2 = \val (P(t) - P(0))$.

Substituting $Q(t)$ for $t$ in $P(t)$ gives
\[
(P \circ Q)(t) = p_0 + p_{a_2}(q_b t^b + \cdots q_n t^n +
o(t^n))^{a_2}
+ \cdots +
p_m Q^m(t) + o(Q^m(t)).
\]
The order of $P\circ Q$ is the minimum of the orders of each of the
terms.  Since $\Ord Q^i(t) \le \Ord Q^j(t)$ for $i \le j$, we have
\[
\Ord P\circ Q = \min(a_2 b + n - b, mb).
\]
Notice that when the valence of $Q$ is zero the order of $P \circ Q$
is also zero, as expected. 

If the valence of $P$ is equal to $a$, $a \not= 0$, we can write
\[
P(t) = t^a \left[\hat{P}(t) + o(t^{m-a})\right],
\]
where $\hat{P}$ has valence zero.  The second smallest exponent of
$\hat{P}$ is $a_2 - a$, where $a_2$ is the second smallest exponent of
$P$.

We can now compute the order of each of these factors, when $t$ is
replaced by $Q(t)$ and then use the multiplication formula of
\figref{TPS:Valences:Fig} to determine the order of $P \circ Q$.  The
first factor is easy:
\[
\begin{aligned}
\Ord Q^a(t) &= ab + n - b, \\
\val Q^a(t) & = a b.
\end{aligned}
\]
But note that this is only valid if $a \not= 0$.  If $a = 0$ then
$\Ord Q^a(t) = \infty$.

For the second factor we can use the earlier results\Marginpar{These
results need to be checked}
\[
\Ord \hat{P} \circ Q = \min((a_2 - a)b + n - b, (m - a)b)
 = \min(a_2 b + n - b, mb) - ab,
\]
and $\val \hat{P}\circ Q = 0$.  So if $a\not=0$
\[
\begin{aligned}
\Ord P \circ Q & =
ab + \min(ab + n - b - ab, \min(a_2 b + n - b, mb) - ab), \\
 & = \min(ab + n - b, a_2 b + n - b, mb).
\end{aligned}
\]
Since $a_2 > a$, we have $\Ord P \circ Q = \min(ab + n - b, a_2 b + n
- b, mb)$.  These results can be summarized as
\[
\Ord P\circ Q = 
\begin{cases}
0 & \mbox{if $\val Q = 0$,} \\
\min(a_2 b + n - b, mb) & \mbox{if $\val P = 0$,} \\
\min(a b + n - b, mb) & \mbox{otherwise.} 
\end{cases}
\]

We can consolidate these cases by defining the {\em second
valence}\index{valence!second} as follows:
\[
\val_2 P = 
\begin{cases}
\val P & \mbox{if $\val P \not= 0$,} \\
\val (P(t) - P(0)) & \mbox{if $\val P = 0$.}
\end{cases}
\]
Then we have
\[
\Ord P \circ Q = \min((\val_2 P) \cdot (\val Q) + \Ord Q - \val Q,
(\Ord P) \cdot (\val Q)),
\]
as shown in \figref{TPS:Valences:Fig}.

\paragraph{Horner's Rule}

We now consider actually computing the coefficients of $P \circ Q$.
For simplicity, assume $\val P = 1$, $\val Q = b$ and $\Ord P = \Ord Q
= n$.  The most direct approach is to use \keyi{Horner's rule}, as was done in
the polynomial case in \sectref{Poly:Subs:Sec}.  We can, however,
reduce the number of coefficient operations somewhat by avoiding
computations that ultimately will not be needed.

Let $Q^{[r]}$ denote the power series of $Q(t)$ truncated to order
$r$
\[
Q^{[r]} = q_1 t^b + q_2 t^{b+1} + \cdots + q_r t^r + o(t^r).
\]
We can write $P\circ Q$ as
\begin{equation} \label{FPS:Horners:Eq}
P\circ Q = Q^{[n]}(p_1 + Q^{[n-b]}(p_2 + Q^{[n-2b]}(p_3 + \cdots ))).
\end{equation}
That this expression correctly computes all of the coefficients of $P
\circ Q$ is clear from its expanded form:
\[
P\circ Q = p_1 Q^{[n]} + p_2 Q^{[n]} Q^{[n-b]} + p_3 Q^{[n]} Q^{[n-b]}
Q^{[n-2b]} + \cdots.
\]
The order of each term is clearly $n$ by the product formula in
\figref{TPS:Valences:Fig}.  

Some extra terms are still computed.  To avoid this we would have to
compute 
\[
P\circ Q = p_1 Q^{[n]} + p_2 (Q^{[n-b]})^2 + p_3 (Q^{[n-2b]})^3 + \cdots,
\]
but this form could not be organized in the same form as ``Horner's rule.''

\paragraph{Matrix Form of Composition}

One way to avoid computing these extra terms is to use
\eqnref{FPS:Composition:Eq} to compute $P \circ Q$.  The space
required by this method is $O(n^2)$, while Horner's rule only requires
$O(n)$.  Thus this approach is not used for computation.  However,
when properly arranged the structure of this computation is quite
illuminating.  

We begin by defining a matrix representation for a formal power
series.\index{power series!matrix representation}  Assume that $P(t)$ has 
valence 1.  Define the infinite matrix 
\[
\mathbf{A}_P = 
\left(
\begin{array}{cccc}
p_1^{(1)} & p_2^{(1)} & p_3^{(1)} &  \cdots \\
 0        & p_2^{(2)} & p_3^{(2)} &  \cdots \\
 0        & 0         & p_3^{(3)} &  \cdots \\
\vdots    &           & \vdots    &  
\end{array}
\right).
\]
We claim that $\mathbf{A}_P \cdot \mathbf{A}_Q = \mathbf{A}_{P\circ Q}$, where the
first product is standard matrix multiplication.  

To see this, let $r_{i1}, r_{i2}, \ldots$ be the elements of the $i$-th row
of the product.  By the matrix product rule we have 
\[
r_{ij} = p^{(i)}_1 q^{(1)}_k + p^{(i)}_2 q^{(2)}_k + p^{(i)}_3 q^{(3)}_k
+ \cdots + p^{(i)}_k q^{(k)}_k.
\]
For the first row, $i = 1$, we have \eqnref{FPS:Composition:Eq}, so
the coefficients of $P \circ Q$ are the elements of the first row of
$\mathbf{A}_{P\circ Q}$.  The elements of lower rows are the coefficients
of $P^n \circ Q = (P \circ Q)^n$.  This proves 
\[
\mathbf{A}_P \cdot \mathbf{A}_Q = \mathbf{A}_{P\circ Q}.
\]
This relation is used in \sectref{FPS:Revert:Sec}.

\paragraph{Differential Equation Techniques}
\index{ordinary differential equations|(}

Finally, if $P(t)$ is a function that satisfies a differential
equation, and $P \circ Q$ needs to be computed frequently, it may be
worth while to derive a recursion relation for the coefficients of
$P\circ Q$ in terms of the coefficients of $Q$ as was done for
exponentiation.

We illustrate this process with the exponential function, $e^{Q(t)}$.
Let 
\[
R(t) = e^{Q(t)} = \sum _{m=0}^\infty r_m t^m,
\]
then 
\[
R'(t) = Q'(t) e^{Q(t)} = Q'(t) R(t).
\]
This is the key differential equation that $e^{Q(t)}$ satisfies.  

Proceeding as before we have 
\[
\sum_{k=0}^\infty k r_k t^{k-1} -
\left(\sum_{k\ge 0} r_k t^k\right)
\left(\sum_{k\ge 0} k q_k t^{k-1}\right) = 0.
\]
Multiplying by $t$ and combining terms we have 
\[
\sum_{k\ge 0}\left(k r_k - \sum_{0 \le \ell \le k} \ell a_{\ell}
b_{l-\ell}\right) = 0
\]
and thus 
\[
r_k = \sum_{1 \le \ell \le k} {\ell \over k} a_{\ell} b_{k-\ell}
\]
since the $\ell = 0$ term in the sum is $0$.  

Again we are unable to compute the $r_0$ term from the formula, but it
is still easy to derive by letting $t = 0$.  Then $R(0) = r_0 =
e^{q_0}$.

Similar formulas can be easily derived for functions that satisfy linear 
(not necessarily constant coefficient) ordinary differential equations.

\index{ordinary differential equations|)}

\section{Reversion of Power Series}
\label{FPS:Revert:Sec}

One operation on formal power series that is not possible with
polynomials is reversion.  Given two formal power series
\[
\begin{aligned}
P(t) & = p_1 t + p_2 t^2 + p_3 t^3 + \cdots, \\
Q(t) & = q_1 t + q_2 t^2 + q_3 t^3 + \cdots,
\end{aligned}
\]
such that $(P \circ Q)(t) = t$, $P$ is said to be the {\em
reversion}\index{reversion} of $Q$.

The simplest way to compute $P$ from $Q$ is to use the method of
undetermined coefficients to set up a system of equations for the
$p_i$ in terms of the $q_j$.  Using \eqnref{FPS:Composition:Eq} we
have
\[
\begin{aligned}
p_1 q_1 &= 1, \\
p_1 q_2 + p_2 q^{(2)}_2 &= 0, \\
p_1 q_3 + p_2 q^{(2)}_3 + p_3 q^{(3)}_3 &= 0, \\
\vdots\\
p_1 q_k + p_2 q^{(2)}_k + p_3 q^{(3)}_k + \cdots + p_k q^{(k)}_k &= 0,\\
\vdots
\end{aligned}
\]
where the $q^{(s)}_i$ is the coefficient of $t^i$ in $Q^s(t)$.
These equations are linear in the $p_i$ and have a unique solution as
long as $q^{(k)}_k = q_1^k$ does not vanish.

A particularly nice way to organize this computation is to use the
matrix representation of a power series of the previous section.  If
$(P\circ Q)(t) = t$ then $\mathbf{A}_P = \mathbf{A}_Q^{-1}$.  The matrix
$\mathbf{A}_Q$ is constructed row by row, multiplying each by $Q(t)$ to
produce the next row.  Since the matrix is upper triangular, the
inverse can be computed by using just the ``back-solving'' phase of
Gaussian elimination.  \Marginpar{Need to put in more detail here.}

\section*{Notes}

\small

This chapter discusses only truncated power series.  {\Koepf} has
discussed a mechanism for generating a general formula for the
$k$-th term in a power series in \cite{Koepf1992-kr}.\Marginpar{An
interesting paper to look at by Niven \cite{Niven1969-sz}?  Also there 
are some results on conjugacy of power series in Scheinberg
\cite{Scheinberg1970-nk}.}

\notesectref{FPS:Intro:Sec} Formal power series have been successfully used to
compute the limits of functions in Macsyma since 1977.  Recent work
on this approach, which is more sophisticated, is contained in the
work of {\Shackell} \cite{Shackell1990-ws}.

The extensions to power series that were introduced where to deal with
problems of singularities.  The negative exponents dealt with poles,
while the fractional exponents dealt with algebraic branch points.
The logarithms cover the issue of logarithmic branch points.  The
remaining type of singularities of functions of a single complex variable
is the essential singularity, which cannot be dealt with easily.  The
essential advances of {\Shackell} and {\SalvyB} mentioned above, is their
manner of dealing with essential singularities.  A uniform mechanism
for dealing with essential singularities in local analysis is a major
remaining problem in symbolic computing.

\notesectref{FPS:Arith:Sec}
If the coefficient field of a power series domain supports the
\key{fast Fourier transform} then an FFT approach to multiplication of
truncated power series of order $n$ can be used as follows: Consider
the two power series as polynomials, and compute their product using
FFT based polynomial multiplication.  This gives a polynomial of
degree $2n$, but the first $n+1$ terms are the coefficients of the
product.  This approach theoretically reduces the cost of power series
multiplication from $O(n^2)$ to $O(n \log n)$ \cite{Knuth1997-tf,Borodin1975-de}.

\notesectref{FPS:Expt:Sec} Formula \eqnref{Powering:Formula:Eq} was
known to {\Euler} \cite{Euler1748-zr} and may be even older.  There
are many derivations known now.  An elegant proof using properties of
double sums is due to {\Holdt} \cite{Von_Holdt1965-rx}.

If the coefficient domain of the power series supports the fast
Fourier transform then FFT polynomial multiplication and repeated
squaring can be used to improve the cost of computing $P^s(t)$ from
$O(n^2)$ for the classical approach to $O(n (\log n)(\log s))$.  In
practice this does not seem to be effective.

A different algorithm for computing the $1/r$-th root of a polynomial 
was introduced by {\Yun} \cite{Yun1976-pr}.  This operation has also been 
used to determine when a polynomial can be written as the composition of 
two polynomials \cite{Kozen1989-ns,Von_zur_Gathen1987-xe}.

\notesectref{FPS:Subst:Sec} and {\bf \S\ref{FPS:Revert:Sec}}
Composition of power series defined as solutions of differential
equations is discussed by {\Norman} \cite{Norman1975-xu}.

\notesectref{FPS:Subst:Sec} and {\bf \S\ref{FPS:Revert:Sec}}
Asymptotically good algorithms are given by {\Brent} and {\Kung}
\cite{Brent1978-zp} for both substitution and composition.

\normalsize

