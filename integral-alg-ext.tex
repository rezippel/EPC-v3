\chapter{Computation in Integral Extensions}
\label{Algebraic:Integers:Chap}

In this section we generalize the concept of the rational integers to the
integers of an algebraic extension. Intuitively, an integer is something
that doesn't have a denominator.  For elements of $\Q$, this means that
they are zeroes of the polynomials $x - a$, where $a$ is an element of
$\Z$.  Consider $k$, an algebraic extension of $\Q$.  Each element in $k$ is
the zero of a minimal polynomial with coefficients in $\Z$:
\[
p_{\alpha}(x) = p_{0} x^{n} + \cdots + p_{n}.
\]
If $p_{\alpha}(x)$ is monic then $\alpha$ is said to be an \keyi{integer}
of $k$.  Denote the set of integers of $k$ by $A$.  It is not hard to show
that $A$ is a ring and that $A \cap \Q = \Z$.

More generally, let $A$ be a ring and $k$ the quotient field of $A$, and
$L$ an extension of $k$.  The minimal polynomial of an element $\alpha$ in
$L$ can be written with coefficients in $A$.  If the minimal polynomial is
monic, then $\alpha$ is said to be {\em integral}\index{integral, algebraic
element} over $A$.  The set of all elements of $L$ that are integral over
$A$ is called the \keyi{integral closure} of $A$ in $L$.  A ring that is
its own integral closure in its quotient field is said to be {\em
integrally closed\/}.  In the following standard diagram $B$ is the
integral closure of $A$ in $L$.
\[
\begin{diagram}
\node{L} \arrow{s,-} \arrow{e,-} \node{B} \arrow{s,-}\\
\node{k} \arrow{e,-} \node{A}
\end{diagram}
\]
There may be several rings between $B$ and $A$ which are integral over $A$.
For instance, we have the following diagram
\[
\begin{diagram}
\node{\Q[\sqrt{5}]} \arrow{s,-} \arrow{e,-} 
    \node{\Z[\phi] = \Z[\frac{1+\sqrt{5}}{2}]} \arrow{s,-} \\
\node{\Q} \arrow{e,-} \node{\Z}
\end{diagram}
\]
Clearly, $\Z[\sqrt{5}]$ is integral over $\Z$, but does not contain $\phi$
which is also integral over $\Z$.  The integral closure of $\Z$ in
$\Q[\sqrt{5}]$ is $\Z[\phi]$.  Generally, a ring $B$ is said to be {\em
integrally closed} in a field $L$ if every element of $L$ that is integral
over $B$ is actually an element of $B$.

The \keyi{different} of $K/k$ if defined to be $p^{\prime}(\alpha)$.  The
\keyi{discriminant} of the extension is defined to be the norm of the
different.  Thus the discriminant is $\res_x(p'(x), p(x))$.  If $L$ is an
algebraic extension of $K$ then 
\[
\Dscr_{L/k} = \Norm_{K/k}(\Dscr_{L/K}) \Dscr_{K/k}.
\]
Similarly, $\Norm_{L/k} = \Norm_{L/K} \Norm_{K/k}$.


\section{Rational Prime Factorization}

The following discussion covers standard ground.  The results are due to
{\Dedekind} \cite{Lang:ANT,Dedekind:Primes} although the treatment here is
different.  If $p$ is a prime element of $\Z$ then it generates a principal
prime ideal $(p) = p \Z$ and all the the prime ideals of $\Z$ are generated
in this manner.  The prime ideals of $\Z[x]$ are:
\begin{list}{$\bullet$}{\parsep=0pt}
\item $(p)$ where $p$ is a prime in $\Z$,
\item $(f(x))$ where $f(x)$ is irreducible in $\Z[x]$, 
\item  $(p, f(x))$ where $f(x)$ is irreducible modulo $p$.
\end{list}
Let $k$ be an algebraic extension of $\Q$ and let $A$ be the integral
closure of $\Z$ in $k$.  Assume $A = \Z[\alpha]$ and that $f(x)$ is the
minimal polynomial for $\alpha$.  There is an isomorphism $\Z[x]/(p(x))
\simeq \Z[\alpha]$ and thus the canonical map $\varphi:\Z[x]/(p(x))
\rightarrow \Z[\alpha]$.  The corresponding map of the spectrums
$\varphi^{\ast}:\Z[\alpha]/(p(x)) \rightarrow \Z[x]$ is a homeomorphism
onto the divisors of
\[
(f(x)) = \left\{\,(p, f_p(x)) \mid
\hbox{$f_p(x)$ irreducible and  $f_{p}(x)$ divides$f(x)$ modulo $p$}\,\right\}.
\]

\index{Dedekind}
Considering the sequence of maps $\Z \rightarrow \Z[x] \rightarrow
\Z[\alpha]$ we see that the factorization of the ideal $p \Z[\alpha]$
corresponds to the factorization of $f(x)$ modulo $p$.   This may be
generalized somewhat.  This is Dedekind's theorem \cite{Lang:ANT}:

\begin{proposition}
Let $A$ be a Dedekind ring with quotient field $K$.  Let $E$ be a
finite separable extension of $K$.  Let $B$ be the integral closure of
$A$ in $E$ and assume $B = A[\alpha]$ for some $\alpha$.  Let $f(X)$
be the irreducible polynomial of $\alpha$ over $K$.  Let $\pger$ be a
prime of $A$.  Let $\bar f$ be the reduction of $f$ with respect to
$\pger$ and let $\bar f(X) = \Pger_1^{e_1}(X) \cdots
\Pger_r^{e_r}(X)$ be the factorization of $\bar f$ into monic
irreducibles over $A = A/\pger$ then $\pger B = \Pger_1^{e_1} \cdots
\Pger_r^{e_r}$ where $\Pger_i = \pger B + P_i(\alpha) B$ where $P_i$
is a monic polynomial whose reduction modulo $p$ is $\bar P_i$.
\end{proposition}

Consider $\alpha$ a root of $f(x) = x^5 - x + 1$.  Its discriminant is
$2869 = 19 \cdot 151$ so $\alpha$ is integral over $\Z$\Marginpar{Why is
$\alpha$ integral?} and generates the integral closure of $\Z$ in
$\Q(\alpha)$.  The only ramified primes are 19 and 151.  Using the
algorithms of \chapref{Poly:Arith:Chap} it is not hard to determine that
modulo 151 $f(x)$ factors into
\[
11 (x - 9) (31 x - 1)^2 (x^2 - 64 x + 61),
\]
which is easily translated into the monic factorization
\[
(x - 9) (x - 39)^2 (x^2 - 64 x + 61).
\]
Thus the prime $(151)$ in $\Z$ splits into three primes in $\Z[\alpha]$:
${\frak P}_{1} = (151, \alpha - 9)$, ${\frak P}_{2} = (151, \alpha-39)$ and
${\frak P}_{3} = (151, \alpha^{2} - 64 \alpha +61)$.  The prime ${\frak
P}_{2}$ has ramification index $2$ over $(151)$.


\section{Integral Bases}

Let $L$ be a separable extension of $k$ of degree $n$, and let $\vec
\omega$ be the generators of a $k$-module $M \subseteq L$.  There are $n$
automorphisms of $L$ that fix $k$, $\sigma_{1}, \ldots, \sigma_{n}$.  Each
of these maps $M$ into a new module generated by $\sigma_{i}(\vec \omega)$.


\section{Kronecker's Theorem}

Let $\alpha$ be an algebraic number over $K \supseteq \Q$ with minimal
polynomial $P(X)$.  We can extend the $M$-norm of
\chapref{PBounds:Chap} to algebraic numbers by $M(\alpha) = M(P)$.
For an algebraic integer, $P(X)$ is monic and $M(\alpha)$ is the
product of the conjugates of $\alpha$ outside of the unit disk.  A
famous theorem of {\Kronecker} \cite{Kronecker57} states taht if all
of the conjugates of an algebraic integer lie inside (or on
the boundary) of the unit disk then $\alpha$ is an $n${\th} root of
unity.  This is easily seen as follows.

\begin{proposition}[Kronecker]
Let $\alpha$ be an algebraic integer of degree $d$ over $\Z$.   If
$M(\alpha) = 1$ then some power of $\alpha$ is equal to $1$. 
\end{proposition}

\begin{proof}
Let $P$ be the minimal polynomial for $\alpha$, so $M(P) = M(\alpha) =
1$.  Define the associated by polynomials $P_k$ as in
\sectref{Graeffe:Bound:Sec}, where their zeroes are the $k${\th} power
of the zeroes of $P(X)$.  We have $M(P_k) = M(P)^k = 1$.   By
\propref{Uni:Coef:MP:Prop}, we have
\[
2^{-d} |P_k| \le M(P_k),
\]
so we have $|P_k| \le 2^d$.  Since the coefficients of the $P_k$ are
bounded, there must be an $i$ and $j$ such that $P_i = P_j$ and thus
$\alpha^i = \alpha^j$ and $\alpha^{|i-j|} = 1$.
\end{proof}


Here is a proof ripped off from Vorlesungen Uber Zahlentheorie Vol. III
Aus der Algebraischen Zahlentheorie by Edmund Landau pg. 225 
(Chelsea Publishing Company)

{\bf Theorem 918} For $m>0$, all coefficients of
$$
x^m + a_1 x^{m-1} + \cdots + a_m
$$
\noindent are integers.  All roots $\xi_1$, \dots, $\xi_m$ of this 
polynomial have absolute value $1$.  The roots are roots of unity.

\vskip .25in
\noindent {\bf Proof:}

For each fixed integer $l$ the numbers $\xi_k^l$ ($k=1, \dots, m$),
by the theorem on symmetric functions, are roots of the polynomial

$$
x^m + a_{l1} x^{m-1} + \cdots + a_{lm} \eqno (1)
$$
\noindent with integer $a$.

Landau expects you to see this immediately.  In case you don't remember
I will give more details.
\[
\begin{aligned}
(x-\xi_1^l)&(x-\xi_2^l)\ldots (x-\xi_m^l)  \\
&= x^m - (\xi_1^l+\xi_2^l+\cdots +\xi_m^l)x^{m-1}
+ ((\xi_1\xi_2)^l+\cdots)x^{m-2}+ \cdots \pm
(\xi_1\xi_2\ldots\xi_m)^l. 
\end{aligned}
\]
\noindent The first term, by the Newton formula for powers, is a combination
of the integers $a_1$, $a_2$, \dots, $a_m$.  The rest of the coefficients
are combinations (with integers) of these same integers.  Now on with
Landau's proof.

\noindent Because 
$$|\xi_k^l| = 1$$
\noindent for $t=1$, \dots, $m$
$$|a_{lt}| \le {m \choose t}.$$
\noindent The number of terms of the symmetric polynomials in
$\xi_1^l$,~$\xi_2^l$, \dots,~$\xi_m^l$ are the numbers ${m \choose t}$.
Since ${m \choose t}$ is free of $l$, (1) 
represents only finitely many polynomials; $\xi_k^l$ represents for each fixed
$k$ only finitely many numbers.  We then have $\xi_k^{l_1}=\xi_k^{l_2}$ for
$1\le l_1 < l_2$.  Finally $\xi_k^{l_2-l_1}=1$.


{\LehmerD} \cite{LehmerD33} asked how close to the unit disk the
conjugates of an algebraic number could be.  More precisely, define
\[
c(d) = \min_{\deg \alpha = d} M(\alpha),
\]
where $\alpha$ ranges over algebraic integers of degree $d$.  The best
current estimate of $c(d)$ is
\[
c(D) -1 > (2.25 - \epsilon) \left(\frac{\log \log d}{\log d}\right)^3
\] 
for sufficiently large $d$, by \cite{Louboutin83}.  Earlier results
are due to \cite{Dobrowolski79,CantorD82,Mignotte77}.
