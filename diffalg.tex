%$Id: /usr/u/rz/AMBook/RCS/diffalg.tex,v 1.1 1992/05/10 19:38:47 rz Exp rz $
\chapter{Differential Algebra}
\label{Diff:Alg:Chap}

\index{derivation}\index{differential field}
Let ${\cal F}$ be a field, with an operator $D : {\cal F} \rightarrow
{\cal F}$.  If, for all $a$ and $b$ in ${\cal F}$ we have
\[
D(a+b) = Da + Db \qquad\mbox{and}\qquad
D(ab) = D(a)b + a D(b)
\]
then $D$ is called a {\em derivation\/}.  A field with a derivation is
called a {\em differential field\/}.  Denote the elements of ${\cal F}$ whose
derivatives are zero by ${\cal C}$.  ${\cal C}$ is a subfield which we
call the {\em field of constants} of ${\cal F}$.

The field of rational functions in $\theta$ over the differential field
${\cal F}$ can be turned into a differential field by adjoining the
elements $\theta' = D(\theta)$, $\theta'' = D(\theta')$, $\theta^{(3)} =
D(\theta'')$ and so on.  This field, which we denote by ${\cal
F}\langle\theta\rangle$ has infinite transcendence degree over ${\cal F}$.
It is conventional to use script letters for differential fields.


\section{Simplification of Transcendental Functions}

\begin{proposition}[Ostrowski-Kolchin]
Let ${\cal F}$ be a differential field with a differention operator
$D$ and field of constants ${\cal C}$.  Also, let ${\cal G} = {\cal
F}\langle \nu_1, \ldots, \nu_m\rangle$, where each of the $\nu_i$ is
primitive.  If the $\nu_1, \ldots, \nu_m$ are algebraically dependent
over ${\cal F}$ then there exist constants $c_1, \ldots, c_m \in {\cal
C}$, not all $0$, such that 
\[
c_1 \nu_1 + \cdots + c_m \nu_m \in {\cal F}
\]
\end{proposition}


\section{Schanuel's Conjectures}

\section{Ax's Result}



